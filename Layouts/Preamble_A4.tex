\documentclass[oneside,open=right,headings=optiontohead]{book}
\renewcommand{\baselinestretch}{1.0}  %行間距倍率
\columnsep 7mm
\usepackage{setspace}

\usepackage[
a4paper=true,
unicode=true,
bookmarksnumbered,
bookmarksopen,
hyperfigures=true,
hyperindex=true,
pdfpagelayout = SinglePage,
pdfpagelabels = true,
pdfstartview = Fit,
colorlinks,
pdfborder=001,
linkcolor=black,
anchorcolor=black,
citecolor=black,
pdftitle={明朝那些事儿},
pdfauthor={当年明月},
pdfsubject={明朝那些事儿(全七卷)},
pdfkeywords={本书主要讲述的是从1344年到1644年这三百年间关于明朝的一些事情,以史料为基础,以年代和具体人物为主线,并加入了小说的笔法,对明朝十七帝和其他王公权贵和小人物的命运进行全景展示,尤其对官场政治、战争、帝王心术着墨最多,并加入对当时政治经济制度、人伦道德的演义。},
pdfcreator={https://m-mono.github.io}
]{hyperref}

\hypersetup{pdfduplex=DuplexFlipLongEdge}% Set print default as duplex
\usepackage{parskip}% replace \\ at end of line to avoid underfull \hbox error

\usepackage{graphics,graphicx,pdfpages}
\usepackage{caption} %用于取消标题编号 \caption*{abc}

%自动加注拼音
\usepackage{xpinyin}
\xpinyinsetup{format={\color{PinYinColor}}}
%手动加注外语及日语振假名
\usepackage{ruby}
\renewcommand\rubysize{0.4} %匹配 xpinyin 默认标注字体大小
\renewcommand\rubysep{-0.3em} %匹配 xpinyin 默认标注高度

\usepackage{xeCJK}
\usepackage{indentfirst}
\setlength{\parindent}{2.0em}

%正文字体
\usepackage{fontspec}
\defaultfontfeatures{Extension = .otf}% adds .otf to end of path when font loaded without ext parameter e.g.
\setCJKmainfont[Path=Fonts/,
BoldFont={SourceHanSerifSC-Bold.otf},
ItalicFont={SourceHanSerifSC-Regular.otf},
BoldItalicFont={SourceHanSerifSC-Bold.otf},
SlantedFont={SourceHanSerifSC-Regular.otf},
BoldSlantedFont={SourceHanSerifSC-Bold.otf},
SmallCapsFont={SourceHanSerifSC-Regular.otf}
]{SourceHanSerifSC-Regular.otf}
\setCJKsansfont[Path=Fonts/]{SourceHanSansSC-Regular.otf}
\setCJKmonofont[Path=Fonts/]{SourceHanSansSC-Regular.otf}
\setmainfont[Path=Fonts/]{SourceHanSerifSC-Regular.otf}
\setsansfont[Path=Fonts/]{SourceHanSansSC-Regular.otf}
\setmonofont[Path=Fonts/]{SourceHanSansSC-Regular.otf}
% Icon 字体
\usepackage{fontawesome5}

% 頁面及文字顏色
\usepackage{xcolor}
\definecolor{TEXTColor}{RGB}{50,50,50} % TEXT Color
\definecolor{PinYinColor}{RGB}{180,180,180} % TEXT Color
\definecolor{NOTEXTColor}{RGB}{0,0,0} % No TEXT Color
\definecolor{BGColor}{RGB}{240,240,240} % BG Color
\definecolor{Gray}{RGB}{246,246,246}

\usepackage{qrcode}
\usepackage{multicol}

\makeindex
\renewcommand{\contentsname}{{明朝那些事儿}}
\usepackage{fancyhdr} % 設置頁眉頁腳
\pagestyle{fancy}
%\addtolength{\headwidth}{\marginparsep}
\addtolength{\headwidth}{\marginparwidth}
\fancyhf{} % 清空當前設置
\renewcommand{\headrulewidth}{0pt}  %頁眉線寬,設為0可以去頁眉線
\renewcommand{\footrulewidth}{0pt}  %頁眉線寬,設為0可以去頁眉線
