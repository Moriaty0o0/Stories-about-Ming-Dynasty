\fancyhead[LO]{{\scriptsize \faBookmark\ 明朝那些事儿 \faAngleRight\ \textbf{\rightmark}}}%奇數頁眉的左邊
\fancyhead[RO]{{\tiny{\textcolor{Gray}{\faQuoteRight\ }}}\thepage}
\fancyhead[LE]{{\tiny{\textcolor{Gray}{\faQuoteRight\ }}}\thepage}
\fancyhead[RE]{{\scriptsize \faBookmark\ 明朝那些事儿 \faAngleRight\ \textbf{\rightmark}}}%偶數頁眉的右邊
\fancyfoot[LE,RO]{}
\fancyfoot[LO,CE]{}
\fancyfoot[CO,RE]{}
\setcounter{part}{5}
\setcounter{chapter}{6}
\setcounter{section}{0}
\part{日暮西山}
\chapter*{朱翊钧篇}
\addcontentsline{toc}{chapter}{朱翊钧篇}
\section{绝顶的官僚}
\ifnum\theparacolNo=2
	\begin{multicols}{\theparacolNo}
		\fi
		在万历执政的前二十多年里,可谓是内忧不止,外患不断,他祖上留传下来的,也只能算是个烂摊子,而蒙古、宁夏、朝鲜、四川,不是叛乱就是入侵,中间连口气都不喘,军费激增,国库难支。

		可是二十年了,国家也没出什么大乱子,所有的困难,他都安然度过。

		因为前十年,他有张居正,后十年,他有申时行。

		若评选明代三百年历史中最杰出的政治家,排行榜第一名非张居正莫属。在他当政的十年里,政治得以整顿,经济得到恢复,明代头号政治家的称谓实至名归。

		但如果评选最杰出的官僚,结果就大不相同了,以张居正的实力,只能排第三。

		因为这两个行业是有区别的。

		从根本上讲,明代政治家和官僚是同一品种,大家都是在朝廷里混的,先装孙子再当爷爷,半斤对八两。但问题在于,明代政治家是理想主义者,混出来后就要干事,要实现当年的抱负。

		而明代官僚是实用主义者,先保证自己的身份地位,能干就干,不能干就混。

		所以说,明代政治家都是官僚,官僚却未必都是政治家。两个行业的技术含量和评定指标各不相同,政治家要能干,官僚要能混。

		张居正政务干得好,且老奸巨滑,工于心计,一路做到首辅,混得也还不错。但他死节不保,死后被抄全家,差点被人刨出来示众,所以只能排第三。

		明代三百年中,在这行里,真正达到登峰造极的水平,混到惊天地、泣鬼神的,当属张居正的老师,徐阶。

		混迹朝廷四十多年,当过宰相培训班学员\footnote{庶吉士。},骂过首辅\footnote{张璁。},发配地方挂职\footnote{延平推官。},好不容易回来,靠山又没了\footnote{夏言。},十几年被人又踩又坑,无怨无悔,看准时机,一锤定音,搞定\footnote{严嵩。}。

		上台之后,打击有威胁的人\footnote{高拱。},提拔有希望的人\footnote{张居正。},连皇帝也要看他的脸色,事情都安排好了,才安然回家欢度晚年,活到了八十一岁,张居正死了他都没死,如此人精,排第一是众望所归。

		而排第二的,就是张居正的亲信兼助手:申时行。

		相信很多人并不认同这个结论,因为在明代众多人物中,申时行并不是个引人瞩目的角色,但事实上,在官僚这行里,他是一位身负绝学,超级能混的绝顶高手。

		无人知晓,只因隐藏于黑暗之中。

		在成为绝顶官僚之前,申时行是一个来历不明的人,具体点讲,是身世不清,父母姓甚名谁,家族何地,史料上一点儿没有,据说连户口都缺,基本属于黑户。

		申时行是一个十分谨小慎微的人,平时有记日记的习惯。即使是微不足道的小事,如今天我和谁说了话,讲了啥,他都要记下来,比如他留下的《召对录》,就是这一类型的著作。

		此外,他也喜欢写文章,并有文集流传后世。

		基于其钻牛角尖的精神,他的记载是研究明史的重要资料。然而奇怪的是,对于自己的身世,这位老兄却是只字不提。

		这是一件比较奇怪的事,而我是一个好奇的人,于是,我查了这件事。

		遗憾的是,虽然我读过很多史书,也翻了很多资料,依然没能找到史料确凿的说法。

		确凿的定论没有,不确凿的传言倒有一个,而在我看来,这个传言可以解释以上的疑问。

		据说\footnote{注意前提。}嘉靖十四年时,有一位姓申的富商到苏州游玩,遇上了一位女子,两人一见钟情,便住在了一起。

		过了一段时间,女方怀孕了,并把孩子生了下来,这个孩子,就是后来的申时行。

		可是在当时,这个孩子不能随父亲姓申,因为申先生有老婆。

		当然了,在那万恶的旧社会,这似乎也不是什么违法行为,以申先生的家产,娶几个老婆也养得起,然而还有一个更麻烦的问题——那位女子不是一般人,确切地说,是一个尼姑。

		所以,在百般无奈之下,这个见不得光的私生子被送给了别人。

		爹娘都没见过,就被别人领养,这么个身世,确实比较不幸。

		但不幸中的万幸是,这个别人,倒也并非普通人,而是当时的苏州知府徐尚珍。他很喜欢这个孩子,并给他取了一个名字——徐时行。

		虽然当时徐知府已离职,但在苏州干过知府,只要不是海瑞,一般都不会穷。

		所以徐时行的童年非常幸福,从小就不缺钱花,丰衣足食,家教良好。而他本人悟性也很高、天资聪慧,二十多岁就考上了举人,人生对他而言,顺利得不见一丝波澜。

		但惊涛骇浪终究还是来了。

		嘉靖四十一年(1562年),徐时行二十八岁,即将上京参加会试,开始他一生的传奇。

		然而就在他动身前夜,徐尚珍找到了他,对他说了这样一句话:

		其实,你不是我的儿子。

		没等徐时行的嘴合上,他已把之前所有的一切都和盘托出,包括他的生父和生母。

		这是一个十分古怪的举动。

		按照现在的经验,但凡考试之前,即使平日怒目相向,这时家长也得说几句好话,天大的事情考完再说,徐知府偏偏选择这个时候开口,实在让人费解。

		然而我理解了。

		就从现在开始吧,因为在你的前方,将有更多艰难的事情在等待着你,到那时,你唯一能依靠的人,只有你自己。

		这是一个父亲,对即将走上人生道路的儿子的最后祝福。

		徐时行沉默地上路了。我相信,他应该也是明白的,因为在那一年会试中,他是状元。

		中了状元的徐时行回到了老家,真相已明,恩情犹在,所以他正式提出要求,希望能够归入徐家。

		辛苦养育二十多年,而今状元及第,衣锦还乡,再认父母,收获的时候到了。

		然而出乎所有人的意料,他的父亲拒绝了这个请求,希望他回归本家,认祖归宗。

		很明显,在这位父亲的心中,只有付出,没有收获。

		无奈之下,徐时行只得怀着无比的歉疚与感动,回到了申家。

		天上终于掉馅饼了,状元竟然都有白捡的。虽说此时他的生父已经去世,但申家的人毫不犹豫地答应了他的请求,敲锣打鼓,张灯结彩地把他迎进了家门。

		从此,他的名字叫做申时行。

		曲折的身世,幸福的童年,从他的养父身上,申时行获取了人生中的第一个重要经验,并由此奠定了他性格的主要特点:

		做人,要厚道。

		然后当厚道的申时行进入朝廷后,才发现原来这里的大多数人都很不厚道。

		在明代,只要进了翰林院,只要不犯什么严重的政治错误,几年之后,运气好的就能分配到中央各部熬资格,有才的入阁当大学士,没才的也能混个侍郎、郎中,就算点背,派到了地方,官也升得极快,十几年下来,做个地方大员也不难。

		有鉴于此,每年的庶吉士都是各派政治势力极力拉拢的对象。申时行的同学里,但凡机灵点的,都已经找到了后台,为锦绣前程做好准备。

		申时行是状元,找他的人自然络绎不绝,可这位老兄却是巍然不动,谁拉都不去,每天埋头读书,毫不顾及将来的仕途。同学们一致公认,申时行同志很老实,而从某个角度讲,所谓老实,就是傻。

		然而事情的发展证明,老实人终究不吃亏。

		要知道,那几年朝廷是不好混的,先是徐阶斗严嵩,过几年,高拱上来斗徐阶,然后张居正又出来斗高拱,总而言之是一塌糊涂。今天是七品言官,明天升五品郎中,后天没准就回家种田去了。

		你方唱罢我登场,上台洗牌是家常便饭,世事无常,跟着谁都不靠谱,所以谁也不跟的申时行笑到了最后。当他的同学纷纷投身朝廷拼杀的时候,他却始终呆在翰林院,先当修撰,再当左庶子。中间除了读书写文件外,还主持过几次讲学\footnote{经筵。},教过一个学生,叫做朱翊钧,又称万历。

		俗语有云,长江后浪推前浪,前浪死在沙滩上。一晃十年过去,经过无数清洗,到万历元年,嘉靖四十一年的这拨人,冲在前面的,基本上都废了。

		就在此时,一个人站到了申时行的面前,对他说,跟着我走。

		这一次,申时行不再沉默,他同意了。

		因为这个人是张居正。

		申时行很老实,但不傻。这十年里,他一直在观察,观察最强大的势力,最稳当的后台,现在,他终于等到了。

		此后他跟随张居正,一路高歌猛进,几年内就升到了副部级礼部侍郎,万历五年(1577年),他又当上了吏部侍郎,一年后,他迎来了自己人生的第二个转折点。

		万历六年(1578年),张居正的爹死了,虽说他已经获准夺情,但也得回家埋老爹。为保证大权在握,他推举年仅四十三岁的申时行进入内阁,任东阁大学士。

		历经十几年的苦熬,申时行终于进入了大明帝国的最高决策层。

		但是当他进入内阁后,他才发现,自己在这里只起一个作用——凑数。

		因为内阁的首辅是张居正,这位仁兄不但能力强,脾气也大,平时飞扬跋扈,是不折不扣的猛人。

		一般说来,在猛人的身边,只有两个选择,要么当敌人,要么当仆人。

		申时行毫不犹豫地选择了后者,他很明白,像张居正这种狠角色,只喜欢一种人——听话的人。

		申时行够意思,张居正也不含糊,三年之内,就把他提为吏部尚书兼建极殿大学士,少傅兼太子太傅\footnote{从一品。}。

		但在此时的内阁里,申时行还只是个小字辈,张居正且不说,他前头还有张四维、马自强、吕调阳,一个个排过去,才能轮到他。距离那个最高的位置,依然是遥不可及。

		申时行倒也无所谓,他已经等了二十年,不在乎再等十年。

		可他万万没有想到,不用等十年,一年都不用。

		万历十年(1582年)张居正死了。

		树倒猢狲散。隐忍多年的张四维接班,开始反攻倒算,重新洗牌,局势对申时行很不利,因为地球人都知道他是张居正的亲信。

		在这关键时刻,申时行第一次展现了他无与伦比的“混功”。

		作为内阁大学士,大家弹劾张居正,他不说话;皇帝下诏剥夺张居正的职务,他不说话;抄张居正的家,他也不说话。

		但不说话,不等于不管。

		申时行是讲义气的,抄家抄出人命后,他立即上书,制止情况进一步恶化。还分了一套房子,十倾地,用来供养张居正的家属。

		此后,他又不动声色地四处找人做工作,最终避免了张先生被人从坟里刨出来示众。

		张四维明知申时行不地道,偏偏拿他没办法。因为此人办事一向是滴水不漏,左右逢源,任何把柄都抓不到。

		但既然已接任首辅,收拾个把人应该也不太难,在张四维看来,他有很多时间。

		然而事与愿违,张首辅还没来得及下手,就得到了一个消息——他的父亲死了。

		死了爹,就得丁忧回家,张四维不愿意。当然,不走倒也可以,夺情就行,但五年前张居正夺情的场景还历历在目。考虑到自己的实力远不如张居正,且不想被人骂死,张四维毅然决定,回家蹲守。

		三年后,又是一条好汉。

		此时,老资格的吕调阳和马自强都走了,申时行奉命代理首辅,等张四维回来。

		一晃两年半过去了,眼看张先生就要功德圆满,胜利出关,却突然病倒了。病了还不算,两个月后,竟然病死了。

		上级都死光了,进入官场二十三年后,厚道的老好人申时行,终于超越了他的所有同学,走上了首辅的高位。

		一个新的时代,将在他的手中开始。

		\subsection{取胜之道}
		就工作能力而言,申时行是十分卓越的,虽说比张居正还差那么一截,但在他的时代,却是最为杰出的牛人。

		因为要当牛人,其实不难,只要比你牛的人死光了,你就是最牛的牛人。

		就好比你上世纪三十年代和鲁迅见过面,给胡适鞠过躬,哪怕就是个半吊子,啥都不精,只要等有学问、知道你底细的那拨人都死绝了,也能弄顶国学大师的帽子戴戴。

		更何况申时行所面对的局面,比张居正时要好得多:首先他是皇帝的老师,万历也十分欣赏这位新首辅;其次,他很会做人,平时人缘也好,许多大臣都拥戴他;加上此时他位极人臣,当上了大领导,一切似乎都在他的掌握之中。

		不过,只是似乎而已。

		所谓朝廷,就是江湖。即使身居高位,扫平天下,也绝不会缺少对手。因为在这个地方,什么都会缺,就是不缺敌人。

		张四维死了,但一个更为强大的敌人,已经出现在他的面前。

		而这个敌人,是万历一手造就的。

		张居正死后,万历得到了彻底的解放。没人敢管他,也没人能管他,所有权力终于回到他的手中。他准备按自己的意愿去管理这个帝国。

		但在此之前,他还必须做一件事。

		按照传统,打倒一个人是不够的,必须把他彻底搞臭,消除其一切影响,才算是善莫大焉。

		于是,一场批判张居正的活动就此轰轰烈烈展开。

		张居正在世的时候,吃亏最大的是言官。不是罢官,就是打屁股,日子很不好过,现在时移势易,第一个跳出来的自然也就是这些人。

		万历十二年(1584年)三月,御史丁此吕首先发难,攻击张居正之子张嗣修当年科举中第,是走后门的关系户云云。

		这是一次极端无聊的弹劾,因为张嗣修中第,已经是猴年马月的事,而张居正死后,他已被发配到边远山区充军。都折腾到这份上了,还要追究考试问题,是典型的没事找事。

		然而事情并非看上去那么简单,事实上,这是一个设计周密的阴谋。

		丁此吕虽说没事干,却并非没脑子,他十分敏锐地察觉到,只要对张居正问题穷追猛打,就能得到皇帝的宠信。

		这一举动还有另一个更阴险的企图:当年录取张嗣修的主考官,正是今天的首辅申时行。

		也就是说,打击张嗣修,不但可以获取皇帝的宠信,还能顺道收拾申时行,把他拉下水,一箭双雕,十分狠毒。

		血雨腥风就此而起。

		申时行很快判断出了对方的意图,他立即上书为自己辩解,说考卷都是密封的,只有编号,没有姓名,根本无法舞弊。

		万历支持了他的老师,命令将丁此吕降职调任外地,大家都松了一口气。

		然而这道谕令的下达,才是暴风雨的真正开端。

		明代的言官中,固然有杨继盛那样的孤胆英雄,但大多数情况下,都是团伙作案。一个成功言官的背后,总有一拨言官。

		丁此吕失败了,于是幕后黑手出场了,合计三双。

		这三个人的名字,分别是李值、江东之,羊可立。在我看来,这三位仁兄是名副其实的“骂仗铁三角”。

		之所以给予这个荣誉称号,是因为他们不但能骂,还很铁。

		李、江、羊三人,都是万历五年(1577年)的进士。原本倒也不熟,自从当了御史后,因为共同的兴趣和事业\footnote{骂人。}走到了一起,在战斗中建立了深厚的友谊,并成为了新一代的搅屎棍。

		之所以说新一代,是因为在他们之前,也曾出过三个极能闹腾的人,即大名鼎鼎的刘台、赵用贤、吴中行。这三位仁兄,当年曾把张居正老师折腾得只剩半条命,十分凑巧的是,他们都是隆庆(1571年)五年的进士,算是老一代的铁三角。

		但这三个老同志都还算厚道人,大家都捧张居正,他们偏骂,这叫义愤。后来的三位,大家都不骂了,他们还骂,这叫投机。

		丁此吕的奏疏刚被打回来,李植就冲了上去,枪口直指内阁的申时行。还把管事的吏部尚书杨巍搭了上去,说这位人事部长逢迎内阁,贬低言官。

		话音没落,江东之和羊可立就上书附和,一群言官也跟着凑热闹,舆论顿时沸沸扬扬。

		对于这些举动,申时行起先并不在意:丁此吕已经滚蛋了,你们去闹吧,还能咋地?

		然而出人意料的事情发生了。几天以后,万历下达了第二道谕令,命令丁此吕留任,并免除应天主考高启愚\footnote{负责出考题。}的职务。

		这是一个十分危险的政治信号。

		其实申时行并不知道,对于张居正,万历的感觉不是恨,而是痛恨。这位曾经的张老师,不但是一个可恶的夺权者,还是笼罩在他心头上的恐怖阴影。

		支持张居正的,他就反对,反对张居正的,他就支持!无论何人、何时、何种动机。

		这才是万历的真正心声,上次赶走丁此吕,不过是给申老师一个面子,现在面子都给过了,该怎么来,咱还怎么来。

		申时行明白,大祸就要临头了:今天解决出考题的,明天收拾监考的,杀鸡儆猴的把戏并不新鲜。

		情况十分紧急,但在这关键时刻,申时行却表现出了让人不解的态度,他并不发文反驳,对于三位御史的攻击,保持了耐人寻味的沉默。

		几天之后,他终于上疏,却并非辨论文书,而是辞职信。

		就在同一天,内阁大学士许国、吏部尚书杨巍同时提出辞呈,希望回家种田。

		这招以退为进十分厉害,刑部尚书潘季驯、户部尚书王璘、左都御史赵锦等十余位部级领导纷纷上疏,挽留申时行。万历同志也手忙脚乱,虽然他很想支持三位骂人干将,把张居正整顿到底,但为维护安定团结,拉人干活,只得再次发出谕令,挽留申时行等人,不接受辞职。

		这道谕令有两个意思,首先是安慰申时行,说这事我也不谈了,你也别走了,老实干活吧。

		此外,是告诉江、羊、李三人,这事你们干得不错,深得我心\footnote{否则早就打屁股了。},但到此为止,以后再说。

		事情就此告一段落,然而之后的发展告诉了我们,这一切,只不过是热身运动。

		问题的根源,在于“铁三角”。科场舞弊事件完结后,这三位拍对了马屁的仁兄都升了官:江东之升任光禄寺少卿,李植任太仆寺少卿,羊可立为尚宝司少卿。

		太仆寺少卿是管养马的,算是助理弼马温,正四品。光禄寺少卿管吃饭宴请,是个肥差,正五品。尚宝司少卿管公章文件,是机要部门,从五品。

		换句话说,这三个官各有各的好处,却并不大,可见万历同志心里有谱:给你们安排好工作,小事来帮忙,大事别掺和。

		这三位兄弟悟性不高,没明白其中的含义,给点颜色就准备开染坊。虽然职务不高,权力不大,却都很有追求,可谓是手攥两块钱,心怀五百万,欢欣鼓舞之余,准备接着干。

		而这一次,他们吸取了上次的教训,打算捏软柿子,将矛头对准了另一个目标——潘季驯。

		可怜潘季驯同志,其实他并不是申时行的人。说到底,不过是个搞水利的技术员,高拱在时,他干,张居正在时,他也干,是个标准的老好人,无非是看不过去,说了几句公道话,就成了打击对象。

		话虽如此,但此人一向人缘不错,又属于特殊科技人才,还干着司法部部长\footnote{刑部尚书。},不是那么容易搞定的。

		可是李植只用了一封奏疏,就彻底终结了他。

		这封奏疏彻底证明了李先生的厚黑水平,非但绝口不提申时行,连潘技术员本人都不骂。只说了两件事——张居正当政时,潘季驯和他关系亲密,经常走动,张居正死后抄家,他曾几次上书说情。

		这就够了。

		申时行的亲信,不要紧;个人问题,不要紧;张居正的同伙,就要命了。

		没过多久,兢兢业业的潘师傅就被革去所有职务,从部长一踩到底,回家当了老百姓。

		这件事干得实在太过龌龊,许多言官也看不下去了。御史董子行和李栋分别上书,为潘季驯求情,却被万历驳回,还罚了一年工资。

		有皇帝撑腰,“铁三角”越发肆无忌惮,把战火直接烧到了内阁的身上,而且下手也特别狠,明的暗的都来。先是写匿名信,说大学士许国安排人手,准备修理李植、江东之。之后又明目张胆地弹劾申时行的亲信,不断发起挑衅。

		部长垮台,首辅被整,闹到这个份上,已经是人人自危,鬼才知道下个倒霉的是谁。连江东之当年的好友,刑科给事中刘尚志也憋不住了,站出来大吼一声:

		“你们要把当年和张居正共事过的人全都赶走,才肯干休吗\footnote{尽行罢斥而后已乎。}?!”

		然而让人费解的是,在这片狂风骤雨之中,有一个人却始终保持着沉默。

		面对漫天阴云,申时行十分之镇定,既不吵,也不闹,怡然自得。

		这事要换在张居正头上,那可就了不得了。以这位仁兄的脾气,免不了先回骂两句,然后亲自上阵,罢官、打屁股,搞批判,不搞臭搞倒誓不罢休。刘台、赵用贤等人,就是先进典型。

		就能力与天赋而言,申时行不如张居正,但在这方面,他却远远地超越了张先生。

		申首辅很清楚,张居正是一个不折不扣的政务天才。而像刘台、江东之这类人,除了嘴皮子利索,口水旺盛外,干工作也就是个白痴水平。和他们去较真,那是要倒霉的,因为这帮人会把对手拉进他们的档次,并凭借自己在白痴水平长期的工作经验,战胜敌人。

		所以在他看来,李植、江东之这类人,不过是跳梁小丑,并无致命威胁,无须等待多久,他们就将露出破绽。

		所谓宽宏大量,胸怀宽广之外,只因对手档次太低。

		然而“铁三角”似乎没有这个觉悟,万历十三年(1585年)八月,他们再一次发动了进攻。

		事情是这样的,为了给万历修建陵墓,申时行前往大峪山监督施工,本打算打地基,结果挖出了石头。

		在今天看来,这实在不算个事,把石头弄走就行了。可在当时,这就是个掉脑袋的事。

		皇帝的陵寝,都是精心挑选的风水宝地,要保证皇帝大人死后,也得躺得舒坦,竟然挑了这么块石头地,存心不让皇上好好死,是何居心?

		罪名有了,可申时行毕竟只是监工,要把他拉下水,必须要接着想办法。

		经过一番打探,办法找到了:原来这块地是礼部尚书徐学谟挑的,这个人不但是申时行的亲家,还是同乡。很明显,他选择这块破地,给皇上找麻烦,是有企图的,是用心不良的,是受到指使的。

		只要咬死两人的关系,就能把申时行彻底拖下水。而这帮野心极大的人,也早已物色好了首辅的继任者,只要申时行被弹劾下台,就立即推荐此人上台,并借此控制朝局,这就是他们的计划。

		然而这个看似万无一失的计划,却有两个致命的破绽。

		几天之后,三人同时上疏,弹劾陵墓用地选得极差,申时行玩忽职守,任用私人,言辞十分激烈。

		在规模空前的攻击面前,申时行却毫不慌张,只是随意上了封奏疏说明情况,因为他知道,这帮人很快就要倒霉了。

		一天之后,万历下文回复:

		“阁臣\footnote{指申时行。}是辅佐政务的,你们以为是风水先生吗\footnote{岂责以堪舆。}!?”

		怒火中烧的万历骂完之后,又下令三人罚俸半年,以观后效。

		三个人被彻底打懵了,他们抓破脑袋,也想不明白这是怎么回事。

		归根结底,还是信息工作没有到位。这几位仁兄晃来晃去,只知道找地的是徐学谟,却不知道拍板定位置的,是万历。

		皇帝大人好不容易亲自出手挑块地,却被他们骂得一无是处,不出口气实在说不过去。

		不过还好,毕竟算是皇帝的人,只是罚了半年的工资,励精图治,改日再整。

		可还没等这三位继续前进,背后却又挨了一枪。

		在此之前,为了确定申时行的接班人选,三个人很是费了一番脑筋,反复讨论,最终拍板——王锡爵。

		这位王先生,之前也曾出过场。张居正夺情的时候,上门逼宫,差点把张大人搞得横刀自尽,是张居正的死对头,加上他还是李植的老师,没有更适合的人选了。

		看上去是那么回事,可惜有两点,他们不知道:

		其一,王锡爵是个很正派的人,他不喜欢张居正,却并非张居正的敌人。

		其二,王锡爵是嘉靖四十一年进士,考试前就认识了老乡申时行,会试,他考第一,申时行考第二,殿试,他考第二,申时行第一。
		\begin{quote}
			\begin{spacing}{0.5}  %行間距倍率
				\textit{{\footnotesize
							\begin{description}
								\item[\textcolor{Gray}{\faQuoteRight}] 没有调查研究,就没有发言权。——毛泽东
							\end{description}
						}}
			\end{spacing}
		\end{quote}

		基于以上两点,得知自己被推荐接替申时行之后,王锡爵递交了辞职信。

		这是一封著名的辞职信,全称为《因事抗言求去疏》,并提出了辞职的具体理由:

		老师不能管教学生,就该走人\footnote{当去。}!

		这下子全完了,这帮人虽说德行不好,但毕竟咬人在行,万历原打算教训他们一下后,该怎么样还怎么样。

		可这仨太不争气,得罪了内阁、得罪了同僚,连自己的老师都反了水,再这么闹腾,没准自己都得搭进去,于是他下令,江东之、李植、羊可立各降三级,发配外地。

		家犬就这么变成了丧家犬,不动声色之间,申时行获得了最终的胜利。
		\ifnum\theparacolNo=2
	\end{multicols}
\fi
\newpage
\section{和稀泥的艺术}
\ifnum\theparacolNo=2
	\begin{multicols}{\theparacolNo}
		\fi
		对申时行而言,江东之这一类人实在是小菜一碟。在朝廷里呆了二十多年,徐阶、张居正这样的超级大腕他都应付过去了,混功已达出神入化的地步,万历五年出山的这帮小喽罗自然不在话下。

		混是一种生活技巧,除个别二杆子外,全世界人民基本都会混。因为混并不影响社会进步,人类发展,该混就混,该干就干,只混不干的,叫做混混。

		申时行不是混混,混只是他的手段,干才是他的目的。

		一般说来,新官上任,总要烧三把火,搞点政绩,大干特干,然而综观申时行当政以来的种种表现,就会惊奇地发现,他的大干,就是不干。他的作为,就是不作为。

		申时行干的第一件事情,是废除张居正的考成法。

		这是极为出人意料的一招,因为在很多人看来,申时行是张居正的嫡系,毫无理由反攻倒算。

		但申时行就这么干了,因为这样干,是正确的。

		考成法,是张居正改革的主要内容,工作指标层层落实,完不成轻则罢官,重则坐牢,令各级官员威风丧胆。

		在很长时间里,这种明代的打考勤,发挥了极大效用,有效提高了官员的工作效率,是张居正的得意之作。

		但张先生并不知道,这种考成法,有一个十分严重的缺陷。

		比如朝廷规定,户部今年要收一百万两税银,分配到浙江,是三十万,这事就会下派给户部浙江司郎中\footnote{正五品。},由其监督执行。

		浙江司接到命令,就会督促浙江巡抚办理。巡抚大人就会去找浙江布政使,限期收齐。

		浙江布政使当然不会闲着,立马召集各级知府,限期收齐。知府大人回去之后召集各级知县,限期收齐。

		知县大人虽然官小,也不会自己动手,回衙门召集衙役,限期收齐。

		最后干活的,就是衙役,他们就没办法了,只能一家一家上门收税。

		明朝成立以来,大致都是这么个办法,就管理学而言,还算比较合理,搞了两百多年,也没出什么大问题。

		考成法一出来,事情就麻烦了。

		原先中央下达命令,地方执行,就算执行不了,也好商量。三年一考核,灾荒大,刁民多,今年收不齐,不要紧,政策灵活掌握,明年努力,接着好好干。

		考成法执行后,就不行了,给多少任务,你就得完成多少,短斤少两自己补上,补不上就下课受罚。

		这下就要了命了,衙役收不齐,连累知县,知县收不齐,连累知府,知府又连累布政使,一层层追究责任,大家同坐一条船,出了事谁也跑不掉。

		与其自下而上垮台,不如自上而下压台。随着一声令下,各级官吏纷纷动员起来,不问理由,不问借口,必须完成任务。

		于是顺序又翻了过来,布政使压知府,知府压知县,知县压衙役,衙役……,就只能压老百姓了。

		接下来的事情就简单了,上级压下级,下级压百姓。一般年景,也还能对付过去,要遇上个灾荒,那就惨了,衙役还是照样上门,说家里遭灾,他点头,说家里死人,他还点头,点完头该交还得交。揭不开锅也好,全家死绝也罢,收不上来官就没了,你说我收不收?

		以上还算例行公事,到后来,事情越发恶劣。

		由于考成法业绩和官位挂钩,工作完成越多,越快,评定就越好,升官就越快。所以许多地方官员开始报虚数,狗不拉屎的穷乡僻壤,也敢往大了报,反正自己也不吃亏。

		可是朝廷不管那些,报了就得拿钱。于是挨家挨户地收,收不上来就逼,逼不出来就打,打急了就跑。而跑掉的这些人,就叫流民。

		流民,是明代中后期的一个严重问题。用今天的话说,就是社会不安定因素,这些人离开家乡,四处游荡,没有户籍,没有住所,也不办暂住证,经常影响社会的安定团结。

		到万历中期,流民数量已经十分惊人。连当时的北京市郊,都盘踞着大量流民。而且这帮人一般都不是什么老实巴交的农民,偷个盗抢个劫之类的,都是家常便饭。朝廷隔三差五就要派兵来扫一次,十分难办。

		而这些情况,是张居正始料未及的。

		于是申时行毅然废除了考成法,并开辟了大量田地,安置各地的流民耕种,社会矛盾得以大大缓解。

		废除考成法,是申时行执政的一次重要抉择。虽然是改革,却不用怎么费力,毕竟张居正是死人兼废人,没人帮他出头,他的条令不废白不废。

		但下一次,就没这么便宜的事了。

		万历十八年(1590年),总兵李联芳带兵在边界巡视的时候,遭遇埋伏,全军覆灭。下黑手的,是蒙古鞑靼部落的扯立克。

		事情闹大了,因为李联芳是明军高级将领,鞑靼部落把他干掉了,是对明朝政府的严重挑衅。所以消息传来,大臣们个个摩拳擦掌,打算派兵去收拾这帮无事生非的家伙。

		无论从哪个角度看,都是非打不可了,堂堂大明朝,被人打了不还手,当缩头乌龟,怎么也说不过去。而且这事闹得皇帝都知道了,连他都觉得没面子,力主出兵。

		老板发话,群众支持,战争已是势在必行,然而此时,申时行站了出来,对皇帝说:

		“不能打。”

		在中国历史上,但凡国家有事,地方被占了,人被杀了,朝廷总就是群情激奋,人人喊打,看上去个个都是民族英雄,正义化身,然而其中别有奥秘:

		临战之时,国仇家恨,慷慨激昂,大家都激动。在这个时候,跟着激动一把,可谓是毫无成本,反正仗也不用自己打,还能落个名声,何乐而不为。

		主和就不同了,甭管真假,大家都喊打,你偏不喊,脱离群众,群众就会把你踩死。

		所以主战者未必勇,主和者未必怯。

		主和的申时行,就是一个勇敢的人。事实证明,他的主张十分正确。

		因为那位下黑手的扯立克,并不是一般人,他的身份,是鞑靼的顺义王。

		顺义王,是当年明朝给俺答的封号,这位扯立克就是俺答的继任者。但此人即不顺,也不义,好好的互市不干,整天对外扩张,还打算联合蒙古、西藏各部落,搞个蒙古帝国出来和明朝对抗。

		对这号人,打是应该的。但普鲁士伟大的军事家克劳塞维茨说过,战争是政治的继续,打仗说穿了,最终的目的就是要对方听话,如果有别的方法能达到目的,何必要打呢?

		申时行找到了这个方法。

		他敏锐地发现,扯立克虽然是顺义王,但其属下却并非铁板一块。由各个部落组成,各有各的主张,大多数人和明朝生意做得好好的,压根不想打仗,如果贸然开战,想打的打了,不想打的也打了,实在是得不偿失。分化瓦解才是上策。

		所以申时行反对。

		当然,以申时行的水平,公开反对这种事,他是不会干的。夜深人静,独自起草,秘密上交,事情干得滴水不漏。

		万历接到奏疏,认可了申时行的意见,同意暂不动兵,并命令他全权处理此事。

		消息传开,一片哗然,但皇帝说不打,谁也没办法找皇帝算帐。申时行先生也是一脸无辜:我虽是朝廷首辅,但皇帝不同意,我也没办法。

		仗是不用打了,但这事还没完。申时行随即下令兵部尚书郑洛,在边界集结重兵,也不大举进攻,每天就在那里蹲着。别的部落都不管,专打扯立克,而且还专挑他的运输车队下手,抢了就跑。

		这种打法毫无成本,且收益率极高,明军乐此不疲,扯立克却是叫苦不迭,实在撑不下去了,只得率部躲得远远的,就这样,不用大动干戈,不费一兵一卒,申时行轻而易举地解决了这个问题,恢复了边境的和平。

		虽然张居正死后,朝局十分复杂,帮派林立,申时行却凭借着无人能敌的“混功”,应对自如,游刃有余。更为难能可贵的是,他不但自己能混,还无私地帮助不能混的同志,比如万历。

		自从登基以来,万历一直在忙两件事,一是处理政务,二是搞臭张居正,从某种意义上讲,这两件事,其实是一件事。

		因为张居正实在太牛了,当了二十六年的官,十年的皇帝\footnote{实际如此。},名气比皇帝还大,虽然人死了,茶还烫的冒泡,所以不搞臭张居正,就搞不好政务。

		但要干这件事,自己是无从动手的,必须找打手,万历很快发现,最好的打手,就是言官和大臣。

		张居正时代,言官大臣都不吃香,被整得奄奄一息,现在万历决定,开闸,放狗。

		事实上,这帮人的表现确实不错,如江东之、李植、羊可立等人,虽说下场不怎么样,但至少在工作期间,都尽到了狗的本分。

		看见张居正被穷追猛打,万历很高兴,看见申时行被牵连,万历也不悲伤,因为在他看来,这不过是轻微的副作用,敲打一下申老师也好,免得他当首辅太久,再犯前任\footnote{张居正。}的错误。

		他解放言官大臣,指挥自若,是因为他认定,这些人将永远听从他的调遣。

		然而他并不知道,自己犯下了一个多么可怕的错误。因为就骂人的水平而言,言官大臣和街头骂街大妈,只有一个区别:大妈是业余的,言官大臣是职业的。

		大妈骂完街后,还得回家洗衣做饭,言官大臣骂完这个,就会骂下一个。所以,当他们足够壮大之后,攻击的矛头将不再是死去的张居正,或是活着的申时行,而是至高无上的皇帝。

		对言官和大臣们而言,万历确实有被骂的理由。

		自从万历十五年(1587年)起,万历就不怎么上朝了,经常是“偶有微疾”,开始还真是“偶有”,后来就变成常有,“微疾”也逐渐变成“头晕眼黑,力乏不兴”,总而言之,大臣们是越来越少见到他了。

		必须说明的是,万历是不上朝,却并非不上班,事情还是要办,就好比说你早上起床,不想去单位,改在家里办公,除了不打考勤,少见几个人外,也没什么不同,后世一说到这位仁兄,总是什么几十年不干活之类,这要么是无意的误解,要么是有意的污蔑。

		在中国当皇帝,收益高,想要啥就有啥,但风险也大,屁股上坐的那个位置,只要是人就想要,但凡在位者,除了个把弱智外,基本上都是怀疑主义者,见谁怀疑谁,今天这里搞阴谋,明天那里闹叛乱,日子过得那叫一个悬,几天不看公文,没准刀就架在脖子上了。

		万历自然也不例外,事实上,他是一个权力欲望极强,工于心计的政治老手,所有的人都只看到他不上朝的事实,却无人察觉背后隐藏的奥秘:

		在他之前,有许多皇帝每日上朝理政,费尽心力,日子过得极其辛苦,却依然是脑袋不保,而他几十年不上朝,谁都不见,却依然能够控制群臣,你说这人厉不厉害?

		但言官大臣是不管这些的,在他们的世界观里,皇帝不但要办事,还要上班,哪怕屁事没有,你也得坐在那,这才叫皇帝。

		万历自然不干,他不干的表现就是不上朝,言官大臣也不干,他们不干的表现就是不断上奏疏。此后的几十年里,他们一直在干同样的事情。

		万历十四年(1586年)十月,这场长达三十余年的战争正式拉开序幕。

		当时的万历,基本上还属于上朝族,只是偶尔罢工而已,就这样,也没躲过去。

		第一个上书的,是礼部祠祭司主事卢洪春,按说第一个不该是他,因为这位仁兄主管的是祭祀,级别又低,平时也不和皇帝见面。

		但这一切并不妨碍他上书提意见,他之所以不满,不是皇帝不上朝,而是不祭祀。

		卢洪春是一个很负责的人,发现皇帝不怎么来太庙,又听说近期经常消极怠工,便上书希望皇帝改正。

		本来是个挺正常的事,却被他搞得不正常。因为这位卢先生除了研究礼仪外,还学过医,有学问在身上,不显实在对不起自己,于是发挥专业特长,写就奇文一篇,送呈御览。

		第二天,申时行奉命去见万历,刚进去,就听到了这样的一句话:

		“卢洪春这厮!肆言惑众,沽名讪上,好生狂妄!着锦衣卫拿在午门前,着实打六十棍!革了职为民当差,永不叙用!”

		以上言辞,系万历同志之原话,并无加工。

		很久很久以前,这厮两个字就诞生了,在明代的许多小说话本中,也频频出现,其意思依照现场情况,有各种不同的解释,从这家伙、这小子、到这混蛋,这王八蛋,不一而同。

		但可以肯定的是,这两字不是好话,是市井之徒的常用语,皇帝大人脱口而出,那是真的急了眼了。

		这是可以理解的,因为卢洪春的那篇奏疏,你看你也急。

		除了指责皇帝陛下不该缺席祭祀外,卢主事还替皇帝陛下担忧其危害:

		“陛下春秋鼎盛,精神强固,头晕眼黑之疾,皆非今日所宜有。”

		年纪轻轻就头晕眼黑,确实是不对的,确实应该注意,到此打住,也就罢了。

		可是担忧完,卢先生就发挥医学特长:

		“医家曰:气血虚弱,乃五劳七伤所致,肝虚则头晕目眩,肾虚则腰痛精泄。”

		气血虚弱,肝虚肾虚,症状出来了,接着就是分析原因:

		“以目前衽席之娱,而忘保身之术,其为患也深。”

		最经典的就是这一句。

		所谓衽席之娱,是指某方面的娱乐,相信大家都能理解,综合起来的意思是:

		皇帝你之所以身体不好,在我看来,是因为过于喜欢某种娱乐,不知收敛保养,如此下去,问题非常严重。

		说这句话的,不是万历他妈,不是他老婆,不是深更半夜交头接耳,天知地知,你知我知,而是一个管礼仪的六品官,在大庭广众之下公开上书,且一言一语皆已千古流传。

		再不收拾他,就真算白活了。

		命令下达给了申时行,于是申时行为难了。

		这位老油条十分清楚,如果按照万历的意思严惩卢洪春,言官们是不答应的;如果不处理,万历又不答应。

		琢磨半天,想了个办法。

		他连夜动笔,草拟了两道文书,第一道是代万历下的,严厉斥责卢洪春,并将其革职查办。第二道是代内阁下的,上奏皇帝,希望能够宽恕卢洪春,就这么算了。

		按照他的想法,两边都不得罪,两边都有交代。

		事实证明,这是幻想。

		首先发作的是万历。这位皇帝又不是傻子,一看就明白申时行耍两面派,立即下令,即刻动手打屁股,不得延误。此外他还不怀好意地暗示,午门很大,多个人不嫌挤。

		午门就是执行廷杖的地方,眼看自己要去垫背,申时行随即更改口风,把卢洪春拉出去结结实实地打了六十棍。

		马蜂窝就这么捅破了。

		言官们很惭愧。一个礼部的业余选手,都敢上书,勇于曝光皇帝的私生活,久经骂阵的专业人才竟然毫无动静,还有没有职业道德?

		于是大家群情激奋,以给事中杨廷相为先锋,十余名言官一拥而上,为卢洪春喊冤翻案。

		面对漫天的口水和奏疏,万历毫不退让,事实上,这是一个极端英明的抉择:一旦让步,从宽处理了卢洪春,那所谓“喜欢某种娱乐,不注意身体”的黑锅,就算是背定了。

		但驳回去一批,又来一批。言官们踊跃发言,热烈讨论,反正闲着也是闲着,不说白不说。

		万历终于恼火了,他决定罚款,带头闹事的主犯罚一年工资,从犯八个月。

		对言官而言,这个办法很有效果。

		在明代,对付不同类别的官员,有不同的方法:要折腾地方官,一般都是降职。罚工资没用,因为这帮人计划外收入多,工资基本不动,罚光了都没事。

		言官就不同了,他们都是靠死工资的,没工资日子就没法过,一家老小只能去喝西北风,故十分害怕这一招。

		于是风波终于平息,大家都消停了。

		但这只是表面现象,对此,申时行有很深的认识。作为天字第一号混事的高手,他既不想得罪领导,又不想得罪同事,为实现安定团结,几十年如一日地和稀泥,然而随着事件的进一步发展,他逐渐意识到,和稀泥的幸福生活长不了。

		因为万历的生活作风,是一天不如一天了。

		事实上,卢洪春的猜测很可能是正确的,二十多岁的万历之所以不上朝,应该是沉迷于某种娱乐。否则实在很难解释,整天在宫里呆着,到底有啥乐趣可言。

		说起来,当年张居正管他也实在管得太紧。啥也不让干,吃个饭喝点酒都得看着。就好比高考学生拼死拼活熬了几年,一朝拿到录取通知书,革命成功,自然就完全解放了。

		万历同志在解放个人的同时,也解放了大家。火烧眉毛的事情\footnote{比如打仗,阴谋叛乱之类。},看一看,批一批,其余的事,能不管就不管,上朝的日子越来越少。

		申时行很着急,但这事又不好公开讲,于是他灵机一动,连夜写就了一封奏疏。在我看来,这封文书的和稀泥技术,已经达到了登峰造极的地步。

		文章大意是这样的:

		皇帝陛下,我听说您最近身体不好,经常头晕眼花\footnote{时作晕眩。},对此我十分担心。我知道,您这是劳累所致啊!由于您经常熬夜工作,亲历亲为\footnote{一语双关,佩服。},才会身体不好。为了国家,希望您能够清心寡欲,养气宁神\footnote{原文用词。},好好保重身体。

		高山仰止,自惭形秽之感,油然而生。

		对于这封奏疏,万历还是很给了点面子。他召见了申时行,表示明白他的苦心,良药虽然苦口,却能治病,今后一定注意。申时行备感欣慰,兴高采烈地走了。

		但这只是错觉,因为在这个世界上,能够药到病除的药只有一种——毒药。

		事实证明,万历确实不是一般人。因为一般人被人劝,多少还能改几天,他却是一点不改,每天继续加班加点,从事自己热爱的娱乐。据说还变本加厉,找来了十几个小太监,陪着一起睡\footnote{同寝。},也算是开辟了新品种。

		找太监这一段,史料多有记载,准确性说不好,但有一点是肯定的,那就是万历同志依旧是我行我素,压根儿不给大臣们面子。

		既然不给脸面,那咱就有撕破脸的说法。

		万历十七年十二月,明代,不,是中国历史上胆最大、气最足的奏疏问世了!其作者,是大理寺官员雒于仁。

		雒于仁,字少泾,陕西泾阳人。纵观明清两代,陕西考试不大行,但人都比较实在。既不慷慨激昂,也不罗罗嗦嗦,说一句是一句,天王老子也敢顶。比如后世的大贪污犯和珅,最得意的时候,上有皇帝撑腰,下有大臣抬轿。什么纪晓岚、刘墉,全都服服帖帖,老老实实靠边站,所谓“智斗”之类,大都是后人胡编的,可谓一呼百应。而唯一不应的,就是来自陕西的王杰。每次和珅说话,文武百官都夸,王杰偏要顶两句,足足恶心了和珅十几年,又抓不到他的把柄,也只能是“厌之而不能去”。\footnote{清史稿。}

		雒于仁就属于这类人,想什么说什么,从不怕得罪人,而且他的这个习惯,还有家族传统:

		雒于仁的父亲,叫做雒遵,当年曾是高拱的学生,干过吏科都给事中。冯保得势的时候,骂过冯保;张居正得势的时候,骂过谭纶\footnote{张居正的亲信。},为人一向高傲,平生只佩服一人,名叫海瑞。

		有这么个父亲,雒于仁自然不是孬种。加上他家虽世代为官,却世代不捞钱,穷日子过惯了,光脚的不怕穿鞋的。不怕罚工资,不怕降职,看不惯皇帝了,就要骂。随即一挥而就,写下奇文一篇,后世俗称为《酒色财气疏》。

		该文主旨明确,开篇即点明中心思想:

		“陛下之恙,病在酒色财气者也,夫纵酒则溃胃,好色则耗精,贪财则乱神,尚气则损肝。”

		这段话用今天的话讲,就是说皇上你确实有病,什么病呢?你喜欢喝酒,喜欢玩女人,喜欢捞钱,还喜欢动怒耍威风,酒色财气样样俱全,自然就病了。

		以上是全文的论点,接下来的篇幅,是论据,描述了万历同志在喝酒玩女人方面的具体体现,逐一论证以上四点的真实性和可靠性,比较长,就不列举了。

		综观此文,下笔之狠,骂法之全,真可谓是鬼哭狼嚎。就骂人的狠度和深度而言,雒于仁已经全面超越了海瑞前辈,雒遵同志如果在天有灵,应该可以瞑目了。

		更缺德的是,雒于仁的这封奏疏是十二月\footnote{农历。}底送上去的,搞得万历自从收到这封奏疏,就开始骂,不停地骂,没日没夜地骂,骂得新年都没过好。

		骂过瘾后,就该办人了。

		万历十八年(1590年)正月初一,按照规矩,内阁首辅应该去宫里拜年。当然也不是真拜,到宫门口鞠个躬就算数。但这一次,申时行刚准备走人,就被太监给叫住了。

		此时,雒于仁的奏疏已经传遍内外,申先生自然知道怎么回事,不用言语就进了宫。看到了气急败坏的皇帝,双方展开了一次别开生面的对话:\footnote{以下言语,皆出自申时行的原始记录。}

		万历:先生看过奏本\footnote{指雒于仁的那份。},说朕酒色财气,试为朕评一评。

		申时行:……\footnote{还没说话,即被打断。}

		万历:“他说朕好酒,谁人不饮酒?……又说朕好色,偏宠贵妃郑氏\footnote{即著名的郑贵妃。},朕只因郑氏勤劳……何曾有偏?”

		喘口气,接着说:

		“他说朕贪财……朕为天子,富有四海之内,普天之下莫非王土,天下之财皆朕之财!又说朕尚气……勇即是气,朕岂不知!人孰无气!”

		这口气出完了,最后得出结论:

		“先生将这奏本去票拟重处!”

		申时行这才搭上话:

		“此无知小臣误听道路之言……\footnote{说到此处,又被打断。}”

		万历大喝一声:

		“他就是出位沽名!”

		申时行傻眼了,他在朝廷混了几十年,从未见过这幅场景,皇帝大人一副吃人的模样,越说越激动,唾沫星子横飞,这样下去,恐怕要出大事。

		于是他闭上了嘴,开始紧张地思索对策。

		既不能让皇帝干掉雒于仁,也不能不让皇帝出气,琢磨片刻,稀泥和好了。

		“他\footnote{指雒于仁。}确实是为了出名\footnote{先打底。},但陛下如果从重处罚他,却恰恰帮他成了名,反损皇上圣德啊!”

		“如果皇上宽容,不和他去一般见识,皇上的圣德自然天下闻名\footnote{继续戴高帽。}!”

		在这堆稀泥面前,万历同志终于消了气:

		“这也说得是,如果和他计较,倒不是损了朕的德行,而是损了朕的气度!”

		上钩了,再加最后一句:

		“皇上圣度如天地一般,何所不容!”\footnote{圆满收工。}

		万历沉默地点了点头。

		话说到这,事情基本就算完了,申时行定定神,突然想起了另一件事,一件极为重要的事。

		他决定趁此机会,解决此事。

		然而他正准备开口,却又听见了一句怒斥:

		“朕气他不过,必须重处!”

		万历到底是年轻人,虽然被申时行和了一把稀泥,依然不肯干休,这会回过味来,又绕回去了。

		这事还他娘没完了,申时行头疼不已,但再头疼事情总得解决,如果任由万历发作胡来,后果将不堪设想。

		在这关键的时刻,申时行再次展现了他举世无双的混事本领,琢磨出了第二套和稀泥方案:

		“陛下,此奏本\footnote{雒于仁。}原本就是讹传,如果要重处雒于仁,必定会将此奏本传之四方,反而做了实话啊!”

		利害关系说完,接下来该掏心窝了:

		“其实原先我等都已知道此奏疏,却迟迟不见陛下发阁\footnote{内阁。}惩处\footnote{学名:留中。},我们几个内阁大学士在私底下都互相感叹,陛下您胸襟宽容,实在是超越千古啊\footnote{马屁与说理相结合。}。”

		“所以以臣等愚见,陛下不用处置此事,奏疏还是照旧留存吧,如此陛下之宽容必定能留存史书,传之后世,千秋万代都称颂陛下是尧舜之君,是大大的好事啊!”

		据说拍马屁这个行当,最高境界是两句古诗,所谓“随风潜入夜,润物细无声”,在我看来,申时行做到了。

		但申先生还是低估了万历的二杆子性格,他话刚讲完,万历又是一声大吼:

		“如何设法处他?只是气他不过!”

		好话说一堆,还这么个态度,那就不客气了:

		“此本不可发出,也无他法处之,还望皇上宽恕,容臣等传谕该寺堂官\footnote{即大理寺高级官员。},使之去任可也。”

		这意思就是,老子不和稀泥了,明白告诉你,骂你的这篇文章不能发,也没办法处理,最多我去找他们领导,把这人免职了事,你别再闹了,闹也没用。

		很明显,万历虽然在气头上,却还是很识趣的,他清楚,目前形势下,自己不能把雒于仁怎么样,半天一言不发。申时行明白,这是默认。

		万历十八年的这场惊天风波就此了解,雒于仁骂得皇上一无是处,青史留名,却既没掉脑袋,也没有挨板子,拍拍屁股就走人了。而气得半死的万历终于认定,言官就是混蛋,此后的几十年里,他都保持着相同的看法。

		最大的赢家无疑是申时行,他保护了卢洪春、保护了雒于仁,安抚了言官大臣,也没有得罪皇帝,使两次危机成功化解,无愧为和稀泥的绝顶高手。

		自万历十一年执政以来,申时行经历了无数考验,无论是上司还是同僚,他都应付自如,七年间,上哄皇帝,下抚大臣,即使有个把不识趣、不配合的,也能被他轻轻松松地解决掉,混得可谓如鱼得水。

		然而正是这一天,万历十八年(1590年)正月初一,在解决完最为棘手的雒于仁问题后,他的好运将彻底结束。

		因为接下来,他说了这样一句话:

		“臣等更有一事奏请。”

		虽然雒于仁的事十分难办,但和申时行即将提出的这件事相比,只能说是微不足道。

		他所讲的事情,影响了无数人的一生,以及大明王朝的国运,而这件事情,在历史上有个专用名词:“争国本”。

		在张居正管事的前十年,万历既不能执政,也不能管事,甚至喝酒胡闹都不行,但他还有一项基本的权力——娶老婆。

		万历六年(1578年),经李太后挑选,张居正认可,十四岁的万历娶了老婆,并册立为皇后。

		不过对万历而言,这不是个太愉快的事情,因为这个老婆是指认的,什么偶然邂逅,自由恋爱都谈不上,某月某天,突然拉来一女的,无需吃饭看电影,就开始办手续,经过无数道繁琐程序仪式,然后正式宣告,从今以后,她就是你的老婆了。

		包办婚姻,纯粹的包办婚姻。

		虽然是凑合婚姻,但万历的运气还不错,因为他的这个老婆相当凑合。

		万历皇后王氏,浙江人,属传统贤妻型,而且为人乖巧,定位明确,善于关键时刻抓关键人,进宫后皇帝都没怎么搭理,先一心一意服侍皇帝他妈,早请示晚汇报,把老太太伺候好了,婆媳问题也就解决了。

		此外她还是皇帝的办公室主任,由于后来万历不上朝,喜欢在家里办公,公文经常堆得到处都是,她都会不动声色地加以整理,一旦万历找不着了,她能够立即说出公文放在何处,何时、由何人送入,在生活上,她对皇帝大人也是关怀备至,是优秀的秘书老婆两用型人才。

		这是一个似乎无可挑剔的老婆,除了一个方面——她生不出儿子。

		古人有云:不孝有三,无后为大,虽说家里有一堆儿子,最后被丢到街上的也不在少数,但既然是古人云,大家就只好人云亦云,生不出儿子,皇后也是白搭。于是万历九年(1581年)的时候,在李太后的授意下,万历下达旨意:命令各地选取女子,以备挑选。

		其实算起来,万历六年两人结婚的时候,万历只有十四岁,到万历九年的时候,也才十七岁,连枪毙都没有资格,就逼着要儿子,似乎有点不地道,但这是一般人的观念,皇帝不是一般人,观念自然也要超前,生儿子似乎也得比一般人急。

		但旨意传下去,被张居正挡了回来,并且表示,此令绝不可行。

		不要误会,张先生的意思并非考虑民间疾苦,不可行,是行不通。

		到底是首辅大人老谋深算,据说他刚看到这道旨意,便下断言:如按此令下达,决然无人可挑。

		俗话说,一入候门深似海,何况是宫门,辛辛苦苦养大的女儿送进去,就好比黄金周的旅游景点,丢进人堆就找不着了,谁也不乐意。那些出身名门、长相漂亮的自然不来,万一拉上来的都是些歪瓜裂枣,恶心了皇帝大人,这个黑锅谁来背?

		可是皇帝不能不生儿子,不能不找老婆,既要保证数量,也要确保质量,毕竟你要皇帝大人将就将就,似乎也是勉为其难。

		事情很难办,但在张居正大人的手中,就没有办不了的事,他脑筋一转,加了几个字:原文是挑选入宫,大笔一挥,变成了挑选入宫册封嫔妃。

		事情就这么解决了,因为说到底,入不入宫,也是个成本问题,万一进了宫啥也混不上,几十年没人管,实在不太值。在入宫前标明待遇,肯定级别,给人家个底线,自然就都来了。

		这就是水平。

		但连张居正都没想到,他苦心琢磨的这招,竟然还是没用上。

		因为万历自己把这个问题解决了。

		就在挑选嫔妃的圣旨下达后,一天,万历闲来无事,去给李太后请安,完事后,准备洗把脸,就叫人打盘水来。

		水端来了,万历一边洗着手,一边四处打量,打量来,打量去,就打量上了这个端脸盆的宫女。

		换在平常,这类人万历是一眼都不看的,现在不但看了,而且还越看越顺眼,顺眼了,就开始搭讪。

		就搭讪的方式而言,皇帝和街头小痞子是没什么区别的,无非是你贵姓,哪里人等等。但差异在于,小痞子搭完话,该干嘛还干嘛,皇帝就不同了。

		几句话搭下来,万历感觉不错,于是乎头一热,就幸了。

		皇帝非凡人,所以幸了之后的反应也不同于凡人,不用说什么一时冲动之类的话,拍拍屁股就走人了。不过万历还算厚道,临走时,赏赐她一副首饰,这倒也未必是他有多大觉悟,而是宫里的规定:但凡临幸,必赐礼物。

		因为遵守这个规定,他后悔了很多年。

		就万历而言,这是一件小事,皇帝嘛,幸了就幸了,感情是谈不上的,事实上,此人姓甚名谁,他都未必记得。

		这个宫女姓王,他很快就将牢牢记住。因为在不久之后,王宫女意外地发现,自己怀孕了。

		这个消息很快就传到了万历那里,他非但不高兴,反而对此守口如瓶,绝口不提。

		因为王宫女地位低,且并非什么沉鱼落雁之类的人物,一时兴起而已,万历不打算认这帐,能拖多久是多久。

		但这位仁兄明显打错了算盘,上朝可以拖,政务可以拖,怀孕拖到最后,是要出人命的。

		随着王宫女的肚子一天天大起来,知道这件事的人也一天天多起来,最后,太后知道了。

		于是,她叫来了万历,向他询问此事。

		万历的答复是沉默,他沉默的样子,很有几分流氓的风采。

		然而李太后对付此类人物,一向颇有心得。当年如高拱、张居正之类的老手都应付过去了,刚入行的新流氓万历自然不在话下。既然不说话,就接着问。

		装哑巴是行不通了,万历随口打哈哈,就说没印象了,打算死不认账。

		万历之所以有持无恐,是因为这种事一般都是你知我知,现场没有证人,即使有证人,也不敢出来\footnote{偷窥皇帝,是要命的。}。

		他这种穿上裤子就不认人的态度彻底激怒了李太后,于是,她找来了证人。

		这个证人的名字,叫内起居注。

		在古代文书中,起居注是皇帝日常言行的记录。比如今天干了多少活,去了多少地方,是第一手的史料来源。

		但起居注记载的,只是皇帝的外在工作情况,是大家都能看见的,而大家看不见的那部分,就是内起居注。

		内起居注记载的,是皇帝在后宫中的生活情况。比如去到哪里,和谁见面,干了些什么。当然,鉴于场所及皇帝工作内容的特殊性,其实际记录者不是史官,而是太监。所谓外表很天真,内心很暴力,只要翻一翻内外两本起居注,基本都能搞清楚。

		由于具有生理优势,太监可以出入后宫,干这类事情也方便得多。皇帝到哪里,就跟到哪里\footnote{当然,不宜太近。},皇帝进去开始工作,太监在外面等着。等皇帝出来,就开始记录,某年某月某日,皇帝来到某后妃处,某时进,某时出,特此记录存入档案。

		皇帝工作,太监记录,这是后宫的优良传统,事实证明,这一规定是极其有效,且合理的。

		因为后宫人太多,皇帝也不计数,如王宫女这样的邂逅,可谓比比皆是。实际上,皇帝乱搞并不重要,重要的,是乱搞之后的结果。

		如果宫女或后妃恰好怀孕,生下了孩子,这就是龙种,要是儿子,没准就是下一任皇帝,万一到时没有原始记录,对不上号,那就麻烦了。

		所以记录工作十分重要。

		但这项工作,还有一个漏洞,因为事情发生的时候,只有皇帝、太监、后妃\footnote{宫女。}三人在场。事后一旦有了孩子,后妃自然一口咬定,是皇帝干的,而皇帝一般都不记得,是不是自己干的。

		最终的确定证据,就是太监的记录。但问题在于,太监也是人,也可能被人收买,如果后妃玩花样,或是皇帝不认账,太监也没有公信力。

		所以宫中规定,皇帝工作完毕,要送给当事人一件物品,而这件物品,就是证据。

		李太后拿出了内起居注,翻到了那一页,交给了万历。

		一切就此真相大白,万历只能低头认账。

		\ifnum\theparacolNo=2
	\end{multicols}
\fi
\newpage
\section{游戏的开始}
\ifnum\theparacolNo=2
	\begin{multicols}{\theparacolNo}
		\fi
		万历十年(1582年),上车补票的程序完成,王宫女的地位终于得到了确认,她挺着大肚子,接受了恭妃的封号。

		两个月后,她不负众望生下了一个儿子,是为万历长子,取名朱常洛。

		消息传来,举国欢腾,老太太高兴,大臣们也高兴,唯一不高兴的,就是万历。

		因为他对这位恭妃,并没有太多感情。对这个意外出生的儿子,自然也谈不上喜欢。更何况,此时他已经有了德妃。

		德妃,就是后世俗称的郑贵妃。北京大兴人,万历初年进宫,颇得皇帝喜爱。

		在后来的许多记载中,这位郑贵妃被描述成一个相貌妖艳,阴狠毒辣的女人。但在我看来,相貌妖艳还有可能,阴狠毒辣实在谈不上。在此后几十年的后宫斗争中,此人手段之拙劣,脑筋之愚蠢,反应之迟钝,实在令人发指。

		综合史料分析,其智商水平,也就能到菜市场骂个街而已。

		可是万历偏偏就喜欢这个女人,经常前去留宿。而郑妃的肚子也相当争气,万历十一年(1583年)生了个女儿,虽然不能接班,但万历很高兴,竟然破格提拔,把她升为了贵妃。

		这是一个不详的先兆,因为在后宫中,贵妃的地位要高于其他妃嫔——包括生了儿子的恭妃。

		而这位郑贵妃的个人素养也实在很成问题,当上了后妃领导后,除了皇后,谁都瞧不上,特别是恭妃,经常被她称作老太婆。横行宫中,专横跋扈,十分好斗。

		难能可贵的是,贵妃同志不但特别能战斗,还特别能生。万历十四年(1586年),她终于生下了儿子,取名朱常洵。

		这位朱常洵,就是后来的福王。按郑贵妃的想法,有万历当靠山,这孩子生出来,就是当皇帝的。但她做梦也想不到,几十年后,自己这个宝贝儿子会死在屠刀之下。挥刀的人,名叫李自成。

		但在当时,这个孩子的出生,确实让万历欣喜异常。他本来就不喜欢长子朱常洛,打算换人,现在替补来了,怎能不高兴?

		然而他很快就将发现,皇帝说话,不一定算数。

		吸取了以往一百多年里,自己的祖辈与言官大臣斗争的丰富经验。万历没敢过早暴露目标,绝口不提换人的事,只是静静地等待时机成熟,再把生米煮成熟饭。

		可还没等米下锅,人家就打上门来了,而且还不是言官。

		万历十四年(1586年)三月,内阁首辅申时行上奏:望陛下早立太子,以定国家之大计,固千秋之基业。

		老狐狸就是老狐狸,自从郑贵妃生下朱常洵,申时行就意识到了隐藏的危险。他知道,自己的这个学生想干什么。

		凭借多年的政治经验,他也很清楚,如果这么干了,迎面而来的,必定是史无前例的惊涛骇浪。从此,朝廷将永无宁日。

		于是他立即上书,希望万历早立长子。言下之意是,我知道你想干嘛,但这事不能干,你趁早断了这念头,早点洗了睡吧。

		其实申时行的本意,倒不是要干涉皇帝的私生活:立谁都好,又不是我儿子,与我何干?之所以提早打预防针,实在是出于好心,告诉你这事干不成,早点收手,免得到时受苦。

		可是他的好学生似乎打定主意,一定要吃苦,收到奏疏,只回复了一句话:

		“长子年纪还小,再等个几年吧。”

		学生如此不开窍,申时行只得叹息一声,扬长而去。

		但这一次,申老师错了,他低估了对方的智商。事实上,万历十分清楚这封奏疏的隐含意义。只是在他看来,皇帝毕竟是皇帝,大臣毕竟是大臣,能坚持到底,就是胜利。此即所谓,明知山有虎,偏向虎山行。

		但一般说来,没事上山找老虎玩的,只有两种人:一种是打猎,一种是自尽。

		话虽如此,万历倒也不打无把握之仗,在正式亮出匕首之前,他决定玩一个花招。

		万历十四年(1586年)三月,万历突然下达谕旨:郑贵妃劳苦功高,升任皇贵妃。

		消息传来,真是粪坑里丢炸弹,分量十足。朝廷上下议论纷纷,群情激奋。

		因为在后宫中,皇贵妃仅次于皇后,算第二把手。且历朝历代,能获此殊荣者少之又少\footnote{生下独子或在后宫服务多年。}。

		按照这个标准,郑贵妃是没戏的。因为她入宫不长,且皇帝之前已有长子,没啥突出贡献,无论怎么算都轮不到她。

		万历突然来这一招,真可谓是煞费苦心。首先可以藉此提高郑贵妃的地位,子以母贵,母亲是皇贵妃,儿子的名分也好办;其次还能借机试探群臣的反应。今天我提拔孩子他妈,你们同意了,后天我就敢提拔孩子。温水煮青蛙,咱们慢慢来。

		算盘打得很好,可惜只是掩耳盗铃。

		要知道,在朝廷里混事的这帮人,个个都不简单:老百姓家的孩子,辛辛苦苦读几十年书,考得死去活来,进了朝廷,再被踩个七荤八素,这才修成正果。生肖都是属狐狸的,嗅觉极其灵敏,擅长见风使舵,无事生非。皇帝玩的这点小把戏,在他们面前也就是个笑话,傻子才看不出来。

		更为难得的是,明朝的大臣们不但看得出来,还豁得出去。第一个出头的,是户部给事中姜应麟。

		相对而言,这位仁兄还算文明,不说粗话,也不骂人,摆事实讲道理:

		“皇帝陛下,听说您要封郑妃为皇贵妃,我认为这是不妥的。恭妃先生皇长子,郑妃生皇三子\footnote{中间还有一个,夭折了。},先来后到,恭妃应该先封。如果您主意已定,一定要封,也应该先封恭妃为贵妃,再封郑妃皇贵妃,这样才算合适。”

		“此外,我还认为,陛下应该尽早立皇长子为太子,这样天下方才能安定。”

		万历再一次愤怒了,这可以理解,苦思冥想几天,好不容易想出个绝招,自以为得意,没想到人家不买账,还一言点破自己的真实意图,实在太伤自尊。

		为挽回面子,他随即下令,将姜应麟免职外放。

		好戏就此开场。一天后,吏部员外郎沈璟上书,支持姜应麟,万历二话不说,撤了他的职。几天后,吏部给事中杨廷相上书,支持姜应麟,沈璟,万历对其撤职处理。又几天后,刑部主事孙如法上书,支持姜应麟、沈璟、杨廷相,万历同志不厌其烦,下令将其撤职发配。

		在这场斗争中,明朝大臣们表现出了无畏的战斗精神:不怕降级,不怕撤职,不怕发配。个顶个地扛着炸药包往上冲,前仆后继,人越闹越多,事越闹越大。中央的官不够用了,地方官也上书凑热闹,搞得一塌糊涂,乌烟瘴气。

		然而事情终究还是办成了,虽然无数人反对,无数人骂仗,郑贵妃还是变成了郑皇贵妃。

		虽然争得天翻地覆,但该办的事还是办了。万历十四年三月,郑贵妃正式册封。

		这件事情的成功解决给万历留下了这样一个印象:自己想办的事情,是能够办成的。

		这是一个错误的判断。

		然而此后,在册立太子的问题上,万历确实消停了——整整消停了四年多。当然,不闹事,不代表不挨骂。事实上,在这四年里,言官们非常尽责。他们找到了新的突破口——皇帝不上朝,并以此为契机,在雒于仁等模范先锋的带领下,继续奋勇前进。

		但总体而言,小事不断,大事没有,安定团结的局面依旧。

		直到这历史性的一天:万历十八年(1590年)正月初一。

		解决雒于仁事件后,申时行再次揭开了盖子:

		“臣等更有一事奏请。”

		“皇长子今年已经九岁,朝廷内外都认为应册立为太子,希望陛下早日决定。”

		在万历看来,这件事比雒于仁的酒色财气疏更头疼,于是他接过了申时行刚刚用过的铁锹,接着和稀泥:

		“这个我自然知道,我没有嫡子\footnote{即皇后的儿子。},长幼有序。其实郑贵妃也多次让我册立长子,但现在长子年纪还小,身体也弱,等他身体强壮些后,我才放心啊。”

		这段话说得很有水平,按照语文学来分析,大致有三层意思。

		第一层先说自己没有嫡子,是说我只能立长子;然后又讲长幼有序,是说我不会插队,但说来说去,就是不说要立谁;接着又把郑贵妃扯出来,搞此地无银三百两。

		最后语气一转,得出结论:虽然我只能立长子、不会插队,老婆也没有干涉此事,但考虑到儿子太小,身体太差,暂时还是别立了吧。

		这招糊弄别人可能还行,对付申时行就有点滑稽了,和了几十年稀泥,哪排得上你小子?

		于是申先生将计就计,说了这样一句话:

		“皇长子已经九岁,应该出阁读书了,请陛下早日决定此事。”

		这似乎是一件完全不相干的事情,但事实绝非如此,因为在明代,皇子出阁读书,就等于承认其为太子,申时行的用意非常明显:既然你不愿意封他为太子,那让他出去读书总可以吧,形式不重要,内容才是关键。

		万历倒也不笨,他也不说不读书,只是强调人如果天资聪明,不读书也行。申时行马上反驳,说即使人再聪明,如果没有人教导,也是不能成才的。

		就这样,两位仁兄从继承人问题到教育问题,你来我往,互不相让,闹到最后,万历烦了:

		“我都知道了,先生你回去吧!”

		话说到这个份上,也只好回去了,申时行离开了宫殿,向自己家走去。

		然而当他刚刚踏出宫门的时候,却听到了身后急促的脚步声。

		申时行转身,看见了一个太监,他带来了皇帝的谕令:

		“先不要走,我已经叫皇长子来了,先生你见一见吧。”

		十几年后,当申时行在家撰写回忆录的时候,曾无数次提及这个不可思议的场景以及此后那奇特的一幕,终其一生,他也未能猜透万历的企图。

		申时行不敢怠慢,即刻回到了宫中,在那里,他看见了万历和他的两个儿子,皇长子朱常洛,以及皇三子朱常洵。

		但给他留下最深刻印象的,却并非这两个皇子,而是此时万历的表情。没有愤怒,没有狡黠,只有安详与平和。

		他指着皇长子,对申时行说:

		“皇长子已经长大了,只是身体还有些弱。”

		然后他又指着皇三子,说道:

		“皇三子已经五岁了。”

		接下来的,是一片沉默。

		万历平静地看着申时行,一言不发。此时的他,不是一个酒色财气的昏庸之辈,不是一个暴跳如雷的使气之徒。

		他是一个父亲,一个看着子女不断成长,无比欣慰的父亲。

		申时行知道机会来了,于是他打破了沉默:

		“皇长子年纪已经大了,应该出阁读书。”

		万历的心意似乎仍未改变:

		“我已经指派内侍教他读书。”

		事到如今,只好豁出去了:

		“皇上您在东宫的时候,才六岁,就已经读书了。皇长子此刻读书,已经晚了!”

		万历的回答并不愤怒却让人哭笑不得:

		“我五岁就已能读书!”

		申时行知道,在他的一生中,可能再也找不到一个更好的机会,去劝服万历,于是他做出了一个惊人的举动。

		他上前几步,未经许可,便径自走到了皇长子的面前,端详片刻,对万历由衷地说道:

		“皇长子仪表非凡,必成大器,这是皇上的福分啊,希望陛下能够早定大计,朝廷幸甚!国家幸甚!”

		万历十八年正月初一日,在愤怒、沟通、争执后,万历终于第一次露出了笑容。

		万历微笑地点点头,对申时行说道:

		“这个我自然知道,其实郑贵妃也劝过我早立长子,以免外人猜疑,我没有嫡子,册立长子是迟早的事情啊。”

		这句和缓的话,让申时行感到了温暖,儿子出来了,好话也说了,虽然也讲几句什么郑贵妃支持,没有嫡子之类的屁话,但终究是表了态。

		形势大好,然而接下来,申时行却一言不发,行礼之后便退出了大殿。

		这正是他绝顶聪明之处,点到即止,见好就收,今天先定调,后面慢慢来。

		但他无论如何也想不到,这次和谐的对话,不但史无前例,而且后无来者。“争国本”事件的严重性,将远远超出他的预料,因为决定此事最终走向的,既不是万历,也不会是他。

		谈话结束后,申时行回到了家中,开始满怀希望地等待万历的圣谕,安排皇长子出阁读书。

		可是一天天过去了,希望变成了失望。到了月底,他也坐不住了,随即上疏,询问皇长子出阁读书的日期。这意思是说,当初咱俩谈好的事,你得守信用,给个准信。

		但是万历似乎突然失忆,啥反应都没有,申时行等了几天,一句话都没有等到。

		既然如此,那就另出新招,几天后,内阁大学士王锡爵上书:

		“陛下,其实我们不求您立刻册立太子,只是现在皇长子九岁,皇三子已五岁,应该出阁读书。”

		不说立太子,只说要读书,而且还把皇三子一起拉上,由此而见,王锡爵也是个老狐狸。

		万历那边却似乎是人死绝了,一点消息也没有,王锡爵等了两个月,石沉大海。

		到了四月,包括申时行在内,大家都忍无可忍了,内阁四名大学士联名上疏,要求册立太子。

		尝到甜头的万历故伎重演:无论你们说什么,我都不理,我是皇帝,你们能把我怎么样?

		但他实在低估了手下的这帮老油条,对付油盐不进的人,他们一向都是有办法的。

		几天后,万历同时收到了四份奏疏,分别是申时行、王锡爵、许国、王家屏四位内阁大学士的辞职报告。理由多种多样,有说身体不好,有说事务繁忙,难以继任的,反正一句话,不干了。

		自万历退居二线以来,国家事务基本全靠内阁,内阁一共就四个人,要是都走了,万历就得累死。

		没办法,皇帝大人只好现身,找内阁的几位同志谈判,好说歹说,就差求饶了,并且当场表态,会在近期解决这一问题。

		内阁的几位大人总算给了点面子,一番交头接耳之后,上报皇帝:病的还是病,忙的还是忙,但考虑到工作需要,王家屏大学士愿意顾全大局,继续干活。

		万历窃喜。

		因为这位兄弟的策略,叫拖一天是一天。拖到这帮老家伙都退了,皇三子也大了,到时木已成舟,不同意也得同意。这次内阁算是上当了。

		然而上当的人,只有他。

		因为他从未想过这样一个问题:为什么留下来的,偏偏是王家屏呢?

		王家屏,山西大同人,隆庆二年进士。简单地说,这是个不上道的人。

		王家屏的科举成绩很好,被选为庶吉士,还编过《世宗实录》,应该说是很有前途的,可一直以来,他都没啥进步。原因很简单,高拱当政的时候,他曾上书弹劾高拱的亲戚,高首辅派人找他谈话,让他给点面子,他说,不行。

		张居正当政的时候,他搞非暴力不合作。照常上班,就是不靠拢上级,张居正刚病倒的时候,许多人都去祈福,表示忠心,有人拉他一起去,他说,不去。

		张居正死了,万历十二年,他进入内阁,成为大学士。此时的内阁,已经有了申时行、王锡爵、许国三个人,他排第四。按规矩,这位甩尾巴的新人应该老实点,可他偏偏是个异类,每次内阁讨论问题,即使大家都同意,他觉得不对,就反对。即使大家都反对,他觉得对,就同意。

		他就这么在内阁里硬挺了六年,谁见了都怕,申时行拿他也没办法。更有甚者,写辞职信时,别人的理由都是身体有病,工作太忙,他却别出一格,说是天下大旱,作为内阁成员,负有责任,应该辞职\footnote{久旱乞罢。}。

		把他留下来,就是折腾万历的。

		几天后,礼部尚书于慎行上书,催促皇帝册立太子,语言比较激烈。万历也比较生气,罚了他三个月工资。

		事情的发生,应该还算正常,不正常的,是事情的结局。

		换在以往,申时行已经开始挥舞铁锹和稀泥了,先安慰皇帝,再安抚大臣,最后你好我好大家好,收工。

		相比而言,王家屏要轻松得多,因为他只有一个意见——支持于慎行。

		工资还没扣,他就即刻上书,为于慎行辩解,说了一大通道理,把万历同志的脾气活活顶了回去。但更让人惊讶的是,这一次,万历没有发火。

		因为他发不了火,事情很清楚,内阁四个人,走了三个,留下来的这个,还是个二杆子,明摆着是要为难自己。而且这位坚持战斗的王大人还说不得,再闹腾一次,没准就走人了,到时谁来收拾这个烂摊子?

		可是光忍还不够,言官大臣赤膊上阵,内阁打黑枪,明里暗里都来,比逼宫还狠,不给个说法,是熬不过去了。

		几天后,一个太监找到了王家屏,向他传达了皇帝的谕令:

		“册立太子的事情,我准备明年办,不要再烦\footnote{扰。}我了。”

		王家屏顿时喜出望外,然而,这句话还没有讲完:

		“如果还有人敢就此事上书,就到十五岁再说!”

		朱常洛是万历十年出生的,万历发出谕令的时间是万历十八年,所以这句话的意思是说,如果你们再敢闹腾,这事就六年后再办!

		虽然不是无条件投降,但终究还是有了个说法,经过长达五年的斗争,大臣们胜利了——至少他们自己这样认为。

		事情解决了,王家屏兴奋了,兴奋之余,就干了一件事。

		他把皇帝的这道谕令告诉了礼部,而第一个获知消息的人,正是礼部尚书于慎行。

		于慎行欣喜若狂,当即上书告诉皇帝:

		“此事我刚刚知道,已经通报给朝廷众官员,要求他们耐心等候。”

		万历气得差点吐了白沫。

		因为万历给王家屏的,并不是正规的圣旨,而是托太监传达的口谕,看上去似乎没区别,但事实上,这是一个有深刻政治用意的举动。

		其实在古代,君无戏言这句话基本是胡扯,皇帝也是人,时不时编个瞎话,吹吹牛,也很正常,真正说了就要办的,只有圣旨。白纸黑字写在上面,糊弄不过去。所以万历才派太监给王家屏传话,而他的用意很简单:这件事情我心里有谱,但现在还不能办,先跟你通个气,以后遇事别跟我对着干,咱们慢慢来。

		皇帝大人原本以为,王大学士好歹在朝廷混了几十年,这点觉悟应该还有,可没想到,这位一根筋的仁兄竟然把事情捅了出去,密谈变成了公告,被逼上梁山了。

		他当即派出太监,前去内阁质问王家屏,却得到了一个让他意想不到的答案。

		王家屏是这样辩解的:

		“册立太子是大事,之前许多大臣都曾因上疏被罚,我一个人定不了,又被许多大臣误会,只好把陛下的旨意传达出去,以消除大家的疑虑\footnote{以释众惑。}。”

		这番话的真正意思大致是这样的:我并非不知道你的用意,但现在我的压力也很大,许多人都在骂我,我也没办法,只好把陛下拉出来背黑锅了。

		虽然不上道,也是个老狐狸。

		既然如此,就只好将错就错了,几天后,万历正式下发圣旨:

		“关于册立皇长子为太子的事情,我已经定了,说话算数\footnote{诚待天下。},等长子到了十岁,我自然会下旨,到时册立出阁读书之类的事情一并解决,就不麻烦你们再催了。”

		长子十岁,是万历十九年,也就是下一年,皇帝的意思很明确,我已经同意册立长子,你们也不用绕弯子,搞什么出阁读书之类的把戏,让老子清净一年,明年就立了!

		这下大家都高兴了,内阁的几位仁兄境况也突然大为改观,有病的病好了,忙的也不忙了,除王锡爵\footnote{母亲有病,回家去了,真的。}外,大家都回来了。

		剩下来的,就是等了。一晃就到了万历二十年,春节过了,春天过了,都快要开西瓜了,万历那里一点消息都没有。

		泱泱大国,以诚信为本,这就没意思了。

		可是万历二十年毕竟还没过,之前已经约好,要是贸然上书催他,万一被认定毁约,推迟册立,违反合同的责任谁都负担不起,而且皇上到底是皇上,你上疏说他耍赖,似乎也不太妥当。

		一些脑子活的言官大臣就开始琢磨,既要敲打皇帝,又不能留把柄,想来想去,终于找到了一个完美的替代目标——申时行。

		没办法,申大人,谁让你是首辅呢?也只好让你去扛了。

		很快,一封名为《论辅臣科臣疏》的奏疏送到了内阁,其主要内容,是弹劾申时行专权跋扈,压制言官,使得正确意见得不到执行。

		可怜,申首辅一辈子和稀泥,东挖砖西补墙,累得半死,临了还要被人玩一把,此文言辞尖锐,指东打西,指桑骂槐,可谓是政治文本的典范。

		文章作者,是南京礼部主事汤显祖,除此文外,他还写过另一部更有名的著作——牡丹亭。

		\subsection{汤显祖}
		汤显祖,字义仍,江西临川人,上书这一年,他四十二岁,官居六品。

		虽说四十多岁才混到六品,实在不算起眼。但此人绝非等闲之辈,早在三十年前,汤先生已天下闻名。

		十三岁的时候,汤显祖就加入了泰州学派\footnote{也没个年龄限制。},成为了王学的门人,跟着那帮“异端”四处闹腾,开始出名。

		二十一岁,他考中举人。七年后,到京城参加会试,运气不好,遇见了张居正。

		之所以说运气不好,并非张居正讨厌他,恰恰相反,张首辅很赏识他,还让自己的儿子去和他交朋友。

		这是件求之不得的好事,可问题在于,汤先生异端中毒太深,瞧不起张居正,摆了谱,表示拒不交友。

		他既然敢跟张首辅摆谱,张首辅自然要摆他一道,考试落榜也是免不了的。三年后,他再次上京赶考,张首辅锲而不舍,还是要儿子和他交朋友,算是不计前嫌。但汤先生依然不给面子,再次摆谱。首辅大人自然再摆他一道,又一次落榜。

		但汤先生不但有骨气,还有毅力,三年后再次赶考,这一次张首辅没有再阻拦他\footnote{死了。},终于成功上榜。

		由于之前两次跟张居正硬扛,汤先生此时的名声已经是如日中天。当朝的大人物张四维、申时行等人都想拉他,可汤先生死活不搭理人家。

		不搭理就有不搭理的去处,名声大噪的汤显祖被派到了南京,几番折腾,才到礼部混了个主事。

		南京本来就没事干,南京的礼部更是闲得出奇,这反倒便宜了汤先生。闲暇之余开始写戏,并且颇有建树,日子过得还算不错,直到万历十九年的这封上疏。

		很明显,汤先生的政治高度比不上艺术高度,奏疏刚送上去,申时行还没说什么,万历就动手了。

		对于这种杀鸡儆猴的把戏,皇帝大人一向比较警觉\footnote{他也常用这招。},立马做出了反应,把汤显祖发配到边远地区\footnote{广东徐闻。}去当典史。

		这是一次极其致命的打击,从此汤先生再也没能翻过身来。

		万历这辈子罢过很多人的官,但这一次,是最为成功的。因为他只罢掉了一个六品主事,却换回一个明代最伟大的戏曲家,赚大发了。

		二十八岁落榜后,汤显祖开始写戏。三十岁的时候,写出了《紫箫记》;三十八岁,写出了《紫钗记》。四十二岁被赶到广东,七年后京察,又被狠狠地折腾了一回,索性回了老家。

		来回倒腾几十年,一无所获。在极度苦闷之中,四十九岁的汤显祖回顾了自己戏剧化的一生,用悲凉而美艳的辞藻写下了他所有的梦想和追求,是为《还魂记》,后人又称《牡丹亭》。

		牡丹亭,全剧共十五出,描述了一个死而复生的爱情故事,\footnote{情节比较复杂,有兴趣自己去翻翻。}。此剧音律流畅,词曲优美,轰动一时,时人传诵:牡丹一出,西厢\footnote{《西厢记》。}失色。此后传唱天下百余年,堪与之媲美者,唯有孔尚任之《桃花扇》。

		为官不济,为文不朽,是以无憾。
		\begin{quote}
			\begin{spacing}{0.5}  %行間距倍率
				\textit{{\footnotesize
							\begin{description}
								\item[\textcolor{Gray}{\faQuoteRight}] 史赞:二百年来,一人而已。
							\end{description}
						}}
			\end{spacing}
		\end{quote}

		总的说来,汤显祖的运气是不错的,因为更麻烦的事,他还没赶上。

		汤先生上书两月之后,福建佥事李琯就开炮了,目标还是申时行。不过这次更狠,用词狠毒不说,还上升到政治高度,一条条列下来,弹劾申时行十大罪,转瞬之间,申先生就成了天字第一号大恶人。

		万历也不客气,再度发威,撤了李琯的职。

		命令一下,申时行却并不高兴,反而唉声叹气,忧心忡忡。

		因为到目前为止,虽然你一刀我一棍打个不停,但都是摸黑放枪,谁也不挑明。万历的合同也还有效,拖到年尾,皇帝赖账就是理亏,到时再争,也是十拿九稳。

		可万一下面这帮愤中愤老忍不住,玩命精神爆发,和皇帝公开死磕,事情就难办了。

		俗语云:怕什么,就来什么。

		工部主事张有德终于忍不住了,他愤然上书,要求皇帝早日册立太子。

		等的就是你。

		万历随即做出反应,先罚了张有德的工资,鉴于张有德撕毁合同,册立太子的事情推后一年办理。

		这算是正中下怀,本来就不大想立,眼看合同到期,正为难呢,来这么个冤大头,不用白不用。册立的事情也就能堂而皇之地往后拖了。

		事实上,这是他的幻想。

		因为在大臣们看来,这合同本来就不合理,忍气吞声大半年,那是给皇帝面子,早就一肚子苦水怨气没处泻,你敢蹦出来,那好,咱们就来真格的!

		当然,万历也算是老运动员了。对此他早有准备,无非是来一群大臣瞎咋呼,先不理,闹得厉害再出来说几句话,把事情熬过去,完事。

		形势的发展和他的预料大致相同,张有德走人后,他的领导,工部尚书曾同亨就上书了,要求皇帝早日册立太子。

		万历对此嗤之以鼻,他很清楚,这不过是个打头的,大部队在后。下面的程序他都能背出来,吵吵嚷嚷,草草收场,实在毫无新鲜可言。

		然而当下一封奏疏送上来的时候,他才知道,自己错了。

		这封奏疏的署名人并不多,只有三个,分别是申时行、许国、王家屏。

		但对万历而言,这是一个致命的打击。

		因为之前无论群臣多么反对,内阁都是支持他的。即使以辞职回家相威胁,也从未公开与他为敌,是他的最后一道屏障,现在竟然公开站出来和他对着干,此例一开,后果不堪设想。

		特别是申时行,虽说身在内阁,时不时也说两句,但那都是做给人看的。平日里忙着和稀泥,帮着调节矛盾,是名副其实的卧底兼间谍。

		可这次,申时行连个消息都没透,就打了个措手不及,实在太不够意思,于是万历私下派出了太监,斥责申时行。

		一问,把申时行也问糊涂了,因为这事他压根就不知道!

		事情是这样的,这封奏疏是许国写的,写好后让王家屏署名,王兄自然不客气,提笔就签了名,而申时行的底细他俩都清楚,这个老滑头死也不会签,于是许大人胆一壮,代申首辅签了名,拖下了水。

		事已至此,申大人只能一脸无辜的表白:

		“名字是别人代签的,我事先真不知道。”

		事情解释了,太监也回去了,可申先生却开始琢磨了:万一太监传达不对怎么办?万一皇帝不信怎么办?万一皇帝再激动一次,把事情搞砸怎么办?

		想来想去,他终于决定,写一封密信。

		这封密信的内容大致是说,我确实不知道上奏的事情,这事情皇上你不要急,自己拿主意就行。

		客观地讲,申时行之所以说这句话,倒不一定是耍两面派,因为他很清楚皇帝的性格:

		像万历这号人,属于死要面子活受罪,打死也不认错的。看上去非常随和,实际上极其固执,和他硬干,是没有什么好处的。

		所以申时行的打算,是先稳住皇帝,再慢慢来。

		事实确如所料,万历收到奏疏后,十分高兴,当即回复:

		“你的心意我已知道,册立的事情我已有旨意,你安心在家调养就是了。”

		申时行总算松了口气,事情终于糊弄过去了。

		但他做梦也想不到,他长达十年的和稀泥生涯,将就此结束——因为那封密信。

		申时行的这封密信,属于机密公文,按常理,除了皇帝,别人是看不见的。

		可是在几天后的一次例行公文处理中,万历将批好的文件转交内阁,结果不留神,把这封密信也放了进去。

		这就好比拍好了照片存电脑,又把电脑拿出去给人修,是个要命的事。

		文件转到内阁,这里是申时行的地盘,按说事情还能挽回。可问题在于申大人为避风头,当时还在请病假,负责工作的许国也没留意,顺手就转给了礼部。

		最后,它落在了礼部给事中罗大纮的手里。

		罗大纮,江西吉水人。关于这个人,只用一句就能概括:一个称职的言官。

		看到申时行的密信后,罗大纮非常愤怒,因为除了耍两面派外,申时行在文中还写了这样一句话:惟亲断亲裁,勿因小臣妨大典。

		这句话说白了,就是你自己说了算,不要理会那些小臣。

		我们是小臣,你是大臣?!

		此时申时行已经发现了密信外泄,他十分紧张,立刻找到了罗大纮的领导,礼部科给事中胡汝宁,让他去找罗大纮谈判。

		可惜罗大纮先生不吃这一套,写了封奏疏,把这事给捅了出去,痛骂申时行两面派。

		好戏就此开场,言官们义愤填膺。吏部给事中钟羽正、候先春随即上书,痛斥申时行,中书黄正宾等人也跟着凑热闹,骂申时行老滑头。

		眼看申首辅吃亏,万历当即出手,把罗大纮赶回家当了老百姓,还罚了上书言官的工资。

		但事情闹到这个份上,已经无法收拾了。

		经历过无数大风大浪的申时行,终究在阴沟里翻了船。自万历十年以来,他忍辱负重,上下协调,独撑大局,打落门牙往肚里吞,至今已整整十年。

		现在,他再也支撑不下去了。

		万历十九年(1591年)九月,申时行正式提出辞职,最终得到批准,回乡隐退。

		大乱就此开始。

		\ifnum\theparacolNo=2
	\end{multicols}
\fi
\newpage
\section{混战}
\ifnum\theparacolNo=2
	\begin{multicols}{\theparacolNo}
		\fi
		申时行在的时候,大家都说朝廷很乱,等申时行走了,大家才知道,什么叫乱。

		首辅走了,王锡爵不在,按顺序,应该是许国当首辅。可这位兄弟相当机灵,一看形势不对,写了封辞职信就跑了。

		只剩王家屏了。

		万历不喜欢王家屏,王家屏也知道皇帝不喜欢他,所以几乎在申时行走人的同时,他就提出辞职。

		然而万历没有批,还把王家屏提为首辅。原因很简单,这么个烂摊子,现在内阁就这么个人,好歹就是他了。

		内阁总算有个人了,但一个还不够,得再找几个。搭个班子,才好唱戏。说起来还是申时行够意思,早就料到有这一天,所以在临走时,他向万历推荐了两个人:一个是时任吏部左侍郎赵志皋,另一个是原任礼部右侍郎张位。

		这个人事安排十分有趣,因为这两个人兴趣不同,性格不同,出身不同,总而言之,就没一点共同语言,但事后证明,就是这么个安排,居然撑了七八年,申先生的领导水平可见一斑。

		班子定下来了,万历的安宁日子也到了头。因为归根结底,大臣们闹腾,还是因为册立太子的事情,申先生不过是帮皇帝挡了子弹,现在申先生走了,皇帝陛下只能赤膊上阵。

		万历二十年(1592年)正月,真正的总攻开始了。

		礼部给事中李献可首先发难,上书要求皇帝早日批准长子出阁读书,而且这位兄台十分机灵,半字不提册立的事,全篇却都在催这事,半点把柄都不留,搞得皇帝陛下十分狼狈,一气之下,借口都不找了:

		“册立已有旨意,这厮偏又来烦扰……好生可恶,降级调外任用!”

		其实说起来,李献可不是什么大人物,这个处罚也不算太重。可万历万没想到,就这么个小人物,这么点小事儿,他竟然没能办得了。

		因为他的圣旨刚下发,就被王家屏给退了回来。

		作为朝廷首辅,如果认为皇帝的旨意有问题,可以退回去,拒不执行,这种权力,叫做封还。

		封还就封还吧,不办就不办吧,更可气的是,王首辅还振振有词:

		这事我没错,是皇帝陛下错了!因为李献可没说册立的事,他只是说应该出阁读书,你应该采纳他的意见,即使不能采纳,也不应该罚他,所以这事我不会办。

		真是要造反了,刚刚提了首辅,这白眼狼就下狠手。万历恨不得拿头撞墙,气急败坏之下,他放了王家屏的假,让他回家休养去了。

		万历的“幸福”生活从此拉开序幕。

		几天后,礼部给事中钟羽正上疏,支持李献可,经典语言如下:

		“李献可的奏疏,我是赞成的,请你把我一同降职吧\footnote{请与同谪。}。”

		万历满足了他的要求。

		又几天后,礼部给事中舒弘绪上疏,发言如下:

		“言官是可以处罚的,出阁读书是不能不办的。”

		发配南京。

		再几天后,户部给事中孟养浩上疏,支持李献可、钟羽正等人。相对而言,他的奏疏更有水平,虽然官很小\footnote{七品。},志气却大,总结了皇帝大人的种种错误,总计五条,还说了一句相当经典的话:

		“皇帝陛下,您坐视皇长子失学,有辱宗社祖先!”

		万历气疯了,当即下令,把善于总结的孟养浩同志革职处理,并拉到午门,打了一百杖。

		暴风雨就是这样诞生的。

		别人也就罢了,可惜孟先生偏偏是言官,干的是本职工作,平白被打实在有点冤。

		于是大家都愤怒了。

		请注意,这个大家是有数的,具体人员及最终处理结果如下所列:

		内阁大学士赵志皋上疏,被训斥。

		吏科右给事中陈尚象上疏,被革职为民。

		御史邹德泳,户科都给事中丁懋逊、兵科都给事中张栋、刑科都给事中吴之佳、工科都给事中杨其休,礼科左给事中叶初春,联名上疏抗议。万历大怒,将此六人降职发配。

		万历终于做了一件了不起的事情。如果加上最初上疏的李献可,那么在短短的几天之内,他就免掉了十二位当朝官员。这一伟大记录,就连后来的急性子崇祯皇帝也没打破。

		事办到这份上,皇帝疯了,大臣也疯了。官服乌纱就跟白送的一样,铺天盖地到处乱扔,大不了就当老子这几十年书白读了。拼个你死我活只为一句话:可以丢官,不能丢人!

		在这一光辉思想的指导下,礼部员外郎董嗣成、御史贾名儒、御史陈禹谟再次上疏,支持李献可。万历即刻反击,董嗣成免职,贾名儒发配,陈禹谟罚工资。

		事情闹到这里,到底卷进来多少人,我也有点乱。但若以为就此打住,那实在是低估了明代官员的战斗力。

		几天后,礼部尚书李长春也上疏了。对这位高级官员,万历也没客气,狠狠地骂了他一顿,谁知没多久,吏部尚书蔡国珍、侍郎杨时乔又上疏抗议,然而这一次,万历没有做出任何反应——实在骂不动了。

		皇帝被搞得奄奄一息,王家屏也坐不住了,他终于出面调停,向皇帝认了错,并希望能够赦免群臣。

		想法本是好的,方法却是错的。好不容易消停下去的万历,一看见这个老冤家,顿时恢复了战斗力,下书大骂:

		“自你上任,大臣狂妄犯上,你是内阁大学士,不但不居中缓和矛盾,反而封还我的批示,故意激怒我!见我发怒,你又说你有病在身,回家休养!国家事务如此众多,你在家躺着\footnote{高卧。},心安吗!?既然你说有病,就别来了,回家养病去吧!”

		王家屏终于理解了申时行的痛苦,万历二十年(1592年)三月,他连上八封奏疏,终于回了家。

		这是一场实力不对等的较量,大臣的一句话,可能毫无作用,万历的一道圣旨,却足以改变任何人的命运。

		然而万历失败了,面对那群前仆后继的人,他虽然竭尽全力,却依然失败了,因为权力并不能决定一切——当它面对气节与尊严的时候。

		王家屏走了,言官们暂时休息了。接班的赵志皋比较软,不说话,万历正打算消停几天,张位又冒出来了。

		这位次辅再接再厉,接着闹,今天闹出阁讲学,明天就闹册立太子。每天变着法地折腾皇帝,万历同志终于顶不住了。如此下去,不被逼死,也被憋死了。

		必须想出对策。

		考虑再三,他决定去找一个人,在他看来,只有这个人才能挽救一切。

		\subsection{王锡爵}
		万历二十一年(1593年),王锡爵奉命来到京城,担任首辅。

		王锡爵,字元驭,苏州太仓人。

		嘉靖四十一年,他二十八岁,赴京赶考,遇见申时行,然后考了第一。

		几天后参加殿试,又遇见了申时行,这次他考了第二。

		据说他之所以在殿试输给申时行,不外乎两点,一是长得不够帅,二是说话不够滑。

		帅不帅不好说,滑不滑是有定论的。

		自打进入朝廷,王锡爵就是块硬骨头。万历五年张居正夺情,大家上书闹,他跑到人家家里闹,逼得张居正大人差点拔刀自尽。吴中行被打得奄奄一息,大家在场下吵,他跑到场上哭。

		万历六年,张居正不守孝回京办公。大家都庆贺,他偏请假,说我家还有父母,实在没有时间工作,要回家尽孝,张居正恨得直磨牙。

		万历九年,张居正病重,大家都去祈福,他不屑一顾。

		万历十年,张居正病逝,反攻倒算开始,抄家闹事翻案,人人都去踩一脚,这个时候,他说:

		“张居正当政时,做的事情有错吗?!他虽为人不正,却对国家有功,你们怎能这样做呢?!”

		万历十三年,他的学生李植想搞倒申时行,扶他上台,他痛斥对方,请求辞职。

		三年后,他的儿子乡试考第一,有人怀疑作弊,他告诉儿子,不要参加会试,回家待业,十三年后他下了台,儿子才去考试,会试第二,殿试第二。

		他是一个经得起时间考验的人。

		所以在万历看来,能收拾局面的,也只有王锡爵了。

		王大人果然不负众望,到京城一转悠,就把情况摸清促了。随即开始工作,给皇帝上了一封密信。大意是说,目前情况十分紧急,请您务必在万历二十一年册立太子,绝不能再拖延了,否则我就是再有能耐,也压制不了!

		吸取了上次的教训,万历没敢再随便找人修电脑,专程派了个太监,送来了自己的回信。

		可王锡爵刚打开信,就傻眼了。

		信上的内容是这样的:

		“看了你的奏疏,为你的忠诚感动!我去年确实说过,今年要举行册立大典,但是\footnote{注意此处。},我昨天晚上读了祖训\footnote{相当于皇帝的家规。},突然发现里面有一句训示:立嫡不立长,我琢磨了一下,皇后现在年纪还不大,万一将来生了儿子,怎么办呢?是封太子,还是封王?”

		“如果封王,那就违背了祖训,如果封太子,那就有两个太子了,我想来想去,想了个办法,要不把我的三个儿子一起封王,等过了几年,皇后没生儿子,到时候再册立长子也不迟。这事我琢磨好了,既不违背祖制,也能把事办了,很好,你就这么办吧。”

		阶级斗争又有新动向了,很明显,万历同志是很动了一番脑筋,觉得自己不够分量,把老祖宗都搬出来了,还玩了个复杂的逻辑游戏,有相当的技术含量,现解析如下。

		按老规矩,要立嫡子\footnote{皇后的儿子。},可是皇后又没生儿子,但皇后今天没有儿子,不代表将来没有。如果我立了长子,嫡子生出来,不就违反政策了吗?但是皇后什么时候生儿子,我也不知道,与其就这么拖着,还不如把现在的三个儿子一起封了了事,到时再不生儿子,就立太子。先封再立,总算对上对下都有了交代。

		王锡爵初一琢磨,就觉得这事有点悬,但听起来似乎又只能这么办,思前想后,他也和了稀泥,拿出了两套方案。

		方案一、让皇长子拜皇后为母亲,这样既是嫡子又是长子,问题就解决了。

		方案二、按照皇帝的意思,三个儿子一起封王,到时再说。

		附注:第二套方案,只有在万不得已的时候,才能使用。

		上当了,彻底上当了。

		清醒了一辈子的王大人,似乎终于糊涂了,他好像并不知道,自己已经跳入了一个陷阱。

		事实上,万历的真正目标,不是皇长子,而是皇三子。

		他喜欢郑贵妃,喜欢朱常洵,压根就没想过要立太子,搞三王并封,把皇长子、三子封了王,地位就平等了,然后就是拖,拖到大家都不闹了,事情也就办成了。

		至于所谓万不得已,采用第二方案,那也是句废话,万历同志这辈子,那是经常地万不得已。

		总之,王锡爵算是上了贼船了。

		万历立即选择了第二种方案,并命令王锡爵准备执行。

		经过长时间的密谋和策划,万历二十一年(1593年)正月二十六日,万历突然下发圣旨:

		“我有三个儿子,长幼有序。但问题是,祖训说要立嫡子,所以等着皇后生子,一直没立太子,为妥善解决这一问题,特将皇长子、皇三子、皇五子全部封王,将来有嫡子,就立嫡子,没嫡子,再立长子,事就这么定了,你们赶紧去准备吧。”

		圣旨发到礼部,当时就炸了锅。这么大的事情,事先竟没听到风声,实在太不正常,于是几位领导一合计,拿着谕旨跑到内阁去问。

		这下连内阁的赵志皋和张位也惊呆了,什么圣旨,什么三王并封,搞什么名堂!?

		很明显,这事就是王锡爵办的。消息传出,举朝轰动,大家都认定,朝廷又出了个叛徒,而且还是主动投靠的。

		所有人都知道,万历已经很久不去找\footnote{幸。}皇后了,生儿子压根就是没影的事。所谓三王并封,就是扯淡,大家都能看出来,王锡爵你混了几十年,怎么看不出来?分明就是同谋,助纣为虐!

		再说皇帝,你都说好了,今年就办,到时候了竟然又不认账。搞个什么三王并封,我们大家眼巴巴地盼着,又玩花样,你当你耍猴子呢?!

		两天之后,算帐的人就来了。

		光禄寺丞朱维京第一个上书,连客套话都不说,开篇就骂:

		“您先前说过,万历二十一年就册立太子,朝廷大臣都盼着,忽然又说要并封,等皇后生子。这种说法,祖上从来就没有过!您不会是想愚弄天下人吧!”

		把戏被戳破了,万历很生气,立即下令将朱维京革职充军。

		一天后,刑部给事中王如坚又来了:

		“十四年时,您说长子幼小,等个两三年;十八年时,您又说您没有嫡子,长幼有序,让我们不必担心;十九年时,您说二十年就册立;二十年时,您又说二十一年举行;现在您竟然说不办了,改为分封,之前的话您不是都忘了吧,以后您说的话,我们该信那一句?”

		这话杀伤力实在太大,万历绷不住了,当即把王如坚免职充军。

		已经没用了,什么罚工资、降职、免职、充军,大家都见识过了,还能吓唬谁?

		最尴尬的,是礼部的头头脑脑们,皇帝下了圣旨,内阁又没有封还,按说是不能不办的。可是照现在这么个局势,如果真要去办,没准自己就被大家给办了。想来想去,搞了个和稀泥方案:三王并封照办,但同时也举行册立太子的仪式。

		方案报上去,万历不干:三王并封,就为不立太子,还想把我绕回去不成?

		既然给面子皇帝都不要,也就没啥说的了。礼部主事顾允成,工部主事岳元声,光禄寺丞王学曾等人继续上书,反对三王并封,这次万历估计也烦了,理都不理,随他们去。

		于是抗议的接着抗议,不理的照样不理,谁也奈何不了谁。

		局面一直僵持不下,大家这才突然发觉,还漏了一个关键人物——王锡爵。

		这事既然是王锡爵和皇帝干的,皇帝又不出头,也只能拿王锡爵开刀了。

		先是顾允成、张辅之等一群王锡爵的老乡上门,劝他认清形势,早日解决问题。然后是吏部主事顾宪成代表吏部全体官员写信给王锡爵,明白无误地告诉他:现在情况很复杂,大家都反对你的三王并封,想糊弄过去是不行的。

		王锡爵终于感受到了当年张居正的痛苦,不问青红皂白,就围上来群殴,没法讲道理,就差打上门来了。

		当然,一点也没差,打上门的终究来了。

		几天之后,礼部给事中史孟麟、工部主事岳元声一行五人,来到王锡爵办公的内阁,过来只干一件事:吵架。

		刚开始的时候,气氛还算不错,史孟麟首先发言,就三王并封的合理性、程序性一一批驳,有理有节,有根有据。

		事情到这儿,还算是有事说事,可接下来,就不行了。

		因为王锡爵自己也知道,三王并封是个烂事,根本就没法辩,心里理亏,半天都不说话。对方一句句地问,他半句都没答,憋了半天,终于忍不住了:

		“你们到底想怎么样?”

		岳元声即刻回答:

		“请你立刻收回那道圣旨,别无商量!”

		接着一句:

		“皇上要问,就说是大臣们逼你这么干的!”

		王锡爵气得不行,大声回复:

		“那我就把你们的名字都写上去,怎么样?!”

		这是一句威胁性极强的话。然而岳元声回答的声音却更大:

		“那你就把我的名字写在最前面!充军也好,廷杖也好,你看着办!”

		遇到这种不要命的二愣子,王锡爵也没办法,只好说了软话:

		“请你们放心,虽然三王并封,但皇长子出阁的时候,礼仪是不一样的。”

		首辅大人认输了,岳元声却不依不饶,跟上来就一句:

		“那是礼部的事,不是你的事!”

		谈话不欢而散,王锡爵虽然狼狈不堪,却也顶住了死不答应。

		因为虽然骂者众多,却还没有一个人能够找到他的死穴。

		这事看起来很简单,万历耍了个计谋,把王锡爵绕了进去,王大人背黑锅,哑巴吃黄连,有苦说不出。

		事实上,那是不可能的,王锡爵先生,虽然人比较实诚,也是在官场打滚几十年的老油条,万历那点花花肠子,他一清二楚,之所以同意三王并封,是将计就计。

		他的真正动机是,先利用三王并封,把皇长子的地位固定下来,然后借机周旋,更进一步逼皇帝册立太子。

		在他看来,岳元声之流都是白颈乌鸦,整天吵吵嚷嚷,除了瞎咋呼,啥事也干不成。所以他任人笑骂,准备忍辱负重,一朝翻身。

		然而这个世界上,终究还是有聪明人的。

		庶吉士李腾芳就算一个。

		李腾芳,湖广湘潭人\footnote{今湖南湘潭。}。从严格意义上讲,他还不是官,但这位仁兄人还没进朝廷,就有了朝廷的悟性,只用一封信就揭破了王锡爵的秘密。

		他的这封信,是当面交给王锡爵的,王大人本想打发这人走,可刚看几行字,就把他给拉住了:

		“公欲暂承上意,巧借王封,转作册立!”

		太深刻了,太尖锐了,于是王锡爵对他说:

		“请你坐下来,好好谈一谈。”

		李腾芳接下来的话,彻底打乱了王锡爵的部署:

		“王大人,你的打算是对的。但请你想一想,封王之后,恐怕册立还要延后,你还能在朝廷呆多久?万一你退了,接替你的人比你差,办不成这件事,负责任的人就是你!”

		王锡爵沉默了,他终于意识到,自己的计划蕴含着极大的风险,但他仍然不打算改正这个错误。因为在这个计划里,还有最后一道保险。

		李腾芳走了,王锡爵没有松口,此后的十几天里,跑来吵架的人就没断过。但王大人心里有谱,打死也不说,直到王就学上门的那一天。

		王就学是王锡爵的门生,自己人当然不用客气,一进老师家门就哭,边哭还边说:

		“这件事情\footnote{三王并封。}大家都说是老师干的,如此下去,恐怕老师有灭门之祸啊!”

		王锡爵却笑了:

		“你放心吧,那都是外人乱说的。我的真实打算,都通过密奏交给了皇上,即使皇长子将来登基,看到这些文书,也能明白我的心意。”

		这就是王先生的保险,然而王就学没有笑,只说了一句话:

		“老师,别人是不会体谅您的!一旦出了事,会追悔莫及啊!”

		王锡爵打了个寒战,他终于发现,自己的思维中,有一个不可饶恕的漏洞:

		如果将来册立失败,皇三子登基,看到了自己拥立长子的密奏,必然会收拾掉自己。

		而如果皇长子登基,即使他知道密奏,也未必肯替自己出头。因为长子登基,本来就是理所当然,犯不着感谢谁,到时,三王并封的黑锅只有他自己背。

		所以结论是:无论谁胜利,他都将失败!

		明知是赔本的生意,还要做的人,叫做傻子。王锡爵不是傻子,自然不做。万历二十一年二月,他专程拜见了万历,只提出了一个要求:撤回三王并封。

		这下万历就不干了,好不容易把你拉上船,现在你要洗手不干,留下我一个人背黑锅,怎么够意思?

		“你要收回此议,即无异于认错,如果你认错,我怎么办?我是皇帝,怎能被臣下挟持?”

		话说得倒轻巧,可惜王大人不上当:你是皇帝,即使不认错,大家也不能把你怎么样,我是大臣,再跟着淌混水,没准祖坟都能让人刨了。

		所以无论皇帝大人连哄带蒙,王锡爵偏一口咬定——不干了。

		死磨硬泡没办法,大臣不支持,内阁不支持,唯一的亲信跑路,万历只能收摊了。

		几天后,他下达谕令:

		“三王都不必封了,再等两三年,如果皇后再不生子,就册立长子。”

		可是大臣们不依不饶,一点也不消停,接着起哄,因为大家都知道,皇帝陛下您多少年不去找皇后了,皇后怎么生儿子,不想立就不想立,你装什么蒜?

		万历又火了,先是辟谣,说今年已经见过皇后,夫妻关系不好,纯属谣传,同时又下令内阁,对敢于胡说八道的人,一律严惩不怠。

		这下子王锡爵为难了,皇帝那里他不敢再去凑热闹了,大臣他又得罪不起,想来想去,一声叹息:我也辞职吧。

		说是这么说,可是皇帝死都不放,因为经历了几次风波之后,他已然明白,在手下这群疯子面前,一丝不挂十分危险,身前必须有个挡子弹的,才好平安过日子。

		于是王锡爵惨了,大臣轰他走,皇帝不让走,夹在中间受气,百般无奈之下,他决定拼一拼——找皇帝面谈。

		可是皇帝大人虽然不上班,却似乎很忙,王锡爵请示了好几个月,始终不见回音。眼看要被唾沫淹死,王大人急眼了,死磨硬泡招数全用上,终于,万历二十一年(1593年)十一月,他见到了万历。

		这是一次十分关键的会面,与会者只有两人,本来是天知地知,你知我知,但出于某种动机\footnote{估计是想保留证据。},事后王锡爵详细地记下了他们的每一句话。

		等了大半年,王锡爵已经毫无耐心:

		“册立一事始终未定,大臣们议论纷纷,烦扰皇上\footnote{包括他自己。},希望陛下早日决断,大臣自然无词。”

		万历倒还想得开:

		“我的主意早就定了,反正早晚都一样,人家说什么不碍事。”

		不碍事?敢情挨骂的不是你。

		可这话又不能明说,于是王大人兜了圈子:

		“陛下的主意已定,我自然是知道的,但外人不知道内情,偏要大吵大嚷,我为皇上受此非议深感不忿,不知道您有什么为难之处,要平白受这份闲气?”

		球踢过来了,但万历不愧为老运动员,一脚传了回去:

		“这些我都知道,我只担心,如果皇后再生儿子,该怎么办?”

		王锡爵气蒙了,就为皇后生儿子的破事,搞了三王并封,闹腾了足足半年,到现在还拿出来当借口,还真是不要脸,既然如此,就得罪了:

		“陛下,您这话几年前说出来,还过得去,现在皇子都十三岁了,还要等到什么时候!从古至今即使百姓家的孩子,十三岁都去读书了,何况还是皇子?!”

		这已经是老子训儿子的口气了,但万历同志到底是久经考验,毫不动怒,只是淡淡地说:

		“我知道了。”

		王锡爵仍不甘心,继续劝说万历,但无论他讲啥,皇帝陛下却好比橡皮糖,全无反应,等王大人说得口干舌燥,气喘吁吁,没打招呼就走人了,只留下王大人,痴痴地看着他离去的背影。

		谈话是完了,但这事没完,王锡爵回家之后,实在是气不过,一怒之下,又写了一封胆大包天的奏疏。

		因为这封奏疏的中心意思只有一个——威胁:

		“皇上,此次召对\footnote{即谈话。},虽是我君臣二人交谈,但此事不久后,天下必然知晓,若毫无结果,将被天下人群起攻之,我即使粉身碎骨,全家死绝,也无济于事!”

		这段话的意思是说,我和你谈过话,别以为大家都不知道,如果没给我一个结果,此事必将公之于天下,我完蛋了,你也得下水!

		这是硬的,还有软的:

		“臣进入朝廷三十余年了,一向颇有名声,现在为了此事,被天下人责难,实在是痛心疾首啊!”

		王锡爵是真没办法了,可万历却是王八吃秤砣,铁了心地对着干,当即写了封回信,训斥了王锡爵,并派人送到了内阁。

		按照常理,王大人看完信后,也只能苦笑,因为他虽为人刚正,却是个厚道人,从来不跟皇帝闹,可这一次,是个例外。

		因为当太监送信到内阁的时候,内阁的张位恰好也在。这人就没那么老实了,是个喜欢惹事的家伙,王锡爵拆信的时候,他也凑过来看。看完后,王锡爵倒没什么,他反而激动了。

		这位仁兄二话不说,当即怂恿王锡爵,即刻上疏驳斥万历。有了张位的支持,王锡爵浑似喝了几瓶二锅头,胆也壮了,针锋相对,写了封奏疏,把皇帝大人批驳得无地自容。

		王锡爵没有想到,他的这一举动,却起到了意想不到的效果。

		因为万历虽然顽固,却很机灵。他之所以敢和群臣对着干,无非是有内阁支持,现在王大人反水了,如果再闹下去,恐怕事情就没法收拾,于是他终于下圣旨:万历二十二年春,皇长子出阁读书。

		胜利在意想不到的时候来临了,王锡爵如释重负,虽然没有能够册立太子,但已出阁读书。无论如何,对内对外,都可以交代了。

		申时行没有办成的事情,王锡爵办成了,按说这也算是个政绩工程,王大人的位置应该更稳才是,然而事实并非如此。

		因为明代的大臣很执着,直来直往,说是册立,就必须册立。别说换名义,少个字都不行!所以出阁读书,并不能让他们满意,朝廷里还是吵吵嚷嚷地闹个不停。

		再加上另一件事,王锡爵就真是无路可走了。

		因为万历二十一年(1593年),恰好是京察年。

		所谓京察,之前已介绍过,大致相当于干部考核,每六年京察一次,对象是全国五品以下官员\footnote{含五品。},包括全国所有的地方知府及下属、以及京城的京官。

		虽然一般说来,明代的考察大都是糊弄事。但京察不同,因为管理京察的,是六部尚书之首的吏部尚书。收拾不了内阁大学士,搞定几个五品官还是绰绰有余的。

		所以每隔六年,大大小小的官员们就要胆战心惊一回。毕竟是来真格的,一旦京察被免官,就算彻底完蛋。

		这还不算,最倒霉的是,如果运气不好,主持考核的是个死脑筋的家伙,找人说情都没用,那真叫玩的就是心跳。

		万历二十一年(1593年)的这次京察,就是一次结结实实的心跳时刻。因为主持者,是吏部尚书孙鑨和考功司郎中赵南星。

		孙鑨倒没什么,可是赵南星先生,就真是个百年难得一遇的顽固型人物。

		赵南星,字梦白,万历二年进士。早在张居正当政时期,他就显示了自己的刺头本色,一直对着干。张居正死后获得提升,也不好好干,几年后就辞职回家了,据他自己说是身体不好,不想干了。

		此人不贪钱,不好色,且认死理,此前不久才再次出山,和吏部尚书一起主持京察。

		这么个人来干这么个事,很明显,就是来折腾人的。

		果不其然,京察刚一开始,他就免了两个人的官,一个是都给事中王三余,另一个是文选司员外郎吕胤昌。

		朝廷顿时一片恐慌。

		因为这两个人的官虽不大,身份却很特殊:王三余是赵南星的亲家,吕胤昌是孙鑨的外甥。

		拿自己的亲戚开刀,意思很明白:今年这关,你们谁也别想轻易过去。

		官不聊生的日子就此开始,六部及地方上的一大批官员纷纷落马,哭天喊地,声震寰宇,连内阁大学士也未能幸免。赵志皋的弟弟被赶回了家,王锡爵的几个铁杆亲信也糟了殃。

		赵志皋是个老实人,也不怎么闹。王锡爵就不同了,他上门逼张居正的时候,赵南星也就是个小跟班,要说闹事,你算老几?

		很快,几个言官便上疏攻击吏部的人事安排,从中挑刺。赵南星自然不甘示弱,上疏反驳,争论了几天,皇帝最后判定:吏部尚书孙鑨罚一年工资,吏部考功司郎中赵南星官降三级。

		这个结果实在不值得惊讶,因为那段时间,皇帝大人正在和王锡爵合伙搞三王并封。

		但王锡爵错了,因为赵南星先生,绝不是一个单纯的人。

		事实上,他之所以被拉到前台,去搞这次京察,是因为在幕后,有个人在暗中操纵着一切。

		这个人的名字,叫顾宪成。

		关于这位仁兄的英雄事迹,后面还要详细介绍,这里就不多说了,但可以确定的是,万历二十一年的这次京察,是在顾宪成的策划下,有预谋,有目的的政治攻击。关于这一点,连修明史的史官都看得清清楚楚。\footnote{明史·顾宪成传。}

		事实印证了这一点,前台刚刚下课,后台就出手了。一夜之间,左都御史李世达、礼部郎中于孔兼等人就冒了出来,纷纷上疏攻击,王大人又一次成为了靶子。

		关键时刻,万历同志再次证明,他是讲义气的,而且也不傻。

		奏疏送上去,他压根就没理,却发布了一道看似毫不相干的命令:

		吏部尚书孙鑨免职,吏部考功司郎中赵南星,削职为民。

		这条圣旨的意思是:别跟我玩花样,你们那点把戏我都明白,再闹,就连你们一起收拾。

		应该说效果十分明显,很快,大家都不闹了。看上去,王锡爵赢了,实际上,他输了,且输得很惨。

		因为孙鑨本就是个背黑锅的角色,官免了也就消停了。赵南星就不同了,硬顶王锡爵后,他名望大增,被誉为不畏强暴,反抗强权的代表人物。虽然打包袱回了老家,却时常有人来拜访,每年都有上百道奏疏送到朝廷,推荐他出来做官。而这位兄弟也不负众望,二十年后再度出山,闹出了更大的动静。

		王锡爵就此完蛋,他虽然赢得了胜利,却输掉了名声,在很多人看来,残暴的王锡爵严酷镇压了开明的赵南星,压制了正直与民意。

		这是一件十分有趣的事情,因为这一切,都似曾相识。

		十六年前,年轻官员王锡爵大摇大摆地迈进了张居正首辅的住所,慷慨激昂,大发议论后,扬长而去,然后名声大噪。

		十六年后,年轻官员赵南星向王锡爵首辅发起攻击,名满天下。

		当年的王锡爵,就是现在的赵南星,现在的王锡爵,就是当年的张居正,很有趣。

		有明一代,所谓的被压制者,未必真被压制,所谓的压制者,未必真能压制。

		遍览明代史料,曾见直言犯上者无数,细细分析之后,方才发觉:犯上是一定的,直言是不一定的。因为在那些直言背后,往往隐藏着不可告人的目的。

		\subsection{最后一根稻草}
		万历二十二年(1594年)五月,王锡爵提出辞呈。

		万历挽留了他很多次,但王锡爵坚持要走。

		自进入朝廷以来,王锡爵严于律己,公正廉洁,几十年来如履薄冰,兢兢业业,终成大器。

		万历二十一年,他受召回到朝廷担任首辅,二十二年离去,总共干了一年。

		但这一年,就毁掉了他之前几十年累积的所有名声。

		虽然他忍辱负重,虽然他尽心竭力,努力维护国家运转,调节矛盾,甚至还完成了前任未能完成的事\footnote{出阁读书。},却再也无法支撑下去。

		因为批评总是容易的,做事总是不容易的。

		王锡爵的离去,标志着局势的进一步失控。从此以后,天下将不可收拾。

		但没有人会料到,王大人辞职,将成为另一事件的导火线。和这件事相比,所谓的朝局纷争,册立太子,都不过是小儿科而已。
		\ifnum\theparacolNo=2
	\end{multicols}
\fi
\newpage
\section{东林崛起}
\ifnum\theparacolNo=2
	\begin{multicols}{\theparacolNo}
		\fi
		首辅走了,日子却还得过,原本排第二的赵志皋应该接班,但这人实在太软,谁都敢欺负他,上到皇帝,下到大臣,都觉得他压不住阵,于是皇帝下令,由大臣推荐首辅。

		幕后人物顾宪成就此出马。

		顾宪成,字叔时,江苏无锡人。万历四年参加乡试,考中第一名解元。三年后去考了进士,成绩平平,分配到户部当了个主事。当官后,最不喜欢的人是张居正,平日怎么别扭怎么来。

		比如张大人病重,大家都去上疏祷告,他不去,别人看他不上路,帮他署了名,他知道后不肯干休,非把自己的名字划掉,那是相当执着。不过这也没什么,当时和张大人对着干的人多了去了,不缺他一个。

		等到张居正死了,他就去了吏部,但也没升官,还接着当六品主事\footnote{正处级。},这中间还请了三年假。

		总之,这是个并不起眼的人。

		万历二十一年京察时,孙鑨是吏部尚书\footnote{正二品。},赵南星是考功司郎中\footnote{相当于司长,正五品。},而顾宪成只是个考功司员外郎\footnote{副手,从五品。}。

		万历八年进入朝廷,就当六品主事,混了十三年,才升了一级,实在有点说不过去。

		但就是这么个说不过去的人,却是这场风暴的幕后操纵者\footnote{实左右之。},不服都不行。

		更为神奇的是,事情闹大了,孙鑨撤职了,赵南星回家了,连王首辅都辞职了,他却是巍然不动。非但不动,还升了一级,当上了吏部文选司郎中。

		之前说过,文选司负责官员人事选拔,是吏部第一肥差。根据史料的记载,顾宪成大致属于性格顽固,遇事不转弯的人,如此个性,竟然能捞到这位置,实在有点不可思议。

		不可思议的事情还在后面,当初孙鑨刚被免职的时候,吏部没有部长,王锡爵打算趁机换人,推荐自己的亲信罗万化接班。顾宪成反对,推荐了右都御史陈有年。

		最终结果:吏部尚书陈有年。

		你要知道,王锡爵大人此时的职务,是内阁首辅、建极殿大学士,领吏部尚书衔兼太子太保,从一品。而顾宪成,是个刚提拔一年的五品郎中。

		王锡爵的后面,有万历撑腰。顾宪成的后面,什么都看不见。

		第一把手加第二把手,对付一个小小的司官,然而事实告诉我们,顾宪成赢了。

		因为在顾宪成的背后,是一片深不可测的黑夜。

		我认为,在那片黑暗中,隐藏着一股强大的力量。

		很快,事实就将再次验证这一点。

		当万历下令大臣推举入阁人选的时候,顾宪成先生又一次冒了出来,上疏推举人选。虽说这事的确归他管,但奇怪的是,如此重大的政治决策,吏部的几位侍郎竟然毫无反应,尚书陈有年也对他言听计从。史料上翻来覆去,只有他的光辉事迹,似乎吏部就他干活。

		而当万历同志看到顾宪成推举的那个名字时,差点没把桌子掀了。

		因为在顾宪成的名单上,第一个就是王家屏。

		作为吏部官员,顾宪成明知这家伙曾把皇帝折腾得七荤八素,竟然还要推荐此人,明摆着就是跟皇帝过不去。

		所以皇帝也忍无可忍了,终于打发顾宪成回了家。

		明代的官员,虽然罢官容易,升官倒也不难,只要过个几年,时局一变,立马就能回到朝廷重新来过。而以顾宪成之前的工作业绩和运动能量,东山再起不过是个时间问题。

		可谁也没想到,顾先生这一走,就再也没回来。

		虽然把这人开了,万历很有点快感,但由此酿成的后果,却是他死都想不到的。

		自明开国以来,无论多大能耐,无论有何背景,包括那位天下第一神算刘伯温,如果下野之后没能重新上台,慢慢地就边缘化了,然后走向同一结局——完蛋,从无例外。

		例外,从顾宪成开始。

		和赵南星一样,自从下野后,顾宪成名气暴涨。大家纷纷推举他再次出山,虽然没啥效果,也算捧了个人场。不久之后,他的弟弟顾允成和同乡高攀龙也辞官回了家,三个人一合计,反正闲着也是闲着,就讲学吧。

		这一讲就是三年,讲着讲着,人越来越多,于是有一天,顾宪成对高攀龙说了这样一句话:

		“我们应该找个固定的讲习场所。”

		地方是有的,在无锡县城的东头,有一个宋代学者杨时讲过学的场地,但年久失修,又太破,实在没法用,所以这事也就搁置了下来。

		七年后,出钱的主终于找到了。常州知府欧阳东凤和顾宪成关系不错,听说此事,大笔一挥就给办了,拨出专款修缮此地。此后,这里就成为了顾宪成等人的活动地点。

		它的名字叫做东林书院,实事求是地讲,确实也就是个书院。但在此后的几十年中,它却焕发了不可思议的魔力,成为了一种威力强大的信念,那些相信或接受的信众,历史上统称为东林党。

		无数人的命运,大明天下的时局,都将由这个看似与朝廷毫无关系的地方,最终确定。

		王锡爵回家去养老,顾宪成回家去讲学,王家屏自然也消停了,于是首辅的位置还是落到了赵志皋同志的身上。

		这就真叫害死人了,因为赵志皋压根就不愿意干!

		赵先生真是老资格了,隆庆二年中进士,先当翰林,再当京官,还去过地方。风风雨雨几十年,苦也吃了,罪也受了,七十多岁才混到首辅,也没啥意思。

		更为重要的是,他个性软弱,既不如申时行滑头,也不如王锡爵强硬。而明代的言官们大都不是什么善茬,一贯欺软怕硬。一旦坐到这个位置上,别说解决册立太子之类的敏感问题,光是来找茬的,都够他喝一壶。

		对此,赵先生十分清楚,所以他主动上疏,不愿意干,情愿回家养老。

		可是万历是不会同意的。好不容易找来个堵枪眼的,你要走了,我怎么办?

		无奈,赵志皋先生虽然廉颇老矣,不太能饭,但还是得死撑下去。

		于是,自万历二十二年起,他开始了四年痛苦而漫长的首辅生涯。具体表现为,不想干,没法干,却又不能走。

		说起来,他还是很敬业的。因为这几年正好是多事之秋,外面打日本,里面闹册立,搞得不可开交,赵大人外筹军备,内搞协调,日夜加班忙碌,干得还不错。

		可下面这帮大臣一点面子都不给,看他好欺负,就使劲欺负。宫里失火了有人骂他,天灾有人骂他,儿子惹事了有人骂他,甚至没事,也有人骂他,说他就该走人\footnote{言志皋宜放。}。

		实在欺人太甚,老实人终于也发火了。

		王锡爵在的时候,平素说一不二,动辄训斥下属,除了三王并封这种惹众怒的事情外,谁也不敢多嘴骂他。到赵志皋这儿,平易近人,待人和气,却老是挨骂,老先生一气之下,也骂人了:

		“都是内阁首辅,势大权重的,你们就争相依附求取进步,势小权轻的,你们就争相攻击,博取名声!”

		骂归骂,可下面这帮人实在啥觉悟也没有,还是喜欢拿老先生开涮。赵老头也真是倒霉,在这紧要关头,偏偏又出了事。

		事情出在兵部尚书石星的身上,如果你还记得,当时正值第一次抗倭援朝战争结束,双方谈判期间,石星最为信任的大忽悠沈惟敬正处于巅峰期,谈判前景似乎很乐观,石大人便通报领导,说和平很有希望。

		他的领导,就是赵志皋。

		赵大爷本来就不爱惹事,听了自然高兴,表示同意谈判。

		结果大家都知道了,所谓和平,全是沈惟敬、小西行长等中日两方的职业骗子们通力协作,忽悠出来的。事情败露后,沈惟敬杀头,石星坐牢。

		按说这事赵先生最多也就是个领导责任,可言官们实在是道德败坏,总找软柿子捏,每次弹劾石星,都要把赵大人稍带上。赵大人气得直喘气,要辞职,皇帝又不许。到万历二十六年,再撑不住了,索性回家养病休息,反正皇帝也不管。

		万历二十九年,赵大人死在了家里,不知是病死,还是老死。但我知道,他确实很累,因为直到他死的那天,辞职都没有批下来,用今天的话说,他应该算是死在了工作岗位上。

		赵志皋日子过得艰难,张位相对好点,因为他的脾气比较厉害,言官们没怎么敢拿他开刀。加上他是次辅,凡事没必要太出头,有赵首辅挡在前面,日子过得也可以。

		他唯一的问题,就是在抗倭援朝战争中,着力推荐了一个人。不但多次上疏保举,而且对其夸奖有加,说此人是不世出之奇才,必定能够声名远播,班师凯旋。

		这个人的名字,叫做杨镐。

		关于此人,我们之前已经说过了。从某个角度讲,他确实不负众望,虽然输了,还是输得声名远播,播到全国人民都晓得。随即开始追究责任。大臣们开骂,骂得张位受不了,就上疏皇帝,说:

		“大家都在骂我\footnote{群言交攻。},但我是忠于国家的,且毫无愧疚,希望皇上体察\footnote{惟上矜察。}。”

		皇帝说:

		“杨镐这个人,就是你暗中密奏,推荐给我的\footnote{密揭屡荐。}!我信了你,才会委派他做统帅,现在败仗打了,国威受损,你还敢说自己毫无愧疚\footnote{犹云无愧。}!?”

		到这个份上,估计也没啥说的了,张位连辞职的资格都没有,就被皇帝免职,走的时候没有一个人帮他说话。

		估计是受刺激太大,这位兄弟回家不久后就死了。

		至万历二十九年,内阁的几位元老全部死光,一个看似微不足道的人,就此踏上这个舞台。

		七年前,王锡爵辞职,朝廷推举阁臣,顾宪成推举了王家屏。但有一点必须说明:当时,顾先生推荐的,并非王家屏一人,而是七个。

		这七个人中,王家屏排第一,可是万历不买账,把顾宪成赶回了家。然而事实上,对顾先生的眼光,皇帝大人还是有所认可的,至少认可排第四的那个。

		南京礼部尚书沈一贯,第四。

		沈一贯,字肩吾,隆庆二年进士。算起来,他应该是赵志皋的同班同学,不过他的成绩比赵大人要好得多,当了庶吉士,后来又去翰林院,给皇帝讲过课。和之前几位类似,他跟张居正大人的关系也相当不好,不过他得罪张先生的原因,是比较搞笑的。

		事情经过是这样,有一天,沈教官给皇帝讲课,说着说着,突然发了个感慨,说自古以来,皇帝托孤,应该找个忠心耿耿的人,如果找不到这种人,还不如多教育自己的子女,亲历亲为。

		要知道,张居正同志的耳目是很多的,很快这话就传到了他的耳朵里,加上他的心胸又不算太宽广,所以张大人当政期间,沈一贯是相当地萧条,从未受到重用。

		相对于直言上疏、痛斥张居正,而落得同样下场的王锡爵等同志,我只能说,其实他不是故意的。

		张居正死后,沈一贯才出头,历任吏部左侍郎、翰林院侍读学士,后来又去了南京当礼部尚书。

		此人平素为人低调,看上去没有什么特点,然而,这只是表面现象而已。

		顾宪成是朝廷的幕后影响者,万历是至高无上的统治者,两人势不两立。

		所以一个既能被顾宪成推荐,又能被皇帝认可的人,是十分可怕的。

		万历二十二年(1594年),沈一贯被任命为吏部尚书兼东阁大学士,进入了帝国的决策层。

		很快,他就展示了他的异常之处,具体表现为,大家都欺负赵志皋,他不欺负。

		赵首辅实在是个彻头彻尾的软柿子,无论大小官员,从他身边过,都禁不住要捏一把,而对赵大人尊敬有加的,只有沈一贯\footnote{事皋甚恭。}。

		但沈一贯先生尊敬赵老头,绝非尊重老人,而是尊重领导,因为排第二的张位、排第三的陈于陛,他都很尊敬。

		沈一贯就这样扎下了根,在此后的七年之中,赵志皋被骂得养了病,陈于陛被骂得辞了官,都没他什么事,他还曾经联同次辅张位保举杨镐,据说还收了钱,可是杨镐事发,张位被弹劾免职,他竟安然无恙。

		到万历二十九年(1601年),死的死了,退的退了,只剩沈一贯,于是这个天字第一号大滑头终于成为了帝国的首辅。

		凭借多年的混事技术,沈先生游刃有余,左推右挡,皇帝信任,大臣也给面子,地位相当稳固,然而在历史上,沈一贯的名声一贯不佳,究其原因,就是他太过滑头。

		因为从某种角度来讲,朝廷首辅就是背黑锅的,国家那么多事,总得找一个负责的,但沈先生全然没有这个概念,能躲就躲能逃就逃,实在不太地道。

		而当时朝廷的局势,却已走到了一个致命的关口。

		万历二十九年,皇长子十九岁,虽然出阁读书,却依然不是太子,而且万历办事不厚道,对教自己儿子的讲官十分刻薄,一般人家请个老师,都要小心伺候,从不拖欠教师工资,万历却连饭都不管,讲官去教他儿子,还得自己带饭,实在太不像话。

		相对而言,皇三子就真舒服得多了,要什么有什么,备受万历宠爱,娇生惯养,啥苦都没吃过,且大有夺取太子之位的势头。

		这些情况大家都看在眼里,外加郑贵妃又是个百年难得一见的蠢人,丝毫不知收敛,极为嚣张,可谓是人见人恨,久而久之,一个父亲偏爱儿子的问题,就变成了恶毒地主婆欺负老实佃户的故事。

		问题越来越严重,舆论越来越激烈,万历是躲一天算一天的主,偏偏又来了这么个首辅,要知道,大臣们不闹事,不代表不敢闹事,一旦他们的怒火到达顶点,国家将陷入前所未有的骚乱。

		然而动乱没有爆发,因为这个曾经搞倒申时行、王锡爵、王家屏等无数政治高手,看似永远无法解决的问题,竟然被解决了。

		而解决它的,就是为人极不地道,一贯滑头的沈一贯。

		说起来,这是个非常玄乎的事。

		万历二十九年(1601年)八月,沈一贯向皇帝上疏,要求册立太子,其大致内容是,皇长子年纪大了,应该册立太子,正式成婚,到时有了孙子,您也能享子孙满堂的福啊。

		无论怎么看,这都是一封内容平平的奏疏,立意不新颖,文采很一般,按照以往的惯例,最终的结局应该是被压在文件堆下几年,再拉出去当柴禾烧。

		可惊喜总是存在的,就在第二天,沈一贯收到了皇帝的回复:

		“即日册立皇长子为太子!”

		沈一贯当时就懵了。

		这绝对不可能。

		争了近二十年,无数猛人因此落马,无数官员丢官发配,皇帝都被折腾得半死不活,却死不松口。

		然而现在,一切都解决了。

		事实摆在眼前,即日册立太子,非常清晰,非常明显。

		沈一贯欣喜若狂,他随即派人出去,通报了这一消息,于是举朝轰动了,所有的人都欢呼雀跃,为这个等待了许久的胜利。

		“争国本”就此落下帷幕,这场万历年间最激烈复杂的政治事件,共逼退首辅四人,部级官员十余人、涉及中央及地方官员人数三百多位,其中一百多人被罢官、解职、发配,闹腾得乌烟瘴气,还搞出了一个叫东林党的副产品,几乎所有人都不相信,它会有解决的一天。

		然而这件事情,却在最意想不到的时候,由最意想不到的人解决了,遭遇父亲冷落的朱常洛终于修成正果,荣登太子。

		但此事之中,仍然存在着一个最大的疑问:为什么那封上疏,能够破解这个残局?

		我不知道沈一贯有没有想过这个问题,但我想了。

		万历并不愚蠢,事实上,从之前的种种表现看,他是一个十分成熟的政治家,没有精神病史,心血来潮或是突发神经,基本都可以排除,而且他的意图十分明显——立皇三子。

		那么到底是什么原因,让他放弃了这个经历十余年的痛骂、折腾,却坚持不懈的企图?

		翻来覆去地审阅沈一贯的那封上疏,并综合此事发生前的种种迹象,我得出了结论:这是压死骆驼的最后一把稻草。

		万历从来就不想立皇长子,这是毫无疑问的,但疑问在于,他知道希望很渺茫,也知道手底下这帮大臣都是死脑筋,为何还要顶着漫天的口水和谩骂,用拖延战术硬扛十几年?

		如果没有充分的把握,皇帝大人是不会吃这个苦的。

		十几年来,他一直在等待两件事情的发生。然而这两件事他都没等到。

		我曾经分析过,要让皇三子超越皇长子继位,修改出生证明之类的把戏自然是没用的,必须有一个理由,一个能够说服所有人的理由,而这个答案只能是:立嫡不立长。

		只有立嫡子,才能压过长子,并堵住所有人的嘴。

		但皇三子就是皇三子,怎样可能变成嫡子呢?

		事实上,是可能的,只要满足一个条件——郑贵妃当皇后。

		只要郑贵妃当上皇后,皇后的儿子自然就是嫡子,皇三子继位也就顺理成章了。

		可是皇后只有一个,所以要让郑贵妃当上皇后,只能靠等,等到王皇后死掉,或是等时机成熟,把她废掉,郑贵妃就能顺利接位。

		可惜这位王皇后身体很好,一直活到了万历四十八年\footnote{这一年万历驾崩。},差点比万历自己活得还长,且她一向为人本分厚道,又深得太后的喜爱,要废掉她,实在没有借口。

		第一件事是等皇后,第二件事是等大臣。

		这事就更没谱了,万历原本以为免掉一批人,发配一批人,再找个和自己紧密配合的首辅,软硬结合就能把事情解决,没想到明代的大臣却是软硬都不吃,丢官发配的非但不害怕,反而很高兴,要知道,因为顶撞皇帝被赶回家,那是光荣,知名度噌蹭地往上涨,值大发了。

		所以他越严厉,越有人往上冲,只求皇帝大人再狠一点,最好暴跳如雷,这样名声会更大,效果会更好。

		而首辅那边,虽然也有几个听话的,无奈都是些老油条,帮帮忙是可以的,跟您老人家下水是不可以的。好不容易拉了个王锡爵下来,搞了三王并封,半路人家想明白了,又跑掉了。

		至于王家屏那类人,真是想起来都能痛苦好几天,十几年磨下来,人换了不少,朝廷越来越闹,皇后身体越来越好,万历同志焦头烂额,开始重新权衡利弊。

		我相信,在他下定决心的过程中,有一件事情起到了关键的作用。

		此事发生的具体时间不详,但应该在万历十四年之后。

		有一天,李太后和万历谈话,说起了皇长子,太后问:你为何不立他为太子?

		万历漫不经心地答道:他是宫女的儿子。

		太后大怒:你也是宫女的儿子!

		这就是活该了,万历整天忙里忙外,却把母亲的出身给忘了,要知道这位李老太太,当年也就是个宫女,因为长得漂亮才被隆庆选中,万历才当上了皇帝,如果宫女的儿子不能继位,那么万历兄是否应该引咎辞职呢?

		万历当即冷汗直冒,跪地给老太太赔不是,好说歹说才糊弄过去。

		这件事情,必定给他留下了极为深刻的印象。

		皇后没指望,老太太反对,大臣不买账,说众叛亲离,丝毫也不过分。万历开始意识到,如果不顾一切,强行立皇三子,他的地位都可能不保。

		在自己的皇位和儿子的皇位面前,所有成熟的政治家都会做出同样的抉择。

		决定政治动向的最终标准是利益,以及利益的平衡。

		这是一条真理。

		就这样,沈一贯捡了个大便宜,成就了册立太子的伟业,他的名声也如日中天,成为了朝廷大臣拥戴的对象。

		可你要说他光捡便宜,不做贡献,那也是不对的,事实上,他确实做了一件了不起的事。

		就在圣旨下达的第二天,万历反悔了,或许是不甘心十几年被人白喷了口水,或许是郑贵妃吹了枕头风,又找了借口再次延期,看那意思是不打算办了。

		但朝廷大臣们并没有看到这封推辞的诏书,因为沈一贯封还了。

		这位一贯滑头的一贯兄,终于硬了一回,他把圣旨退了回去,还加上了这样一句话:

		“万死不敢奉诏!”

		沈一贯的态度,深深地震慑了万历,他意识到,自己已经无路可退。

		万历二十九年十月,皇帝陛下正式册立皇长子朱常洛为太子,“争国本”事件正式结束。

		被压了十几年的朱常洛终于翻身,然而他的母亲,那位恭妃,却似乎永无出头之日。

		按说儿子当上太子,母亲至少也能封个贵妃,可万历压根就没提这件事,一直压着,直到万历三十四年,朱常洛的儿子出世,她才被封为皇贵妃。

		但皇贵妃和皇贵妃不一样,郑贵妃有排场,有派头,而王贵妃不但待遇差,连儿子来看他,都要请示皇帝,经批准才能见面。

		但几十年来,她没有多说过一句话,直到万历三十九年的那一天。

		她已经病入膏肓,不久于人世,而朱常洛也获准去探望他,当那扇大门洞开时,她再次见到了自己的儿子。

		二十九年前的那次偶遇,造就了她传奇的一生,从宫女到贵妃,再到未来的太后\footnote{死后追封。}。

		但是同时,这次偶遇也毁灭了她,因为万历同志很不地道,几十年如一日对她搞家庭冷暴力,既无恩宠,也无厚待,生不如死。

		然而她并不落寞,也无悔恨。

		因为她看到了自己的儿子,已经长大成人的儿子。

		青史留名的太后也好,籍籍无名的宫女也罢,都不重要。重要的是,作为一个母亲,在临终前看到了自己的儿子,看到他经历千难万苦,终于平安成人,这就足够了。

		所以,在这生命的最后一刻,她拉着儿子的衣角,微笑着说:

		“儿长大如此,我死何恨。”

		这里使用的是史料原文,因为感情,是无法翻译的。

		还有,其实这句话,她是哭着说的,但我认为,当时的她,很高兴。

		王宫女就此走完了她的一生,虽然她死后,万历还是一如既往地混账,竟然不予厚葬,经过当时的首辅叶向高反复请求,才得到了一个谥号。

		虽然她这一生,并没有什么可供传诵的事迹,但她已然知足。

		在这个世界上,所有的爱都是为了相聚,只有母爱,是为了分离。

		接受了母亲最后祝福的朱常洛还将继续走下去,在他成为帝国的统治者前,必须接受更为可怕的考验。

		\subsection{梃击}
		朱常洛是个可怜人,具体表现为出身低,从小就不受人待见,身为皇子,别说胎教,连幼儿园都没上过,直到十二岁才读书,算半个失学儿童。身为长子,却一直位置不稳,摇摇摆摆到了十九岁,才正式册立为太子。

		读书的时候,老师不管饭,册立的时候,仪式都从简,混到这个份上,怎个惨字了得。

		他还是个老实人,平时很少说话,也不闹事,待人也和气,很够意思,但凡对他好的,他都报恩。比如董其昌先生,虽被称为明代最伟大的天才画家,但人品极坏,平日欺男霸女,鱼肉百姓,闹得当地百姓都受不了,但就是这么个人,因为教过他几天,辞官后还特地召回,给予优厚待遇。

		更为难得的是,对他不好的,他也不记仇,最典型的就是郑贵妃,这位妇女的档次属于街头大妈级,不但多事,而且闹事,屡次跟他为难,朱常洛却不以为意,还多次替其开脱。

		无论从哪个角度看,他都是一个不折不扣的好人。

		但历史已经无数次证明,在皇权斗争中,好人最后的结局,就是废人。

		虽然之前经历风风雨雨,终于当上太子,但帝国主义亡我之心不死,只要万历一天不死,他一天不登基,幕后的阴谋将永不停息,直至将他彻底毁灭。

		现实生活不是电影,坏人总是赢,好人经常输,而像朱常洛这种老好人,应该算是稳输不赢。

		可是这一次,是个例外。

		事实证明,万历二十九年,朱常洛被册立为太子,不过是万里长征走完了第一步,两年后,麻烦就来了。

		这是一个很大的麻烦,大到国家动荡,皇帝惊恐,太子不安,连老滑头沈一贯都被迫下台。

		但有趣的是,惹出麻烦的,既不是朱常洛,也不是郑贵妃,更不是万历,事实上,幕后黑手到底是谁,直至今日,也无人知晓。

		万历三十一年十一月,一篇文章在朝野之间开始流传,初始还是小范围内传抄,后来索性变成了大字报,民居市场贴得到处都是,识字不识字都去看,短短十几天内朝廷人人皆知,连买菜的老大娘都知道了,在没有互联网和手机短信的当年,传播速度可谓惊人。

		之所以如此轰动,是因为这篇文章的内容,实在是太过火爆。

		此文名叫《续忧危竑议》,全篇仅几百字,但在历史上,它却有一个诡异的名字——“妖书”。

		在这份妖书中,没有议论,没有叙述,只有两个人的对话,一个人问,一个人答。问话者的姓名不详,而回答的那个人,叫做郑福成。这个名字,也是文中唯一的主角。

		文章一开始,是两个人在谈事。一个说现在天下太平,郑福成当即反驳,说目前形势危急。因为皇帝虽然立了太子,但那是迫于沈一贯的要求,情非得已,很快就会改立福王。

		这在当年,就算是反动传单了,而且郑福成这个名字,也很有技术含量,郑贵妃、福王、成功三合一,可谓言简意赅。

		之所以被称为妖书,只说皇帝太子,似乎还不合格,于是内阁的两位大人,也一起下了水。

		当时的内阁共有三人,沈一贯是首辅,另外两人是沈鲤和朱赓。妖书的作者别出心裁,挑选了沈一贯和朱赓,并让他们友情客串,台词如下:

		问:你怎么知道皇帝要改立福王呢?

		郑福成答:你看他用朱赓,就明白了。朝中有这么多人,为什么一定要用朱赓呢?因为他姓朱,名赓,赓者,更也。真正的意思,就是改日更立啊\footnote{佩服,佩服。}。

		这是整朱赓,还有沈一贯同志:

		问:难道沈一贯不说话吗?

		郑福成答:沈一贯这个人阴险狡诈,向来是有福独享,有难不当,是不会出头的。

		闹到这个份上,作者还不甘心,要把妖书进行到底,最后还列出了朝廷中的几位高官,说他们都是改立的同党,是大乱之源。

		更为搞笑的是,这篇妖书的结尾,竟然还有作者署名!

		落款者分别是吏科都给事中项应祥,四川道御史乔应甲。

		这充分说明,妖书作者实在不是什么良民,临了还要耍人一把,难能可贵的是,他还相当有版权意识,在这二位黑锅的名下还特别注明,项应祥撰\footnote{相当于原著。},乔应甲书\footnote{相当于执笔。}。

		这玩意一出来,大家都懵了。沈一贯当即上书,表示自己非常愤怒,希望找出幕后主使人,与他当面对质,同时他还要求辞官,以示清白以及抗议。

		而妖书上涉及的其他几位高级官员也纷纷上书,表示与此事无关,并要求辞职。

		最倒霉的人是朱赓,或许是有人恶搞他,竟然把一份妖书放在了他的家门口。这位朱先生是个厚道人,吓得不行,当即把这份妖书和自己的奏疏上呈皇帝,还一把鼻涕一把泪地哭诉,说我今年都快七十了,有如此恩宠已是意外,也没啥别的追求,现在竟然被人诬陷,请陛下让我告老还乡。

		朝廷一片混乱,太子也吓得不行。他刚消停两年,就出这么个事,闹不好又得下去,整日坐卧不安,担惊受怕。

		要说还是万历同志久经风雨,虽然愤怒,倒不怎么慌。先找太子去聊天,说我知道这不关你的事,好好在家读书,别出门。

		然后再发布谕令,安抚大臣,表示相信大家,不批准辞职,一个都别走。

		稳定情绪后,就该破案了。像这种天字第一号政治案件,自然轮不上衙门捕快之类的角色,东厂锦衣卫倾巢而出,成立专案组,没日没夜地查,翻天覆地地查。

		万历原本以为,来这么几手,就能控制局势,然而这场风暴,却似乎越来越猛烈。

		首先是太子,这位仁兄原本胆小,这下更是不得了,窝在家里哪里都不去,唯恐出事。而郑贵妃那边也不好受,毕竟妖书针对的就是她,千夫所指,舆论压力太大,每日只能以泪洗面,不再出席任何公开活动。

		内阁也消停了,沈一贯和朱赓吓得不行,都不敢去上班,呆在家里避风头。日常工作只有沈鲤干,经常累得半死。大臣们也怕,因为所有人都知道,平时争个官位,抢个待遇的没啥,这个热闹却凑不得。虽说皇帝大人发话,安抚大家不让辞职,可这没准是放长线钓大鱼,不准你走,到时候来个一锅端,那就麻烦大了。

		总而言之,从上到下,一片人心惶惶。很多人都认定,在这件事情的背后,有很深的政治背景。

		确实如此。

		这是一件明代历史上著名的政治疑案,至今仍无答案,但从各种蛛丝马迹之中,真相却依稀可辨。

		可以肯定的是,这件事情应该与郑贵妃无关,因为她虽然蠢,也想闹事,却没必要闹出这么大动静,把自己挤到风口浪尖受罪,而太子也不会干这事,以他的性格,别人不来惹他就谢天谢地,求神拜佛了。

		作案人既不是郑贵妃,也不是太子,但可以肯定的是,作案者,必定是受益者。

		在当时的朝廷中,受益者不外乎两种,一种是精神受益者,大致包括看不惯郑贵妃欺压良民,路见不平也不吼,专门暗地下黑手的人,写篇东西骂骂出口气。

		这类人比较多,范围很大,也没法子查。

		第二种是现实受益者。就当时的朝局而言,嫌疑人很少——只有两个。

		这两个人,一个是沈一贯,另一个是沈鲤。

		这二位仁兄虽然是本家,但要说他们不共戴天,也不算夸张。

		万历二十九年,沈一贯刚刚当首辅的时候,觉得内阁人太少,决定挑两个跑腿的,一个是朱赓,另一个是沈鲤。

		朱赓是个老实人,高高兴兴地上班了,沈鲤却不买账,推辞了很多次,就是不来。沈一贯以为他高风亮节,也就没提这事。

		可两年之后,这位仁兄竟然又入阁了。沈一贯同志这才明白,沈鲤不是不想入阁,而是不买他的帐。因为这位本家资历老,名望高,还给皇帝讲过课,关系很好,压根就看不起自己。

		看不起自然就不合作,外加沈鲤也不是啥善人,两人在内阁里一向是势不两立。

		而现在妖书案发,内阁三个人,偏偏就拉上了沈一贯和朱赓,毫无疑问,沈鲤是有嫌疑的。

		这是我的看法,也是沈一贯的看法。

		这位老油条在家呆了好几天,稳定情绪之后,突然发现这是一个绝佳的机会。

		他随即恢复工作,以内阁首辅的身份亲自指挥东厂锦衣卫搜捕,而且还一反往日装孙子的常态,明目张胆对沈鲤的亲信,礼部侍郎郭正域下手,把他的老乡、朋友、下属、仆人全都拉去审问。

		在这个不寻常的行动背后,是一个不寻常的算盘:

		如果事情是沈鲤干的,那么应该反击,这叫报复,如果事情不是沈鲤干的,那么也应该反击,这叫栽赃。

		在这一光辉思想的指导下,斗争愈演愈烈,沈鲤的亲信被清算,他本人也未能幸免,锦衣卫派了几百人到他家,也不进去,也不闹事,就是不走,搞得沈鲤门都出不去,十分狼狈。

		但沈先生如果没两把刷子,是不敢跟首辅叫板的,先是朱常洛出来帮忙叫屈,又传话给东厂的领导,让他们不要乱来,后来连万历都来了,直接下令不得骚扰沈鲤。

		沈一贯碰了钉子,才明白这个冤家后台很硬,死拼是不行的,他随即转换策略,命令锦衣卫限期破案——抓住作案人,不怕黑不了你。

		可是破案谈何容易,妖书满街都是,传抄者无数,鬼才知道到底哪一张纸才是源头,十一月十日案发,查到二十日,依然毫无进展。

		东厂太监陈矩,锦衣卫都督王之桢急得直跳脚,如果还不破案,这官就算当到头了。

		二十一日,案件告破。

		说起来,这起妖书案是相当的妖,案发莫名其妙不说,破案也破得莫名其妙。二十一日这天,先是锦衣卫衙门收到一份匿名检举信,后又有群众举报,锦衣卫出动,这才逮住了那个所谓的真凶:皦生光。

		皦生光先生是什么人呢?

		答案是——什么人都不是。

		这位仁兄既不是沈鲤的人,也不是沈一贯的人,他甚至根本就不是官员,而只是一个顺天府的秀才。

		真凶到案,却没有人心大快,恰恰相反,刚刚抓到他的时候,朝廷一片哗然,大家都说锦衣卫和东厂太黑,抓不到人了弄这么个人来背锅。

		这种猜测很有道理,因为那封妖书,不是一个秀才能写得出来的。

		那年头,群众参政议政积极性不高,把肚子混饱就行,谁当太子鬼才关心。更何况沈一贯和朱赓的关系,以及万历迫不得已才同意立长子这些情况,地方官都未必知道,一个小秀才怎么可能清楚?

		但细细一查,才发现这位仁兄倒还真有点来头。

		原来皦生光先生除了是秀才外,还兼职干过诈骗犯。具体方法是欺负人家不识字,帮人写文章,里面总要带点忌讳,不是用皇帝的避讳字,就是加点政治谣言。等人家用了,再上门勒索,说你要不给钱,我就跑去报官云云。

		后来由于事情干得多了,秀才也被革了,发配到大同当老百姓,最近才又潜回北京。

		可即便如此,也没啥大不了,归根结底,他也就是个普通混混,之所以被确定为重点嫌疑人,是因为他曾经敲诈过一个叫郑国泰的人。

		郑国泰,是郑贵妃的弟弟。

		一个穷秀才,又怎么诈骗皇亲国戚呢?

		按照锦衣卫的笔录,事情大致是这样的:有个人要去郑国泰家送礼,要找人写文章,偏偏这人不知底细,找到了皦生光。皦秀才自然不客气,发挥特长,文章里夹了很多私货,一来二去,东西送进去了。

		一般说来,以郑国泰的背景,普通的流氓是不敢惹的,可皦生光不是普通的流氓,胆贼大,竟然找上了门,要郑大人给钱。至于此事的结局,说法就不同了,有的说郑国泰把皦生光打了一顿,赶出了门,也有的说郑国泰胆小,给钱私了。

		但无论如何,皦秀才终究和此事搭上了边。有了这么个说法,事情就好办了,侦查工作随即开始,首先是搜查,家里翻个底朝天,虽说没找到妖书,但发现了一批文稿,据笔迹核对\footnote{司法学名:文检。},与妖书的初期版本相似\footnote{注意,是相似。}。

		之后是走访当地群众,以皦秀才平日的言行,好话自然没有,加上这位兄弟又有前科,还进过号子,于是锦衣卫最后定案:有罪。

		案子虽然定了,但事情还没结。因为明朝的司法制度十分严格,处决人犯必须经过司法审讯。即便判了死罪,还得由皇帝亲自进行死刑复核,这才能把人拉出去咔嚓一刀。

		所以万历下令,鉴于案情重大,将此案送交三法司会审。

		之前提过,三法司,即是明朝的三大司法机关:大理寺、都察院、刑部,大致相当于今天的司法部、监察部、最高人民检察院、最高人民法院等若干部门。

		三法司会审,是明代最高档次的审判,也是最为公平的审判。倒不是三法司这帮人有啥觉悟,只是因为参与部门多,把每个人都搞定,比较难而已。例如当年的严世藩,人缘广,关系硬,都察院、大理寺都有人,偏偏刑部的几个领导是徐阶的人,最后还是没躲过去。

		相比而言,像皦秀才这种要钱没钱要权没权的人,死前能捞个三司会审,也就不错了,结案只是时间问题。

		可是这起案件,远没有想象中那么简单。

		一到三法司,皦秀才就不认账了。虽说之前他曾招供,说自己是仇恨郑国泰,故意写妖书报复,但那是在锦衣卫审讯时的口供。锦衣卫是没有善男信女的,也不搞什么批评教育,政策攻心,除了打就是打,口供是怎么来的,大家心里都有数。现在进了三法司,看见来了文明人,不打了,自然就翻了案。

		更麻烦的是,沈一贯和朱赓也不认。

		这二位明显是被妖书案整惨了,心有不甘,想借机会给沈鲤点苦头吃。上疏皇帝,说证词空泛,不可轻信,看那意思,非要搞出个一二三才甘心。

		所以在审讯前,他们找到了萧大亨,准备做手脚。

		萧大亨,时任刑部尚书,是沈一贯的亲信,接到指令后心领神会,在审讯时故意诱供,让皦秀才说出幕后主使。

		可是皦秀才还真够意思,问来问去就一句话:

		“无人主使!”

		萧大亨没办法,毕竟是三法司会审,搞得太明显也不好,就给具体负责审案的下属,刑部主事王述古写了张条子,还亲自塞进了他的袖口,字条大意是,把这件事情往郭正域、沈鲤身上推。

		没想到王述古接到条子,看后却大声反问领导:

		“案情不出自从犯人口里,却要出自袖中吗?!”

		萧大亨狼狈不堪,再也不敢掺和这事。

		沈鲤这边也没闲着,他知道沈一贯要闹事,早有防备:你有刑部帮忙,我有都察院撑腰。一声令下,都察院的御史们随即开动,四下活动,灭火降温,准备冷处理此事。

		其中一位御史实在过于激动,竟然在审案时,众目睽睽之下,对皦秀才大声疾呼:

		“别牵连那么多人了,你就认了吧。”

		审案审到这个份上,大家都是哭笑不得,要结案,结不了;不结案,又没个交代,皇帝、太子、贵妃、内阁,谁都不能得罪。万一哪天皦秀才吃错了药,再把审案的诸位领导扯进去,那真是哭都没眼泪。

		三法司的人急得不行,可急也没用,于是有些不地道的人就开始拿案件开涮。

		比如有位审案御史,有一天突然神秘地对同事说,他已经确定,此案一定是皦秀才干的。

		大家十分兴奋,认定他有内部消息,纷纷追问他是怎么知道的。

		御史答:

		“昨天晚上我做梦,观音菩萨告诉我,这事就是他干的。”

		当即笑瘫一片。

		没办法,就只能慢慢磨,开审休审,休审开审,周而复始,终于有一天,事情解决了。

		皦生光也受不了了,天天审问,天天用刑,天天折腾,还不如死了好,所以他招供了:

		“是我干的,你们拿我去结案吧。”

		这个世界清净了。

		万历三十二年(1604年)四月,皦生光被押赴刑场,凌迟处死。

		妖书案就此结束,虽说闹得天翻地覆,疑点重重,但有一点是肯定的,那就是:皦生光很冤枉。

		因为别的且不谈,单说妖书上列出的那些官员,就皦秀才这点见识,别说认识,名字都记不全。找这么个人当替死鬼,手真狠,心真黑。

		妖书何人所写,目的何在,没人知道,似乎也没人想知道。

		因为有些时候,真相其实一点也不重要。

		妖书案是结了,可轰轰烈烈的斗争又开始了。沈一贯被这案子整得半死不活,气得不行,卯足了劲要收拾沈鲤。挖坑、上告、弹劾轮番上阵,可沈鲤同志很是强悍,怎么搞都没倒。反倒是沈一贯,由于闹得太过,加上树大招风,竟然成为了言官们的新目标。骂他的人越来越多,后来竟然成了时尚\footnote{弹劾日众。}。

		沈一贯眼看形势不妙,只好回家躲起来,想要避避风头,没想到这风越刮越大,三年之间,弹劾他的奏疏堆起来足有一人高,于是他再也顶不住了。

		万历三十四年(1606年),沈一贯请求辞职,得到批准。

		有意思的是,这位仁兄走之前,竟然还提了一个要求:我走,沈鲤也要走。

		恨人恨到这个份上,也不容易。

		而更有意思的是,万历竟然答应了。

		这是一个不寻常的举动,因为沈鲤很有能力,又是他的亲信。而沈一贯虽说人滑了点,办事还算能干,平时朝廷的事全靠这两人办,万历竟然让他们全都走人,动机就一个字——烦。

		自打登基以来,万历就没过几天清净日子。先被张居正压着,连大气都不敢出,等张居正一死,言官解放,吵架的来了,天天闹腾。到生了儿子,又开始争国本,堂堂皇帝,竟然被迫就范。

		现在太子也立了,某些人还不休息,跟着搞什么妖书案,打算混水摸鱼,手下这两人还借机斗来斗去,时不时还以辞职相威胁,太过可恶。

		既然如此,你们就都滚吧,有多远滚多远,让老子清净点!

		沈一贯和沈鲤走了,内阁只剩下了朱赓。

		这一年,朱赓七十二岁。

		朱赓很可怜,他不但年纪大,而且老实,老实到他上任三天,就有言官上书骂他,首辅大人心态很好,统统不理。

		可让他无法忍受的是,他不理大臣,皇帝也不理他。

		内阁人少,一个七十多的老头起早贪黑熬夜,实在扛不住,所以朱赓多次上书,希望再找几个人入阁。

		可是前后写了十几份报告,全都石沉大海,到后来,朱大人忍不住了,可怜七十多岁的老大爷,亲自跑到文华门求见皇帝,等了半天,却还是吃了闭门羹。

		换在以前,皇帝虽然不上朝,但大臣还是要见的,特别是内阁那几个人,这样才能控制朝局。比如嘉靖,几十年不上朝,但没事就找严嵩、徐阶聊天,后来索性做了邻居,住到了一起\footnote{西苑。}。

		但万历不同,他似乎是不想干了。在他看来,内阁一个人不要紧,没有人也不要紧,虽然朱首辅七十多了,也还活着嘛。能用就用,累死了再说,没事就别见了,也不急这几天,会有人的,会见面的,再等等吧。

		就这样,朱老头一边等一边干,一个人苦苦支撑,足足等了一年,既没见到助手,也没见过皇帝。

		这一年里朱老头算被折腾惨了,上书国政,皇帝不理,上书辞职,皇帝也不理,到万历三十四年(1607年),朱赓忍无可忍,上书说自己有病,竟然就这么走了。

		皇帝还是不理。

		最后一个也走了。

		内阁没人呆,首辅没人干,经过万历的不懈努力,朝廷终于达到了传说中的最高境界——千山鸟飞绝,万径人踪灭。

		自明代开国以来,只有朱元璋在的时候,既无宰相,也无内阁,时隔多年,万历同志终于重现往日荣光。

		而对于这一空前绝后的盛况,万历很是沉得住气,没人就没人,日子还不是照样过?

		但很快,他就发现这日子没法过了。

		因为内阁是联系大臣和皇帝的重要渠道,而且内阁有票拟权,所有的国家大事,都由其拟定处理意见,然后交由皇帝审阅批准。所以即使皇帝不干活,国家也过得去。

		朱元璋不用宰相和内阁,原因在于他是劳模,什么都能干。而万历先生连文件都懒得看,你要他去干首辅的活,那就是白日做梦。

		朝廷陷入了全面瘫痪,这么下去,眼看就要破产清盘,万历也急了,下令要大臣们推举内阁人选。

		几番周折后,于慎行、叶向高、李廷机三人成功入阁,班子总算又搭起来了。

		但这个内阁并没有首辅,因为万历特意空出了这个位置,准备留给一个熟人。

		这个人就是王锡爵,虽说已经告老还乡,但忆往昔,峥嵘岁月稠。之前共背黑锅的革命友谊,给万历留下了深刻的印象,所以他派出专人,去请王锡爵重新出山,并同时请教他一个问题。

		王锡爵不出山。

		由于此前被人坑过一次,加上都七十四岁了,王锡爵拒绝了万历的下水邀请,但毕竟是多年战友,还教过人家,所以,他解答了万历的那个疑问。

		万历的问题是,言官太过凶悍,应该如何应付。

		王锡爵的回答是,他们的奏疏你压根别理\footnote{一概留中。},就当是鸟叫\footnote{禽鸟之音。}!

		我觉得,这句话十分之中肯。

		此外,他还针对当时的朝廷,说了许多意见和看法,为万历提供了借鉴。

		然后,他把这些内容写成了密疏,派人送给万历。

		这是一封极为机密的信件,其内容如果被曝光,后果难以预料。

		所以王锡爵很小心,不敢找邮局,派自己家人携带这封密信,并反复嘱托,让他务必亲手交到朝廷,绝不能流入任何人的手中,也算是吸取之前申时行密疏走光的经验。

		但他做梦也没想到,这一次,他的下场会比申时行还惨。

		话说回来,这位送信的同志还是很敬业的,拿到信后立即出发,日夜兼程赶路,一路平安,直到遇见了一个人。

		当时他已经走到了淮安,准备停下来歇脚,却听说有个人也在这里,于是他便去拜访了此人。

		这个人的名字,叫做李三才。

		李三才,字道甫,陕西临潼人,时任都察院右佥都御史,凤阳巡抚。

		这个名字,今天走到街上,问十个人估计十个都不知道,但在当年,却是天下皆知。

		关于此人的来历,只讲一点就够了:

		二十年后,魏忠贤上台时,编了一本东林点将录,把所有跟自己作对的人按照水浒一百单八将称号,以实力排序,而排在此书第一号的,就是托塔天王李三才。

		总而言之,这是一个十分厉害的人物。

		因为淮安正好归他管,这位送信人原本认识李三才,到了李大人的地头,就去找他叙旧。

		两人久别重逢,聊着聊着,自然是要吃饭,吃着吃着,自然是要喝酒,喝着喝着,自然是要喝醉。

		送信人心情很好,聊得开心,多喝了几杯,喝醉了。

		李三才没有醉,事实上,他非常清醒,因为他一直盯着送信人随身携带的那口箱子。

		在安置了送信人后,他打开了那个箱子,因为他知道,里面必定有封密信。

		得知信中内容之后,李三才大吃一惊,但和之前那位泄露申时行密疏的罗大纮不同,他并不打算公开此信,因为他有更为复杂的政治动机。

		手握着这封密信,李三才经过反复思考,终于决定:篡改此信件。

		在他看来,篡改信件,更有利于达到自己的目的。

		所谓篡改,其实就是重新写一封,再重新放进盒子里,让这人送过去,神不知鬼不觉。

		可是再一细看,他就开始感叹:王锡爵真是个老狐狸。

		古代没有加密电报,所以在传送机密信件时,往往信上设有暗号,两方约定,要么多写几个字,要么留下印记,以防被人调包。

		李三才手中拿着的,就是一封绝对无法更改的信,倒不是其中有什么密码,而是他发现,此信的写作者,是王时敏。

		王时敏,是王锡爵的孙子,李三才之所以认定此信系他所写,是因为这位王时敏还有一个身份——著名书法家。

		这是真没法了,明天人家就走了,王时敏的书法天下皆知,就自己这笔字,学都没法学,短短一夜时间,又练不出来。

		无奈之下,他只好退而求其次,抄录了信件全文,并把信件放了回去。

		第二天,送信人走了,他还要急着把这封密信交给万历同志。

		当万历收到此信时,绝不会想到,在他之前,已经有很多人知道了信件的内容,而其中之一,就是远在无锡的普通老百姓顾宪成。

		这件事可谓疑团密布,大体说来,有几个疑点:

		送信人明知身负重任,为什么还敢主动去拜会李三才,而李三才又为何知道他随身带有密信,之后又要篡改密信呢?

		这些问题,我可以回答。

		送信人去找李三才,是因为李大人当年的老师,就是王锡爵。

		非但如此,王锡爵还曾对人说,他最喜欢的学生,就是李三才。两人关系非常的好,所以这位送信人到了淮安,才会去找李大人吃饭。

		作为凤阳巡抚,李三才算是封疆大吏,而且他本身就是都察院的高级官员,对中央的政治动向十分关心,皇帝为什么找王锡爵,找王锡爵干什么,他都一清二楚,唯一不清楚的,就是王锡爵的答复。

		最关键的问题来了,既然李三才是王锡爵的学生,还算他的亲信,李三才同志为什么要背后一刀,痛下杀手呢?

		因为在李三才的心中,有一个人,比王锡爵更加重要,为了这个人,他可以出卖自己的老师。

		万历二年(1574年),李三才考中了进士,经过初期培训,他分到户部,当上了主事,几年之后,另一个人考中进士,也来到了户部当主事,这个人叫顾宪成。

		这之后他们之间发生了什么事情,史书上没有写,我也不知道,但是我惊奇地发现,当顾宪成和李三才在户部做主事的时候,他们的上司竟然叫赵南星。

		联想到这几位后来在朝廷里呼风唤雨的情景,我们有理由相信,在那些日子里,他们谈论的应该不仅仅是仁义道德,君子之交,暗室密谋之类的把戏也没少玩。

		李三才虽然是东林党,但道德水平明显一般,他出卖王老师,只是因为一个目的——利益。

		只要细细分析一下,就能发现,李三才涂改信件的真正动机。

		当时的政治形势看似明朗,实则复杂,新成立的这个三人内阁,可谓凶险重重,杀机无限。

		李廷机倒还好说,这个人性格软弱,属于和平派,谁也不得罪,谁也不搭理,基本可以忽略。

		于慎行就不同了,这人是朱赓推荐的,算是朱赓的人,而朱赓是沈一贯的人,沈一贯和王锡爵又是一路人,所以在东林党的眼里,朱赓不是自己人。

		剩下的叶向高,则是一个非同小可的人,此后一系列重大事件中,他起到了极为关键的作用,此人虽不是东林党,却与其有着千丝万缕的联系,是个合格的地下党。

		这么一摆,你就明白了,内阁三个人,一个好欺负,两个搞对立,遇到事情,必定会僵持不下。

		僵持还算凑合,可要是王锡爵来了,和于慎行团结作战,东林党就没戏了。

		虽然王锡爵的层次很高,公开表明自己不愿去,但东林党的同志明显不太相信,所以最好的办法,就是打开那封信,看个究竟。

		在那封信中,李三才虽然没有看到重新出山的许诺,却看到了毫无保留的支持,为免除后患,他决定篡改。

		然而由于写字太差,没法改,但也不能就此算数,为了彻底消除王锡爵的威胁,他抄录并泄露了这封密信,而且特意泄露给言官。

		因为在信中,王锡爵说言官发言是鸟叫,那么言官就是鸟人了。鸟人折腾事,是从来不遗余力的。

		接下来的事情可谓顺其自然,舆论大哗,言官们奋笔疾书,把吃奶的力气拿出来痛骂王锡爵,言辞极其愤怒,怎么个愤怒法,举个例子你就知道了。

		我曾翻阅过一位言官的奏疏,内容就不说了,单看名字,就很能提神醒脑——巨奸涂面丧心比私害国疏。

		如此重压之下,王锡爵没有办法,只好在家静养,从此不问朝政,后来万历几次派人找他复出,他见都不见,连回信都不写,估计是真的怕了。

		事情的发展,就此进入了顾宪成的轨道。

		王锡爵走了,朝廷再也没有能担当首辅的人选,于是李廷机当上了首辅,这位兄弟不负众望,上任后不久就没顶住骂,回家休养,谁叫也没用,基本算是罢工了。

		而异类于慎行也不争气,刚上任一年就死了,就这样,叶向高成为了内阁的首辅,也是唯一的内阁大臣。

		对手被铲除了,这是最好的结局。

		必须说明的是,所谓李三才和顾宪成的勾结,并不是猜测,因为在史料翻阅中,我找到了顾宪成的一篇文章。

		在文章中,有这样几句话:

		“木偶兰溪、四明、婴儿山阴、新建而已,乃在遏娄江之出耳。”

		“人亦知福清之得以晏然安于其位者,全赖娄江之不果出……密揭传自漕抚也,岂非社稷第一功哉?”

		我看过之后,顿感毛骨悚然。

		这是两句惊天动地的话,却不太容易看懂,要看懂这句话,必须解开几个密码。

		第一句话中,木偶和婴儿不用翻译,关键在于新建、兰溪、四明、山阴、以及娄江五个词语。

		这五个词,是五个地名,而在这里,则是暗指五个人。

		新建,是指张位\footnote{新建人。}、兰溪,是指赵志皋\footnote{兰溪人。}、四明,是指沈一贯\footnote{四明人。},山阴,是指朱赓\footnote{山阴人。}。

		所以前半句的意思是,赵志皋和沈一贯不过是木偶,张位和朱赓不过是婴儿!

		而后半句中的娄江,是指王锡爵\footnote{娄江人。}。

		连接起来,我们就得到了这句话的真实含义:

		赵志皋、沈一贯、张位、朱赓都不要紧,最为紧要的,是阻止王锡爵东山再起!

		顾宪成,时任南直隶无锡县普通平民,而赵、张、沈、朱四人中,除张位外,其余三人都当过首辅,首辅者,宰相也,一人之下,万人之上!

		然而这个无锡的平民,却在自己的文章中,把这些不可一世的人物,称为木偶、婴儿。

		而从文字语气中可以看出,他绝非单纯发泄,而是确有把握,似乎在他看来,除了王锡爵外,此类大人物都不值一提。

		一个普通老百姓能牛到这个份上,真可谓是前无古人后无来者。

		第二句话的玄机在于两个关键词语:福清和漕抚。

		福清所指的,就是叶向高,而漕抚,则是李三才。

		叶向高是福建福清人,李三才曾任漕运总督,把这两个词弄清楚后,我们就明白了这句话的意思:

		“大家都知道叶向高能安心当首辅,是因为王锡爵不出山……密揭这事是李三才捅出来的,可谓是为社稷立下第一功!”

		没有王法了。

		一个平民,没有任何职务,远离京城上千里,但他说,内阁大臣都是木偶婴儿。而现在的朝廷第一号人物能够坐稳位置,全都靠他的死党出力。

		纵观二十四史,这种事情我没有听过,没有看过。

		但现在我知道了,在看似杂乱无章的万历年间,在无休止的争斗和吵闹里,一股暗流正在涌动、在黑暗中集结,慢慢地伸出手,操纵所有的一切。
		\ifnum\theparacolNo=2
	\end{multicols}
\fi
\newpage
\section{谋杀}
\ifnum\theparacolNo=2
	\begin{multicols}{\theparacolNo}
		\fi
		\subsection{疯子}
		王锡爵彻底消停了,万历三十六年,叶向高正式登上宝座,成为朝廷首辅,此后七年之中,他是内阁第一人,也是唯一的人,史称“独相”。

		时局似乎毫无变化,万历还是不上朝,内阁还是累得半死,大臣还是骂个不停,但事实真相并非如此。

		在表象之下,政治势力出现了微妙的变化,新的已经来了,旧的赖着不走,为了各自利益,双方一直在苦苦地寻觅,寻觅一个致对方于死地的机会。

		终于,他们找到了那个最好、最合适的机会——太子。

		太子最近过得还不错,自打妖书案后,他很是清净了几年,确切地说,是九年。

		万历四十一年(1613年),一个人写的一封报告,再次把太子拖下了水。

		这个人叫王曰乾,时任锦衣卫百户,通俗点说,是个特务。

		这位特务向皇帝上书,说他发现了一件非常离奇的事情:有三个人集会,剪了三个纸人,上面分别写着皇帝、皇太后、皇太子的名字,然后在上面钉了七七四十九个铁钉\footnote{真是不容易。}。钉了几天后,放火烧掉。

		这是个复杂的过程,但用意很简单——诅咒,毕竟把钉子钉在纸人上,你要说是祈福,似乎也不太靠谱。

		这也就罢了,更麻烦的是,这位特务还同时报告,说这事是一个太监指使的,偏偏这个太监,又是郑贵妃的太监。

		于是事情闹大了,奏疏送到皇帝那里,万历把桌子都给掀了,深更半夜睡不着觉,四下乱转,急得不行。太子知道后,也是心急火燎,唯恐事情闹大,郑贵妃更是哭天喊地,说这事不是自己干的。

		大家都急得团团转,内阁的叶向高却悄无声息,万历气完了,也想起这个人了,当即大骂:

		“出了这么大的事,这人怎么不说话!?”\footnote{此变大事,宰相何无言。}

		此时,身边的太监递给他一件东西,很快万历就说了第二句话:

		“这下没事了。”

		这件东西,就是叶向高的奏疏,事情刚出,就送上来了。

		奏疏的内容大致是这样的:

		陛下,此事的原告\footnote{指王曰乾。}和被告\footnote{指诅咒者。}我都知道,全都是无赖混混,之前也曾闹过事,还被司法部门\footnote{刑部。}处理过,这件事情和以往的妖书案很相似,但妖书案是匿名,无人可查,现在原告被告都在,一审就知道,皇上你不要声张就行了。

		看完这段话,我的感觉是:这是个绝顶聪明的人。

		叶向高的表面意思,是说这件事情,是非曲折且不论,但不宜闹大,只要你不说,我不说,把这件事情压下去,一审就行。

		这是一个不符合常理的抉择。因为叶向高,是东林党的人,而东林党,是支持太子的,现在太子被人诅咒,应该一查到底,怎能就此打住呢?

		事实上,叶向高是对的。

		第二天,叶向高将王曰乾送交三法司审讯。

		这是个让很多人疑惑的决定,这人一审,事情不就闹大了吗?

		如果你这样想,说明你很单纯,因为就在他吩咐审讯的后一天,王曰乾同志就因不明原因,不明不白地死在了监牢里,死因待查。

		什么叫黑?这就叫黑。

		而只要分析当时的局势,揭开几个疑点,你就会发现叶向高的真实动机:

		首先,最大的疑问是:这件事情是不是郑贵妃干的,答案:无所谓。

		自古以来,诅咒这类事数不胜数,说穿了就是想除掉一个人,又没胆跳出来,在家做几个假人,骂骂出出气,是纯粹的阿Q精神。一般也就是老大妈干干\footnote{这事到今天还有人干,有多种形式,如“打小人”。},而以郑贵妃的智商,正好符合这个档次,说她真干,我倒也信。

		但问题在于,她干没干并不重要,反正铁钉扎在假人上,也扎不死人,真正重要的是,这件事不能查,也不能有真相。

		追查此事,似乎是一个太子向郑贵妃复仇的机会,但事实上,却是不折不扣的陷阱。

		原因很简单,此时朱常洛已经是太子,只要没有什么大事,到时自然接班,而郑贵妃一哭二闹三上吊之类的招数,闹了十几年,早没用了。

		但如若将此事搞大,再惊动皇帝,无论结果如何,对太子只好坏处,没有好处。因为此时太子要做的,只有一件事情——等待。

		事实证明,叶向高的判断十分正确,种种迹象表明,告状的王曰乾和诅咒的那帮人关系紧密,此事很可能是一个精心策划的阴谋,某些人\footnote{不一定是郑贵妃。},为了某些目的,想把水搅浑,再混水摸鱼。

		久经考验的叶向高同志识破了圈套,危机成功度过了。

		但太子殿下一生中最残酷的考验即将到来,在两年之后。

		万历四十三年(1615年)五月初四日,黄昏。

		太子朱常洛正在慈庆宫中休息,万历二十九年他被封为太子,住到了这里,但他爹人品差,基础设施一应具缺,要啥都不给,连身边的太监都是人家淘汰的,皇帝不待见,大臣自然也不买账,平时谁都不上门,十分冷清。

		但这一天,一个特别的人已经走到他的门前,并将以一种特别的方式问候他。

		他手持一根木棍,进入了慈庆宫。

		此时,他与太子的距离,只有两道门。

		第一道门无人看守,他迈了过去。

		在第二道门,他遇到了阻碍。

		一般说来,重要国家机关的门口,都有荷枪实弹的士兵站岗,就算差一点的,也有几个保安,实在是打死都没人问的,多少还有个老大爷。

		明代也是如此,锦衣卫、东厂之类的自不必说,兵部吏部门前都有士兵看守,然而太子殿下的门口,没有士兵,也没有保安,甚至连老大爷都没有。

		只有两个老太监。

		于是,他挥舞木棍,打了过去。

		众所周知,太监的体能比平常人要差点\footnote{练过宝典除外。},更何况是老太监。

		很快,一个老太监被打伤,他越过了第二道门,向着目标前进。

		目标,就在前方的不远处。

		然而太监虽不能打,却很能喊,在尖利的呼叫声下,其他太监们终于出现了。

		接下来的事情还算顺理成章,这位仁兄拿的毕竟不是冲锋枪,而他本人不会变形,不会变身,也没能给我们更多惊喜,在一群太监围攻下,终于束手就擒。

		当时太子正在慈庆宫里,接到报告后并不惊慌,毕竟人抓住了,也没进来,他下令将此人送交宫廷守卫处理,在他看来,这不过是个小事。

		但接下来发生的一切,将远远超出他的想象。

		人抓住了,自然要审,按照属地原则,哪里发案由哪里的衙门审,可是这个案子不同,皇宫里的案子,难道你让皇帝审不成?

		推来推去,终于确定,此案由巡城御史刘廷元负责审讯。

		审了半天,刘御史却得出个让人啼笑皆非的结论——这人是个疯子。

		因为无论他好说歹说,利诱威胁,这人的回答却是驴唇不对马嘴,压根就不对路,还时不时蹦出几句谁也听不懂的话,算是个彻头彻尾的疯子。

		于是几轮下来,刘御史也不审了,如果再审下去,他也得变成疯子。

		但要说一点成就没有,那也不对,这位疯子交代,他叫张差,是蓟州人,至于其他情况,就一无所知了。

		这个结果虽然不好,却很合适,因为既然是个疯子,自然就能干疯子的事,他闯进皇宫打人的事情就有解释了,没有背景、没有指使,疯子嘛,也不认路,糊里糊涂到皇宫,糊里糊涂打了人,很好,很好。

		不错,不错,这事要放在其他朝代,皇帝一压,大臣一捧,也就结了。

		可惜,可惜,这是在明朝。

		这事刚出,消息就传开了,街头巷尾人人议论,朝廷大臣们更不用说,每天说来说去就是这事,而大家的看法也很一致:这事,就是郑贵妃干的。

		所谓舆论,就是群众的议论,随着议论的人越来越多,这事也压不下去了,于是万历亲自出马,吩咐三法司会审此案。

		说是三法司,其实只有刑部,审讯的人档次也不算高,尚书侍郎都没来,只是两个郎中\footnote{正厅级。}。

		但这二位的水平,明显比刘御史要高,几番问下来,竟然把事情问清楚了。

		侦办案件,必须找到案件的关键,而这个案子的关键,不是谁干了,而是为什么干,也就是所谓的:动机。

		经过一番询问,张差说出了自己的动机:在此前不久,他家的柴草堆被人给烧了,他气不过,到地方衙门伸冤,地方不管,他就到京城来上访,结果无意中闯入了宫里,心里害怕,就随手打人,如此而已。

		如果用两个字来形容张差的说法,那就是扯淡。

		柴草被人烧了,就要到京城上访,这个说法充分说明了这样一点:张差即使不是个疯子,也是个傻子。

		因为这实在不算个好理由,要换个人,怎么也得编一个房子烧光,恶霸鱼肉百姓的故事,大家才同情你。

		况且到京城告状的人多了去了,有几个能进宫,宫里那么大,怎么偏偏就到了太子的寝宫,您还一个劲地往里闯?

		对于这一点,审案的两位郎中心里自然有数,但领导意图他们更有数,这件事,只能往小了办。

		这两位郎中的名字,分别是胡士相、岳骏声,之所以提出他们的名字,是因为这两个人,绝非等闲之辈。

		于是在一番讨论之后,张差案件正式终结,犯人动机先不提,犯人结局是肯定的——死刑\footnote{也算杀人灭口。}。

		但要杀人,也得有个罪名,这自然难不倒二位仁兄,不愧是刑部的人,很有专业修养,从大明律里,找到这么一条:宫殿射箭、放弹、投砖石伤人者,按律斩。

		为什么伤人不用管,伤什么人也不用管,案件到此为止,就这么结案,大家都清净了。

		如此结案,也算难得糊涂,事情的真相,将就此被彻底埋葬。

		然而这个世界上,终究还是有不糊涂,也不愿意装糊涂的人。

		五月十一日,刑部大牢。

		七天了,张差已经完全习惯了狱中的生活,目前境况,虽然和他预想的不同,但大体正常,装疯很有效,真相依然隐藏在他的心里。

		开饭时间到了,张差走到牢门前,等待着今天的饭菜。

		但他并不知道,有一双眼睛,正在黑暗中注视着他。

		根据规定,虽然犯人已经招供,但刑部每天要派专人提审,以防翻供。

		五月十一日,轮到王之寀。

		王之寀,字心一,时任刑部主事。

		主事,是刑部的低级官员,而这位王先生虽然官小,心眼却不小,他是一个坚定的阴谋论者,认定这个疯子的背后,必定隐藏着某些秘密。

		凑巧的是,他到牢房里的时候,正好遇上开饭,于是他没有出声,找到一个隐蔽的角落,静静地注视着那个疯子。

		因为在吃饭的时候,一个人是很难伪装的。

		之后一切都很正常,张差平静地领过饭,平静地准备吃饭。

		然而王之寀已然确定,这是一个有问题的人。

		因为他的身份是疯子,而一个疯子,是不会如此正常的。

		所以他立即站了出来,打断了正在吃饭的张差,并告诉看守,即刻开始审讯。

		张差非常意外,但随即镇定下来,在他看来,这位不速之客和之前的那些大官,没有区别。

		审讯开始,和以前一样,张差装疯卖傻,但他很快就惊奇地发现,眼前这人一言不发,只是静静地看着他。

		他表演完毕后,现场又陷入了沉寂,然后,他听到了这样一句话:

		“老实说,就给你饭吃,不说就饿死你。”\footnote{实招与饭,不招当饿死。}

		在我国百花齐放的刑讯逼供艺术中,这是一句相当搞笑的话,但凡审讯,一般先是民族大义、坦白从宽,之后才是什么老虎凳、辣椒水。即使要利诱,也是升官发财,金钱美女之类。

		而王主事的诱饵,只是一碗饭。

		无论如何,是太小气了。

		事实证明,张差确实是个相当不错的人,具体表现为头脑简单,思想朴素,在吃一碗饭和隐瞒真相、保住性命之间,他毫不犹豫地选择了前者。

		于是他低着头,说了这样一句话:

		“我不敢说。”

		不敢说的意思,不是不知道,也不是不说,而是知道了不方便说。

		王之寀是个相当聪明的人,随即支走了所有的人,然后他手持那碗饭,听到了事实的真相:

		“我叫张差,是蓟州人,小名张五儿,父亲已去世。”

		“有一天,有两个熟人找到我,带我见了一个老公公\footnote{即太监。},老公公对我说,你跟我去办件事,事成后给你几亩地,保你衣食无忧。”

		“于是我就跟他走,初四(五月四日)到了京城,到了一所宅子里,遇见另一个老公公。”

		“他对我说,你只管往里走,见到一个就打死一个,打死了,我们能救你。”

		“然后他给我一根木棍,带我进了宫,我就往里走,打倒了一个公公,然后被抓住了。”

		王之寀惊呆了。

		他没有想到,外界的猜想竟然是真的,这的的确确,是一次策划已久的政治暗杀。

		但他更没有想到的是,这起暗杀事件竟然办得如此愚蠢,眼前这位仁兄,虽说不是疯子,但说是傻子倒也没错,而且既不是武林高手,也不是职业杀手,最多最多,也就是个彪悍的农民。

		作案过程也极其可笑,听起来,似乎是群众推荐,太监使用,顺手就带到京城,既没给美女,也没给钱,连星级宾馆都没住,一点实惠没看到,就答应去打人,这种傻冒你上哪去找?

		再说凶器,一般说来,刺杀大人物,应该要用高级玩意,当年荆轲刺秦,还找来把徐夫人的匕首,据说是一碰就死,退一万步讲,就算是杀个老百姓,多少也得找把短刀,可这位兄弟进宫时,别说那些高级玩意,菜刀都没一把,拿根木棍就打,算是怎么回事。

		从头到尾,这事怎么看都不对劲,但毕竟情况问出来了,王之寀不敢怠慢,立即上报万历。

		可是奏疏送上去后,却没有丝毫回音,皇帝陛下一点反应都没有。

		但这早在王之寀的预料之中,他老人家早就抄好了副本,四处散发,本人也四处鼓捣,造舆论要求公开的审判。

		他这一闹,另一个司法界大腕,大理寺丞王士昌跳出来了,也跟着一起嚷嚷,要三法司会审。

		可万历依然毫无反应,这是可以理解的,要知道,人家当年可是经历过争国本的,上百号人一拥而上,那才是大世面,这种小场面算个啥。

		照此形势,这事很快就能平息下去,但皇帝陛下没有想到,他不出声,另一个人却跳了出来。

		这个人,就是郑贵妃的弟弟郑国泰。

		事情的起因,只是一封奏疏。

		就在审讯笔录公开后的几天,司正陆大受上了一封奏疏,提出了几个疑问:

		既然张差说有太监找他,那么这个太监是谁?他曾到京城,进过一栋房子,房子在哪里?有个太监和他说过话,这个太监又是谁?

		这倒也罢了,在文章的最后,他还扯了句无关痛痒的话,大意是,以前福王册封的时候,我曾上疏,希望提防奸邪之人,今天果然应验了!

		这话虽说有点指桑骂槐,但其实也没说什么,可是郑国泰先生偏偏就蹦了出来,写了封奏疏,为自己辩解。

		这就是所谓对号入座,它形象地说明,郑国泰的智商指数,和他的姐姐基本属同一水准。

		这还不算,在这封奏疏中,郑先生又留下了这样几句话:

		有什么推翻太子的阴谋?又主使过什么事?收买亡命之徒是为了什么?……这些事我想都不敢想,更不敢说,也不忍听。

		该举动生动地告诉我们,原来蠢字是这么写的。

		郑先生的脑筋实在愚昧到了相当可以的程度,这种货真价实的此地无银三百两,言官们自然不会放过,很快,工科给事中何士晋就做出了反应,相当激烈的反应:

		“谁说你推翻太子!谁说你主使!谁说你收买亡命之徒!你既辩解又招供,欲盖弥彰!”

		郑国泰哑口无言,事情闹到这个地步,已经收不住了。

		此时,几乎所有的人都认为,事实真相即将大白于天下,除了王之寀。

		初审成功后,张差案得以重审,王之寀也很是得意了几天,然而不久之后,他才发现,自己忽视了一个很重要的问题:

		张差装疯非常拙劣,为碗饭就开口,为何之前的官员都没看出来呢?

		思前想后,他得出了一个非常可怕的结论:他们是故意的。

		第一个值得怀疑的,就是首先审讯张差的刘廷元,张差是疯子的说法,即源自于此,经过摸底分析,王之寀发现,这位御史先生,是个不简单的角色。

		此人虽然只是个巡城御史,却似乎与郑国泰有着紧密的联系,而此后复审的两位刑部郎中胡士相、岳骏声,跟他交往也很密切。

		这似乎不奇怪,虽然郑国泰比较蠢,实力还是有的,毕竟福王受宠,主动投靠的人也不少。

		但很快他就发觉,事情远没有他想象的那么简单。

		因为几天后,刑部决定重审案件,而主审官,正是那位曾认定刘廷元结论的郎中,胡士相。

		胡士相,时任刑部山东司郎中,就级别而言,他是王之寀的领导,而在审案过程中,王主事惊奇地发现,胡郎中一直闪烁其辞,咬定张差是真疯,迟迟不追究事件真相。

		一切的一切,给了王之寀一个深刻的印象:在这所谓疯子的背后,隐藏着一股庞大的势力。

		而刘廷元、胡士相,只不过是这股势力的冰山一角。

		但让他疑惑不解的是,指使这些人的,似乎并不是郑国泰,虽然他们拼命掩盖真相,但郑先生在朝廷里人缘不好,加上本人又比较蠢,要说他是后台老板,实在是抬举了。

		那么这一切,到底是怎么回事呢?

		王之寀的感觉是正确的,站在刘廷元、胡士相背后的那个影子,并不是郑国泰。

		这个影子的名字,叫做沈一贯。

		就沈一贯的政绩而言,在史书中也就是个普通角色,但事实上,这位仁兄的历史地位十分重要,是明朝晚期研究的重点人物。

		因为这位兄弟的最大成就,并不是搞政治,而是搞组织。

		我们有理由相信,在工作期间,除了日常政务外,他一直在干一件事——拉人。

		怎么拉,拉了多少,这些都无从查证,但有一点我们是确定的,那就是这个组织的招人原则——浙江人。

		沈一贯,是浙江四明人,在任人唯亲这点上,他和后来的同乡蒋介石异曲同工,于是在亲信的基础上,他建立了一个老乡会。

		这个老乡会,在后来的中国历史上,被称为浙党。

		这就是沈一贯的另一面,他是朝廷的首辅,也是浙党的领袖。

		应该说,这是一个明智的决定,因为你必须清楚地认识到这样一点:

		在万历年间,一个没有后台\footnote{皇帝。},没有亲信\footnote{死党。}的首辅,是绝对坐不稳的。

		所以沈一贯干了五年,叶向高干了七年,所以赵志皋被人践踏,朱赓无人理会。

		当然,搞老乡会的绝不仅仅是沈一贯,除浙党外,还有山东人为主的齐党,湖广人\footnote{今湖北湖南。}为主的楚党。

		此即历史上著名的齐、楚、浙三党。

		这是三个能量极大、战斗力极强的组织,因为组织的骨干成员,就是言官。

		言官,包括六部给事中,以及都察院的御史,给事中可以干涉部领导的决策,和部长\footnote{尚书。}平起平坐,对中央事务有很大的影响。

		而御史相当于特派员,不但可以上书弹劾,还经常下到各地视察,高级御史还能担任巡抚。

		故此,三党的成员虽说都是些六七品的小官,拉出来都不起眼,却是相当的厉害。

		必须说明的是,此前明代二百多年的历史中,虽然拉帮结派是家常便饭,但明目张胆地搞组织,并无先例,先例即由此而来。

		这是一个很有趣的谜团。

		早不出来,晚不出来,为何偏偏此时出现?

		而更有趣的是,三党之间并不敌对,也不斗争,反而和平互助,这实在是件不符合传统的事情。

		存在即是合理,一件事情之所以发生,是因为它有发生的理由。

		有一个理由让三党陆续成立,有一个理由让他们相安无事。是的,这个理由的名字,叫做东林党。

		无锡的顾宪成,只是一个平民,他所经营的,只是一个书院,但几乎所有人都知道,这个书院可以藐视当朝的首辅,说他们是木偶、婴儿,这个书院可以阻挡大臣复起,改变皇帝任命。

		大明天下,国家决策,都操纵在这个老百姓的手中。从古至今,如此牛的老百姓,我没有见过。

		无论是在野的顾宪成、高攀龙、赵南星,还是在朝的李三才,叶向高,都不是省油的灯,东林党既有社会舆论,又有朝廷重臣,要说它是纯道德组织,鬼才信,反正我不信。

		连我都不信了,明朝朝廷那帮老奸巨滑的家伙怎么会信,于是,在这样一个足以影响朝廷,左右天下的对手面前,他们害怕了。

		要克服畏惧,最有效、最快捷的方法,就是找一个人来和你一起畏惧。

		史云:明朝亡于党争。我云:党争,起于此时。

		刘廷元、胡士相不是郑国泰的人,郑先生这种白痴是没有组织能力的,他们真正的身份,是浙党成员。

		但疑问在于,沈一贯也拥立过太子,为何要在此事上支持郑国泰呢?

		答案是,对人不对事。

		沈一贯并不喜欢郑国泰,更不喜欢东林党,因为公愤。

		所谓公愤,是他在当政时,顾宪成之类的人总在公事上跟他过不去,他很愤怒,故称公愤。

		不过,他最不喜欢的那个人,却还不是东林党——叶向高,因为私仇,三十二年的私仇。

		三十二年前(万历十一年,1583年)叶向高来到京城,参加会试。

		叶向高,字进卿,福建福清人,嘉靖三十八年生人。

		必须承认,他的运气很不好,刚刚出世,就经历了生死考验。

		因为在嘉靖三十八年,倭寇入侵福建,福清沦陷,确切地说,沦陷的那一天,正是叶向高的生日。

		据说他的母亲为了躲避倭寇,躲在了麦草堆里,倭寇躲完了,孩子也生出来了,想起来实在不容易。

		大难不死的叶向高,倒也没啥后福,为了躲避倭寇,一两岁就成了游击队,鬼子一进村,他就跟着母亲躲进山里,我相信,几十年后,他的左右逢源,机智狡猾,就是在这打的底。

		倭寇最猖獗的时候,很多人都丢弃了自己的孩子\footnote{累赘。},独自逃命,也有人劝叶向高的母亲,然而她说:

		“要死,就一起死。”

		但他们终究活了下来,因为另一个伟大的明代人物——戚继光。

		\subsection{考试}
		嘉靖四十一年(1562年),戚继光发动横屿战役,攻克横屿,收复福清,并最终平息了倭患。

		必须说明,当时的叶向高,不叫叶向高,只有一个小名,这个小名在今天看来不太文雅,就不介绍了。

		向高这个名字,是他父亲取的,意思是一步一步,向高处走。

		事实告诉我们,名字这个东西,有时候改一改,还是很有效的。

		隆庆六年(1572年),叶向高十四岁,中秀才。

		万历七年(1579年),叶向高二十一岁,中举人。

		万历十一年(1583年),叶向高二十五岁,第二次参加会试。考试结束,他的感觉非常好。

		结果也验证了他的想法,他考中了第七十八名,成为进士。现在,在他的面前,只剩下最后一关——殿试。

		殿试非常顺利,翰林院的考官对叶向高十分满意,决定把他的名次排为第一,远大前程正朝着叶向高招手。

		然而,接下来的一切,却发生了出人意料的变化。

		因为从此刻起,叶向高就与沈一贯结下了深仇大恨,虽然此前,他们从未见过。

		要解释清楚的是,叶向高的第七十八名,并非全国七十八名,而是南卷第七十八名。

		明代的进士,并不是全国统一录取,而是按照地域,分配名额,具体分为三个区域,南、北、中,录取比例各有不同。

		所谓南,就是淮河以南各省,比例为55%。北,就是淮河以北,比例为35%。而中,是指云贵川三省,以及凤阳,比例为10%。

		具体说来是这么个意思,好比朝廷今年要招一百个进士,那么分配到各地,就是南部五十五人,北部三十五人,中部十人。这就意味着,如果你是南部人,在考试中考到了南部第五十六名,哪怕你成绩再好,文章写得比北部第一名还好,你也没法录取。

		而如果你是中部人,哪怕你文章写得再差,在南部只能排到几百名后,但只要能考到中部卷前十名,你就能当进士。

		这是一个历史悠久的规定,从二百多年前,朱元璋登基时,就开始执行了,起因是一件非常血腥的政治案件——南北榜案件。这个案件是笔糊涂账,大体意思是一次考试,南方的举人考得很好,好到北方没几个能录取的,于是有人不服气,说是考官舞弊,事情闹得很大,搞到老朱那里,他老人家是个实在人,也不争论啥,大笔一挥就干掉了上百人。

		可干完后,事情还得解决,因为实际情况是,当年的北方教学质量确实不如南方,你把人杀光了也没辙。无奈之下,只好设定南北榜,谁都别争了,就看你生在哪里,南方算你倒霉,北方算你运气。

		到明宣宗时期,事情又变了,因为云贵川一带算是南方,可在当年是蛮荒之地,别说读书,混碗饭吃都不容易,要和南方江浙那拨人对着考,就算是绝户。于是皇帝下令,把此地列为中部,作为特区,而凤阳,因为是朱元璋的老家,还特别穷,特事特办,也给列了进去。

		当然了,这也是没办法的事,毕竟基础不同,底子不同,在考试上,你想一夜之间人类大同,那是不可能的,所以现在这套理论还在用。我管这个,叫考试地理决定论。

		这套理论很残酷,也很真实,主要是玩机率,看你在哪投胎。

		比如你要是生在山东、江苏、湖北之类的地方,就真是阿弥陀佛了,这些地方经常盘踞着一群读书不要命的家伙,据我所知,有些“乡镇中学”\footnote{地图上都找不到。}的学生,高二就去高考\footnote{不记成绩。},大都能考六百多分\footnote{七百五十分满分。},美其名曰:锻炼素质,明年上阵。

		每念及此,不禁胆战心惊,跟这帮人做邻居的结果是:如果想上北大,六百多分,只是个起步价。

		应该说,现在还是有所进步的,逼急还能玩点阴招,比如说……更改户口。

		不幸的是,明代的叶向高先生没法玩这招,作为南卷的佼佼者,他有很多对手,其中的一个,叫做吴龙徴。

		这位吴先生,也是福建人,但他比其他对手厉害得多,因为他的后台叫沈一贯。

		按沈一贯的想法,这个人应该是第一,然后进入朝廷,成为他的帮手,可是叶向高的出现,却打乱了沈一贯的部署。

		于是,沈一贯准备让叶向高落榜,至少也不能让他名列前茅。

		而且他认定,自己能够做到这一点,因为他就是这次考试的主考官。

		但是很可惜,他没有成功,因为一个更牛的人出面了。

		主考官固然大,可再大,也大不过首辅。

		叶向高虽然没有关系,却有实力。文章写得实在太好,好到其他考官不服气,把这事捅给了申时行,申大人一看,也高兴得不行,把沈一贯叫过去,说这是个人才,必定录取!

		这回沈大人郁闷了,大老板出面了,要不给叶向高饭碗,自己的饭碗也难保,但他终究是不服气的,于是最终结果如下:

		叶向高,录取,名列二甲第十二名。

		这是一个出乎很多人意料的结果,因为若要整人,大可把叶向高同志打发到三甲,就此了事,不给状元,却又给个过得去的名次,实在让人费解。

		告诉你,这里面学问大了。

		叶向高黄了自己的算盘,自然是要教训的。但问题是,这人是申时行保的,申首辅也是个老狐狸,如果要敷衍他,是没有好果子吃的,所以这个面子不但要给,还要给足。而二甲十二名,是最恰当的安排。

		因为根据明代规定,一般说来,二甲十二名的成绩,可以保证入选庶吉士,进入翰林院,但这个名次离状元相当远,也不会太风光,恶心下叶向高,的确是刚刚好。

		但不管怎么说,叶向高还是顺顺当当地踏上了仕途。此后的一切都很顺利,直到十五年后。

		万历二十六年(1598年),就在这一年,叶向高的命运被彻底改变,因为他等到了一个千载难逢的机会。

		此时皇长子朱常洛已经出阁读书,按照规定,应该配备讲官,人选由礼部确定。

		众所周知,虽说朱常洛不受待见,但按目前形势,登基即位是迟早的事,只要拉住这个靠山,自然不愁前程。所以消息一出,大家走关系拉亲戚,只求能混到这份差事。

		叶向高走不走后门我不敢说,运气好是肯定的,因为决定人选的礼部侍郎郭正域,是他的老朋友。

		名单定了,报到了内阁,内阁压住了,因为内阁里有沈一贯。

		沈一贯是个比较一贯的人,十五年前那档子事,他一直记在心里,讲官这事是张位负责,但沈大人看到叶向高的名字,便心急火燎跑去高声大呼:

		“闽人岂可作讲官?!”

		这句话是有来由的,在明代,福建一向被视为不开化地带,沈一贯拿地域问题说事,相当阴险。

		张位却不买账,他也不管你沈一贯和叶向高有什么恩怨,这人我看上了,就要用!

		于是,在沈一贯的磨牙声中,叶向高正式上任。

		叶讲官不负众望,充分发挥主观能动,在教书的同时,和太子建立了良好的私人关系。

		根据种种史料反映,叶先生应该是个相当灵活的人,我们有理由相信,在教书育人的同时,他还广交了不少朋友,比如顾宪成,比如赵南星。

		老板有了,朋友有了,地位也有了,万事俱备,要登上拿最高的舞台,只欠一阵东风。

		一年后,风来了,却是暴风。

		万历二十七年(1601年),首辅赵志皋回家了,虽然没死,也没退,但事情是不管了,张位也走了,内阁,只剩下了沈一贯。

		缺了人就要补,于是叶向高的机会又来了。

		顾宪成是他的朋友,朱常洛是他的朋友,他所欠缺的,只是一个位置。

		他被提名了,最终却未能入阁,因为内阁,只剩下了沈一贯。

		麻烦远未结束,内阁首辅沈一贯大人终于可以报当年的一箭之仇了,不久后,叶向高被调出京城,到南京担任礼部右侍郎。

		南京礼部主要工作,除了养老就是养老,这就是四十岁的叶向高的新岗位,在这里,他还要呆很久。

		很久是多久?十年。

		这十年之中,朝廷里很热闹,册立太子、妖书案,搞得轰轰烈烈。而叶向高这边,却是太平无事。

		整整十年,无人理,无人问,甚至也无人骂、无人整。

		叶向高过得很太平,也过得很惨,惨就惨在连整他的人都没有。

		对于一个政治家而言,最痛苦的惩罚不是免职、不是罢官,而是遗忘。

		叶向高,已经被彻底遗忘了。

		一个前程似锦的政治家,在政治生涯的黄金时刻,被冷漠地抛弃,对叶向高而言,这十年中的每一天,全都是痛苦的挣扎。

		但十余年之后,他将感谢沈一贯给予他的痛苦经历,要想在这个冷酷的地方生存下去,同党是不够的,后台也是不够的,必须亲身经历残酷的考验和磨砺,才能在历史上写下自己的名字。

		因为他并不是一个普通的首辅,在不久的未来,他将超越赵志皋、张位、甚至申时行、王锡爵。他的名字将比这些人更为响亮夺目。

		因为一个极为可怕的人,正在前方等待着他。而他,将是唯一能与之抗衡的人。这个人,叫做魏忠贤。

		万历三十五年(1607年),沈一贯终于走了,年底,叶向高终于来了。

		但沈一贯的一切,都留了下来,包括他的组织,他的势力,以及他的仇恨。

		所以刘廷元、胡士相也好,疯子张差也罢,甚至这件事情是否真的发生过,根本就不要紧。

		梃击,不过是一个傻子的愚蠢举动,并不重要,重要的是,通过这件事情,能够打倒什么,得到什么。

		东林党的方针很明确,拥立朱常洛,并借梃击案打击对手,掌控政权。

		所以浙党的方针是,平息梃击案,了结此事。

		而王之寀,是一个找麻烦的人。

		这才是梃击案件的真相。

		对了,还忘了一件事:虽然没有迹象显示王之寀和东林党有直接联系,但此后东林党敌人列出的两大名单\footnote{点将录、朋党录。}中,他都名列前茅。

		\subsection{再审}
		王之寀并不简单,事实上,是很不简单。

		当他发现自己的上司胡士相有问题时,并没有丝毫畏惧,因为他去找了另一个人——张问达。

		张问达,字德允,时任刑部右侍郎,署部事。

		所谓刑部右侍郎、署部事,换成今天的话说,就是刑部常务副部长。也就是说,他是胡士相的上司。

		张问达的派系并不清晰,但清晰的是,对于胡士相和稀泥的做法,他非常不满。接到王之寀的报告后,他当即下令,由刑部七位官员会审张差。

		这是个有趣的组合,七人之中,既有胡士相,也有王之寀,可以听取双方意见,又不怕人捣鬼,而且七个人审讯,可以少数服从多数。

		想法没错,做法错了。因为张问达远远低估了浙党的实力。

		在七个主审官中,胡士相并不孤单,大体说来,七人之中,支持胡士相,有三个人,支持王之寀的,有两个。

		于是,审讯出现了戏剧化的场景。

		张差恢复了理智,经历了王之寀的突审和反复,现在的张差,已经不再是个疯子,他看上去,十分平静。

		主审官陆梦龙发问:

		“你为什么认识路?”

		这是个关键的问题,一个平民怎样来到京城,又怎样入宫,秘密就隐藏在答案背后。

		顺便说明一下:陆梦龙,是王之寀派。

		出乎所有人的意料,没有等待,没有反复,他们很快就听到了这个关键的答案:

		“我是蓟州人,如果没有人指引,怎么进得去?”

		此言一出,事情已然无可隐瞒。

		再问:

		“谁指引你的?”

		答:

		“庞老公,刘老公。”

		完了,完了。

		虽然张差没有说出这两个人的名字,但大家的人心中,都已经有了确切的答案。

		庞老公,叫做庞保,刘老公,叫做刘成。

		大家之所以知道答案,是因为这两个人的身份很特殊——他们是郑贵妃的贴身太监。

		陆梦龙呆住了,他知道答案,也曾经想过无数次,却没有想到,会如此轻易地得到。

		就在他惊愕的那一瞬间,张差又说出了更让人吃惊的话:

		“我认识他们三年了,他们还给过我一个金壶,一个银壶。”\footnote{予我金银壶各一。}

		陆梦龙这才明白,之前王之寀得到的口供也是假的,真相刚刚开始!

		他立即厉声追问道:

		“为什么\footnote{要给你。}?!”

		回答干净利落,三个字:

		“打小爷!”

		声音不大,如五雷轰顶。

		因为所有人都知道,所谓小爷,就是太子爷朱常洛。

		现场顿时大乱,公堂吵作一团,交头接耳,而此时,一件更诡异的事情发生了。

		作为案件的主审官,胡士相突然拍案而起,大喝一声:

		“不能再问了!”

		这一下大家又懵了,张差招供,您激动啥?

		但他的三位同党当即反应过来,立刻站起身,表示审讯不可继续,应立即结束。

		七人之中,四对三,审讯只能终止。

		但形势已不可逆转,王之寀、陆梦龙立即将案件情况报告给张问达,张侍郎十分震惊。

		与此同时,张差的口供开始在朝廷内外流传,舆论大哗,很多人纷纷上书,要求严查此案。

		郑贵妃慌了,天天跑到万历那里去哭,但此时,局势已无法挽回。

		然而,此刻压力最大的人并不是她,而是张问达,作为案件的主办人,他很清楚,此案背后,是两股政治力量的死磕,还搭上太子、贵妃、皇帝,没一个省油的灯。

		案子如果审下去,审出郑贵妃来,就得罪了皇帝,可要不审,群众那里没法交代,还会得罪东林、太子,小小的刑部右侍郎,这拨人里随便出来一个,就能把自己整死。

		总而言之,不能审,又不能不审。

		无奈之下,他抓耳挠腮,终于想出了一个绝妙的解决方案。

		在明代的司法审讯中,档次最高的就是三法司会审,但最隆重的,叫做十三司会审。

		明代的六部,长官为尚书、侍郎,部下设司,长官为郎中、员外郎,一般说来是四个司,比如吏部、兵部、工部、礼部都是四个司,分管四大业务,而刑部,却有十三个司。

		这十三个司,分别是由明朝的十三个省命名,比如胡士相,就是山东司的郎中,审个案子,竟然把十三个司的郎中全都找来,真是煞费苦心。

		此即所谓集体负责制,也就是集体不负责,张问达先生水平的确高,看准了法不责众,不愿意独自背黑锅,毅然决定把大家拉下水。

		大家倒没意见,反正十三个人,人多好办事,打板子也轻点。

		可到审讯那天,人们才真切地感受到,中国人是喜欢热闹的。

		除了问话的十三位郎中外,王之寀还带了一批人来旁听,加上看热闹的,足有二十多人,人潮汹涌,搞得跟菜市场一样。

		这次张差真的疯了,估计是看到这么多人,心有点慌,主审官还没问,他就说了,还说得特别彻底,不但交代了庞老公就是庞保,刘老公就是刘成,还爆出了一个惊人的内幕:

		按张差的说法,他绝非一个人在战斗,还有同伙,包括所谓马三舅、李外父,姐夫孔道等人,是货真价实的团伙作案。

		精彩的还没完,在审讯的最后,张差一鼓作气,说出了此案中最大的秘密:红封教。

		红封教,是个邪教,具体组织结构不详,据张差同志讲,组织头领有三十六号人,他作案,就是受此组织指使。

		一般说来,凑齐了三十六个头领,就该去当强盗了,这话似乎太不靠谱,但经事后查证,确有其事,刑部官员们再一查,就不敢查了,因为他们意外发现,红封教的起源地,就是郑贵妃的老家。

		而据某些史料反映,郑贵妃和郑国泰,就是红封教的后台。这一点,我是相信的,因为和同时期的白莲教相比,这个红封教发展多年,却发展到无人知晓,有如此成就,也就是郑贵妃这类脑袋缺根弦的人才干得出来。

		张差确实实在,可这一来,就害苦了浙党的同胞们,审案时丑态百出,比如胡士相先生,负责做笔录,听着听着写不下去了,就把笔一丢了事,还有几位浙党郎中,眼看这事越闹越大,竟然在堂上大呼一声:

		“你自己认了吧,不要涉及无辜!”

		但总的说来,浙党还是比较识相的,眼看是烂摊子,索性不管了,同意如实上报。

		上报的同时,刑部还派出两拨人,一拨去找那几位马三舅、李外父,孔道姐夫,另一拨去皇宫,找庞保、刘成。

		于是郑贵妃又开始哭了,几十年来的保留剧目,屡试不爽,可这一次,万历却对她说:

		“我帮不了你了。”

		这是明摆着的,张差招供了,他的那帮外父、姐夫一落网,再加上你自己的太监,你还怎么跑?

		但老婆出事,不管也是不行的,于是万历告诉郑贵妃,而今普天之下,只有一个人能救她,而这个人不是自己。

		“唯有太子出面,方可了解此事。”

		还有句更让人难受的话:

		“这事我不管,你要亲自去求他。”

		郑贵妃又哭了,但这次万历没有理她。

		于是不可一世的郑贵妃收起了眼泪,来到了宿敌的寝宫。

		事实证明,郑小姐装起孙子来,也是巾帼不让须眉,进去看到太子,一句不说就跪,太子也客气,马上回跪,双方爬起来后,郑贵妃就开始哭,一边哭一边说,我真没想过要害你,那都是误会。

		太子也不含糊,反应很快,一边做垂泪状\footnote{真哭是个技术活。},一边说,我明白,这都是外人挑拨,事情是张差自己干的,我不会误会。

		然后他叫来了自己的贴身太监王安,让他当即拟文,表明自己的态度。随即,双方回顾了彼此间长达几十年的传统友谊,表示今后要加强沟通,共同进步,事情就此圆满结束。

		这是一段广为流传的史料,其主题意境是,郑贵妃很狡诈,朱常洛很老实,性格合理,叙述自然,所以我一直深信不疑,直到我发现了另一段史料,一段截然不同的史料:

		开头是相同的,郑贵妃去向万历哭诉,万历说自己没办法,但接下来,事情出现变化——他去找了王皇后。

		这是一个很聪明的举动,因为皇后没有帮派,还有威望,找她商量是再合适不过了。

		皇后的回答也直截了当:

		“此事我也无法,必须找太子面谈。”

		很快,老实太子来了,但他给出的,却是一个截然不同的答案:

		“此事必有主谋!”

		这句话一出来,明神宗脸色就变了,郑贵妃更是激动异常,伸个指头出来,对天大呼:

		“如果这事是我干的,我就全家死光\footnote{奴家赤族。}!”

		这句话说得实在太绝,于是皇帝也吼了一句:

		“这是我的大事,你全家死光又如何\footnote{稀罕汝家。}?!”

		贵妃发火了,皇帝也发火了,但接下来的一句话,却浇灭了所有人的激情:

		“我看,这件事情就是张差自己干的。”

		说这句话的人,就是太子朱常洛。虽然几秒钟之前,他还曾信誓旦旦地要求追查幕后真凶。

		于是大家都满意了,为彻底平息事端,万历四十三年(1615年)五月二十八日,二十多年不上朝的万历先生终于露面了。他召来了内阁大臣、文武百官,以及自己的太子,皇孙,当众训话,大致意思是:自己和太子关系很好,你们该干嘛就干嘛,少来瞎搅和,此案是张差所为,把他干掉了事,就此定案,谁都别再折腾。

		太子的表现也很好,当众抒发父子深情,给这出闹剧画上了圆满句号。

		一天后,张差被凌迟处死,十几天后,庞保和刘成不明不白地死在了刑部大牢里,就杀人灭口而言,干得也还算相当利落。

		轰动天下的疯子袭击太子事件就此结束,史称明宫三大案之“梃击”。

		梃击是一起复杂的政治案件,争议极大,有很多疑点,包括幕后主使人的真实身份。

		因为郑贵妃要想刺杀太子,就算找不到绝顶高手,到天桥附近找个把卖狗皮膏药的,应该也不是问题,选来选去就找了个张差,啥功夫没有,还养了他三年。这且不论,动手时连把菜刀都没有,拿根木棍闯进宫,就想打死太子,相当无聊。

		所以有些人认为,梃击案是朝廷某些党派所为,希望混水摸鱼,借机闹事,甚至有人推测此事与太子有关。因为这事过于扯淡,郑贵妃不傻,绝不会这么干。

		但我的看法是,这事是郑贵妃干的,因为她的智商,就是傻子水平。

		对于梃击案,许多史书的评价大都千篇一律,郑贵妃狡猾,万历昏庸,太子老实,最后老实的太子在正义的东林官员支持下,战胜了狡猾的郑贵妃。

		这都是蒙人的。

		仔细分析就会发现,郑贵妃是个蠢人,万历老奸巨滑,太子也相当会来事,而东林官员们,似乎也不是那么单纯。

		所以事实的真相应该是,一个蠢人办了件蠢事,被一群想挑事的人利用,结果被老滑头万历镇了下来,仅此而已。

		之所以详细介绍此事,是因为我要告诉你:在接下来的叙述中,你将逐渐发现,许多你曾无比熟悉的人,其实十分陌生,许多你曾坚信的事实,其实十分虚伪,而这,不过是个开头。

		\ifnum\theparacolNo=2
	\end{multicols}
\fi
\newpage
\section{不起眼的敌人}
\ifnum\theparacolNo=2
	\begin{multicols}{\theparacolNo}
		\fi
		以上,就是万历同志执政四十余年的大致成就,具体说来,就是斗争、斗争、再斗争。

		先斗倒张居正,再斗争国本、妖书、梃击,言官、大臣、首辅轮番上阵,一天到晚忙活这些事,几十年不上朝,国家是不怎么管了,山东、山西、河南、江西及大江南北相继告灾,文书送上去,理都不理。而更滑稽的是,最大的受害者不是老百姓,而是官员。

		在万历年间,如果你考上进士,也别高兴,因为考上了,未必有官做。

		一般说来,朝代晚期,总会出现大量贪官污吏,欺压百姓,摊派剥削,但我可以很负责地讲,万历年间这个问题很不严重,因为压根就没官。

		老子曾经说过,最好的国家,是老百姓不知道统治者是谁,从某个角度讲,万历同志做到了。

		按照以往制度,六部给事中的名额,应该是五十余人,而都察院的名额,应该是一百余人。可到了万历三十五年,六部给事中只有四个人,而且其中五个部没有都给事中,连个管事的都没有,都察院的十三道御史,竟然只剩下五个人,干几十个人的活,累得要死。

		更要命的是,都察院是监察机构,经常要到全国各地视察,五个人要巡全国十三个省,一年巡到头,连家都回不去,其中最惨的一位兄弟,足足在外巡了六年,才找到个替死鬼,回了京城。

		基层御史只有五个,高层御史却是一个都没有,左都御史、右都御史经常空缺,都察院考勤都没人管,来不来,干不干,全都靠自觉。

		最惨的,还是中央六部,当时的六部,部长副部长加起来,一共只有四个。礼部没有部长,户部只有一个副部长,工部连副部长都没有,只有几个郎中死顶。

		其实候补进士很多,想当官的人也多,可是万历同志就是不批,你能咋办?

		最搞笑的是,即使万历批了,发了委任状,你也当不了官。

		比如万历三十七年(1609年),朝廷实在顶不住了,死磨硬泡,才让万历先生批了几百名官员的上任凭证。可是几个月过去了,竟然无人上任,再一查才知道,凭证压根就没发。

		因为根据规定,发放凭证的是吏部都给事中,可这个职位压根就没人,鬼来发证?

		官员倒霉不说,还连累了犯人,到万历三十八年(1610年),刑部大牢里已经关了上千名犯人,一直没人管,有些小偷小摸的,审下来也就是个治安处罚,却被关了好几年,原因很简单,刑部长官退了,又没人接,这事自然无人理。

		不过犯人还是应该感到幸运,毕竟管牢房伙食的人还在。

		当官很难,辞官也难,你今天上完班,说明天我不干了,谁都不拦你,但要等你的辞职报告批下来,估计也得等个几年。如果你等不及了,就这么走也行,没人追究你。

		总而言之,万历的这个政府,基本属于无政府,如此看来,他应该属于无政府主义者,思想如此超前,着实不易。

		一般说来,史料写到这段,总是奋笔疾书,痛斥万历昏庸腐朽,政府实效,人民生活在水深火热之中。

		而在我看来,持这种看法的,不是装蒜,就是无知。

		因为事实绝非如此。万历年间,恰恰是明代经济最发达的时期,所谓资本主义萌芽,正是兴盛于此。

		而老百姓的生活,那真是滋润,想干什么就干什么,明初的时候,出去逛要村里开介绍信,未经许可乱转,抓住就是充军。万历年间,别说介绍信,连户口\footnote{黄册。}都不要了,你要有本事,跑到美国都没人管你。

		至于日常活动,那就更不用说了,许多地方衙门里压根就没官,也没人收苛捐杂税,贪污受贿,许多农民涌入城市打工,成为明代的农民工。

		这帮人也很自由,今天给你干几天,明天给他干几天,雇主大都是江浙一带的老板,虽说也有些不厚道的老板拖欠民工工资,但大体而言,还算是守规矩。

		久而久之,城市的人越来越多,这些人就是所谓的市民,明代著名的市民文化由此而起,而最受广大市民欢迎的文化读物,就是《金瓶梅》、三言等等。

		按照现在的说法,这些书籍大都含有封建糟粕,应该限制传播,至少也要写个此处划掉多少字之类的说明,但当时连政府都没人管,哪有人理这个,什么足本善本满天飘,肆无忌惮。

		穿衣服也没谱,朱元璋那时候,衣服的材料、颜色,都要按身份定,身份不到就不能穿,穿了就要打屁股,现在是没人管了,想穿什么穿什么,还逐渐出现了性别混装,也就是男人穿女装,涂脂抹粉,搞女性化\footnote{不是太监。},公然招摇过市,还大受欢迎。

		穿女装还好,而更耸人听闻的是,经常有些人\footnote{不是个把。},什么都没穿,光着身子在市面上走来走去,即所谓裸奔。刚奔的时候还有人喊,奔久了也就见怪不怪了。

		至于思想,那更是没法说,由于王守仁的心学大量传播,特别是最为激进的泰州学派,狂得没边,什么孔子孟子,三纲五常,那都是“放屁”、“假道学”,总而言之,打倒一切权威,藐视一切准则。

		封建礼教也彻底废了,性解放潮流席卷全国,按照“二拍”的说法,女人离异再嫁,是再寻常不过的事情,青楼妓院如雨后春笋,艳情小说极其流行,涌现了许多优秀作者和忠实读者群。今天流传下来的所谓明代艳情文学,大都是那时的产物。

		说到这个份上,我也无话可说了。

		自然经济,这是纯粹的自然经济。

		万历年间的真相大抵如此,一个政治纷乱,经济繁荣、文化灿烂、生机勃勃的世界。

		然而这个世界,终究被毁灭了。

		毁灭的起因,是一个人。这人的名字,叫李成梁。

		\subsection{不世之功臣}
		李成梁,是一个猛人,还不是一般的猛。

		他出生于嘉靖五年(1526年),世袭铁岭卫指挥佥事,算是高级军官,可到他这辈,混得相当差劲,家里能卖的都卖了,非常穷,穷得连进京继承官职的路费都没有。

		他本人也混得很差,直到四十岁,还是个穷秀才。后来找人借钱,好歹凑了个数\footnote{继承官职,是要行贿的。},这才捞到官位,还真不是一般的惨。

		但此后,他便一发不可收拾。

		当时的辽东很乱,虽然俺答部落改行做了生意,不抢了,但其他部落看俺答发了财,自己又没份,更不消停,一窝蜂地来抢,什么插汉部、泰宁部、朵颜部、王杲部,乱得一塌糊涂,乱到十年之内,竟然有三位明朝大将战死。

		然后李成梁来了,然后一切都解决了。

		打仗,实际上和打麻将差不多,排兵布阵,这叫洗牌,掷色子,就是开打,战况多变,就是不知道下一张摸什么牌,而要想赢牌,一靠技术,二靠运气。

		靠死运气,怎么打怎么赢,所谓福将。

		靠死里打,怎么打怎么赢,所谓悍将。

		李成梁,应该是福将加悍将。

		隆庆四年(1570年),李成梁到辽东接任总兵,却没人办交接手续,因为前任总兵王首道,是被蒙古人干掉的。

		当时辽东的形势很乱,闹事的部落很多,要全列出来,估计得上百字,大致说来,闹得最凶的有如下几个:

		蒙古方面:插汉部,首领土蛮。泰宁部,首领速巴亥。朵颜部,首领董狐狸。

		女真方面:建州女真,王杲部。海西女真,叶赫部、哈达部,首领清佳努、孟格部禄。

		这些名字很难记,也全都不用记,因为他们很快就会被李成梁干掉。

		以上这些人中,最不消停的,是土蛮。他的部落最大,人最多,有十几万人,比较团结,具体表现为抢劫时大家一起来,每次抢的时候,都是漫天烟尘,铺天盖地,明军看到就跑,压根无法抵挡。

		所以李成梁来后,第一个要打的,就是这只出头鸟。

		自从李大人出马后,土蛮就从没舒坦过。从万历元年起,李成梁大战五次,小战二十余次,基本算是年年打,月月打。

		总打仗不奇怪,奇怪的是,李成梁每次都打赢。

		其实他的兵力很少,也就一两万人,之所以每战必胜,大致有两个原因:首先是技术问题,他属下的辽东铁骑,每人配发三眼火铳,对方用刀,他用火枪,明明白白就欺负你。

		其次是战术问题,李成梁不但骁勇善战,还喜欢玩阴招,对手来袭时,准备大堆财物,摆在外面,等蒙古人下马抢东西,他就发动攻击。此外,他还不守合同,经常偷袭对手,靠这两大优势,十年之内,他累计斩杀敌军骑兵近五万人,把土蛮折腾得奄奄一息。

		看到这段史料,再回忆起他儿子李如松同志的信用问题,不禁感叹:家庭教育,是很重要的。

		土蛮歇了,泰宁也很惨,被打得到处跑不说,万历十年(1582年),连首领速巴孩都中了埋伏,被砍了脑袋。

		蒙古休息了,女真精神了。

		女真,世代居住于明朝辽东一带,到万历年间,主要分为四个部落:海西女真、建州女真、黑龙江女真、东海女真。

		黑龙江和东海的这两拨人,一直比较穷,吃饭都成问题,连抢劫的工具都没有,基本上可以忽略。

		而最让人头疼的,是建州女真。

		当时的建州女真,头领叫做王杲,这人用今天的话说,是个给脸不要脸的人。

		他原本在这里当地主,后来势力大了,明朝封他当建州卫指挥使,官位不低,这人不满意,自封当了都督。

		王杲的地盘靠近抚顺,明朝允许他和抚顺做生意,收入很高,这人不满意,诱杀了抚顺的守将,非要去抢一把。

		因为他经常不满意,所以李成梁对他也不满意,万历元年(1573年),找个机会打了一仗。

		开始明军人少,王杲占了便宜,于是他又不满意了,拼命地追,追到后来,进了李成梁的口袋,又拼命跑,从建州跑到海西,李将军也是个执着的人,从建州追到海西,王杲束手无策,只能投降。

		投降后,属下大部被杀,他本人被送到京城,剐了。

		但在乱军之中,有一个人跑了,这个人叫阿台,是王杲的儿子。十年后,祸患即由此而起。

		建州女真完了,下一个要解决的,是海西女真。

		海西女真中,第一个被解决的,是叶赫部。

		应该承认一点:李成梁除掉叶赫部的方法,是相当无耻的。

		万历十一年(1583年),叶赫部首领,贝勒清佳努率两千余人来到开原,准备进行马市贸易。在这里,他们将用牲畜换取自己所需的各种物资。

		高兴而来,满载而归,过去无数次,他们都是这样做的。

		然而这次不同。

		当他们准备进入开原城时,守城明军拦住了他们,说:

		“你们人太多了,不能全部入城。”

		清佳努想了一下,回答:

		“好的,我只带三百人进城。”

		但当他入城后,才惊奇地发现,这里没有商人,没有小贩,没有拥挤的人流,只是一片空地。

		然后,他听到了炮声。

		炮声响起的同时,城外的李成梁下达了攻击令,数千名明军蜂拥而起,短短几分钟之内,清佳努和三百随从全部被杀,城外的明军也很有效率,叶赫部只跑掉了四百四十人。

		然后是哈达部。

		相对而言,哈达部人数少,也不怎么惹事,李成梁本来也没打算收拾他们。但不幸的是,哈达部有个孟格部禄,孟格部禄又有个想法:和叶赫部联合。

		这就有点问题了,因为李成梁先生的目标,并不是蒙古,甚至也不是女真,他选择敌人的唯一标准,就是强大。

		强大,强大到足以威胁帝国的程度,就必须消灭。

		本着这一指导原则,李成梁偷袭了哈达部,将部落主力歼灭,解决了这个问题。

		自隆庆四年至万历十九年,在二十二年的时间里,李成梁把辽东变成了静土,并不干净,却很安静。

		如果各部落团结,他就挑事,挑出矛盾后,就开始分类。听话的,就给胡萝卜吃;不听话的,就用大棒。多年来,他作战上百次,大捷十余次,歼敌十多万人,年年立功受奖,年年升官发财,连戚继光都要靠边站,功绩彪炳,无懈可击。

		除了万历十一年的那一场战役。

		万历十一年(1583年),李成梁得到了一个消息:阿台出现了。

		从战火中逃离的阿台,带着对明朝的刻骨仇恨,开始了他的二次创业。经过十年不懈的杀人抢劫,他成功地由小土匪变成了大强盗,并建立了自己的营寨,继续与明朝对抗。

		对付这种人,李成梁的办法有,且只有一个。

		万历十一年(1583年)二月,他自抚顺出兵,攻击阿台的营寨。

		攻击没有想象中顺利,阿台非常顽强,李成梁竭尽全力,放火强攻全用上,竟然未能攻克,无奈之下,他找来了两个帮手。

		这两个帮手,实际上是帮他带路的向导,一个叫尼堪外兰,另一个,叫觉昌安。

		这两位都是当地部落首领,所以李成梁希望他们出面,去找阿台谈判,签个合同把事情结了。

		当然了,遵不遵守合同,那就另说了,先把人弄出来。

		两个人就这么去了,但是,李成梁疏漏了一个重要的细节——动机。

		同为建州女真,这两个人有着不同的动机,和不同的身份。

		尼堪外兰是附近的城主,之所以帮助李成梁,是因为除掉阿台,他就能够获得利益。

		而觉昌安跑过来,只是为了自己的孙女——阿台是他的孙女婿。

		当两人来到城寨下时,不同的动机,终将导致不同的行为。

		觉昌安对尼堪外兰说,我进去劝降,你在外面等着,先不要动手。

		尼堪外兰同意。

		觉昌安进入城内,见到了阿台,开始游说。

		很可惜,他的口才实在不怎么样,说得口干舌燥,阿台压根就没反应。

		时间不断逝去,等在城外的尼堪外兰开始不耐烦了。

		但他很明白,觉昌安还在里面,无论如何不能动手。

		正在这个关键的时刻,李成梁的使者来了,只传达了一句话:

		“为何还未解决?”

		对李成梁而言,这只是个普通的催促。

		但这句话,在尼堪外兰的脑海中,变成了命令。

		他之所以跑来,不是为了觉昌安,更不是为阿台,只是为了利益和地盘,为了李成梁的支持。

		于是,他打算用自己的方式去解决。

		他走到城寨边,用高亢的声音,开始了自己的谈判:

		“天朝大军已经到了,你们已经没有出路,太师\footnote{指李成梁。}有令,若杀掉阿台者,就是此地之主!”

		这是一个谎言。

		所谓封官许愿,是尼堪外兰的创造,因为李成梁虽不守信用,但一个小小的营寨,打了就打了,还犯不着许愿开支票。

		但事实证明,人穷志短,空头支票,也是很有号召力的。

		应该说,游牧民族是比较实诚的,喊完话后,没有思想斗争,没有激烈讨论,就有人操家伙奔阿台去了。

		谁先砍的第一刀无人知晓,反正砍他的人是争先恐后,络绎不绝,最后被乱刀砍死,连觉昌安也未能幸免。

		虽然城外的李成梁不知道怎么回事,但他知道该干什么,趁乱带兵杀了进去。

		因为他不知道尼堪外兰的那个合同\footnote{估计知道了也没用。},所以也就没有什么顾忌,办事也绝了点——城内共计两千三百人,无一生还。

		和觉昌安一起进城的,还有他的儿子塔克世,同样死在城里。

		不过对于李成梁而言,这实在无关紧要,多死个把人无所谓,在他的战斗生涯中,这只是次微不足道的战斗,打扫战场,捡完人头报功,回家睡觉。

		尼堪外兰倒是高兴,虽然觉昌安是惨了点,毕竟讨好了李成梁,也算大功告成。

		但在他们看不见的地方,有一个人已经点燃了火种,燎原冲天的烈焰,终将由此而起。他是觉昌安的孙子,他是塔克世的儿子,他的名字,叫做努尔哈赤。

		\subsection{万世之罪首}
		努尔哈赤很气愤——他应该气愤,他的祖父、父亲死了,而且死得很冤枉,看起来,李成梁害死了他的两位亲人,实际上,是五个。

		如果你还记得,觉昌安所以入城,是为了阿台的妻子,自己的孙女,当然,也就是努尔哈赤的堂姐,她也死在乱军之中,这是第三个。

		而阿台,自然就是努尔哈赤的堂姐夫,他是第四个,然而,他和努尔哈赤的关系,远比你想象得复杂得多。

		嘉靖三十八年(1559年),努尔哈赤生于赫图阿拉,他的祖父觉昌安和父亲塔克世都是女真世袭贵族,曾任建州左卫指挥使。

		滑稽的是,虽说家里成分很高,努尔哈赤的生活档次却很低,家里五兄弟,他排行老大,却很像小弟,从小就要帮着干活,要啥没啥。

		原因很简单,当时的女真部落,大都穷得掉渣,所谓女真贵族,虽说不掉渣,但也很穷,所以为了生计,小时候的努尔哈赤曾到他的外祖父家暂住。

		他的外祖父,就是我们的老朋友,王杲。

		现在,先洗把脸,整理一下他们之间的关系:

		努尔哈赤的母亲是王杲的女儿,也就是说,阿台是努尔哈赤的舅舅,但是阿台又娶了努尔哈赤的堂姐,所以他又是努尔哈赤的堂姐夫,这还好,要换到努尔哈赤他爹塔克世这辈,就更乱了,因为阿台既是他的侄女婿,又是他的小舅子。

		乱是乱了点,考虑到当时女真族的生存状态,反正都是亲戚,也算将就了。

		你应该能理解努尔哈赤有多悲痛了,在李成梁的屠刀之下,他失去了祖父觉昌安、外祖父王杲、父亲塔克世、堂姐某某\footnote{对不起,没查到。}以及舅舅阿台\footnote{兼堂姐夫。}。

		悲痛的努尔哈赤找到了明朝的官员,愤怒地质问道:

		“我的祖父、父亲何故被害,给我一个说法!”

		明朝的官员倒还比较客气,给了个说法:

		“对不住,我们不是故意的,误会!”

		很明显,这个说法不太有说服力,所以明朝官员还准备了一份礼物,以安抚努尔哈赤受伤的心灵。

		这份礼物是三十份敕书,三十匹马、一份都督的任免状。

		马和任免状大家都知道,我解释一下这敕书是个什么玩意。

		所谓敕书,用今天的话说,就是贸易许可证。

		当时的女真部落,住在深山老林,除了狗熊啥都缺,过日子是过不下去了,要动粗,抢劫的经验又比不上蒙古,明朝不愿开放互市,无奈之下,只好找到了这个折衷的方式,一道敕书,就能做一笔生意。三十分敕书,就是三十笔生意。

		明朝的意思很明白,人死了,给点补偿费,你走人吧。

		客观地讲,这笔补偿费实在有点低,似乎无法平息努尔哈赤的愤怒。

		然而他接受了。

		他接受了所有的一切,回到了自己的家乡。

		然后,他召集了族人,杀死了一头牛,举行了祭天仪式,拿出了祖上流传下来的十三副铠甲,宣布,起兵。

		收了赔偿金再起兵,和收了钱不办事,似乎是异曲同工。但无论如何,努尔哈赤向着自己的未来迈出了第一步。这一年,他二十五岁。

		按照许多史料书籍的说法,下面将是努尔哈赤同志的光荣创业史,先起兵杀死尼堪外兰,然后统一建州女真,打败海西女真最强的叶赫部落,至万历四十六年(1618年),统一女真。

		最后是基本类同的几句评价:非常光辉、非常励志、非常艰苦等等。

		本人同意以上评语,却也要加上四个字:非常诡异。

		据说努尔哈赤从小住在林子里,自己打猎、采集蘑菇,到市集上换东西,生活艰苦,所以意志坚定,渴了喝泉水,饿了啃人参,所以身体强壮,天赋异禀,无师自通,所以极会打仗。

		有以上几大优惠条件,所以十三副铠甲起兵,便不可收拾。

		这绝不可能。

		努尔哈赤起兵时,他的武器是弓箭,不是导弹,他带着十三副铠甲,不是十三件防弹衣,在当时众多的女真部落中,他只不过是个小人物。

		然而这个小人物,只用了三十多年,就统一了女真、建立了政权,占据了原本重兵集结的辽东,并正式向明朝挑战。

		于是,我得出了一个结论:他得到了帮助。

		而帮助他的这个人,就是李成梁。

		我并不是阴谋论者,却惊奇地发现,无数的清代史料书籍中,都详细地描述了祖父觉昌安的惨死、李成梁的冷漠残酷、努尔哈赤的无助,却不约而同地忽略了这样一个细节——努尔哈赤的祖父觉昌安,是李成梁的朋友、好朋友。

		据某些笔记的记载,努尔哈赤和李成梁之前很早就认识了,不但认识,努尔哈赤还给李成梁打过下手,他们之间,还有一段极为神秘的纠葛。

		据说努尔哈赤少年时,曾经因为闹事,被李成梁抓回来管教,不久之后,努尔哈赤被释放了,不是李成梁放的。

		放走努尔哈赤的,是李成梁的老婆\footnote{小妾。},而她放走努尔哈赤的理由也很简单——这人长得好\footnote{奇其貌,阴纵之出。}。至于他俩有无其他纠葛,我不知道,也不想知道。

		相关的说法还有很多,什么努尔哈赤跟李成梁打过仗,一同到过京城,凡此种种,更不可思议的是,据说努尔哈赤和李成梁还是亲家:努尔哈赤的弟弟,叫做舒尔哈齐,这位舒尔哈齐有个女儿,嫁给了李成梁的儿子李如柏,做妾。

		而种种迹象表明,勇敢而悲痛的努尔哈赤,除了会打仗、身体好外,似乎还很会来事儿。他经常给李成梁送礼,东西是一车车地拉,拍起马屁来,可谓“无所不用其极”。\footnote{明史学者孟森语。}

		所以,我们有理由认为,努尔哈赤和李成梁家族,有着某种不可告人的联系。

		当你知道了这一点,再回头审视此前的几条记录,你就会发现,这个流传久远的故事的第二版本,以及隐藏其后的真正秘密。

		万历十一年(1583年)二月,努尔哈赤祖父、父亲被误杀,努尔哈赤接受委任,管理部落。

		万历十一年(1583年)十二月,努尔哈赤部的死敌,海西女真中最强大的叶赫部贝勒清佳努被讨伐,所部两千余人全部被杀,势力大减。

		此后不久,努尔哈赤率兵攻打尼堪外兰,尼堪外兰自认有功,投奔李成梁,李成梁把他交给了努尔哈赤。

		万历十五年(1587年),海西女真哈达部孟格部禄联合叶赫,被李成梁发现,随即攻打,斩杀五百余人。

		万历十六年(1588年),叶赫部再度强大,李成梁再次出击,杀死清佳努的儿子那林脖罗,斩杀六百余人,叶赫部实力大损,只得休养生息。

		万历二十一年(1593年),努尔哈赤终于统一建州女真,成为了女真最强大的部落。

		万历二十一年(1593年)九月,面对越来越强大的努尔哈赤,海西女真叶赫部联合哈达部、蒙古科尔沁部等九大部落,组成联军,攻击努尔哈赤,失败,被杀四千余人,史称“古勒山之战”。

		战后,努尔哈赤将叶赫部首领分尸,一半留存,一半交叶赫部。自此,叶赫部与爱新觉罗部不共戴天。据说其部落首领于战败之时,曾放言如下:

		“我叶赫部若只剩一女子,亦将倾覆之!”

		叶赫部居住于那拉河畔,故又称叶赫那拉。

		李成梁做了件不公道的事情,他扶植了努尔哈赤,培养了明朝的敌人。

		但公道地讲,他并不是故意的,更不是所谓的汉奸。

		因为在他看来,所谓努尔哈赤,不过是一只柔弱的猫,给他吃穿,让他成长,最后成为一只温顺、听话的猫。

		这只猫逐渐长大了,它的身躯变得强壮,叫声变得凄厉,脚掌长出了利爪,最后它亮出了獠牙。至此,我们终于知道,它不是猫,而是老虎,它不是宠物,而是野兽。

		但李成梁的观察能力,那真不是普通的差。

		万历十九年(1591年)李成梁退休,在此之前,他已打垮了蒙古、叶赫、哈达以及所有强大的部落,除了努尔哈赤。

		非但不打,还除掉了他的对手,李成梁实在是个很够意思的人。

		十年后,李成梁再次上任,此时的努尔哈赤已经统一了建州女真,极其壮大,但在李成梁看来,他似乎还是那只温顺的猫,于是,他做出了一个错误的抉择——放弃六堡。

		六堡,是明代在辽东一带的军事基地,是遏制女真的重要堡垒,也是辽东重镇抚顺、清河的唯一屏障,若丢失此处,女真军队将纵横辽东、不可阻挡。

		而此时的六堡,没有大兵压境,没有粮食饥荒,无论如何,都不应该、不需要、不能放弃。

		然而李成梁放弃了。

		万历三十四年(1606年),李成梁正式放弃六堡,并迁走了这里的十余万居民,将此地拱手让给了努尔哈赤。

		这是一个错误的抉择,也是一个无耻的抉择,李成梁将军不但丢失了战略重地,毁灭了十余万人的家园,还以此向朝廷报功,所谓“招抚边民十余万”,实在不知世上有羞耻二字。

		努尔哈赤毫无代价地占领六堡,明朝的繁荣、富饶,以及虚弱全部暴露在他的面前,那一刻,他终于看到了欲望,和欲望实现的可能。

		万历四十三年(1615年),李成梁去世,年九十,不世之功臣,千秋之罪首。

		建功一世,祸患千秋,万死不足恕其罪!

		几个月后,万历四十四年(1616年),努尔哈赤在赫图阿拉建立政权,年号天命,史称后金,努尔哈赤称天命汗。这说明他还是很给李成梁面子的,至少给了几个月的面子。

		海西女真、叶赫部、哈达部,这些名词已不复存在,现在的女真,是唯一的女真,是努尔哈赤的女真,是拥有自己文字\footnote{努尔哈赤找人造出来的。}的女真,是拥有八旗制度,和精锐骑兵部队的女真。

		辽东已经容不下努尔哈赤了,他从来不是一个老实本分的老百姓,也不是遵纪守法的好公民,当现有的财富和土地无法满足他的欲望时,眼前这个富饶的大明帝国,将是他的唯一选择。

		好了,面具不需要了,伪装也不需要了,唯一要做的,是抽出屠刀,肆无忌惮地砍杀他们的士兵,掳掠他们的百姓,抢走他们的所有财富。

		杀死士兵,可以得到装备马匹,掳掠百姓,可以获得奴隶,抢夺财富,可以强大金国。

		当然了,这些话是不能明说的,因为一个强盗,杀人放火是不需要借口的,但对一群强盗而言,理由,是很有必要的。

		万历四十六年(1618年)正月,努尔哈赤在赫图阿拉,发出了战争的宣告:

		“今岁(年),必征大明国!”

		光叫口号是不够的,无论如何,还得找几个开战的理由。

		四月,努尔哈赤找到了理由,七个。

		此即所谓七大恨,在文中,努尔哈赤先生列举了七个明朝对不住他的地方,全文就不列了,但值得表扬的是,在挑事方面,这篇文章,还真是下了点功夫。

		祖父、父亲被杀,自然是要讲下的,李成梁的庇护,自然是不会提的,某些重大事件,也不能放过。比如边界问题:擅自进入我方边界。经济问题:割了我们这边的粮食。外交问题:十名女真人在边界被害\footnote{这个理由好像很眼熟。}。

		其中,最有意思的理由是:明朝偏袒叶赫、哈达部,对自己不公。

		对于这句话,明朝有什么看法不好说,但被李成梁同志打残无数次的叶赫和哈达部,应该是有话要讲的。

		这个七大恨,后来被包括袁崇焕在内的许多人驳斥过,凑热闹的事我就不干了。我只是认为,努尔哈赤先生有点多余,想抢,抢就是了,想杀,杀就是了,何苦费那么大劲呢?

		杀死一切敢于抵抗的人,抢走一切能够抢走的东西,占领一切能够占领的土地,目的十分明确。

		抢掠,其实无须借口。

		万历四十六年(1618年)四月,努尔哈赤将他的马刀指向了第一个目标——抚顺。

		有一位古罗马的将领,在与日耳曼军队征战多年后,发出了这样的感叹:

		他们不懂军事,却很彪悍,不懂权谋,却很狡猾。

		这句简单的话,蕴藏着深厚的哲理。

		很多人说过,最好的老师,不是特级教师,不是名牌学校,而是兴趣。

		但我要告诉你,这个答案是错误的。

		在这个世界上,最优秀的老师,是生存。

		为了一块土地,为了一座房子,为了一块肉,为了在这个世界上多活一天,熟悉杀戮的技巧、掌握抢劫的诀窍,无须催促、无须劝说,在每一天生与死的较量中,懂得生存,懂得如何去生存。

		生存很困难,所以为了生存,必须更加狡诈、必须更加残暴。

		所以在抚顺战役中,我们看到的,并不是纵横驰骋的游牧骑兵,光明正大的英勇冲锋,而是更为阴险狡诈的权谋诡计。

		万历四十六年(1618年)四月十五日,努尔哈赤抵达抚顺近郊。

		但他并没有发动进攻,却派人向城里散布了一个消息。

		这个消息的内容是,明天,女真部落三千人,将携带大量财物来抚顺交易。

		抚顺守将欣然应允,承诺打开城门,迎接商队的到来。

		第二天(十五日)早晨,商队来了,抚顺打开了城门,百姓商贩走出城外,准备交易。

		然后,满脸笑容的女真商队拿出了他们携带的唯一交易品——屠刀。

		贸易随即变成了抢掠,商队变成了军队,很明显,女真人做无本生意的积极性要高得多。

		努尔哈赤的军队再无须隐藏,精锐的八旗骑兵,在“商队”的帮助下,向抚顺城发动了进攻。

		守城明军反应很快,开始组织抵抗,然而没过多久,抵抗就停止了,城内一片平静。

		对于这个不同寻常的变化,努尔哈赤并不惊讶,因为这一切,都在他的计划之中。

		很快,他就见到了计划中的那个关键棋子——李永芳。

		李永芳,是抚顺城的守将之一,简单介绍下——是个叛徒。

		他出卖抚顺城,所换来的,是副将的职称,和努尔哈赤的一个孙女。

		抚顺失陷了,努尔哈赤抢到了所有能够抢到的财物、人口,明朝遭受了重大损失。

		明军自然不肯干休,总兵张承胤率军追击努尔哈赤,却遭遇皇太极的伏兵,阵亡,全军覆没。

		抚顺战役,努尔哈赤掠夺了三十多万人口、牛马,获得了前所未有的财富,但这一切,只是个开始。

		对努尔哈赤而言,继续抢下去,有很多的理由。

		女真部落缺少日常用品,拿东西去换太麻烦,发展手工业不靠谱,抢来得最快。而更重要的是,当时的女真正在闹灾荒,草地荒芜,野兽数量大量减少,这帮大爷又不耕地,粮食不够,搞得部落里怨声载道,矛盾激化。

		所以继续抢,那是一举多得,既能够填补产业空白,又能解决吃饭问题,而且还能转嫁矛盾。

		于是,万历四十六年(1618年)七月,他再次出击,这次,他的目标是清河。

		清河,就是今天的辽宁本溪,此地是通往辽阳、沈阳的必经之地,战略位置十分重要。

		而清河的失陷过程也再次证明,努尔哈赤,实在是个狡猾狡猾的家伙。

		七月初,他率军出征,却不打清河,反而跑到相反方向去闹腾,对外宣称是去打叶赫部,然后调转方向,攻击清河。

		到了清河,也不开打,又是老把戏,先派奸细,打扮成商贩进了城,然后发动进攻,里应外合,清河人少势孤,守军一万余人全军覆没。

		之后的事情比较雷同,城内的十几万人口被努尔哈赤全数打包带走,有钱、有奴隶、有粮食,空白填补了,粮食保证了,矛盾缓和了。

		但他留下的,是一片彻底的白地,是无数被抢走口粮而饿死的平民,是无数家破人亡的惨剧,痛苦、无助。

		无论什么角度、什么立场、什么观点、什么利益、什么目的、什么动机、什么想法、什么情感、什么理念、都应该承认一点,至少一点:

		这是抢掠,是自私、无情、带给无数人痛苦的抢掠。
		\begin{quote}
			\begin{spacing}{0.5}  %行間距倍率
				\textit{{\footnotesize
							\begin{description}
								\item[\textcolor{Gray}{\faQuoteRight}] 征服的荣光背后,是无数的悲泣与哀嚎。——本人语
							\end{description}
						}}
			\end{spacing}
		\end{quote}

		\ifnum\theparacolNo=2
	\end{multicols}
\fi
\newpage
\section{萨尔浒}
\ifnum\theparacolNo=2
	\begin{multicols}{\theparacolNo}
		\fi
		\subsection{会战}
		努尔哈赤是一位伟大的军事家,至少我是这样认为。

		作为一名没有进过私塾,没有上过军校,没有受过系统军事训练的游牧民族首领,努尔哈赤懂得什么是战争,也懂得如何赢得战争。他的战役指挥水平,已经达到了炉火纯青的地步。

		在抚顺、清河以及之后一系列战役中,他表现出了惊人的军事天赋,无论是判断对方动向,选择战机、还是玩阴耍诈,都可谓是无懈可击。

		毫无疑问,他是这个时代最杰出的军事将领——在那两个人尚未出现之前。

		但对明朝而言,这位十分优秀的军事家,只是一名十分恶劣的强盗。不仅恶劣,而且残忍。

		清河、抚顺战役结束后,抢够杀完的努尔哈赤非但没有歉意,不打收条,还做了一件极其无耻的事情。

		他挑选了三百名当地平民,在抚顺关前,杀死了二百九十九人,只留下了一个。

		他割下了这个人的耳朵,并让他带回一封信,以说明自己无端杀戮的理由:

		“如果认为我做的不对,就约定时间作战!如果认为我做得对,你就送金银布帛吧,可以息事宁人!”

		绑匪见得多了,但先撕票再勒索的绑匪,倒还真是第一次见。

		明朝不是南宋,没有送礼的习惯。他们的方针,向来是不向劫匪妥协,何况是撕了肉票的劫匪。既然要打,那咱就打真格的。

		万历四十七年(1619年)三月,经过长时间的准备,明军集结完毕,向赫图阿拉发起进攻。

		明军共分东、西、南、北四路,由四位总兵率领,统帅及进攻路线如下:

		东路指挥刘綎,自朝鲜进攻。

		西路指挥官杜松,自抚顺进攻。

		北路指挥官马林,自开原进攻。

		南路指挥官李如柏,自清河进攻。

		进攻的目标只有一个,赫图阿拉。

		以上四路明军,共计十二万人,系由各地抽调而来,而这四位指挥官,也都大有来头。

		李如柏的身份最高,他是李成梁的儿子,李如松的弟弟,但水平最低,你要说他不会打仗,比较冤枉,你要说他很会打仗,比较扯淡。

		马林的父亲,是马芳,这个人之前没提过,但很厉害,厉害到他的儿子马林,本来是个文人,都当上了总兵。至于马先生的作战水平,相信你已经清楚。

		这两路的基本情况如此,就指挥官来看,实在没什么戏。

		但另外两路,就完全不同了。

		东路指挥官刘綎,也是老熟人了。使六十多斤的大刀,还“轮转如飞”,先打日本,后扫西南,“万历三大征”打了两大征,让他指挥东路,可谓志在必得。

		但四路军中,最大的主力却并不是东路,最猛的将领也并不是刘綎。这两大殊荣,都属于西路军,以及它的指挥官,杜松。

		杜松,陕西榆林人,原任陕西参将,外号杜太师。

		前面提过,太师是朝廷的正一品职称,拿到这个头衔的,很少很少,除了张居正外,其他获得者一般都是死人、追认。

		但杜将军得到的这个头衔,确确实实是别人封的,只不过……不是朝廷。

		他在镇守边界的时候,经常主动出击蒙古,极其生猛,前后共计百余战,无一败绩。蒙古人被他打怕了,求饶又没用,听说明朝官员中太师最大,所以就叫他太师。

		而杜将军不但勇猛过人,长相也过人,因为他常年冲锋肉搏,所以身上脸上到处都是伤疤,面目极其狰狞,据说让人看着就不住地打哆嗦。

		但这位刘綎都甘拜下风的猛人,这次前来上任,居然是带着镣铐来的,因为在不久之前,他刚犯了错误。

		杜松虽然很猛,却有个毛病:小心眼。

		所谓小心眼,一般是生气跟别人过不去,可是让人哭笑不得的是,杜松先生小心眼,总是跟自己过不去。

		比如之前,他曾经跟人吵架,以武将的脾气,大不了一气之下动家伙砍人,可是杜兄一气之下,竟然出家当和尚了。

		这实在是个奇怪的事,让人怎么都想不明白,可还没等别人想明白,杜松就想明白了,于是又还俗,继续干他的杀人事业。

		后来他升了官,到辽东当上了总兵,可是官升了,脾气一点没改,上阵打仗吃了亏\footnote{不算败仗。},换了别人,无非写了检讨,下次再来。

		可这位兄弟不知那根筋不对,竟然要自杀,好歹被人拦住还是不消停,一把火把军需库给烧了,论罪被赶回了家,这一次是重返故里。

		虽说过了这么多年,经历了这么多事,但他的同事们惊奇地发现,这人一点没改,刚到沈阳\footnote{明军总营。}报到,就开始咋呼:

		“我这次来,就是活捉努尔哈赤的,你们谁都别跟我抢!”

		又不是什么好事,谁跟你抢?

		事实也证明,这个光荣任务,没人跟他抢,连刘綎都不敢,于是最精锐的西路军,就成为了他的部属。

		以上四路明军,共计十二万人,大致情况也就是这样,大明人多,林子太大,什么人都有,什么鸟都飞,混人、文人、猛人,一应俱全。

		说漏了,还有个鸟人——辽东经略杨镐。

		杨镐,是一个出过场的人,说实话,我不太想让这人再出来,但可惜的是,我不是导演,没有换演员的权力。

		作为一个无奈的旁观者,看着它的开幕和结束,除了叹息,只有叹息。

		参战明军由全国七省及朝鲜、叶赫部组成,并抽调得力将领指挥。全军共十二万人,号称四十七万,这是自土木堡之变以来,明朝最大规模的军事行动。

		要成事,需要十二万人,但要坏事,一个人就够了。

		从这个角度讲,杨镐应该算是个很有成就的人。

		自从朝鲜战败后,杨镐很是消停了一阵。但这个人虽不会搞军事,却会搞关系,加上他本人还比较老实,二十年后,又当上了兵部左侍郎兼都察院右都御史。此外,他还加入了组织——浙党。

		当时的朝廷首辅,是浙党的铁杆方从哲,浙党的首辅,自然要用浙党的将领,于是这个光荣的任务,就落在了杨镐的身上。

		虽然后来许多东林党拿杨镐说事,攻击方从哲,但公正地讲,在这件事上,方先生也是个冤大头。

		我查了一下,杨镐兄的出生年月日不详,但他是万历八年(1580年)的进士,考虑到他的智商和表现,二十岁之前考中的可能性实在很小,三十而立、四十不惑都是有可能的。

		如此算来,万历四十七年(1619年)的时候,杨大爷至少也有六十多了。在当时的武将中,资历老、打过仗的,估计也就他了。

		方首辅没有选择的余地。

		所以,这场战争的结局,也没有选择的余地。

		万历四十七年(1619年)二月二十一日,杨镐坐镇沈阳,宣布出兵。

		下令后不久,回报:

		今天下大雨,走不了。

		走不了,那就休息吧。

		这一休息就是四天,二月二十五日,杨镐说,今天出兵。

		下令后不久,又回报:

		辽东地区降雪,行军道路泥泞,请求延后。

		几十年来,杨镐先生虽说打仗是不太行,做人倒还行,很少跟人红脸,对于合理化建议,他也比较接受,既然下大雨延期他能接受,下大雪延期,似乎也没什么问题。

		在这个世界上,好人不怕,坏人也不怕,就怕时好时坏、无端抽风的人。

		杨镐偏偏就是个抽风的人,不知是那根筋有问题,突然发火了:

		“国家养士,只为今日,若临机推阻,军法从事!”

		完事还把尚方宝剑挂在门外,那意思是,谁敢再说话,来一个干一个。

		窝囊了几十年,突然雄起,也算可喜可贺。

		然而接下来发生的一幕,就让杨先生雄不起来了。

		按照惯例,出师之前,要搞个仪式,一般是找个叛徒、汉奸类的人物杀掉祭旗,然后再杀几头牲口祭天。

		祭旗的时候,找了抚顺的一个逃兵,一刀下去,干掉了,可祭天的时候,却出了大问题。

		事实证明,有时候,宰牲口比宰人要难得多,祭天的这头牛,不知是神牛下凡,还是杀牛刀太糙,反正是用刀捅、用脚揣,折腾了好几次,才把这牛干掉。

		封建社会,自然要搞点封建迷信,祭天的时候出了这事,大家都议论纷纷,然而杨镐先生却突然超越了时代,表现出了不信鬼神的大无畏精神。他坚定地下达了命令:

		出征!

		然后,他就干了件蠢事,一件蠢得让人毛骨悚然的事。

		在出征之前,杨镐将自己的出征时间、出征地点、进攻方向写成一封信,并托人送了出去,还反复叮嘱,必定要保证送到。

		收信人的名字,叫努尔哈赤。

		对于他的这一举动,许多后人都难以理解,还有人认为,他有汉奸的嫌疑。

		但我认为,以杨镐的智商,做出这样的事情,实在是不奇怪的。

		在杨镐看来,自己手中有十二万大军,努尔哈赤下属的全部兵力,也只有六万,手下的杜松、刘綎,身经百战,经验丰富,要对付山沟里的这帮游击队,毫无问题。

		基于这种认识,杨镐认为,作为天朝大军,写这封信,是很有必要的。

		在成功干掉一头牛,以及写信示威之后,四路大军正式出征,史称“萨尔浒之战”,就此拉开序幕。

		但在序幕拉开之前,战役的结局,实际上已经注定。

		因为几百年来几乎所有的人,都忽略了一个基本的问题:单凭这支明军,是无法消灭努尔哈赤的。

		努尔哈赤的军队,虽然只有六万人,却身经百战,极其精锐,且以骑兵为主,明军就不同了,十二万人,来自五湖四海,那真叫一个东拼西凑,除杜松、刘綎部外,战斗力相当不靠谱。

		以指挥水平而论,就更没法说了,要知道,这努尔哈赤先生并不是山寨的土匪,当年跟着李成梁混饭吃,那是见过大世面的,加上这位仁兄天赋异禀,极具军事才能,如果李如松还活着,估计还有一拼,以杜松、刘綎的能力,是顶不住的。

		实力,这才是失败的真相。

		杨镐的错误,并不是他干了什么,而是他什么也没干。

		其实从他接手的那天起,失败就已注定。因为以当时明军的实力,要打赢是不容易的,加上他老人家,那就变成不可能了。

		可惜这位大爷对此毫无意识,还把军队分成了四部。

		在这四支部队中,他把最精锐的六万余人交给了杜松,由其担任先锋。其余三部各两万人,围攻努尔哈赤。

		这个想法,在理论上是很合理的,但在实践中,是很荒谬的。

		按照杨镐的想法,仗是这么打的:努尔哈赤要呆在赫图阿拉,不许随便乱动,等到明朝四路大军压境,光荣会师,战场上十二万对六万,\footnote{最好分配成两个对一个。},也不要骑马,只能步战,然后决一死战,得胜回朝。

		有这种脑子的人,只配去撞墙。

		要知道,努尔哈赤先生的日常工作是游击队长,抢了就分,打了就跑,也从来不修碉堡炮楼,严防死守。

		这就意味着,如果努尔哈赤集中兵力,杜松将不具备任何优势,再加上杜将军的脑筋向来缺根弦,和努尔哈赤这种老狐狸演对手戏,必败无疑。

		而当努尔哈赤听到明军四路进军的消息后,只说了一句话:

		“凭尔几路来,我只一路去。”

		我仿佛看见,一出悲剧正上演,剧中没有喜悦。

		二月二十八日,明军先锋杜松抵达抚顺近郊。

		为了抢头功,他命令士兵日夜不停行军,但由于路上遭遇女真部队阻击,辎重落后,三月一日,他终于停下了脚步,就地扎营。

		他扎营的地点,叫做萨尔浒。

		\subsection{死战}
		此时的杜松,已经有点明白了,自他出征以来,大仗没有,小仗没完,今天放火明天偷袭,后勤也被切断,只能扎营固守。

		多年的战争经验告诉他,敌人就在眼前,随时可能发动进攻,情况非常不利,部下建议,应撤离此地。

		但他并未撤退,却将手下六万人分为两部,分别驻守于吉林崖和萨尔浒。

		杜松并未轻敌,事实上,他早已判定,隐藏在自己附近的,是女真军队的主力,且人数至少在两万以上。

		以自己目前的兵力,攻击是不可能的,但防守还是不成问题的,所以没有撤退的必要。

		应该说,他的判断是准确的,只有一点不同——埋伏在这里的,并不是女真部队的主力,而是全部。

		刘綎的运气相当不好\footnote{或者说是相当好。},由于他的行军道路比较偏,走后不久就迷了路,敌人没找着他,当然,他也没找到敌人。

		但这种摸黑的游戏没能持续多久。努尔哈赤已经擦掉了刀上的血迹,开始专心寻找刘綎。

		三月初四,他找到了。

		此时,刘綎的兵力只有一万余人,是努尔哈赤的四分之一。胜负未战已分。

		然而还在山谷中转悠的刘綎并没有听到震耳的冲杀声,却等来了一个使者,杜松的使者。

		使者的目的只有一个:传达杜松的命令,希望刘綎去与他会合。

		此时,杜松已经死去,所以这个使者,是努尔哈赤派人假冒的。

		但是刘綎并没有上当,他当即回绝了使者的要求。

		不过他回绝的理由,确实有点搞笑:

		“我是总兵,杜松也是总兵,他凭什么命令我!”

		这下连假使者也急了,连说带比划,讲了一堆好话,刘綎才最终同意,前去与杜松会师。

		然后,他依据指引,来到了一个叫阿布达里岗的地方,这里距离赫图阿拉只有几十里。

		在这里,他看见了杜松的旗帜和军队。

		但当这支军队冲入队列,发动攻击时,他才知道自己上当了。

		寡不敌众、深陷重围,必败无疑,必死无疑。

		但刘綎仍然镇定地拔出了刀,开始奋战。

		之后的一切,史书上是这样介绍的:
		\begin{quote}
			\begin{spacing}{0.5}  %行間距倍率
				\textit{{\footnotesize
							\begin{description}
								\item[\textcolor{Gray}{\faQuoteRight}] 阵乱,綎中流矢,伤左臂,又战,
								\item[\textcolor{Gray}{\faQuoteRight}] 复伤右臂、犹鏖战不已,
								\item[\textcolor{Gray}{\faQuoteRight}] 内外断绝,面中一刀,截去半颊,犹左右冲突,
								\item[\textcolor{Gray}{\faQuoteRight}] 手歼数十人而死。
							\end{description}
						}}
			\end{spacing}
		\end{quote}

		用今天的话说,大致是这样:

		阵乱了,刘綎中箭,左臂负伤,继续作战。

		在战斗中,他的右臂也负伤了,依然继续奋战。

		身陷重围无援,他的脸被刀砍掉了一半,依然继续奋战,左冲右杀。

		最后,他杀死了数十人,战死。

		这就是一个身陷绝境的将领的最后记录。

		这是一段毫无感情,也无对话的文字,但在冷酷的文字背后,我听了刘綎最后的遗言和呼喊:

		宁战而死,绝不投降!

		刘綎战死,东路军覆灭。

		现在,只剩下南路军了。

		南路军的指挥官,是李如柏。

		因为他的部队速度太慢,走了几天,才到达预定地点,此时其他三路军已经全军覆没。

		于是在坐等一天之后,他终于率领南路军光荣回朝,除因跑得过快,自相践踏死了点人外,毫发无伤。

		就军事才能而言,他是四人之中最差的一个,但他的运气却实在很好,竟然能够全身而退。

		或许这一切,并不是因为运气。

		因为许多人都依稀记得,他是李成梁的儿子,而且他还曾经娶过一个女子,可这位女子偏偏就是努尔哈赤的弟弟,舒尔哈齐的女儿。

		无论是运气太好还是太早知道,反正他是回来了。

		但在战争,尤其是败仗中,活下来的人是可耻的,李如柏终究还是付出了代价。

		回来后,他受到了言官的一致弹劾,而对于这样一个独自逃跑的人,所有人的态度都是一致的——鄙视。

		偷生的李如柏终于受不了了,在这种生不如死的环境中,他选择了自尽,结束自己的生命。

		萨尔浒大战就此结束,此战明军大败,死伤将领共计三百一十余人,士兵死伤四万五千八百七十余人,财物损失不计其数。

		消息传回京城,万历震怒了。

		我说过,万历先生不是不管事,是不管小事,打了这么个烂仗,实在太过窝囊。

		觉得窝囊了,自然要找人算帐,几路总兵都死光了,自然要找杨镐。

		杨镐倒是相当镇定,毕竟他的关系搞得好,自他回来后,言官弹劾不绝于耳,但有老上级兼老同党方从哲保着,他也不怎么慌。

		可这事实在是太大了,皇帝下旨追查,言官拼命追打,特别是一个叫杨鹤的御史,三天两头上书,摆明了是玩命的架势,那边努尔哈赤还相当配合,又攻陷了铁岭,几棍子抡下来,实在是扛不住了。

		不久后,他被逮捕,投入诏狱,经审讯判处死刑,数年后被斩首。

		责任追究完了,但就在追究责任的时候,努尔哈赤也没歇着,还乘势攻下了全国比较大的城市——铁岭。

		至此,辽东北部全部被努尔哈赤占领,明朝在辽东的根据地,只剩下了沈阳和辽阳。

		看上去,局势十分危急,但事实上,是万分危急。

		萨尔浒之战后,明军陷入了彻底的混乱,许多地方不见敌人,听到风声就跑,老百姓跑,当兵的也跑,个别缺德的骑兵为了不打仗,竟然主动把马饿死。

		而由于指挥系统被彻底打乱,朝廷的军饷几个月都无法发放,粮食也没有,对努尔哈赤而言,此地已经唾手可得。

		但他终究没有得到,因为接替杨镐的人已经到任。他的名字,叫做熊廷弼。

		熊廷弼,是个不讨人喜欢的家伙。

		熊廷弼,字飞白,江夏\footnote{今湖北武汉。}人,自小聪明好学,乡试考中第一,三十岁就成为进士,当上了御史。

		可此人脾气太坏,坏到见谁和谁过不去,坏到当了二十年的御史都没升官。

		他还有个嗜好——骂人,且骂得很难听,后来连他都察院的同事都受不了,压根不搭理他,基本算是人见人厌。

		但如果没有这个人见人厌的家伙,相信明朝差不多就可以收摊,下场休息去了。

		万历四十七年(1619年),萨尔浒大战后,在一片混乱之中,新任经略熊廷弼带着几个随从,进入了辽东。

		他从京城出发的时候,开原还没有失陷,但当他到达辽东的时候,连铁岭都丢掉了。

		等他到达辽阳的时候,才发现,明朝仅存的沈阳和辽阳,已几乎是一座空城。

		他命令下属前往沈阳,稳定局势,叫来一个,竟然吓得直哭,打死都不敢去,再换一个,刚刚走出城,就跑回来了,说打死也不敢再走。

		于是熊廷弼说:

		“我自己去。”

		他从辽阳出发,一路走一路看,遇到逃跑的百姓,就劝他们回去,遇到逃跑的士兵,就收编他们,遇到逃跑的将领,就抓起来。

		就这样,到沈阳的时候,他已经集结了上万平民,数千名士兵,还有王捷、王文鼎等几位逃将。

		安置了平民,整顿了士兵,就让人把逃将拉出去,杀头。

		逃将求饶,说我们逃出来已经不容易了,何必要杀我们。

		熊廷弼说:如果不杀你们,怎么对得起那些没有逃跑的人?

		然后,他去见了李如桢。

		李如桢是铁岭的守将,但后金军队进攻的时候,他却一直呆在沈阳。

		不但一直呆在沈阳,铁岭被敌军攻击的时候,他连救兵都不派,坐视铁岭失守,让人十分费解,不知是反应迟钝,还是另有密谋。

		熊廷弼倒不打算研究这个问题,他只是找来这位仁兄,告诉他:你给我滚。

		李如桢当时还是总兵,不是说免就能免的,可熊廷弼实在太过凶恶,李总兵当即就滚了,回去后又挨了熊廷弼的弹劾,最后被关入监狱,判处死刑\footnote{后改充军。}。

		至此,一代名将李成梁的光荣世家彻底完结,除李如松外,都没啥好下场,连老家铁岭都被当年手下的小喽罗努尔哈赤占据,可谓是干干净净、彻彻底底。

		在当年的史料记载中,李成梁的事迹可谓数不胜数,和他同时期的戚继光,几乎完全被他的光芒所掩盖。

		但几百年后,戚继光依然光耀史册,万人景仰,而李成梁,却几乎已不为人知。

		我知道,历史只会夸耀那些值得夸耀的人。

		当所有人都认为,熊廷弼的行动已告一段落时,他却又说了一句话:

		“我要去抚顺。”

		大家认为熊廷弼疯了。

		当时的抚顺,已经落入努尔哈赤的手中,以目前的形势,带几个人去抚顺,无疑就是送死。

		但熊廷弼说,努尔哈赤认定我不敢去,所以我现在去,反而是最安全的。

		说是这么说,但敢不敢去,那是另外一码事。

		熊廷弼去了,大家战战兢兢,他却毫不惊慌,优哉游哉地转了一圈。

		当所有人都胆战心惊的时候,他又下了个让人抓狂的命令:吹号角。

		随行人员快要疯了,这就好比是孤身闯进山贼的山寨,再大喊抓贼,偷偷摸摸地来,你还大声喧哗,万一人家真的冲出来,你怎么办?

		但命令是必须执行的,人来了,号角吹了,后金军却一动不动。熊廷弼大摇大摆回了家。

		几天后,努尔哈赤得知了事情的真相,非但不恼火发动进攻,反而派人堵住了抚顺进出的关口,严令死守,不得随意出击。

		努尔哈赤之所以表现如此低调,只是因为他和头号汉奸李永芳的一次对话。

		当熊廷弼到来的消息传到后金时,李永芳急忙跑去找努尔哈赤,告诉他,这是个猛人。

		努尔哈赤不以为然:辽东已经到了这个地步,这蛮子\footnote{后金对明朝将领的通称。}就是再厉害,也只有一个人,如何挽回危局?

		李永芳回答:只要有他,就能挽回危局!

		此后发生的一切,都证明了李永芳的判断,只用了短短几个月,熊廷弼就稳定了局势,此后他一反常态,除了防御外,还组织了许多游击队,到后金占领地区进行骚扰,搞得对方疲于奔命,势头非常凶猛。

		于是,努尔哈赤决定,暂时停止对明朝的进攻,休养生息,等待时机。

		这个时机的期限,只有一年。

		然而正是这关键的一年挽救了明朝。因为此时的朝廷,即将发生几件惊天动地的大事。
		\ifnum\theparacolNo=2
	\end{multicols}
\fi
\newpage
\section{东林党的实力}
\ifnum\theparacolNo=2
	\begin{multicols}{\theparacolNo}
		\fi
		在很多的史书中,万历中后期的历史基本上是这个样子:皇帝老休息,朝政无人管,大臣无事干。

		前两头或许是正确的,但第三条是绝对不正确的。

		隐藏在平静外表下的,是无比激烈的斗争。而斗争的主角,是东林党。

		在许多人的印象中,东林是道德与正义的象征,一群胸怀理想的知识分子,为了同一个目标,走到一起来了。他们怀揣着抱负参与政治,并曾一度掌控政权,却因为被邪恶的势力坑害,最终失败。

		我认为,这是一个比较客观的说法。但是,很多人都忽略了一个问题,一个很有趣的问题:

		一群只会读书的书呆子、知识分子,是如何掌控政权的呢?

		正义和道德是值得景仰的,值得膜拜的,值得三拜九叩的,但是,正义和道德不能当饭吃,不能当衣服穿,更不可能掌控政权。

		因为掌控政权的唯一方式,就是斗争。

		道德文章固然有趣,却是无法解决问题的。

		最先认识到这一点的人,应该是顾宪成。

		在万历二十一年(1593年)的那次京察中,吏部尚书孙鑨——撤职了,考功司郎中赵南星——回家了,首辅王锡爵——辞职了,而这事幕后的始作俑者,从五品的小官,考功司员外郎顾宪成——升官了\footnote{吏部文选司郎中。}。

		升官了还不说,连他的上级,继任吏部尚书陈有年,也都是他老人家安排的,甚至后来回无锡当老百姓,他依然对朝廷动向了如指掌。李三才偷看信件,王锡爵打道回府,朝廷的历任首辅,在他眼中不是木偶,就是婴儿。

		这是一团迷雾,迷雾中的一切,似乎和他有关系,又似乎没有关系。

		拨开这团迷雾之后,我看到了一样东西——实力。

		顾宪成的实力,来自于他的官职。

		在吏部中,最大的是尚书\footnote{部长。}、其次是侍郎\footnote{副部长。},再往下就是四个司的郎中\footnote{司长。},分别是文选司、验封司、稽勋司、考功司。

		但是,这四个司的地位是不同的,而其中最厉害的,是文选司和考功司,文选司负责人事任免,考功负责官员考核,这两个司的官员向来无人敢惹,升官还是免职,发达还是破产,那就是一句话的事。

		相对而言,验封司、稽勋司就一般了,一般到不用再介绍。

		有鉴于此,明代的吏部尚书和侍郎,大都由文选司和考功司的郎中接任。

		而顾宪成先生的升迁顺序是:吏部考功司主事——考功司员外郎\footnote{副职。}——文选司郎中。

		这就意味着,那几年中,大明的所有官员\footnote{除少数高官。},无论是升迁,还是考核,都要从顾宪成手底下过,即使不过,也要打个招呼,就不打招呼,也得混个脸熟。

		此外,我们有理由相信,顾宪成大人也是比较会来事的,因为一个不开窍的书呆子,是混不了多久的。

		现在你应该明白了。

		在这个世界上,实力和道德,经常是两码事。

		东林之中,类似者还有很多,比如李三才。

		李三才先生的职务,之前已经说过,是都察院佥都御史,巡抚凤阳,兼漕运总督。

		都察院佥都御史多了去了,凤阳是个穷地方,不巡也罢,真正关键的职务,是最后那个。

		自古以来,漕运就是经济运转的主要途径,基本算是坐地收钱,肥得没边,普天之下,唯一可以与之相比的,只有盐政。

		坐在这个位置上,要想不捞外快,一靠监督,二靠自觉。

		很可惜,李三才不自觉,从种种史料分析,他很有钱,有钱得没个谱,请客吃饭,都是大手笔。

		至于监督,那就更不用说了,这位李先生本人就是都察院的御史,自己去检举自己,估计他还没这个觉悟。

		作为东林党的重量级人物,李三才在这方面的名声,那真是相当的大,大到几十年后,著名学者夏允彝到凤阳寻访,还能听到相关事迹,最后还叹息一声,给了个结论——负才而守不洁。

		列举以上两人,只是为了说明一点:

		东林,是书院,但不仅仅是书院,是道德,但不仅仅是道德。它是一个有实力,有能力,有影响力、有斗争意识的政治组织。

		事实上,它的能量远远超出你的想象。

		明白了这一点,你就会发现,那段看似平淡无奇的历史,每一分、每一秒,都是你死我活的争斗。

		争斗的方式,是京察。

		万历二十一年(1593年),顾宪成失望地回家了,他虽费劲气力,却终究未能解决对手,京察失败。

		但这一切,仅仅是个开始。

		十二年后(万历三十三年),京察开始,主持者杨时乔,他的公开身份,是吏部左侍郎,他的另一个公开身份,是东林党。

		当时的首辅,是浙党首领沈一贯,对于这位东林党下属,自然很不待见,于是,他决定换人。

		沈一贯是朝廷首辅,杨时乔只是吏部二把手,然而意外发生了,虽然沈大人上窜小跳,连皇帝的工作都做了,却依然毫无用处。杨侍郎该怎么来,还怎么来,几板斧抡下来,浙党、齐党、楚党、宣党……反正非东林党的,统统下课,沈一贯拼了老命,才算保住几个亲信。

		那么现在,请你再看一遍之前列举过的几条史料,玄机就在其中:

		万历三十三年(1605年),京察,沈一贯亲信以及三党干将被逐。

		万历三十五年(1607年),沈一贯退休回家。

		同年,王锡爵的密信被李三才揭发,复出无望。

		一年后,东林派叶向高成为首辅,开始执掌朝廷大权。

		是的,这一切的一切,不是偶然。

		而最终要获得的,正是权力。

		权力已经在握,但还需要更进一步。

		万历三十九年(1611年),辛亥京察,主持人吏部尚书孙丕杨,东林党。

		此时的首辅已经是叶向高了,东林党人遍布朝廷,对于那些非我族群而言,清理回家之类的待遇估计是免不了了。

		然而一个人的掺和,彻底改变了这一切。这个人就是李三才。

		此时的李三才已经升到了户部尚书,作为东林党的干将,他将进入内阁,更进一步。

		算盘大致如此,可打起来就不是那么回事了。

		听说李三才要入阁,朝廷顿时一片鸡飞狗跳,闹翻了天,主要原因在于李先生的底子不算干净,许多人对他有意见。

		而更重要的是,这人实在太猛,太有能力。东林党已经如此强大,如果再让他入阁,三党的人估计就只能集体歇业了。

		于是,一场空前猛烈的反击开始。

		明代的京察,按照地域,分为南察和北察,北察由尚书孙丕杨负责,而南察的主管者,是吏部侍郎史继楷,三党成员,他选定的考察对象都是同一个类型——支持李三才的人。

		很快,浙、楚、齐三党轮番上阵,对李三才发起了最后的攻击,他们的动机十分明确,明确到《明神宗实录》都写了出来——“攻淮\footnote{李三才。}则东林必救,可布一网打尽之局”。

		在集中火力打击之下,李三才没能顶住,回家养老去了。

		但就整体而言,此时的东林党依然占据着优势,叶向高执政,东林党掌权,非常强大,强大得似乎不可动摇。

		然而就在此时,强大的东林党,犯了一个致命的错误。

		一直以来,东林党的指导思想,是我很道德。强大之后,就变成了你不道德,工作方针,原先是党同伐异,强大之后,就变成了非我族类,其心必异。

		总而言之,不是我的同党,就是我的敌人。

		这种只搞单边主义的混账做法,最终导致了一个混账的结果:

		在东林党人的不懈努力下,齐、浙、楚三党终于抛弃了之前的成见,团结一致跟东林党死磕了。

		他们的折腾,得到了立竿见影的回报:

		万历四十二年(1614年),叶向高退休回家。

		万历四十五年(1617年),京察开始,主持京察的,分别是吏部尚书郑继之、刑部尚书李志。

		郑继之是楚党,李志是浙党。

		有冤报冤,有仇报仇的时候到了,但凡是东林党,或者与东林党有关的人,二话不说,收包袱走人。这其中,还包括那位揭发了梃击案真相的王之寀。

		萨尔浒之战前,朝廷斗争情况大致如此,这场斗争的知名度相当小,但在历史上的地位相当重要。对明朝而言,其重要程度,基本等于努尔哈赤+皇太极+李自成+张献忠。

		因为这是一场延续了几十年的斗争,是一场决定明朝命运的斗争。

		因为在不久之后,东林党将通过一个人的帮助,彻底击败浙、齐、楚三党。

		然后,土崩瓦解的三党将在另一个人的指挥下,实现真正的融合,继续这场斗争,而那时,他们将有一个共同的名字——阉党。

		万历四十五年的京察,标志着东林党的没落,所谓东林党三大巨头,顾宪成已经死了,邹元标到处逛,赵南星家里蹲。

		两大干将也全部消停,叶向高提早退休,李三才回家养老。

		此时的首辅,是浙党的方从哲,此时的朝廷,是三党的天下。对东林党而言,前途似乎一片黑暗。

		但新生的机会终会到来,因为一个人的死去。

		万历四十八年(1620年)七月二十一日,万历不行了。

		高拱、张居正、申时行、李成梁、东林党、朝鲜、倭寇、三大征、萨尔浒、资本主义萌芽、不上朝、太子、贵妃、国本、打闷棍。

		我只能说,他这辈子应该比较忙。

		关于这位兄弟的评论,我想了很久,很久,却是很久,很久,也想不出来。

		你说他没干过好事吧,之前二十多年,似乎干得也不错,你说他软弱吧,他还搞了三大征,把日本鬼子赶回了老家,你说他不理朝政吧,这几十年来哪件大事他不知道?

		一个被张居正压迫过的人,一个勤于政务的人,一个被儿子问题纠缠了几十年的人,一个许多年不见大臣、不上班的人,一个终生未出京城,生于深宫、死于深宫的人。

		一个复杂得不能再复杂的人,一个简单得不能再简单的人。

		于是,我最终懂得了这个人。

		一个热血沸腾的青年,一个励精图治的君主,一个理想主义者,在经历残酷的斗争,无休止的吵闹,无数无效的抗争,无数无奈的妥协后,最终理解了这个世界,理解了现实的真正意义,并最终成为了这个世界的牺牲品。

		大致如此吧。

		明神宗朱翊钧,万历四十八年逝世,年五十八。

		在这个残酷的世界面前,他还不够勇敢。

		\subsection{明光宗朱常洛}
		虽然几十年来,万历都不喜欢自己的长子朱常洛,但在生命的最后一刻,他终于做出了抉择,将皇位传给了这个久经考验的儿子。

		担惊受怕几十年的朱常洛终于熬出头了,万历四十八年(1620年)八月一日,朱常洛正式登基,即后世所称之明光宗,定年号为泰昌。

		由于此时还是万历年间,按照惯例,要等老爹这一年过完,明年才能另起炉灶,用自己的年号。

		可几乎所有的人都没有想到,这个年号,竟然没能用上。

		因为朱常洛活了三十八年,明光宗却只能活一个月。

		一个撑了三十八年,经历无数风雨险阻到达目标的人,却在一个月中意外死亡,是很不幸的。

		导致死亡与不幸的罪魁祸首,是郑贵妃。

		\subsection{红丸}
		应该说,朱常洛是个好孩子,至少比较厚道。

		几十年来,他一直夹着尾巴做人,亲眼目睹了父亲的冷漠、朝廷的冷清,感受到了国家的凋敝,时局的危险。

		他不愿意再忍受下去,于是,当政后的第一天,他用几道谕令显示了自己的决心。

		大致说来,他是把他爹没办的事给办了,包括兑现白条——给辽东前线的士兵发工资,废除各地矿税,以及补充空缺的官员。

		这几件事情,办得很好,也很及时,特别是最后一条,把诸多被万历同志赶下岗的仁兄们拉了回来,实在是大快官心,于是一时之间,光宗的人望到达了顶点,朝廷内外无不感恩戴德,兴高采烈。

		但有一个人不高兴,非但不高兴,而且很害怕。

		万历死后,郑贵妃终于明白,自己是多么的虚弱,今日之城内,已是敌人之天下。所谓贵妃,其实也不贵,如果明光宗要对付她,贱卖的可能性是相当的大。

		很快,一件事情就证明了她的判断。

		考虑到万历死后不好办,之前郑贵妃软磨硬泡,让万历下了道遗嘱,讲明,一旦自己死后,郑贵妃必须进封皇后。

		如此一来,等万历死后,她就成了太后,无论如何,铁饭碗是到手了。

		明光宗看上去倒也老实,丝毫不赖帐,当即表示,如果父皇如此批示,那就照办吧。

		但他同时表示,这是礼部的事,我批下去,让他们办吧。

		按说皇帝批下来就没问题了,可是礼部侍郎孙如游不知怎么回事,非但不办,还写了个奏疏,从理论、辈分、名分上论证了这件事,最后得出结论——不行。

		光宗同志似乎也不生气,还把孙侍郎的奏疏压了下来,但封皇后这事再也没提。

		郑贵妃明白了,这就是个托。

		很明显,这位看上去很老实的人,实际上不怎么老实。既然如此,必须提前采取行动。

		经过深思熟虑,她想出了一个计划,而这个计划的第一步,是一件礼物。

		十天之后,她将这件礼物送给了朱常洛,朱常洛很高兴地收下了。

		光宗皇帝的性命,就丢在了这份礼物上。

		这份礼物,是八个美女。

		对于常年在宫里坐牢,哪都不能去,啥也没有的朱常洛而言,这是一份丰厚的礼物,辛辛苦苦、畏畏缩缩了几十年,终于可以放纵一下了。

		古语有云:一口吃不成胖子,但朱常洛应该算是不同凡响,他几天就变成了瘦子,在史料上,含蓄的文言文是这样描述的:

		“是夜,连幸数人,圣容顿减。”

		白天日理万机,晚上还要辛勤工作,身体吃不消,实在是件十分自然的事情。于是不久之后,朱常洛就病倒了。

		这一天是万历四十八年(1620年)八月十日。

		计划的第二步即将开始,四天之后。

		万历四十八年(1620年)八月十四日。

		皇帝的身体依然很差,身体差就该看医生,崔文升就此出了场。

		崔文升,时任司礼监秉笔太监。前面曾经讲过,这是一个十分重要的职务,仅次于司礼监掌印太监。

		可是这人来,并不是要给皇帝写遗嘱,而是看病,因为这位崔兄多才多能,除了能写外,还管着御药房,搞第二产业。

		后来的事情告诉我们,第二产业是不能随便乱搞的。

		诊断之后,崔大夫胸有成竹,给病人开了一副药,并且乐观地表示,药到病除。

		他开的这幅药,叫泻药。

		一个夜晚辛勤工作,累垮了身体的人,怎么能服泻药呢?

		所以后来很多史书都十分肯定地得出了结论:这是个“蒙古大夫”。

		虽然我不在现场,也不懂医术,但我可以认定:崔文升的诊断,是正确的。

		因为之前的史料中,有这样六个字:是夜,连幸数人。

		这句话的意思大家应该知道,就不解释了,但大家也应该知道,要办到这件事情,难度是很大的。对光宗这种自幼体弱的麻杆而言,基本就是个不可能的任务。

		但是他完成了。

		所以唯一的可能性是,他找了帮手,而这个帮手,就是药物。

		是什么药物,大家心里也有数,我就不说了,这类药物在明代宫廷里,从来就是必备药,从明宪宗开始,到天天炼丹的嘉靖,估计都没少用。明光宗初来乍到,用用还算正常。

		可这位兄弟明显是用多了,加上身体一向不好,这才得了病。

		在中医理论中,服用了这种药,是属于上火,所以用泻药清火,也还算对症下药。

		应该说,崔文升是懂得医术的,可惜,是半桶水。

		根据当时史料反映,这位仁兄下药的时候,有点用力过猛,手一哆嗦,下大了。

		错误是明显的,后果是严重的,光宗同志服药之后,一晚上拉了几十次,原本身体就差,这下子更没戏了,第二天就卧床不起,算彻底消停了。

		蒙古的崔大夫看病经历大致如此,就这么看上去,似乎也就是个医疗事故。虽说没法私了,但毕竟大体上没错,也没在人家身体里留把剪刀、手术刀之类的东西当纪念品,态度还算凑合。

		可问题是,这事一冒出来,几乎所有的人都立刻断定,这是郑贵妃的阴谋。

		因为非常凑巧,这位下药的崔文升,当年曾经是郑贵妃的贴身太监。

		这真是跳进黄河也洗不清,要看病,不找太医,偏找太监,找了个太监,偏偏又是郑贵妃的人,这太监下药,偏又下猛了,说他没问题,实在有点困难。

		对于这件事情,你说它不是郑贵妃的计划,我信,因为没准就这么巧;说它是郑贵妃的计划,我也信,因为虽说下药这招十分拙劣,谁都知道是她干的,但以郑贵妃的智商,以及从前表现,这种蠢事,她是干得出来的。

		无论动机如何,结果是肯定的,明光宗已经奄奄一息,一场惊天大变即将拉开序幕。

		但这一切还不够,要达到目的,这些远远不够,即使那个人死去,也还是不够。

		必须把控政权,把未来所有的一切,都牢牢抓在手中,才能确保自己的利益。

		于是在开幕之前,郑贵妃找到了最后一个同盟者。

		这位同盟者的名字,不太清楚。

		目前可以肯定的是,她姓李,是太子的嫔妃。

		当时太子的嫔妃有以下几种:大老婆叫太子妃,之后分别是才人、选侍、淑女等。

		而这位姓李的女人,是选侍,所以在后来的史书中,她被称为“李选侍”。

		李选侍应该是个美女,至少长得还不错,因为皇帝最喜欢她,而且皇帝的儿子,那个未来的天才木匠——朱由校,也掌握在她的手中,正是因为这一点,郑贵妃找上了她。

		就智商而言,李选侍还算不错\footnote{相对于郑贵妃。},就人品而言,她和郑贵妃实在是相见恨晚,经过一番潜规则后,双方达成协议,成为了同盟,为了不可告人的目的。

		现在一切已经齐备,只等待着一个消息。

		所有的行动,将在那一刻展开,所有的野心,将在那一刻实现。

		\subsection{小人物}
		目标就在眼前,一切都很顺利。

		皇帝的身体越来越差,同党越来越多,帝国未来的继承人尽在掌握之中,在郑贵妃和李选侍看来,前方已是一片坦途。

		然而她们终究无法前进,因为一个微不足道的小人物。

		明光宗即位后,最不高兴的是郑贵妃,最高兴的是东林党。

		这是很正常的,从一开始,东林党就把筹码押在这位柔弱的太子身上,争国本、妖书案、梃击案,无论何时何地,他们都坚定地站在这一边。

		现在回报的时候终于到了。

		明光宗非常够意思,刚上任,就升了几个人的官,这些人包括刘一璟、韩旷、周嘉谟、邹元标、孙如游等等。

		这几个人估计你不知道,其实也不用知道,只要你知道这几个人的职务,就能明白,这是一股多么强大的力量。

		刘一璟、韩旷,是东阁大学士,内阁成员,周嘉谟是吏部尚书,邹元标是大理寺丞,孙如游是礼部侍郎。当然,他们都是东林党。

		在这群人中,有内阁大臣、人事部部长、法院院长,部级高官,然而,在后来那场你死我活的斗争中,他们只是配角。真正力挽狂澜的人,是一个看似微不足道的小人物。

		这个人的名字,叫做杨涟。

		杨涟,字文孺,号大洪,湖广\footnote{湖北。}应山人,万历三十五年(1607年)进士,任常熟知县,后任户科给事中、兵科给事中。

		这是一份很普通的履历,因为这人非但当官晚,升得也不快,明光宗奄奄一息的时候,也才是个七品给事中。

		但在这份普通履历的后面,是一个不普通的人。

		上天总是不公平的,有些人天生就聪明,天生就牛,天生就是张居正、戚继光,而绝大多数平凡的人,天生就不聪明,天生就不牛,天生就是二傻子,没有办法。

		但上天依然是仁慈的,他给出了一条没有天赋,也能成功的道路。

		对于大多数平凡的人而言,这是最好的道路,也是唯一的道路,它的名字,叫做纯粹。

		纯粹的意思,就是专心致志、认真、一根筋、二杆子等等等等。

		纯粹和执着,也是有区别的,所谓执着,就是不见棺材不掉泪,而纯粹,是见了棺材,也不掉泪。

		纯粹的人,是这个世界上最可怕的人,他们的一生,往往只有一个目标,为了达到这个目标,他们可以不择手段,不顾一切,他们无法被收买,无法被威逼,他们不要钱,不要女色,甚至不要权势和名声。

		在他们的世界里,只有一个目标,以及坚定的决心和意志。

		杨涟,就是一个纯粹的人。

		他幼年的事迹并不多,也没有什么砸水缸之类的壮举,但从小就为人光明磊落,还很讲干净,干净到当县令的时候,廉政考核全国第一。此外,这位仁兄也是个不怕事的人,比如万历四十八年(1620年),万历生病,半个月不吃饭,杨涟听说了,也不跟上级打招呼,就跑去找首辅方从哲:

		“皇上生病了,你应该去问安。”

		方首辅胆子小,脾气也好,面对这位小人物,丝毫不敢怠慢:

		“皇上一向忌讳这些问题,我只能去问宫里的内侍,也没消息。”

		朝廷首辅对七品小官,面子是给足了,杨先生却不要这个面子,他先举了个例子,教育了首辅大人,又大声强调:

		“你应该多去几次,事情自然就成了\footnote{自济。}!”

		末了,还给首辅大人下了个命令:

		“这个时候,你应该住在内阁值班,不要到处走动!”

		毫无惧色。

		根据以上史料,以及他后来的表现,我们可以认定:在杨涟的心中,只有一个目标——为国尽忠,匡扶社稷。

		事实上,在十几天前的那个夜晚,这位不起眼的小人物,就曾影响过这个帝国的命运。

		万历四十八年(1620年)七月二十一日,夜,乾清宫。

		万历就快撑不住了,在生命的最后时刻,他反省了自己一生的错误,却也犯下了一个十分严重的错误——没有召见太子。

		一般说来,皇帝死前,儿子应该在身边,除了看着老爹归西、嚎几声壮胆以外,还有一个重要意义——确认继位。

		虽说太子的名分有了,但中国的事情一向难说,要不看着老爹走人,万一隔天突然冒出几份遗嘱、或是几个顾命大臣,偏说老头子临死前改了主意,还找人搞了公证,这桩官司可怎么打?

		但不知万历兄是忘了,还是故意的,反正没叫儿子进来。

		太子偏偏是个老实孩子,明知老头子不行了,又怕人搞鬼,在宫殿外急得团团转,可就是不敢进去。

		关键时刻,杨涟出现了。

		在得知情况后,他当机立断,派人找到了一个极为重要的人物——王安。

		王安,时任太子侍读太监,在明代的历史中,这是一个重量级人物。此后发生的一系列事件里,他都起着极为关键的作用。

		而在那个夜晚,杨涟只给王安带去了一句话,一句至关紧要的话:

		“皇上已经病得很重了\footnote{疾甚。},不召见太子,并不是他的本意。太子应该主动进宫问候\footnote{尝药视膳。},等早上再回去。”

		这就是说,太子您之所以进宫,不是为了等你爹死,只是进去看看,早上再回去嘛。

		对于这个说法,太子十分满意,马上就进了宫,问候父亲的病情。

		当然,第二天早上,他没回去。

		朱常洛就此成为了皇帝,但杨涟并没有因此获得封赏,他依然是一个不起眼的给事中。不过,这对于杨先生而言,实在是个无所谓的事。

		他平静地回到暗处,继续注视着眼前的一切。他很清楚,真正的斗争刚刚开始。

		事情正如他所料,蒙古崔大夫开了泻药,皇帝陛下拉得七荤八素,郑贵妃到处活动,李选侍经常串门。

		当这一切被组合起来的时候,那个无比险恶的阴谋已然暴露无遗。

		形势十分危急,不能再等待了。

		杨涟决定采取行动,然而现实很残酷:他的朋友虽然多,却很弱小,他的敌人虽然少,却很强大。

		周嘉谟、刘一璟、韩爌这拨人,级别固然很高,但毕竟刚上来,能量不大,而郑贵妃在宫里几十年,根基极深,一手拉着李选侍,一手抓着皇长子,屁股还拼命往皇太后的位置上凑。

		按照规定,她应该住进慈宁宫,可这女人脸皮相当厚,死赖在乾清宫不走,看样子是打算长住。

		因为乾清宫是皇帝的寝宫,可以监视皇帝的一举一动,一旦光宗同志有啥三长两短,她必定是第一个采取行动的人,那时,一切都将无可挽回。

		而要阻止这一切,杨涟必须做到两件事情:首先,他要把郑贵妃赶出乾清宫;其次,他要把郑贵妃当太后的事情彻底搅黄。

		这就是说,先要逼郑老寡妇搬家,再把万历同志临死前封皇后的许诺当放屁,把郑贵妃翘首企盼的申请拿去垫桌脚。

		杨涟先生的职务,是七品兵科给事中,不是皇帝。

		事实上,连皇帝本人也办不了,光宗同志明明不喜欢郑贵妃,明明不想给她名分,也没法拍桌子让她滚。

		这就是七品芝麻官杨涟的任务,一个绝对、绝对无法完成的任务。

		但是他完成了,用一种匪夷所思的方式。

		他的计划是,让郑贵妃自己搬出去,自己撤回当皇太后的申请。

		这是一个看上去绝不可能的方案,却是唯一可能的方案。因为杨涟已经发现,眼前的这个庞然大物,有一个致命的弱点,只要伸出手指,轻轻地点一下,就够了。

		这个弱点有个名字,叫做郑养性。

		郑养性,是郑贵妃哥哥郑国泰的儿子,郑国泰死后,他成为了郑贵妃在朝廷中的联系人,平日十分嚣张。

		然而杨涟决定,从这个人入手,因为经过细致的观察,他发现,这是一个外强中干,性格软弱的人。

		万历四十八年(1620年)八月十六日。杨涟直接找到了郑养性,和他一同前去的,还有周嘉谟等人。

		一大帮子人上门,看架势很像逼宫,而事实上,确实是逼宫。

		进门也不讲客套,周嘉谟开口就骂:

		“你的姑母\footnote{指郑贵妃。}把持后宫多年,之前争国本十几年,全都是因为她,现在竟然还要封皇太后,赖在乾清宫不走,还给皇上奉送美女,到底有什么企图?!”

		刚开始时,郑养性还不服气,偶尔回几句嘴,可这帮人都是职业选手,骂仗的业务十分精湛,说着说着,郑养性有点扛不住了。

		白脸唱完了,接下来是红脸:

		“其实你的姑母应该也没别的意思,不过是想守个富贵,现在朝中的大臣都在这里,你要听我们的话,这事就包在我们身上。”

		红脸完了,又是唱白脸:

		“要是不听我们的话,总想封太后,不会有人帮你,你总说没这想法,既然没这想法,就早避嫌疑!”

		最狠的,是最后一句:

		“如此下去,别说富贵,身家性命能否保得住,都未可知!”

		郑养性彻底崩溃了。眼前的这些人,听到的这些话,已经打乱了他的思维。于是,他去找了郑贵妃。

		其实就时局而言,郑贵妃依然占据着优势,她有同党,有帮手,如果赖着不走,谁也拿她没办法。什么富贵、性命,这帮闹事的书呆子,也就能瞎嚷嚷几句而已。

		然而关键时刻,郑贵妃不负白痴之名,再次显露她的蠢人本色,在慌乱的外甥面前,她也慌乱了。

		经过权衡利弊,她终于做出了决定:搬出乾清宫,不再要求当皇太后。

		至此,曾经叱诧风云的郑贵妃,正式退出了历史舞台,这位大妈费劲心机,折腾了三十多年,却啥也没折腾出来。此后,她再也没能翻过身来。

		这个看似无比强大的对手,就这样,被一个看似微不足道的人,轻而易举地解决了。

		但在杨涟看来,这还不够,于是三天之后,他把目标对准了另一个人。

		万历四十八年(1620年)八月十九日,杨涟上书,痛斥皇帝。

		杨先生实在太纯粹,在他心中,江山社稷是第一位的,所以在他看来,郑大妈固然可恶,崔大夫固然可恨,但最该谴责的,是皇帝。

		明知美女不应该收,你还要收,明知春药不能多吃,你还要吃,明知有太医看病,你还要找太监,不是脑袋有病吧。

		基于愤怒,他呈上了那封改变他命运的奏疏。

		在这封奏疏里,他先谴责了蒙古大夫崔文升,说他啥也不懂就敢乱来,然后笔锋一转,对皇帝提出了尖锐的批评——勤劳工作,不爱惜自己的身体。

		必须说明的是,杨先生不是在拍马屁,他的态度是很认真的。

		因为在文中,他先暗示皇帝大人忙的不是什么正经工作,然后痛骂崔文升,说他如何没有水平,不懂医术。最后再转回来:就这么个人,但您还是吃他的药。

		这意思是说,崔大夫已经够没水平了,您比他还要差。

		所以这奏疏刚送上去,内阁就放出话来,杨先生是没有好下场的。

		三天后,这个预言得到了印证。

		明光宗突然派人下令,召见几位大臣,这些人包括方从哲、周嘉谟、孙如游,当然,还有杨涟。此外,他还命令,锦衣卫同时进宫,听候指示。

		命令一下来,大家就认定,杨涟要完蛋了。

		因为在这拨人里,方从哲是首辅,周嘉谟是吏部尚书,孙如游是礼部尚书,全都是部级干部,只有杨涟先生,是七品给事中。

		而且会见大臣的时候,召集锦衣卫,只有一种可能——收拾他。

		由于之前的举动,杨涟知名度大增,大家钦佩他的人品,就去找方从哲,让他帮忙求个情。

		方从哲倒也是个老好人,找到杨涟,告诉他,等会进宫的时候,你态度积极点,给皇上磕个头,认个错,这事就算过去了。

		但是杨涟的回答,差点没让他一口气背过去:

		“死就死\footnote{死即死耳。},我犯了什么错?!”

		旁边的周嘉谟连忙打圆场:

		“方先生\footnote{方从哲。}是好意。”

		可到杨先生这里,好意也不好使:

		“知道是好意,怕我被人打死,要得了伤寒,几天不出汗,也就死了,死有什么可怕!但要我认错,绝无可能!”

		就这样,杨涟雄赳赳气昂昂地进了宫,虽然他知道,前方等待着他的,将是锦衣卫的大棍。

		可是他错了。

		那位躺在床上,病得奄奄一息的皇帝陛下非但没有发火,反而和颜悦色说了这样一句话:

		“国家的事情,全靠你们尽心为我分忧了。”

		虽然称呼是复数,但他说这句话的时候,眼睛只看着杨涟。

		这之后,他讲了许多事情,从儿子到老婆,再到郑贵妃,最后,他下达了两条命令:

		一、赶走崔文升。

		二、收回封郑贵妃为太后的谕令。

		这意味着,皇帝陛下听从了杨涟的建议,毫无条件,毫无抱怨。

		当然,对于他而言,这只是个顺理成章的安排。

		但他绝不会想到,他这个无意间的举动,将对历史产生极重要的影响。

		因为他并不知道,此时此刻,在他对面的那个人心中的想法。

		从这一刻起,杨涟已下定了决心——以死相报。

		一直以来,他都只是个小人物,虽然他很活跃,很有抱负,声望也很高,他终究只是小人物。

		然而眼前的这个人,这个统治天下的皇帝,却毫无保留地尊重,并认可了自己的情感、抱负,以及纯粹。

		所以他决定,以死相报,致死不休。

		这种行为,不是愚忠,不是效命,甚至也不是报答。

		它起源于一个无可争议,无可辩驳的真理:

		士为知己者死。

		这一天是万历四十八年(1620年)八月二十二日,明光宗活在世上的时间,还有十天。

		这是晚明历史上最神秘莫测的十天。一场更为狠毒的阴谋,即将上演。
		\ifnum\theparacolNo=2
	\end{multicols}
\fi
\newpage

\chapter*{朱常洛篇}
\addcontentsline{toc}{chapter}{朱常洛篇}
\section{小人物的奋斗}
\ifnum\theparacolNo=2
	\begin{multicols}{\theparacolNo}
		\fi
		八月二十三日。

		内阁大学士刘一璟、韩旷照常到内阁上班,在内阁里,他们遇见了一个人。

		这个人的名字叫李可灼,时任鸿胪寺丞,他来这里的目的,是要进献“仙丹”。

		此时首辅方从哲也在场,他对这玩意兴趣不大,毕竟皇帝刚吃错药,再乱来,这个黑锅就背不起了。

		刘一璟和韩旷更是深恶痛绝,但也没怎么较真,直接把这人打发走了。

		很明显,这是一件小事,而小事是不应该过多关注的。

		但某些时候,这个理论是不可靠的。

		两天后,八月二十五日。

		明光宗下旨,召见内阁大臣、六部尚书等朝廷重臣,此外,他特意叫上了杨涟。

		对此,所有的人都很纳闷。

		更让人纳闷的是,此后直至临终,他召开的每一次会议,都叫上杨涟,毫无理由,也毫无必要。或许是他的直觉告诉他,这个叫杨涟的人,非常之重要。

		他的直觉非常之准。

		此时的光宗,已经是奄奄一息,所以,几乎所有的大臣都认定,今天的会议,将要讨论的,是关乎国家社稷的重要问题。

		然而他们没有想到,这次内阁会议的议题,只有一个——老婆。

		光宗同志的意思是,自己的后妃李选侍,现在只有一个女儿,伺候自己那么多年,太不容易,考虑给她升官,封皇贵妃。

		此外,他还把皇长子朱由校领了出来,告诉诸位大人,这孩子的母亲也没了,以后,就让李选侍照料他。

		在场的所有人都目瞪口呆。

		明明您都没几天蹦头了,趁着脑袋还管用,赶紧干点实事,拟份遗嘱,哪怕找口好棺材,总算有个准备。竟然还想着老婆的名分,实在令人叹服。

		在现场的人们看来,这是一个尊重妇女,至死不渝的模范丈夫。

		但是事实并非如此。

		八月二十六日。

		出乎许多人的意料,明光宗再次下旨,召开内阁会议,与会人员包括内阁大臣及各部部长,当然还有杨涟。

		会议与昨天一样,开得十分莫名其妙。这位皇帝陛下把人叫进来,竟然先拉一通家常,又把朱由校拉进来,说我儿子年纪还小,你们要多照顾等等。

		这么东拉西扯,足足扯了半个时辰\footnote{一个小时。},皇上也扯累了,正当大家认为会议即将结束的时候,扯淡又开始了。

		如昨天一样,光宗再次提出,要封李选侍为皇贵妃,大家这才明白,扯来扯去不就是这件事吗?

		礼部尚书孙如游当即表示,如果您同意,那就办了吧\footnote{亦无不可。}。

		然而就在此时,一件令人震惊的事情发生了。

		一个人突然闯了进来,公然打断了会议,并在皇帝、内阁、六部尚书的面前,拉走了皇长子朱由校。

		这个人,就是李选侍。

		所有人都懵了,没有人去阻拦,也没有人去制止。原因很简单,这位李选侍毕竟是皇帝的老婆,皇帝大人都不管,谁去管。

		而更让人难以置信的是,很快,他们就听见了严厉的斥责声,李选侍的斥责声,她斥责的,是皇帝的长子。

		于是,一个空前绝后的场面出现了。

		大明帝国未来的继承人,被一个女人公然拉走,当众责骂,而皇帝,首辅、各部尚书,全部毫无反应,放任这一切的发生。

		所有的人静静地站在那里,听着那个女人的责骂,直到骂声结束为止。

		然后,尚未成年的朱由校走了出来,他带着极不情愿的表情,走到了父亲的身边,说出了这样一句话:

		“要封皇后!”

		谜团就此解开,莫名其妙的会议,东拉西扯的交谈,终于有了一个明确的答案——胁迫。

		开会是被胁迫的,闲扯是被胁迫的,一个奄奄一息的丈夫,一个年纪幼小的孩子,要不胁迫一把,实在有点说不过去。

		李选侍很有自信,因为她很清楚,这个软弱的丈夫不敢拒绝她的要求。

		现在,她距离自己的皇后宝座,只差一步。

		但是这一步,到死都没迈过去。

		因为就在皇长子刚说出那四个字的时候,另一个声音随即响起:

		“皇上要封皇贵妃,臣必定会尽快办理!”

		说这句话的人,是礼部尚书孙如游。

		李选侍太过天真了,和朝廷里这帮老油条比起来,她也就算个学龄前儿童。

		孙尚书可谓聪明绝顶,一看情形不对,知道皇上顶不住了,果断出手,只用了一句话,就把皇后变成皇贵妃。

		光宗同志也很机灵,马上连声回应:好,就这么办。

		李小姐的皇后梦想就此断送,但她是不会放弃的,因为她很清楚,在自己的手中,还有一张王牌——皇长子。

		只要那个奄奄一息的人彻底死去,一切都将尽在掌握。

		但她并不知道,此时,一双眼睛已经死死地盯住了她。

		杨涟已经确定,眼前这个飞扬跋扈的女人,不久之后,将是一个十分可怕的敌人。而在此之前,必须做好准备。

		八月二十九日。

		此前的三天里,光宗的身体丝毫不见好转,于是在这一天,他再次召见了首辅方从哲等朝廷重臣。

		光宗同志这次很清醒,一上来就直奔主题:

		寿木如何?寝地如何?

		寿木就是棺材,寝地就是坟,这就算是交代后事了。

		可是方从哲老先生不知是不是老了,有点犯糊涂,张口就是一大串,什么你爹的坟好、棺材好请你放心之类的话。

		光宗同志估计也是哭笑不得,只好拿手指着自己,说了一句:

		是我的\footnote{朕之寿宫。}。

		方首辅狼狈不堪,可还没等他缓过劲来,就听到了皇帝陛下的第二个问题:

		“听说有个鸿胪寺的医官进献金丹,他在何处?”

		对于这个问题,方从哲并未多想,便说出了自己的回答:

		“这个人叫李可灼,他说自己有仙丹,我们没敢轻信。”

		他实在应该多想想的。

		因为金丹不等于仙丹,轻信不等于不信。

		正是这个模棱两可的回答,导致了一个错误的判断:

		“好吧,召他进来。”

		于是,李可灼进入了大殿,他见到了皇帝,他为皇帝号脉,他为皇帝诊断,最后,他拿出了仙丹。

		仙丹的名字,叫做红丸。

		此时,是万历四十八年(1620年)八月二十九日上午,明光宗服下了红丸。

		他的感觉很好。

		按照史书上的说法,吃了红丸后,浑身舒畅,且促进消化,增加食欲\footnote{思进饮膳。}。

		消息传来,宫外焦急等待的大臣们十分高兴,欢呼雀跃。

		皇帝也很高兴,于是,几个时辰后,为巩固疗效,他再次服下了红丸。

		下午,劳苦功高的李可灼离开了皇宫,在宫外,他遇见了等待在那里的内阁首辅方从哲。

		方从哲对他说:

		“你的药很有效,赏银五十两。”

		李可灼高兴地走了,但他并没有领到这笔赏银。

		方从哲以及当天参与会议的人都留下了,他们住在了内阁,因为他们相信,明天,身体好转的皇帝将再次召见他们。

		六个时辰之后。

		凌晨,住在内阁的大臣们突然接到了太监传达的谕令:

		即刻入宫觐见。

		所有的人都明白,这意味着什么,但当他们尚未赶到的时候,就已得到了第二个消息——皇上驾崩了。

		万历四十八年(1620年)九月初一,明光宗在宫中逝世,享年三十九,享位一月。

		皇帝死了,这十分正常,皇帝吃药,这也很正常,但吃药之后就死了,这就不正常了。

		明宫三大案之“红丸案”,就此拉开序幕。

		没有人知道,所谓的红丸,到底是什么药,也没有人知道,在死亡的背后,到底隐藏着什么样的阴谋。

		此时向乾清宫赶去的人,包括内阁大臣、各部长官,共计十三人。在他们的心中,有着不同的想法和打算,因为皇帝死了,官位、利益、权力,一切的一切都将改变。

		只有一个人例外。

		杨涟十分悲痛,因为那个赏识他的人,已经死了,而且死得不明不白。此时此刻,他只有一个念头。

		查出案件的真相,找出幕后的黑手,揭露恶毒的阴谋,让正义得以实现,让死去的人得以瞑目。

		这就是杨涟的决心。

		但此时,杨涟即将面对的,却是一个更为复杂,更为棘手的问题。

		虽然大家都住在内阁,同时听到消息,毕竟年纪不同,体力不同,比如内阁的几位大人,方从哲老先生都七十多了,刘一璟、韩旷年纪也不小,反应慢点、到得晚点十分正常。

		所以首先到达乾清宫的,只有六部的部长、都察院左都御史,当然还有杨涟。

		这几个人已经知道了皇帝去世的消息,既然人死了,那就不用急了,就应该考虑尊重领导了,所以他们决定,等方首辅到来再进去。

		进不了宫,眼泪储备还不能用,而且大清早的,天都没亮,反正是等人,闲着也是闲着,于是,他们开始商讨善后事宜。

		继承皇位的,自然是皇长子朱由校了,但问题是,他的父亲死了,母亲也死了,而且年纪这么小,宫里没有人照顾,怎么办呢?

		于是,礼部尚书孙如游、吏部尚书周嘉谟、左都御史张问达提出:把朱由校交给李选侍。

		这个观点得到了绝大多数人的支持,事实上,反对者只有一个。

		然后,他们就听到了这个唯一反对者的声音:

		“万万不可!”

		其实就官职和资历而言,杨涟没有发言的资格,因为他此时他不过是个小小的七品给事中,说难听点,他压根就不该呆在这里。

		然而在场的所有人,都保持了沉默,静静地等待着他的发言,因为他是皇帝临死前指定的召见者,换句话说,他是顾命大臣。

		杨涟十分激动,他告诉所有的人,朱由校很幼稚,如果把他交给一个女人,特别是一个用心不良的女人,一旦被人胁迫,后果将不堪设想。

		这几句话,彻底唤起了在场朝廷重臣们的记忆,因为就在几天前,他们亲眼目睹了那个凶恶女人的狰狞面目。

		他们同意了杨涟的意见。

		但事实上,皇帝已经死了,未来的继承人,已在李选侍掌握之中。

		所以,杨涟说出了他的计划:

		“入宫之后,立刻寻找皇长子,找到之后,必须马上带出乾清宫,脱离李选侍的操纵,大事可成!”

		十三位顾命大臣终于到齐了,在杨涟的带领下,他们走向了乾清宫。

		一场你死我活的斗争即将开始。

		\subsection{战斗,从大门口开始}
		当十三位顾命大臣走到门口的时候,被拦住了。

		拦住他们的,是几个太监。毫无疑问,这是李选侍的安排。

		皇帝去世的时候,她就在宫内,作为一位智商高于郑贵妃的女性,她的直觉告诉她,即将到来的那些顾命大臣,将彻底毁灭她的野心。

		于是她决定,阻止他们入宫。

		应该说,这个策略是成功的,太监把住大门,好说歹说就不让进,一帮老头加书呆子,不懂什么枪杆子里出政权的深刻道理,只能干瞪眼。

		幸好,里面还有一个敢玩命的:

		“皇上已经驾崩,我们都是顾命大臣,奉命而来!你们是什么东西!竟敢阻拦!且皇长子即将继位,现情况不明,你们关闭宫门,到底想干什么?!”

		对付流氓加文盲,与其靠口,不如靠吼。

		在杨涟的怒吼之下,吃硬不吃软的太监闪开了,顾命大臣们终于见到了已经歇气的皇上。

		接下来是例行程序,猛哭猛磕头,哭完磕完,开始办正事。

		大学士刘一璟首先发问:

		“皇长子呢?他人在哪里?”

		没人理他。

		“快点交出来!”

		还是没人理他。

		李选侍清醒地意识到,她手中最重要的棋子,就是皇长子,只要控制住这个未来的继承人,她的一切愿望和野心,都将得到满足。

		这一招很绝,绝到杨涟都没办法,宫里这么大,怎么去找,一帮五六十岁的老头,哪有力气玩捉迷藏?

		杨涟焦急万分,毕竟这不是家里,找不着就打地铺,明天接着找,如果今天没戏,明天李选侍一道圣旨下来,是死是活都不知道!

		必须找到,现在,马上,必须!

		在这最为关键的时刻,一个太监走了过来,在大学士刘一璟的耳边,低声说出了两个字:

		“暖阁。”

		这个太监的名字,叫做王安。

		王安,河北雄县人,四十多年前,他进入皇宫,那时,他的上司叫冯保。

		二十六年前,他得到了新的任命,到一个谁也不愿意去的地方,陪一个谁也不愿意陪的人,这个人就是没人待见,连名分都没有的皇长子朱常洛。

		王安是个好人,至少是个识货的人,当朱常洛地位岌岌可危的时候,他坚定且始终站在了原地,无论是“争国本”,还是“梃击”都竭尽全力,证明了他的忠诚。

		朱常洛成为明光宗之后,他成为了司礼秉笔太监,掌控宫中大权。

		这位仁兄最喜欢的人,是东林党,因为一直以来,东林党都是皇帝陛下的朋友。

		而他最不喜欢的人,就是李选侍,因为这个女人经常欺负后宫的一位王才人,而这位王才人,恰好就是皇长子朱由校的母亲。

		此刻还不下烂药,更待何时?

		刘一璟大怒,大吼一声:

		“谁敢藏匿天子!”

		可是吼完了,就没辙了,因为这毕竟是宫里,人躲在里面,你总不能破门而入去抢人吧。

		所以最好的方法,是让李选侍心甘情愿地交人,然后送到门口,挥手致意。

		这似乎绝不可能,但是王安说,这是可能的。随后,他进入了暖阁。

		面对李选侍,王安体现出了一个卓越太监的素质,他虽没有抢人的体力,却有骗人的智力。

		他对李选侍说,现在情况特殊,必须让皇长子出面,安排先皇的丧事,安抚大家的情绪,事情一完,人就能回来。

		其实这谎扯得不圆,可是糊弄李选侍是够了。

		她立即叫出了朱由校。

		然而,就在她把人交给王安的那一瞬间,却突然醒悟了过来!她随即拉住了朱由校的衣服,死死拉住,不肯松手。

		王安知道,动粗的时候到了,他决定欺负眼前这个耍赖的女人。因为太监虽说不男不女,可论力气,比李小姐还是要大一些。

		王安一把拉过朱由校,抱起就走,冲出了暖阁。当门外的顾命大臣们看见皇长子的那一刻,他们知道,自己胜利了。

		于是,在先皇的尸体\footnote{估计还热着。}旁,新任皇帝接受了顾命大臣们的齐声问候:万岁!

		万岁喊完了,就该跑了。

		在人家的地盘上,抢了人家的人,再不跑就真是傻子了。

		具体逃跑方法是,王安开路,刘一璟拉住朱由校的左手,英国公张维贤拉住朱由校的右手,包括方从哲在内的几个老头走中间,杨涟断后。就这样,朱由校被这群活像绑匪\footnote{实际上也是。}的朝廷大臣带了出去。

		事情正如所料,当他们刚刚走出乾清宫的时候,背后便传来了李选侍尖利的叫喊声:

		“哥儿\footnote{指朱由校。},回来!”

		李大姐这嗓子太突然了,虽然没要人命,却把顾命大臣们吓了一跳,他们本来在乾清宫外准备了轿子,正在等轿夫来把皇子抬走,听到声音后,脚一跺,不能再等了!

		不等,就只能自己抬,情急之下,几位高干一拥而上,去抬轿子。

		这四位高级轿夫分别是吏部尚书周嘉谟,给事中杨涟,内阁大学士刘一璟,英国公张维迎。

		前面几位大家都熟,而最后这位张维迎,是最高世袭公爵,他的祖先,就是跟随明成祖朱棣靖难中阵亡的第一名将张玉。

		也就是说,四个人里除杨涟外,职务最低的是部长,我又查了下年龄,最年轻的杨涟,当时也已经四十八岁了,看来人急眼了,还真敢拼命。

		就这样,朱由校在这帮老干部的簇拥下,离开了乾清宫,他们的目标,是文华殿,只要到达那里,完成大礼,朱由校就将成为新一代的皇帝。

		而那时,李选侍的野心将彻底破灭。

		当然,按照最俗套的电视剧逻辑,坏人们是不会甘心失败的,真实的历史也是如此。

		毕竟老胳膊老腿,走不快,很快,大臣们就发现,他们被人追上了。

		追赶他们的,是李选侍的太监。一个带头的二话不说,恶狠狠地拦住大臣,高声训斥:

		“你们打算把皇长子带到哪里去?”

		一边说,还一边动手去拉朱由校,很有点动手的意思。

		对于这帮大臣而言,搞阴谋、骂骂人是长项,打架是弱项。于是,杨涟先生再次出场了。

		他大骂了这个太监,并且鼓动朱由校:

		“天下人都是你的臣子,何须害怕!”

		一顿连骂带捧,把太监们都镇住了,领头的人见势不妙,就撤了。

		这个被杨涟骂走的领头太监,名叫李进忠,是个不出名的人。但不久之后,他将更名改姓,改为另一个更有名的名字——魏忠贤。

		在杨涟的护卫下,朱由校终于来到了文华殿,在这里,他接受了群臣的朝拜,成为了新的皇帝,史称明熹宗。

		\subsection{明熹宗朱由校}
		这就算即位了,但问题在于,毕竟也是大明王朝,不是杂货铺,程序还要走,登基还得登。

		有人建议,咱就今天办了得了,可是杨涟同志不同意,这位仁兄认定,既然要登基,就得找个良辰吉日,一查,那就九月初六吧。

		这是一个极为错误的决定。

		今天是九月初一,只要皇长子没登基,乾清宫依然是李选侍的天下,而且,她依然是受命照顾皇长子的人,对于她而言,要翻盘,六天足够了。

		然而杨涟本人,却没有意识到这一点。

		就在他即将步入深渊的时候,一个人拉住了他,并且把一口唾沫吐在了他的脸上。

		这个人的名字,叫做左光斗。

		左光斗,字遗直,安徽桐城人。万历三十五年进士。现任都察院巡城御史,杨涟最忠实的战友,东林党最勇猛的战士。

		虽然他的职位很低,但他的见识很高,刚一出门,他就揪住了杨涟,对着他的脸,吐了口唾沫:

		“到初六登基,今天才初一,如果有何变故,怎么收拾,怎么对得起先皇?!”

		杨涟醒了,他终于明白,自己犯下了一个不可饶恕的错误。

		皇长子还在宫内,一旦李选侍掌握他,号令群臣,到时必定死无葬身之地!

		但事已至此,只能明天再说,毕竟天色已晚,皇宫不是招待所,杨大人不能留宿,无论如何,必须等到明天。

		杨涟走了,李选侍的机会来了。

		当天傍晚,朱由校再次来到乾清宫,他不能不来,因为他父亲的尸体还在这里。

		可是他刚踏入乾清宫,就被李选侍扣住了,尸体没带走,还搭进去一个活人。

		眼看顾命大臣们就要完蛋,王安又出马了。

		这位太监可谓是智慧与狡诈的化身,当即挺身而出,去和李选侍交涉,按说被人抢过一次,总该长点记性,可是王安先生几番忽悠下来,李选侍竟然又交出了朱由校。

		这是个很难理解的事,要么是李小姐太弱智,要么是王太监太聪明,无论如何最终的结果是,李选侍失去了一个机会,最后的机会。

		因为第二天,杨涟将发起最为猛烈的进攻。

		九月初二。

		吏部尚书周嘉谟和御史左光斗同时上书,要求李选侍搬出乾清宫。

		这是一个十分聪明的战略,因为乾清宫是皇帝的寝宫,只要李选侍搬出去,她将无法制约皇帝,失去所有政治能量。

		但要赶走李选侍,自己动手是不行的,毕竟这人还是后妃,拉拉扯扯成何体统?

		经过商议,杨涟等人统一意见:让她自己走。

		左光斗主动承担了这个艰巨的任务,为了彻底赶走这个女人,他连夜写出了一封奏疏,一封堪称恶毒无比的奏疏。

		文章大意是说,李小姐你不是皇后,也没人选你当皇后,所以你不能住乾清宫,而且这里也不需要你。

		然后他进一步指出,朱由校才满十六岁,属于青春期少年,容易冲动,和你住在一起是不太合适的。

		话说到这里,已经比较露骨了。

		别慌,更露骨的还在后面。

		在文章的最后,左光斗写出了一句画龙点睛的话:

		“武氏之祸,再现于今,将来有不忍言者!”

		所谓武氏,就是武则天,也就是说,左光斗先生担心,如此下去,武则天夺位的情形就会重演。

		如果你认为这是一句非常过分的话,那你就错了,事实上,是非常非常过分,因为左光斗是读书人,有时候,读书人比流氓还流氓。

		希望你还记得,武则天原先是唐太宗的妃子,高宗是太宗的儿子,后来,她又成了唐高宗的妃子。

		现在,李选侍是明光宗的妃子,熹宗是光宗的儿子,后来……

		所以左光斗先生的意思是,李选侍之所以住在乾清宫,是想趁机勾引她的儿子\footnote{名义上的。}。

		李选侍急了,这很正常,你看你也急,问题在于,你能咋办?

		李选侍想出的主意,是叫左光斗来谈话。事实证明,这是个不折不扣的馊主意,因为左光斗的回答是这样的:

		“我是御史,天子召见我才会去,你算是个什么东西\footnote{若辈何为者。}?”

		九月初三。

		左光斗的奏疏终于送到了皇帝的手中,可是皇帝的反应并不大,原因简单:他看不懂。

		拜他父亲所赐,几十年来躲躲藏藏,提心吊胆,儿子的教育是一点没管,所以朱由校小朋友不怎么读书,却很喜欢做木匠,常年钻研木工技巧。

		幸好,他的身边还有王安。

		王太监不负众望,添油加醋解说一番,略去儿童不宜的部分,最后得出结论:李选侍必须滚蛋。

		朱由校决定,让她滚。

		很快,李选侍得知了这个决定,她决定反击。

		九月初四。

		李选侍反击的具体形式,是谈判。

		她派出了一个使者,去找杨涟,希望这位钢铁战士会突然精神失常,放弃即将到手的胜利,相信她是一个善良、无私的女人,并且慷慨大度的表示,你可以继续住在乾清宫,继续干涉朝政。

		人不能愚蠢到这个程度。

		但她可以。

		而她派出的那位使者,就是现在的李进忠,将来的魏忠贤。

		这是两位不共戴天的死敌第一次正面交锋。

		当然,当时的杨涟并没有把这位太监放在眼里,见面二话不说:

		“她\footnote{指李选侍。}何时移宫?”

		李进忠十分客气:

		“李选侍是先皇指定的养母,住在乾清宫,其实并没有什么问题。”

		杨涟很不客气:

		“你给我记好了,回去告诉李选侍,现在皇帝已经即位,让她立刻搬出来,如果乖乖听话,她的封号还能给她,如果冥顽不灵,就等皇帝发落吧!”

		最后还捎带一句:

		“你也如此!”

		李进忠沉默地走了,他很清楚,现在自己还不是对手,在机会到来之前,必须等待。

		李选侍绝望了,但她并不甘心,在最后失败之前,她决心最后一搏,于是她去找了另一个人。

		九月初五,登基前最后一日。

		按照程序规定,明天是皇帝正式登基的日期,但是李选侍却死不肯搬,摆明了要耍赖,于是,杨涟去找了首辅方从哲,希望他能号召群臣,逼李选侍走人。

		然而,方从哲的态度让他大吃一惊,这位之前表现积极的老头突然改了口风:

		“让她迟点搬,也没事吧\footnote{迟亦无害。}。”

		杨涟愤怒了:

		“明天是皇上登基的日子,难道要让他躲在东宫,把皇宫让给那个女人吗?!”

		方从哲保持沉默。

		李选侍终于聪明了一次,不能争取杨涟,就争取别人,比如说方从哲。

		因为孤独的杨涟,是无能为力的。

		但她错了,孤独的杨涟依然是强大的,因为在他的心中,始终都留存着一个信念:

		当我只是个小人物的时候,你体谅我的激奋,接受我的意见,相信我的才能,将你的身后之事托付于我。

		所以,我会竭尽全力,战斗至最后一息,绝不放弃。

		因为你的信任,和尊重。

		在这最后的一天里,杨涟不停地到内阁以及各部游说,告诉大家形势危急,必须立刻挺身而出,整整一天,即使遭遇冷眼,被人讥讽,他依然不断地说着,不断地说着。

		最终,许多人被他打动,并在他的率领下,来到了宫门前。

		面对着阴森的皇宫,杨涟喊出了执着而响亮的宣言:

		“今日,除非你杀掉我,若不移宫,宁死不离\footnote{死不去。}!”

		由始至终,李选侍都是一个极为贪婪的女人,为达到目的,可以不择手段,不顾一切,虐待朱由校的母亲,逼迫皇帝,责骂皇长子,只为她的野心和欲望。

		但现在,她退缩了,她决定放弃。因为她已然发现,这个叫杨涟的人,是很勇敢的,敢于玉石俱焚、敢于同归于尽。

		无奈地叹息之后,她退出了乾清宫,从此,她消失了,消失得无影无踪,她或许依然专横、撒泼,却已无人知晓,因为,她已无关紧要。

		随同她退出的,还有她的贴身太监们,时移势易,混口饭吃也不容易。

		然而一位太监留了下来,他知道,自己的命运还未终结,因为他已经发现了一个新的目标——另一个女人。

		从这个女人的身上,他将得到新的前途,以及新的名字。
		\ifnum\theparacolNo=2
	\end{multicols}
\fi
\newpage

\chapter*{朱由校篇}
\addcontentsline{toc}{chapter}{朱由校篇}
\section{强大,无比强大}
\ifnum\theparacolNo=2
	\begin{multicols}{\theparacolNo}
		\fi
		万历四十八年(1620年)九月初六,明熹宗朱由校在乾清宫正式登基,定年号为天启。

		一个复杂无比,却又精彩绝伦的时代就此开始。

		杨涟终于完成了他的使命,自万历四十八年(1620年)八月二十二日起,在短短十五天之内,他无数次绝望,又无数次奋起,召见、红丸、闯宫、抢人、拉拢、死磕,什么恶人、坏人都遇上了,什么阴招、狠招都用上了。

		最终,他成功了。

		据史料记载,在短短十余天里,他的头发已变成一片花白。

		当天启皇帝朱由校坐在皇位上,看着这个为他的顺利即位费尽心血的人时,他知道,自己应该回报。

		几日后,杨涟升任兵科都给事中,一年后,任太常少卿,同年,升任都察院佥都御史,后任左副都御史。短短一年内,他从一个从七品的芝麻官,变成了从二品的部级官员。

		当然,得到回报的,不仅是他。

		东林党人赵南星,退休二十多年后,再度复出,任吏部尚书。

		东林党人高攀龙,任光禄丞。后升任光禄少卿。

		东林党人邹元标,任大理寺卿,后任刑部右侍郎,都察院左都御史。

		东林党人孙慎行,升任礼部尚书。

		东林党人左光斗,升任大理寺少卿,一年后,升任都察院左佥都御史。

		以下还有若干官,若干人,篇幅过长,特此省略。

		小时候,老师告诉我,个人是渺小的,集体才是伟大的,现在,我相信了。

		当皇帝的当皇帝,升官的升官,滚蛋的滚蛋,而那个曾经统治天下的人,却似乎已被彻底遗忘。

		明光宗朱常洛,作为明代一位极具特点\footnote{短命。}的皇帝,他的人生可以用四个字来形容——苦大仇深。

		出生就不受人待见,母亲被冷遇,长大了,书读不上,太子立不了,基本算三不管,吃穿住行级别很低,低到连刺杀他的人,都只是个普通农民,拿着根木棍,就敢往宫里闯。

		好不容易熬到登基,还要被老婆胁迫,忍了几十年,放纵了一回,身体搞垮了,看医生,遇见了蒙古大夫,想治病,就去吃仙丹,结果真成仙了。

		更搞笑的是,许多历史书籍到他这里,大都只讲三大案,郑贵妃、李选侍,基本上没他什么事,原因很简单,他只当了一个月皇帝。

		在他死后,为了他的年号问题,大臣们展开了争论,因为万历四十八年七月,万历死了,八月,他就死了。而他的年号泰昌,没来得及用。

		问题来了,如果把万历四十八年(1620年)当作泰昌元年,那是不行的,因为直到七月,他爹都还活着。

		如果把第二年(1621年)当作泰昌元年,那也是不行的,因为他去年八月,就已经死了。

		这是一个无法解决的问题。

		问题终究被解决了,凭借大臣们无比高超的和稀泥技巧,一个前无古人、后无来者的处理方案隆重出场:

		万历四十八年(1620年)一月到七月,为万历四十八年。八月,为泰昌元年。明年(1621年),为天启元年。

		这就是说,在这一年里,前七个月是他爹的,第二年是他儿子的,而他的年份,只有一个月。

		原因很简单,他只当了一个月皇帝。

		他很可怜,几十年来畏畏缩缩,活着没有待遇,死了没有年号,事实上,他人才刚死,就有一堆人在他尸体旁边你死我活,抢儿子抢地方,忙得不亦乐乎。

		原因很简单,他只当了一个月皇帝。

		有人曾对我说,原来,历史很有趣。但我对他说,其实,历史很无趣。

		因为在绝大多数情况下,历史没有正恶,只有成败。

		左都御史、左副都御史、吏部尚书、刑部侍郎、大理寺丞等等等等,政权落入了东林党的手中。

		它很强大,强大到无以复加的地步。对于这一现象,史称“众正盈朝”。

		按照某些史书的传统解释,从此,在东林党人的管理下,朝廷进入了一个公正、无私的阶段,许多贪婪的坏人被赶走,许多善良的好人留下来。

		对于这种说法,用两个字来评价,就是胡说。

		用四个字来评价,就是胡说八道。

		之前我曾经说过,东林党不是善男信女,现在,我再说一遍。

		掌权之后,这帮兄弟干的第一件事,就是追查红丸案。

		追查,是应该的,毕竟皇帝死得蹊跷,即使里面没有什么猫腻,但两位蒙古大夫,一个下了泻药,让他拉了几十次,另一个送仙丹,让他飞了天,无论如何,也应该追究责任。

		退一万步讲,就算你追究责任后还不过瘾,非要搞几个幕后黑手出来,郑贵妃、李选侍这几位重点嫌疑犯,名声坏,又歇了菜,要打要杀,基本都没个跑。

		可是现成的偏不找,找来找去,找了个老头——方从哲。

		天启元年(1621年),礼部尚书孙慎行上疏,攻击方从哲。大致意思是说,方从哲和郑贵妃有勾结,而且他还曾经赏赐过李可灼,出事后,只把李可灼赶回了家,没有干掉,罪大恶极,应予严肃处理。

		这就真是有点无聊恶搞了,之前说过,李可灼最初献药,还是方老头赶回去的,后来赏钱那是皇帝同意的,所谓红丸到底是什么玩意,鬼才知道,稀里糊涂把人干掉,也不好。

		所以无论从哪个角度看,方从哲都没错,而且此时东林党掌权,方老头识时务,也不打算呆了,准备回家养老去了。

		可孙部长用自己的语言,完美地解释了强词夺理这个词的含义:

		“从哲\footnote{方从哲。}纵无弑之心,却有弑之罪,纵辞弑之名,难免弑之实。”

		这意思是,你老兄即使没有干掉皇帝的心思,也有干掉皇帝的罪过,即使你退休走人,也躲不过去这事。

		强词夺理还不算,还要赶尽杀绝:

		“陛下宜急讨此贼,雪不共之仇!”

		所谓此贼,不是李可灼,而是内阁首辅,他的顶头上司方从哲。

		很明显,他很激动。

		孙部长激动之后,都察院左都御史邹元标也激动了,跟着上书过了把瘾,不搞定方从哲,誓不罢休。

		这是一件十分奇怪的事。

		七十多岁的老头,都快走人了,为什么就是揪着不放呢?

		因为他们有着一个不可告人的目的。

		郑贵妃不重要,李选侍不重要,甚至案件本身也不重要。之所以选中方从哲,把整人进行到底,真正的原因在于,他是浙党。

		只要打倒了方从哲,借追查案件,就能解决一大批人,将政权牢牢地抓在手中。

		他们的目的达到了,不久之后,崔文升被发配南京,李可灼被判流放,而方从哲,也永远地离开了朝廷。

		明宫三大案就此结束,东林党大获全胜。

		局势越来越有利,天启元年(1621年)十月,另一个重量级人物回来了。

		这个人就是叶向高。

		东林党之中,最勇猛的,是杨涟,最聪明的,就是这位仁兄了。而他担任的职务,是内阁首辅。

		作为名闻天下的老滑头,他的到来,标志着东林党进入了全盛时期。

		内忧已除,现在,必须解决外患。

		因为他们还没来得及庆祝,就得知了这样一个消息——沈阳失陷。

		沈阳是在熊廷弼走后,才失陷的。

		熊廷弼驻守辽东以来,努尔哈赤十分消停,因为这位熊大人做人很粗,做事很细,防守滴水不漏,在他的管理下,努尔哈赤成了游击队长,只能时不时去抢个劫,大事一件没干成。

		出于对熊廷弼的畏惧和愤怒,努尔哈赤给他取了个外号:熊蛮子。

		这是一个名副其实的外号,不但对敌人蛮,对自己人也蛮。

		熊大人的个性前面说过了,彪悍异常,且一向不肯吃亏,擅长骂人,骂完努尔哈赤,还不过瘾,一来二去,连兵部领导、朝廷言官也骂了。

		这就不太好了,毕竟他还归兵部管,言官更不用说,平时只有骂人,没有被人骂,索性敞开了对骂,闹到最后,熊大人只好走人。

		接替熊廷弼的,是袁应泰。

		在历史中,袁应泰是个评价很高的人物,为官廉洁,为人清正,为政精明,只有一个缺点,不会打仗。

		这就没戏了。

		他到任后,觉得熊廷弼很严厉,很不近人情,城外有那么多饥民\footnote{主要是蒙古人。},为什么不放进来呢?就算不能打仗,站在城楼上充数也不错嘛。

		于是他打开城门,放人入城,亲自招降。

		一个月后,努尔哈赤率兵进攻,沈阳守将贺世贤拼死抵抗,关键时刻,之前招安的蒙古饥民开始大肆破坏,攻击守军,里应外合之下,沈阳陷落。贺世贤战死,七万守军全军覆没。

		这一天,是天启元年(1621年)三月十二日。

		袁应泰没有时间后悔,因为他只多活了六天。

		攻陷沈阳后,后金军队立刻整队,赶往下一个目标——辽阳。

		当年,辽阳的地位,大致相当于今天的沈阳,是辽东地区的经济、文化、军事中心,也是辽东的首府。此地历经整修,壕沟围绕,防守严密,还有许多火炮,堪称辽东第一坚城。

		守了三天。

		战斗经过比较简单,袁应泰率三万军队出战,被努尔哈赤的六万骑兵击败,退回坚守,城内后金奸细放火破坏,大乱,后金军乘虚而入,辽阳陷落。

		袁应泰看见了城池的陷落,他非常镇定,从容穿好官服,佩带着宝剑,面向南方,自缢而死。

		他不是一个称职的大明将领,却是一个称职的大明官员。

		辽阳的丢失,标志着局势的彻底崩溃,标志着辽东成为了后金的势力范围,标志着从此,他们想去哪里,就去哪里,想抢哪里,就抢哪里。

		局势已经坏得不能再坏了,所以,不能用的人,也不能不用了。

		天启元年(1621年)七月,熊廷弼前往辽东。

		在辽东,他遇见了王化贞。

		他不喜欢这个人,从第一次见面开始。因为他发现,这人不买他的帐。

		熊廷弼此时的职务是辽东经略,而王化贞是辽东巡抚。从级别上看,熊廷弼是王化贞的上级。
		\begin{quote}
			\begin{spacing}{0.5}  %行間距倍率
				\textit{{\footnotesize
							\begin{description}
								\item[\textcolor{Gray}{\faQuoteRight}] 角色并不重要,关键在于会不会抢戏。——小品演员陈佩斯
							\end{description}
						}}
			\end{spacing}
		\end{quote}

		王化贞就是一个很会抢戏的人,因为他有后台,所以他不愿意听话。

		关于这两个人的背景,有些历史书上的介绍大概如此:熊廷弼是东林党支持的,王化贞是阉党支持的。最终结局也再次证明,东林党是多么地明智,阉党是多么地愚蠢。

		胡扯。

		不是胡扯,就是装糊涂。

		因为最原始的史料告诉我们,熊廷弼是湖广人,他是楚党的成员,而在大多数时间里,楚党是东林党的敌人。

		至于王化贞,你说他跟阉党有关,倒也没错,可是他还有个老师,叫做叶向高。

		天启元年的时候,阉党都靠边站,李进忠还在装孙子,连名字都没改,要靠这帮人,王化贞早被熊先生赶去看城门了。

		他之所以敢嚣张,敢不听话,只是因为他的老师,是朝廷首辅,朝中的第一号人物。

		熊廷弼是对的,所以他是东林党,或至少是东林党支持的,王化贞是错的,所以他是阉党,或至少是阉党赏识的。大致如此。

		我并非不能理解好事都归自己,坏事都归别人的逻辑,也并不反对,对某些坏人一棍子打死再踩上一只脚的行为,我只是认为,做人,还是要厚道。

		王化贞不听熊廷弼的话,很正常,因为他的兵,比熊廷弼的多。

		当时明朝在辽东的剩余部队,大约有十五万,全都在王化贞的手中。而熊廷弼属下,只有五千人。

		所以每次王化贞见熊廷弼时,压根就不听指挥,说一句顶一句,气得熊大人恨不能拿刀剁了他。

		但事实上,王化贞是个很有能力的人。

		王化贞,山东诸城人。万历四十一年进士。原先是财政部的一名处级干部\footnote{主事。},后来不知怎么回事,竟然被调到了辽东广宁\footnote{今辽宁北宁。}。

		此人极具才能,当年蒙古人闹得再凶,到他的地头,都不敢乱来。后来辽阳、沈阳失陷,人心一片慌乱,大家都往关内跑,他偏不跑。

		辽阳城里几万守军,城都丢了,广宁城内,只有几千人,还是个破城,他偏要守。

		他非但不跑,还招集逃兵,整顿训练,居然搞出了上万人的队伍,此外,他多方联络,稳定人心,坚守孤城,稳定了局势。所谓“提弱卒,守孤城,气不慑,时望赫然”,天下闻名,那也真是相当的牛。

		熊廷弼也是牛人,但对于这位同族,他却十分不感冒,不仅因为牛人相轻,更重要的是,此牛非彼牛也。

		很快,熊大人就发现,这位王巡抚跟自己,压根不是一个思路。

		按他自己想法,应该修筑堡垒,严防死守,同时调集援兵,长期驻守。

		可是王化贞却认定,应该主动进攻,去消灭努尔哈赤,他还说,只要有六万精兵,他就可以一举荡平。

		熊廷弼觉得王化贞太疯,王化贞觉得熊廷弼太熊。

		最后王化贞闭口了,他停止了争论,因为争论没有意义。

		兵权在我手上,我想干嘛就干嘛,和你讨论,是给你个面子,你还当真了?

		一切都按照王化贞的计划进行着,准备粮草,操练士兵,寻找内应,调集外援,忙得不亦乐乎。

		忙活到一半,努尔哈赤来了。

		天启二年(1622年)正月十八日,努尔哈赤亲率大军,进攻广宁。

		之前半年,努尔哈赤听说熊廷弼来了,所以他不来。后来他听说,熊廷弼压根没有实权,所以他来了。

		实践证明,王巡抚胆子很大,脑子却很小,面对努尔哈赤的进攻,他摆出了一个十分奇怪的阵型,先在三岔河布阵,作为第一道防线,然后在西平堡设置第二道防线,其余兵力退至广宁城。

		就兵力而言,王化贞大概是努尔哈赤的两倍,可大敌当前,他似乎不打算“一举荡平”,也不打算御敌于国门之外,因为外围两道防线的总兵力也才三万人,是不可能挡住努尔哈赤的。

		用最阴暗的心理去揣摸,这个阵型的唯一好处,是让外围防线的三万人和努尔哈赤死拼,拼完,努尔哈赤也就差不多了。

		事实确实如此,正月二十日,努尔哈赤率军进攻第一道防线三岔河,当天即破。

		第二天,他来到了第二道防线西平堡,发动猛烈攻击,但这一次,他没有如愿。

		因为西平堡守将罗一贯,是个比较一贯的人,努尔哈赤进攻,打回去,汉奸李永芳劝降,骂回去,整整一天,后金军队毫无进展。

		王化贞的反应还算快,他立即派出总兵刘渠、祁秉忠以及他的心腹爱将孙得功,分率三路大军,增援西平堡。

		努尔哈赤最擅长的,就是围点打援。所以明军的救援,早在他意料之中。

		但在他意料之外的,是明军的战斗力。

		总兵刘渠、祁秉忠率军出战,两位司令十分勇猛,亲自上阵,竟然打得后金军队连连败退,于是,作为预备队的孙得功上阵了。

		按照原先的想法,孙得功上来,是为了加强力量,可没想到的是,这位兄弟刚上阵,却当即溃败,惊慌之余,孙大将还高声喊了一嗓子:

		“兵败了!兵败了!”

		您都兵败了,那还打什么?

		后金军随即大举攻击,明军大败,刘渠阵亡,祁秉忠负伤而死,孙得功逃走,所属数万明军全军覆没。

		现在,在努尔哈赤面前的,是无助、毫无遮挡的西平堡。

		罗一贯很清楚,他的城池已被团团包围,不会再有援兵,不会再有希望,对于胜利,他已无能为力。

		但他仍然决定坚守,因为他认为,自己有这个责任。

		正月二十二日,努尔哈赤集结所属五万人,发动总攻。

		罗一贯率三千守军,拼死守城抵抗。

		双方激战一天,后金军以近二十倍的兵力优势,发起了无数次进攻,却无数次败退,败退在孤独却坚定的罗一贯眼前。

		明军凭借城堡大量杀伤敌军,后金损失惨重,毫无进展,只得围住城池,停止进攻。

		但出乎他们意料的是,城头突然陷入了死一般的寂静,没有了呐喊,没有了杀声。

		因为城内的士兵,已经放出了最后一支弓箭,发射了最后一发火炮。

		在这最后的时刻,罗一贯站在城头,向着京城的方向,行叩拜礼,说出了他的遗言:

		“臣力竭矣!”

		然后,他自刎而死。

		这是努尔哈赤自起兵以来,损失空前惨重的一战,据史料记载,和西平堡三千守军一同阵亡的,有近七千名后金军。

		罗一贯尽到了自己的职责,王化贞也准备这样做。

		得知西平堡失陷后,他连夜督促加强防守,并对逃回来的孙得功既往不咎,鼓励守城将士众志成城,击退后金军队。

		然后,他就去睡觉了。

		王化贞不是个怕事的人,当年辽阳失守,他无兵无将都敢坚守,现在手上有几万人,自然敢睡觉。

		但还没等他睡着,就听见了随从的大叫:

		“快跑!”

		王化贞跑出卧房。

		他看见无数百姓和士兵丢弃行李兵器,夺路而逃,原本安静祥和的广宁城,已是一片混乱,彻底的混乱。

		而此时的城外,并没有努尔哈赤,也没有后金军,一个都没有。

		这莫名其妙的一切,起源于两个月前的一个决定。

		王化贞不是白痴,他很清楚努尔哈赤的实力,在那次谈话中,他之所以告诉熊廷弼,说六万人一举荡平,是因为他已找到了努尔哈赤的弱点。

		这个弱点,叫做李永芳。

		李永芳是明朝叛将,算这一带的地头蛇,许多明军将领跟他都有交情,毕竟还是同胞兄弟,所以在王化贞看来,这是一个可以争取的人。

		于是,他派出了心腹孙得功,前往敌营,劝降李永芳。

		几天后,孙得功回报,李永芳深明大义,表示愿意归顺,在进攻时作为内应。

		王化贞十分高兴。

		两个月后,孙得功西平堡战败,惊慌之下,大喊“兵败”,导致兵败。

		是的,你的猜测很正确,孙得功是故意的,他是个叛徒。

		孙得功去劝降李永芳,却被李永芳劝降,原因很简单,不是什么忠诚、爱国、民族、大同之类的话,只是他出价更高。

		为了招降李永芳,努尔哈赤送了一个孙女,一个驸马\footnote{额驸。}的头衔,还有无数金银财宝,很明显,王化贞出不起这个价。

		努尔哈赤从来不做赔本买卖,他得到了极为丰厚的回报。

		孙得功帮他搞垮了明朝的援军,但这还不够,这位誓把无耻进行到底的败类,决定送一份更大的礼物给努尔哈赤——广宁城。

		因为自信的王化贞,将城池的防守任务交给了他。

		接下来的事顺理成章,从被窝里爬起来的王大人慌不择路,派人去找马,准备逃走,可是没想到,孙心腹实在太抠门,连马都弄走了,搞得王大人只找到了几头骆驼,最后,他只能骑着骆驼跑路。

		还好,那天晚上,孙心腹忙着带领叛军捣乱,没顾上逃跑的王巡抚,否则以他的觉悟,拿王大人的脑袋去找努尔哈赤换个孙女,也是不奇怪的。

		第二天,失意的王巡抚在逃走的路上,遇到了一个让他更为失意的人。

		熊廷弼用实际行动证明,他不是一个慈悲的人,至少不会放过落水狗。

		当王巡抚痛哭流涕,反复检讨错误时,他用一句话表示了他的同情:

		“六万大军一举荡平?现在如何?”

		王化贞倒还算认账,关键时刻,也不跟熊廷弼吵,只是提出,现在应派兵,坚守下一道防线——宁远。

		这是一个十分明智的判断,可是熊大人得理不饶人,还没完了:

		“现在这个时候,谁肯帮你守城?晚了!赶紧掩护百姓和士兵入关,就足够了!”

		这句话的潜台词是,当初不听我的,现在我也不听你的。

		事情到这份上,就没什么可说的了,作为丧家犬,王化贞没有发言权。

		于是,战局离开了王化贞的掌控,走上了熊廷弼的轨道。

		从王化贞到熊廷弼,从掌控到轨道,这是一个有趣的变化。

		变化的前后有很多不同点,也有一个共同点:都是错误的。

		虽然敌情十分紧急,城池空虚,但此时明军主力尚存,若坚定守住,估计也没什么问题。可是熊先生来了牛脾气,不由分说,宁远也不守了,把辽东的几十万军民全部撤回关\footnote{山海关。}内,放弃了所有据点。

		熊大人没有意识到,他已经做到了无数敌人、无数汉奸、无数叛徒想做却做不到的事情,因为事实上,他已放弃整个辽东。

		自明朝开国以来,稳固统治两百余年的辽东,就这么丢了。无论从哪个角度看,熊廷弼都没有理由、没有借口、没有道理这样做。

		但是他做了。

		我认为,他是为了一口气。

		当初不听我的话,现在看你怎么办?

		就是这口气,最后要了他的命。

		率领几十万军民,成功撤退的两位仁兄终于回京了,明朝政府对他们俩的处理,是相当一视同仁的——撤职查办。

		无论谁对谁错,你们把朝廷在辽东的本钱丢得精光,还有脸回来?这个黑锅你们不背,谁背?

		当然,最后处理结果还是略有不同,熊大人因为脾气不好,得罪人多,三年后(天启五年)就被干掉了。

		相对而言,王大人由于关系硬,人缘好,又多活了七年,崇祯五年才正式注销户口。

		对于此事,许多史书都说,王化贞死得该,熊廷弼死得冤。

		前者我同意,后者,我保留意见。

		事实上,直到王化贞逃走后的第三天,努尔哈赤才向广宁进发,他没有想到,明军竟然真的不战而逃,而且以他的兵力,并不足以占据辽东。

		然而当他到达广宁,接受孙得功投降之时,才发现,整个辽东,已经没有敌人。

		因为慷慨的熊蛮子,已把这片广阔的土地毫无保留地交给了他。

		白给的东西不能不要,于是在大肆抢掠之后,他率军向新的目标前进——山海关。

		可是走到半路,他发现自己的算盘打错了。

		因为熊蛮子交给他的,不是辽东,而是一个空白的辽东。

		为保证不让敌人抢走一粒粮,熊先生干得相当彻底,房子烧掉,水井埋掉,百姓撤走,基本上保证了千里无鸡鸣,万里无人烟。

		要这么玩,努尔哈赤先生就不干了,他辛苦奔波,最终的目的是为了抢东西,您把东西都搬走了,我还去干嘛?

		而且从广宁到山海关,几百里路空无一人,很多坚固的据点都无人看守,别说抢劫,连打仗的机会都没有。

		于是,当军队行进到一个明军据点附近时,努尔哈赤决定:无论这些地方有多广袤,无论这些据点有多重要,都不要了,撤退。

		努尔哈赤离开了这里,踏上了归途,但他不会想到,自己已经犯下了一个致命的错误。

		因为四年之后,他将再次回到这里,并为争夺这个他曾轻易放弃的小地方,失去所有的一切。

		这个他半途折返的地点,叫做宁远。

		\subsection{堪与匹敌者,此人也}
		自万历四十六年,努尔哈赤起兵以来,短短三年时间,抚顺、铁岭、开原、辽阳、沈阳,直至整个辽东,全部陷落。

		从杨镐、刘綎到袁应泰、王化贞、熊廷弼,不能打的完了,能打的也完了,熊人死了,牛人也死了。

		辽东的局势,说差,那是不恰当的,应该说,是差得不能再差,差到官位摆在眼前,都没人要。

		比如总兵,是明军的高级将领,全国不过二十人左右,用今天话说,是军区司令员。要想混到这个职务,不挤破头是不大可能的。

		一般说来,这个职务相当安全,平日也就是看看地图,指手划脚而已。然而这几年情况不同了,辽东打仗,明朝陆续派去了十四位总兵,竟然全部阵亡,无一幸免。

		总兵越来越少,而且还在不断减少,因为没人干,某些在任总兵甚至主动辞职,宁可回家种田,也不干这份工作。

		但公认最差的职业,还不是总兵,是辽东经略。

		总兵可以有几十个,辽东经略只有一个。总兵可以不干,辽东经略不能不干。

		可是连傻子都知道,辽东都没了,人都撤回山海关了,没兵没地没百姓,还经略个啥?

		大家不是傻子,大家都不去。

		接替辽东经略的第一人选,是兵部尚书张鹤鸣,天启为了给他鼓劲,先升他为太子太保\footnote{从一品。},又给他尚方宝剑,还亲自送行。

		张尚书没说的,屁股一拍,走了。

		走是走了,只是走得有点慢,从京城到山海关,他走了十七天。

		这条路线上星期我走过,坐车三个钟头。

		张大人虽说没车,马总是有的,就两百多公里,爬也爬过去了。

		这还不算,去了没多久,这位大人又说自己年老力衰,主动辞职回家了。

		没种就没种,装什么蒜?

		相比而言,接替他的宣府巡抚就好得多了。

		这位巡抚大人接到任命后,连上三道公文,明白跟皇帝讲:我不去。

		天启先生虽说是个木匠,也还有点脾气,马上下达谕令:不去,就滚\footnote{革职为民,永不叙用。}。

		不想去也好,不愿去也好,替死鬼总得有人当,于是,兵部侍郎王在晋出场了。

		王在晋,字明初,江苏太仓人。万历二十年进士。这位仁兄从没打过仗,之所以让他去,是因为他不能不去。

		张尚书跑路的时候,他是兵部副部长,代理部长\footnote{署部事。},换句话说,轮也轮到他了。

		史书上对于这位仁兄的评价大都比较一致:什么废物、愚蠢,不一而同。

		对此,我都同意,但我认为,他至少是个勇敢的人。

		明知是黑锅,依然无怨无悔、义无反顾地去背,难道不勇敢吗?

		而他之所以失败,实在不是态度问题,而是能力问题。

		因为他面对的敌人,是努尔哈赤。

		努尔哈赤,明朝最可怕的敌人,战场应变极快,骑兵战术使用精湛,他的军事能力,可与大明历史上的任何一位名将相媲美。

		毫无疑问,他是这个时代最为强悍、最具天赋的军事将领,之一。

		他或许很好,很强大,却绝非没有对手。

		事实上,他宿命的克星已然出现,就在他的眼前——不只一个。

		王在晋到达辽东后,非常努力,非常勤奋,他日夜不停地勘查地形,考量兵力部署,经过几天几夜的刻苦专研,终于想出了一个防御方案。

		具体方案是这样的,王在晋认为,光守山海关是不够的,为了保证防御纵深,他决定再修一座新城,用来保卫山海关,而这座新城就在山海关外八里的八里铺。

		王在晋做事十分认真,他不但选好了位置,还拟好了预算,兵力等等,然后一并上交皇帝。

		天启皇帝看后大为高兴,立即批复同意,还从国库中拨出了工程款。

		应该说,王在晋的热情是值得肯定的,态度是值得尊重的,创意是值得鼓励的,而全盘的计划,是值得唾弃的。

		光守山海关是不够的,因为一旦山海关被攻破,京城就将毫无防卫,唾手可得,虽说山海关沿线很坚固,很结实,但毕竟是砖墙,不是高压电网,如果努尔哈赤玩一根筋,拼死往城墙上堆人,就是用嘴啃,估计也啃穿了。

		在这一点上,王在晋的看法是正确的。

		但这也是他唯一正确的地方,除此之外,都是胡闹。

		哪里胡闹,我就不说了,等一会有人说。

		总之,如按此方案执行,山海关破矣,京城丢矣,大明亡矣。

		对于这一结果,王在晋不知道,天启自然也不知道,而更多的人,是知道了也不说。

		就在一切几乎无可挽回的时候,一封群众来信,彻底改变了这个悲惨的命运。

		这封信是王在晋的部下写的,并通过朝廷渠道,直接送到了叶向高的手中,文章的主题思想只有一条:王在晋的方案是错误的。

		这下叶大人头疼了,他干政治是老手,干军事却是菜鸟,想来想去,这个主意拿不了,于是他跑去找皇帝。

		可是皇帝大人除了做木匠是把好手,基本都是抓瞎,他也吃不准,于是,他又去找了另一个人。

		天惊地动,力挽狂澜,由此开始。
		\begin{quote}
			\begin{spacing}{0.5}  %行間距倍率
				\textit{{\footnotesize
							\begin{description}
								\item[\textcolor{Gray}{\faQuoteRight}] “夫攻不足者守有余,度彼之才,恢复固未易言,令专任之,犹足以慎固封守。”
							\end{description}
						}}
			\end{spacing}
		\end{quote}

		这句话,来自于一个人的传记。

		这句话的大致意思是:以此人的才能,恢复失去的江山,未必容易,但如果信任他,将权力交给他,稳定固守现有的国土,是可以的。

		这是一个至高无上的评价。

		因为这句话,出自于《明史》。说这句话的人,是清代的史官。

		综合以上几点,我们可以认定,在清代,这是一句相当反动的话。

		因为它的隐含意思是:

		如果此人一直在任,大清是无法取得天下的。

		在清朝统治下,捧着清朝饭碗,说这样的话,是要掉脑袋的。

		可是他们说了,他们不但说了,还写了下来,并且流传千古,却没有一个人,因此受到任何惩罚。

		因为他们所说的,是铁一般的事实,是清朝统治者无法否认的事实。

		与此同时,他们还用一种十分特殊的方式,表达了对此人的崇敬。

		在长达二百二十卷、记载近千人事迹的明史传记中,无数为后人熟知的英雄人物,都要和别人挤成一团。

		而在这个人的传记里,只有他自己和他的子孙。

		这个人不是徐达,徐达的传记里,有常遇春。

		不是刘伯温,刘伯温的传记里,有宋濂、叶琛、章溢。

		不是王守仁,王守仁的传记里,还搭配了他的门人冀元亨。

		也不是张居正,张大人和他的老师徐阶、老对头高拱在一个传记里。

		当然,更不是袁崇焕,袁将军住得相当挤,他的传记里,还有十个人。

		这个人是孙承宗。

		明末最伟大的战略家,努尔哈赤父子的克星,京城的保卫者,皇帝的老师,忠贞的爱国者。

		举世无双,独一无二。

		在获得上述头衔之前,他是一个不用功的学生,一个讨生活的教师,一个十六年都没有考上举人的落魄秀才。

		\ifnum\theparacolNo=2
	\end{multicols}
\fi
\newpage
\section{天才的敌手}
\ifnum\theparacolNo=2
	\begin{multicols}{\theparacolNo}
		\fi
		嘉靖四十二年(1563年),孙承宗出生在北直隶保定府高阳\footnote{今河北省高阳县。}。

		生在这个地方,不是个好事。

		作为明朝四大防御要地,蓟州防线的一部分,孙承宗基本是在前线长大的。

		这个地方不好,或者说是太好,蒙古人强大的时候,经常来,女真人强大的时候,经常来,后来改叫金国,也常来,来抢。

		来一次,抢一次,打一次。

		这实在不是个适合人类居住的地方,别的小孩都怕,可孙承宗不怕。

		非但不怕,还过得特别滋润。

		他喜欢战争,喜欢研究战争,从小,别人读四书,他读兵书。成人后,别人往内地跑,他往边境跑,不为别的,就想看看边界。

		万历六年(1578年),保定府秀才孙承宗做出了一个决定——外出游学。这一年,他十六岁。在此后十余年的时间里,孙秀才游历四方,努力向学,练就了一身保国的本领。

		当然,这是史料里正式的说法。

		实际上,这位仁兄在这十几年来,大都是游而不学,要知道,他当年之所以考秀才,不是为了报国,说到底,是混口饭吃,游学?不用吃饭啊?

		还好,孙秀才找到了一份比较好的工作——老师,从此,他开始在教育战线上奋斗,而且越奋斗越好,好到名声传到了京城。

		万历二十年(1592年),在兵部某位官员的邀请下,孙秀才来到京城,成为了一位优秀的私人教师。

		但是慢慢地,孙秀才有思想活动了,他发现,光教别人孩子是不够的,能找别人教自己的孩子,才是正道。

		于是第二年(1593年),他进入了国子监,刻苦读书,再一年后(1594年),他终于考中了举人,这一年,他三十二岁。

		一般说来,考上举人,要么去考进士,要么去混个官,可让人费解的是,孙举人却依然安心当他的老师,具体原因无人知晓,估计他的工资比较高。

		但事实证明,正是这个奇怪的决定,导致了他奇特的人生。

		万历二十七年(1599年),孙承宗的雇主奉命前往大同,就任大同巡抚。官不能丢,孩子的教育也不能丢,于是孙承宗跟着去了。

		我记得,在一次访谈节目中,有一名罪犯说过:无论搞多少次普法教育,都是没用的,只要让大家都去监狱住两天,亲自实践,就不会再犯罪了。

		我同意这个说法,孙承宗应该也同意。

		在那个地方,孙承宗发现了一个陌生而又熟悉的世界,拼死的厮杀,血腥的战场,智慧的角逐,勇气的考验。

		战争,是这个世界上最神秘莫测,最飘忽不定,最残酷,最困难,最考验智商的游戏。在战场上,兵法没有用,规则没有用,因为在这里,最好的兵法,就是实战,唯一的规则,就是没有规则。

		大同的孙老师没有实践经验,也无法上阵杀敌。然而一件事情的发生却足以证实,他已经懂得了战争。

		在明代,当兵是一份工作,是工作,就要拿工资,拿不到工资,自然要闹。一般人闹,无非是堵马路,喊几句,当兵的闹,就不同了,手里有家伙,要闹就往死里闹,专用名词叫做“哗变”。

		这种事,谁遇上谁倒霉,大同巡抚运气不好,偏赶上了。有一次工资发得迟了点,当兵的不干,加上有人挑拨,于是大兵们二话不说,操刀就奔他家去了。

		巡抚大人慌得不行,里外堵得严严实实,门都出不去,想来想去没办法,寻死的心都有了。

		关键时刻,他的家庭教师孙承宗先生出马了。

		孙老师倒也没说啥,看着面前怒气冲冲,刀光闪闪的壮丽景象,他只是平静地说:

		“饷银非常充足,请大家逐个去外面领取,如有冒领者,格杀勿论。”

		士兵一哄而散。

		把复杂的问题弄简单,是一个优秀将领的基本素质。

		孙承宗的镇定、从容、无畏表明,他有能力,用最合适的方法,处理最纷乱的局势,应对最凶恶的敌人。

		大同,在长达五年的时间里,孙承宗看到了战争,理解了战争,懂得了战争,并最终掌握了战争。他的掌握,来自他的天赋、理论以及每一次感悟。

		辽东,大他三岁的努尔哈赤正在讨伐女真哈达部的路上,此时的他,已经是一位精通战争的将领,他的精通,来自于砍杀、冲锋以及每一次拼死的冒险。

		两个天赋异禀的人,以他们各自不同的方式,进入了战争这个神秘的领域,并获知了其中的奥秘。

		二十年后,他们将相遇,以实践来检验他们的天才与成绩。

		\subsection{相遇}
		万历三十二年(1604年),孙承宗向他的雇主告别,踏上了前往京城的道路。他的目标,是科举。这一年,他四十二岁。

		经过几十年的风风雨雨,秀才、落魄秀才,教师、优秀教师、举人、军事观察员,目睹战争的破坏、聆听无奈的哀嚎、体会无助的痛苦,孙承宗最终确定了自己的道路。

		他决定放弃稳定舒适的生活,他决定,以身许国。

		于是在几十年半吊子生活之后,考场老将孙承宗打算认真地考一次。

		这一认真,就有点过了。

		放榜的那天,孙承宗得知了自己的考试名次——第二,全国第二。

		换句话说,他是榜眼。

		按照明朝规定,榜眼必定是庶吉士,必定是翰林,于是在上岗培训后,孙承宗进入翰林院,成为了一名正七品编修。

		之前讲过,明代朝廷是讲出身的,除个别特例外,要想进入内阁,必须是翰林出身,否则,即使你工作再努力,能力再突出,也是白搭。这是一个公认的潜规则。

		但请特别注意,要入内阁,必须是翰林,是翰林,却未必能入内阁。

		毕竟翰林院里不只一个人,什么学士、侍读学士、侍讲、修撰、检讨多了去了,内阁才几个人,还得排队等,前面的人死一个才能上一个,实在不易。

		孙承宗就是排队等的人之一,他的运气不好,等了足足十年,都没结果。

		第十一年,机会来了。

		万历四十二年(1614年),孙承宗调任詹事府谕德。

		这是一个小官,却有着远大的前程,因为它的主要职责是给太子讲课。

		从此,孙承宗成为了太子朱常洛的老师,在前方等待着他的,是无比光明的未来。

		光明了一个月。

		万历四十八年(1620年),即位仅一个月的明光宗朱常洛去世。

		但对于孙承宗而言,这没有什么影响,因为他已经找到了一个新的学生——朱由校。

		教完了爹再教儿子,真可谓是诲人不倦。

		天启皇帝朱由校这辈子没读过什么书,就好做个木工,所以除木匠师傅外,他对其它老师极不感冒。

		孙承宗是唯一的例外。

		由于孙老师长期从事儿童\footnote{私塾。}教育,对于木头型、愚笨型、死不用功型的小孩,一向都有点办法,所以几堂课教下来,皇帝陛下立即喜欢上了孙老师,他从没有叫过孙承宗的名字,而代以一个固定的称谓:“吾师”。

		这个称呼,皇帝陛下叫了整整七年,直到去世为止。

		他始终保持对孙老师的信任,无论何人,以何种方式,挑拨、中伤,都无济于事。

		我说的这个“何人”,是指魏忠贤。

		正因为关系紧,后台硬,孙老师的仕途走得很快,近似于飞,一年时间,他就从五品小官,升任兵部尚书,进入内阁,成为东阁大学士。

		所以,当那封打小报告的信送上来后,天启才会找到孙承宗,征询他的意见。

		可孙承宗同志的回答,却出乎皇帝的意料:

		“我也不知如何决断。”

		幸好后面还有一句:

		“让我去看看吧。”

		天启二年(1622年),兵部尚书兼东阁大学士孙承宗来到山海关。

		孙承宗并不了解王在晋,但到山海关和八里铺转了一圈后,他对王大人便有了一个直观且清晰的判断——这人是个白痴。

		他随即找来了王在晋,开始了一段在历史上极其有名的谈话。

		在谈话的开头,气氛是和谐的,孙承宗的语气非常客气:

		“你的新城建成之后,是要把旧城的四万军队拉过来驻守吗?”

		王在晋本以为孙大人是来找麻烦的,没想到如此友善,当即回答:

		“不是的,我打算再调集四万人来守城。”

		但王大人并不知道,孙先生是当过老师的人,对笨人从不一棍子打死,总是慢慢地折腾:

		“照你这么说,方圆八里之内,就有八万守军了,是吗?”

		王大人还没回过味来,高兴地答应了一声:

		“是的,没错啊。”

		于是,张老师算帐的时候到了:

		“只有八里,竟然有八万守军?你把新城修在旧城前面,那旧城前面的地雷、绊马坑,你打算让我们自己人去趟吗?!”

		“新城离旧城这么近,如果新城守得住,还要旧城干什么?!”

		“如果新城守不住,四万守军败退到旧城城下,你是准备开门让他们进来,还是闭关守城,看着他们死绝?!”

		王大人估计被打懵了,半天没言语,想了半天,才憋出来一句话:

		“当然不能开门,但可以让他们从关外的三道关进来,此外,我还在山上建好了三座军寨,接应败退的部队。”

		这么蠢的孩子,估计孙老师还没见过,所以他真的发火了:

		“仗还没打,你就准备接应败军?不是让他们打败仗吗?而且我军可以进入军寨,敌军就不能进吗?现在局势如此危急,不想着恢复国土,只想着躲在关内,京城永无宁日!”

		王同学彻底无语了。

		事实证明,孙老师是对的,如果新关被攻破,旧关必定难保,因两地只隔八里,逃兵无路可逃,只能往关里跑,到时逃兵当先锋,努尔哈赤当后队,不用打,靠挤,就能把门挤破。

		这充分说明,想出此计划的王在晋,是个不折不扣的蠢货。

		但聪明的孙老师,似乎也不是什么善类,他没有帮助迟钝生王在晋的耐心,当即给他的另一个学生——皇帝陛下写了封信,直接把王经略调往南京养老去了。

		赶走王在晋后,孙承宗想起了那封信,便向身边人吩咐了这样一件事:

		“把那个写信批驳王在晋的人叫来。”

		很快,他就见到了那个打上级小报告的人,他与此人彻夜长谈,一见如故,感佩于这个人的才华、勇气和资质。

		这是无争议的民族英雄孙承宗,与有争议的民族英雄袁崇焕的第一次见面。

		孙承宗非常欣赏袁崇焕,他坚信,这是一个必将震撼天下的人物,虽然当时的袁先生,只不过是个正五品兵备佥事。

		事实上,王在晋并不是袁崇焕的敌人,相反,他一直很喜欢袁崇焕,还对其信任有加,但袁崇焕仍然打了他的小报告,且毫不犹豫。

		对于这个疑问,袁崇焕的回答十分简单:

		“因为他的判断是错的,八里铺不能守住山海关。”

		于是孙承宗问出了第二个问题:

		“你认为,应该选择哪里?”

		袁崇焕回答,只有一个选择。

		然后,他的手指向了那个唯一的地点——宁远。

		宁远,即今辽宁兴城,位居辽西走廊中央,距山海关二百余里,是辽西的重要据点,位置非常险要。

		虽然几乎所有的人都认为,宁远很重要,很险要,但几乎所有的人也都认为,坚守宁远,是一个愚蠢的决定。

		因为当时的明朝,已经丢失了整个辽东,手中仅存的只有山海关,关外都是敌人,跑出二百多里,到敌人前方去开辟根据地,主动深陷重围,让敌人围着打,这不是勇敢,是缺心眼。

		我原先也不明白,后来我去了一趟宁远,明白了。

		宁远是一座既不大,也不起眼的城市,但当我登上城楼,看到四周地形的时候,才终于确定,这是个注定让努尔哈赤先生欲哭无泪的地方。

		因为它的四周三面环山,还有一面,是海。

		说宁远是山区,其实也不夸张。它的东边是首山,西边是窟窿山,中间的道路很窄,是个典型的关门打狗地形,努尔哈赤先生要从北面进攻这里,是很辛苦的。

		当然了,有人会说,既然难走,那不走总行了吧。

		很可惜,虽然走这里很让人恶心,但不恶心是不行的,因为辽东虽大,要进攻山海关,必须从这里走。

		此路不通让人苦恼,再加个别无他路,就只能去撞墙了。

		是的,还会有人说,辽东都丢了,这里只是孤城,努尔哈赤占有优势,兵力很强,动员个几万人把城团团围住,光是围城,就能把人饿死。

		这是一个理论上可行的方案,仅仅是理论。

		如果努尔哈赤先生这样做了,那么我可以肯定,最先被拖垮的一定是他自己。

		因为宁远最让人绝望的地方,并不是山,而是海。

		明朝为征战辽东,在山东登州地区修建了仓库,如遇敌军围城,船队就能将粮食装备源源不断地送到沿海地区,当然也包括宁远。

		而努尔哈赤先生,只能眼睁睁地看着这一切的发生,要知道,他的军队里,没有海军这个兵种。

		更为重要的是,距离宁远不远的地方,有个觉华岛,在岛上有明军的后勤仓库,可以随时支援宁远。

		之所以把仓库建在岛上,原因很简单,明朝人都知道,后金没有海军,没有翅膀,飞不过来。

		但有些事,是说不准的。

		上个月,我从宁远坐船,前往觉华岛\footnote{现名菊花岛。},才发现,原来所谓不远,也挺远,海上走了半个多钟头才到。

		上岸之后,宁远就只能眺望了,于是,我问了当地人一个问题:你们离陆地这么远,生活用品用船运很麻烦吧。

		他回答:我们也用汽车拉,不麻烦。

		然后补充一句:冬天,海面会结冰。

		我又问:这么宽的海面\footnote{我估算了一下,大概有近十公里。},都能冻住吗?

		他回答:一般情况下,冻不住。

		接着还是补充:去年,冻住了。

		去年,是2007年,冬天很冷。

		于是,我想起了三百八十一年前,发生在这里的那场惊天动地的战争,我知道,那一年的冬天,也很冷。

		\subsection{学生}
		孙承宗接受了袁崇焕的意见,他决定,在宁远筑城。

		筑城的重任,他交给了袁崇焕。

		但要准备即将到来的战争,这些还远远不够,还有很多事情要做。

		孙承宗最先做的一件事,就是练兵。

		当时他手下的士兵,总数有七万多人,数字挺大,但也就是个数,一查才发现,有上万人压根没有,都是空额,工资全让老领导们拿走了。

		这是假人,留下来的真人也不顶用,很多兵都是老兵油子,领饷时带头冲,打仗时带头跑,特别是关内某些地方的兵,据说逃跑时的速度,敌人骑马都赶不上。

		对于这批人,孙承宗用一个字就都打发了:滚。

		他遣散了上万名撤退先锋,因为他已经找到了一个极具战斗力的群体——难民。

		难民,就是原本住得好好的人,突然被人赶走,地被占了,房子被烧,老婆孩子被杀,求生不得,求死不能。让这样的人去参军打仗,是不需要动员的。

		孙承宗从难民中挑选了七千人,编入了自己的军队,四年后,他们的仇恨将成为战胜敌人的力量。

		除此之外,他还做了很多事,大致如下:

		修复大城九,城堡四十五;练兵十一万,训练弓弩、火炮手五万;立军营十二、水营五、火营二、前锋后劲营八;造甲胄、军事器械、弓矢、炮石、渠答\footnote{守城的擂石。}、卤盾等数万具。另外,拓地四百里;招集辽人四十余万,训练辽兵三万;屯田五千顷,岁入十五万两白银。

		具体细节不知道,看起来确实很多。

		应该说,孙承宗所做的这些工作非常重要,但绝不是最重要的。

		十七世纪最重要的是什么?是人才。

		天启二年(1622年),孙承宗已经六十岁了,他很清楚,虽然他熟悉战争,精通战争,有着挽救危局的能力,但他毕竟老了。

		为了大明江山,为了百姓的安宁,为了报国的理想,做了一辈子老师的孙承宗决定,收下最后一个学生,并把自己的谋略、战法、无畏的信念,以及永不放弃希望的勇气,全部传授给他。

		他很欣慰,因为他已经找到了一个合适的人选——袁崇焕。

		在他看来,袁崇焕虽然不是武将出身\footnote{进士。},也没怎么打过仗,但这是一个具备卓越军事天赋的人,能够在复杂形势下,作出正确的判断。

		更重要的是,他有着战死沙场的决心。

		因为战场之上,求生者死,求死者生。

		在之后的时间里,他着力培养袁崇焕,巡察带着他,练兵带着他,甚至机密决策也都让他参与。

		当然,孙老师除了给袁同学开小灶外,还让他当了班干部。从宁前兵备副使、宁前道,再到人事部\footnote{吏部。}的高级预备干部\footnote{巡抚。},只用了三年。

		袁崇焕用实际行动证明,他是个不折不扣的优等生。三年里,他圆满完成了自己的工作,并熟练掌握了孙承宗传授的所有技巧、战术与战略。

		在这几年中,袁崇焕除学习外,主要的工作是修建宁远城,加强防御,然而有一天,他突然意识到了一个问题:

		后金军以骑兵为主,擅长奔袭,行动迅猛,抢了就能跑,而明军以步兵为主,骑兵质量又不行,打到后来,只能坚守城池,基本上是敌进我退,敌退我不追,这么下去,到哪儿才是个头?

		是的,防守是不够的,仅凭城池、步兵坚守,是远远不够的。

		彻底战胜敌人强大骑兵唯一方式,就是建立一支同样强大的骑兵。

		所以,在孙老师的帮助下,他开始召集难民,仔细挑选,进行严格训练,只有最勇猛精锐,最苦大仇深的士兵,才有参加这支军队的权力。

		同时,他饲养优良马匹,大量制造明朝最先进的火器三眼神铳,配发到每个人的手中,并反复操练骑兵战法,冲刺砍杀,一丝不苟。

		因为他所需要的,是这样一支军队:无论面临绝境,或是深陷重围,这支军队都能够战斗到最后一刻,绝不投降。

		他成功了。

		他最终训练出了一支这样的军队,一支努尔哈赤、皇太极父子终其一生,直至明朝灭亡,也未能彻底战胜的军队。

		在历史上,这支军队的名字,叫做关宁铁骑。

		袁崇焕的成长,远远超出了孙承宗的预料,无论是练兵、防守、战术,都已无懈可击。虽然此时,他还只是个无名小卒。

		对这个学生,孙老师十分满意。

		但他终究还是发现了袁崇焕的一个缺点,一个看似无足轻重的缺点,从一件看似无足轻重的小事上。

		天启三年(1623年),辽东巡抚阎鸣泰接到举报,说副总兵杜应魁冒领军饷。

		要换在平时,这也不算是个事,但孙老师刚刚整顿过,有人竟然敢顶风作案,必须要严查。

		于是他派出袁崇焕前去核实此事。

		袁崇焕很负责任,到地方后不眠不休,开始查账清人数,一算下来,没错,杜总兵确实贪污了,叫来谈话,杜总兵也认了。

		按规定,袁特派员的职责到此结束,就该回去报告情况了。

		可是袁大人似乎太过积极,谈话刚刚结束,他竟然连个招呼都不打,当场就把杜总兵给砍了,被砍的时候,杜总兵还在做痛哭流涕忏悔状。

		事发太过突然,在场的人都傻了,等大家回过味来,杜总兵某些部下已经操家伙,准备奔着袁大人去了。

		毕竟是朝廷命官,你又不是直属长官,啥命令没有,到地方就把人给砍了,算是怎么回事?

		好在杜总兵只是副总兵,一把手还在,好说歹说,才把群众情绪安抚下去,袁特派员这才安然返回。

		返回之后的第一个待遇,是孙承宗的一顿臭骂:

		“杀人之前,竟然不请示!杀人之后,竟然不通报!士兵差点哗变,你也不报告!到现在为止,我还不知道,你到底杀了什么人!以何理由要杀他!”

		“据说你杀人的时候,只说是奉了上级的命令,如果你凭上级的命令就可以杀人,那还要尚方宝剑\footnote{皇帝特批孙承宗一柄。}干什么?!”

		袁崇焕没有吱声。

		就事情本身而言,并不大,却相当恶劣,既不是直系领导,又没有尚方宝剑,竟敢擅自杀人,实在太过嚣张。

		但此刻人才难得,为了这么个事,把袁崇焕给办了,似乎也不现实,于是孙承宗把这件事压了下去,他希望袁崇焕能从中吸取教训:意气用事,胡乱杀人,是绝对错误的。

		事后证明,袁崇焕确实吸取了教训,当然,他的认识和孙老师的有所不同:

		不是领导,没有尚方宝剑,擅自杀人,是不对的,那么是领导,有了尚方宝剑,再擅自杀人,就该是对的。

		从某个角度讲,他这一辈子,就栽在这个认识上。

		不过局部服从整体,杜总兵死了也就死了,无所谓,事实上,此时辽东的形势相当的好,宁远以及附近的松山、中前所、中后所等据点已经连成了一片,著名的关宁防线\footnote{山海关——宁远。}初步建成,驻守明军已达十一万人,粮食可以供应三年以上,关外两百多公里土地重新落入明朝手中。

		孙承宗修好了城池、整好了军队,找好了学生,恢复了国土,但这一切还不够。

		要应对即将到来的敌人,单靠袁崇焕是不行的,必须再找几个得力的助手。

		\subsection{助手}
		袁崇焕刚到宁远时,看到的是破墙破砖,一片荒芜,不禁感叹良多。

		然而很快就有人告诉他,这是刚修过的,事实上,已有一位将领在此筑城,而且还筑了一年多。

		修了一年多,就修成这个破样,袁崇焕十分恼火,于是他把这个人叫了过来,死骂了一顿。

		没想到,这位仁兄全然没有之前被砍死的那位杜总兵的觉悟,非但不认错,竟然还跳起来,跟袁大人对骂,张口就是老子打了多少年仗,你懂个屁之类的混话。

		这就是当时的懒散游击将军,后来的辽东名将祖大寿的首次亮相。

		祖大寿,是一个很有名的人,有名到连在他家干活的仆人祖宽都进了明史列传,然而这位名人本人的列传,却在清史稿里,因为他最终还换了老板。

		但奇怪的是,和有同样遭遇的吴某某、尚某某、耿某某比起来,他的名声相当好,说他是X奸的人,似乎也不多。原因在于,几乎所有的人都认为,他已尽到了自己的本分。

		祖大寿,字复宇,辽东宁远人,生在宁远,长在宁远,参军还在宁远。此人脾气暴躁,品性凶狠,好持刀砍人,并凭借多年砍人之业绩,升官当上了游击,熊廷弼在的时候很赏识他。

		后来熊廷弼走了,王化贞来了,也很赏识他,并且任命他为中军游击,镇守广宁城。

		再后来,孙得功叛乱,王化贞逃跑了,关键时刻,祖大寿二话不说,也跑了。

		但他并没有跑回去,而是率领军队跑到了觉华岛继续坚守。

		坚守原则,却不吃眼前亏,从后来十几年中他干过的那些事来看,这是他贯彻始终的人生哲学。

		对一个在阎王殿参观过好几次的人而言,袁崇焕这种进士出身,连仗都没打过的人,竟然还敢跑来抖威风,是纯粹的找抽,不骂是不行的。

		这场对骂的过程并不清楚,但结果是明确的,袁大人虽然没当过兵,脾气却比当兵的更坏,正如他的那句名言:“你道本部院是个书生,本部院却是一个将首!”双方你来我往,几个回合下来,祖大寿认输了。

		从此,他成为了袁崇焕的忠实部下,大明的优秀将领,后金骑兵不可逾越的铜墙铁壁。

		祖大寿,袁崇焕的第一个助手。

		其实祖大寿这个名字,是很讨巧的,因为用当地口音,不留神就会读成祖大舅。为了不至于乱辈分,无论上级下属,都只是称其职务,而不呼其姓名。

		只有一个人,由始至终、坚定不移地称其为大舅,原因很简单,祖大寿确实是他的大舅。

		这个人名叫吴三桂。

		当时的吴三桂不过十一二岁,尚未成年,既然未成年,就不多说了。事实上,在当年,他的父亲吴襄,是一个比他重要得多的人物。

		吴襄,辽宁绥中人,祖籍江苏高邮,武举人。

		其实按史料的说法,吴襄先生的祖上,本来是买卖人,从江苏跑到辽东,是来做生意的。可是到他这辈,估计是兵荒马乱,生意不好做了,于是一咬牙,去考了武举,从此参加军队,迈上了丘八的道路。

		由于吴先生素质高,有文化\footnote{至少识字吧。},和兵营里的那些傻大粗不一样,祖大寿对其比较赏识,刻意提拔,还把自己的妹妹嫁给了他。

		吴襄没有辜负祖大寿的信任,在此后十余年的战斗中,他和他的儿子,将成为大明依靠的支柱。

		吴襄,袁崇焕的第二个助手。

		在逃到宁远之前,吴襄和祖大寿是王化贞的下属,在王化贞到来之前,他们是毛文龙的下属。

		现在看来,毛文龙,似乎并不有名,也不重要,但在当时,他是个非常有名,且极其重要的人,至少比袁崇焕要重要得多。

		天启初年的袁崇焕,是宁前道,毛文龙,是皮岛总兵。

		准确地说,袁崇焕,是宁前地区镇守者,朝廷四品文官。

		而毛文龙,是左都督、朝廷一品武官、平辽将军、尚方宝剑的持有者、辽东地区最高级别军事指挥官。

		换句话说,毛总兵比袁大人要大好几级,与毛文龙相比,袁崇焕只是一个微不足道的无名小卒,双方根本就不在同一档次上。

		因为毛总兵并不是一个普通的总兵。

		明代总兵,是个统称,大致相当于司令员,但管几个省的,可以叫司令员,管一个县的,也可以叫司令员。比如,那位吃空额贪污的杜应魁,人家也是个副总兵,但袁特派说砍,就把他砍了,眼睛都不眨,检讨都不写。

		总而言之,明代总兵是分级别的,有分路总兵、协守总兵等等,而最高档次的,是总镇总兵。

		毛文龙,就是总镇总兵,事实上,他是大明在关外唯一的总镇级总兵。

		总镇总兵,用今天的话说,是大军区司令员,地位十分之高,一般都附带将军头衔\footnote{相当于荣誉称号,如平辽、破虏等。},极个别的还兼国防部长\footnote{兵部尚书。}。

		明朝全国的总镇总兵编制,有二十人,十四个死在关内,现存六人,毛文龙算一个。

		但在这些幸存者之中,毛总兵是比较特别的,虽然他的级别很高,但他管的地盘很小——皮岛,也就是个岛。

		皮岛,别名东江,位处鸭绿江口,位置险要,东西长十五里,南北宽十二里,毛总兵就驻扎在上面,是为毛岛主。

		这是个很奇怪的事,一般说来,总镇总兵管辖的地方很大,不是省军区司令,也是地区军区司令,只有毛总兵,是岛军区司令。

		但没有人觉得奇怪,因为其他总兵的地盘,是接管的,毛总兵的地盘,是自己抢来的。

		毛文龙,万历四年(1576年)生人,浙江杭州人,童年的主要娱乐是四处蹭饭吃。

		由于家里太穷,毛文龙吃不饱饭,自然上不起私塾,考不上进士。而就我找到的史料看,他似乎也不是斗狠的主,打架撒泼的功夫也差点,不能考试,又不能闹腾,算是百无一用,比书生还差。

		但要说他什么都没干,那也不对,为了谋生,他开始从事服务产业——算命。

		算命是个技术活,就算真不懂,也要真能忽悠,于是毛文龙开始研究麻衣相术、测字、八卦等等。

		但我们有理由相信,他在这方面的学问没学到家,给人家算了几十年的命,就没顾上给自己算一卦。

		不过,他在另一方面的造诣,是绝对值得肯定的——兵法。

		在平时只教语文,考试只考作文的我国古代,算命、兵法、天文这类学科都是杂学,且经常扎堆,还有一个莫名其妙的统称——阴阳学。

		而迫于生计,毛先生平时看的大都是这类杂书,所以他虽没上过私塾,却并非没读过书。据说他不但精通兵法理论,还经常用于实践——聊天时用来吹牛。

		就这么一路算,一路吹,混到了三十岁。

		不知是哪一天,哪根弦不对,毛文龙突然决定,结束自己现在的生活,毅然北上寻找工作。

		他一路到了辽东,遇见当时的巡抚王化贞,王化贞和他一见如故,认为他是优秀人才,当即命他为都司,进入军队任职。

		这个世界上似乎没有这样的好事,没错,前面两句话是逗你们玩的。

		毛文龙先生之所以痛下决心北上求职,是因为他的舅舅时来运转,当上了山东布政使,跟王化贞关系很好,并向王巡抚推荐了自己的外甥。

		王巡抚给了面子,帮毛文龙找了份工作,具体情况就是如此。

		在王化贞看来,给安排工作,是挣了毛文龙舅舅的一个人情,但事实证明,办这件事,是挣了大明的一个人情。

		毛文龙就这样到部队上班了,虽说只是个都司,但在地方而言,也算是高级干部了,至少能陪县领导吃饭,问题在于,毛都司刚去的时候,不怎么吃得开,因为大家都知道他是关系户,都知道他没打过仗,所以,都瞧不起他。

		直到那一天的到来——天启元年(1621年)三月二十一日。

		这一天,辽阳陷落,辽东经略袁应泰自尽,数万守军全军覆没,至此,广宁之外,明朝在辽东已无立足之地。

		难民携家带口,士兵丢弃武器,大家纷纷向关内逃窜。

		除了毛文龙。

		毛文龙没有跑,但必须说明的是,他之所以不跑,不是道德有多高尚,而是实在跑不掉了。

		由于辽阳失陷太快,毛先生反应不够快,没来得及跑,落在了后面,被后金军堵住,没辙了。

		如果只有他一个人,化化妆,往脸上抹把土,没准还能顺过去。不幸的是,他的手下还有两百来号士兵。

		带着这么群累赘,想溜,溜不掉;想打,打不过。明军忙着跑,后金军忙着追,敌人不管他,自己人也不管他。毛文龙此时的处境,可以用一个词完美地概括——弃卒。

		当众人一片哀鸣,认定走投无路之际,毛文龙找到了一条路——下海。

		他找来了船只,将士兵们安全撤退到了海上。

		然而很快,士兵们就发现,他们行进的方向不是广宁,更不是关外。

		“我们去镇江。”毛文龙答。

		于是大家都傻了。

		所谓镇江,不是江苏镇江,而是辽东的镇江堡,此地位于鸭绿江入海口,与朝鲜隔江而立,战略位置十分重要,极其坚固,易守难攻。

		但大家之所以吃惊,不是由于它很重要,很坚固,而是因为它压根就不在明朝手里。

		辽阳、沈阳失陷之前,这里就换地主了,早就成了后金的大后方,且有重兵驻守,这个时候去镇江堡,动机只有两个:投敌,或是找死。

		然而毛文龙说,我们既不投敌,也不寻死,我们的目的,是攻占镇江。

		很明显,这是在开玩笑,辽阳已经失陷了,没有人抵抗,没有人能够抵抗。大家的心中,有着共同且唯一的美好心愿——逃命。

		但是毛文龙又说,我没有开玩笑。

		我们要从这里出发,横跨海峡,航行上千里,到达敌人重兵集结的坚固堡垒,凭借我们这支破落不堪、装备不齐、刚刚一败涂地,只有几百人的队伍,去攻击装备精良、气焰嚣张、刚刚大获全胜的敌人,以寡敌众。

		我们不逃命,我们要攻击,我们要彻底地击败他们,我们要收复镇江,收复原本属于我们的土地!

		没有人再惊讶,也没有人再反对,因为很明显,这是一个合理的理由,一个足以让他们前去攻击镇江,义无反顾的理由。

		在夜幕的掩护下,毛文龙率军抵达了镇江堡。

		事实证明,他或许是个冲动的人,但绝不是个愚蠢的人,如同预先彩排的一样,毛文龙发动了进攻,后金军队万万想不到,在大后方竟然还会被人捅一刀,没有丝毫准备,黑灯瞎火的,也不知到底来了多少人,从哪里来,只能惊慌失措,四散奔逃。

		此战明军大胜,歼灭后金军千余人,阵斩守将佟养真,收复镇江堡周边百里地域,史称“镇江堡大捷”。

		这是自努尔哈赤起兵以来,明朝在辽东最大,也是唯一的胜仗。

		消息传来,王化贞十分高兴,当即任命毛文龙为副总兵,镇守镇江堡。

		后金丢失镇江堡后,极为震惊,派出大队兵力,打算把毛文龙赶进海里喂鱼。

		由于敌太众,我太寡,毛文龙丢失了镇江堡,被赶进了海里,但他没有喂鱼,却开始钓鱼——退守皮岛。

		毕竟只是个岛,所以刚开始时,谁也没把他当回事,可不久之后,他就用实际行动,让努尔哈赤先生领会了痛苦的真正含义。

		自天启元年以来,毛文龙就没休息过,每年派若干人,出去若干天,干若干事,不是放火,就是打劫,搞得后金不得安生。

		更烦人的是,毛岛主本人实在狡猾无比,你没有准备,他就上岸踢你一脚,你集结兵力,设好埋伏,他又不来,就如同耳边嗡嗡叫的蚊子,能把人活活折磨死。

		后来努尔哈赤也烦了,估计毛岛主也只能打打游击,索性不搭理他,让他去闹,没想到,毛岛主又给了他一个意外惊喜。

		天启三年(1623年),就在后金军的眼皮底下,毛岛主突然出兵,一举攻占金州\footnote{今辽宁金州。},而且占住就不走了,在努尔哈赤的后院放了把大火。

		努尔哈赤是真没法了,要派兵进剿,却是我进敌退,要登陆作战,又没有那个技术,要打海战,又没有海军,实在头疼不已。

		努尔哈赤是越来越头疼,毛岛主却越来越折腾,按电视剧里的说法,住孤岛上应该是个很惨的事,要啥啥没有,天天坐在沙滩上啃椰子,眼巴巴盼着人来救。

		可是毛文龙的孤岛生活过得相当充实,照史书上的说法,是“召集流民,集备军需,远近商贾纷至沓来,货物齐备捐税丰厚”。

		这就是说,毛岛主在岛上搞得很好,大家都不在陆地上混了,跟着跑来讨生活,岛上的商品经济也很发达,还能抽税。

		这还不算,毛岛主除了搞活内需外,还做进出口贸易,日本、朝鲜都有他的固定客商,据说连后金管辖区也有人和他做生意,反正那鬼地方没海关,国家也不征税,所以毛岛主的收入相当多,据说每个月都有十几万两白银。

		有钱,自然就有人了,在高薪的诱惑下,上岛当兵的越来越多,原本只有两百多,后来袁崇焕上岛清人数时,竟然清出了三万人。

		值得夸奖的是,在做副业的同时,毛岛主没有忘记本职工作,在之后的几年中,他创造了很多业绩,摘录如下:

		天启三年,文龙占金州。

		四年五月,文龙遣将沿鸭绿江越长白山,侵大清国东偏。

		八月,遣兵从义州城西渡江,入岛中屯田。

		五年六月,遣兵袭耀州之官屯寨。

		六年五月,遣兵袭鞍山驿,越数日又遣兵袭撤尔河,攻城南。

		乱打一气不说,竟然跑到人家地面上屯田种粮食,实在太嚣张了。

		努尔哈赤先生如果不恨他,那是不正常的。

		可是恨也白恨,科技跟不上,只能眼睁睁看着毛岛主胡乱闹腾。

		拜毛文龙同志所赐,后金军队每次出去打仗的时候,很有一点惊弓之鸟的感觉,唯恐毛岛主在背后打黑枪,以至于长久以来不能安心抢掠,工作精力和情绪受到极大影响,反响极其恶劣。

		如此成就,自然无人敢管,朝廷哄着他,王化贞护着他,后来,王在晋接任了辽东经略,都得把他供起来。

		毛文龙,袁崇焕的第三个帮助者,现在的上级、未来的敌人。

		天启三年(1623年),袁崇焕正热火朝天地在宁远修城墙的时候,另一个人到达宁远。

		这个人是孙承宗派来的,他的职责,是与袁崇焕一同守护宁远。这个人的名字叫满桂。

		满桂,宣府人,蒙古族。很穷,很勇敢。

		满桂同志应该算是个标准的打仗苗子,从小爱好打猎。长大参军了,就爱好打人,在军队中混了很多年,每次出去打仗,都能砍死几个,可谓战功显赫,然而战功如此显赫,混到四十多岁,才是个百户。

		倒不是有人打压他,实在是因为他太实在。

		明朝规定,如果你砍死敌兵一人\footnote{要有首级。},那么恭喜你,接下来你有两种选择,一、升官一级。二、得赏银五十两。

		每次满桂都选第二种,因为他很缺钱。

		我不认为满桂很贪婪,事实上,他很老实。

		因为他并不知道,选第二种的人,能拿钱,而选第一种的,既能拿权,也能拿钱。

		就这么个混法,估计到死前,能混到个千户,就算老天开眼了。

		然而数年之后一个人的失败,造就了他的成功,这个失败的人,是杨镐。

		万历四十七年(1619年),杨镐率四路大军,在萨尔浒全军覆没,光将领就死了三百多人,朝廷没人了,只能下令破格提拔,满桂同志就此改头换面,当上了明军的高级将领——参将。

		但真正改变他命运的,是另一个成功的人——孙承宗。

		天启二年(1622年),在巡边的路上,孙承宗遇见了满桂,对这位老兵油子极其欣赏\footnote{大奇之。},高兴之余,就给他升官,把他调到山海关,当上了副总兵,一年后,满桂被调往宁远,担任守将。

		满桂是一个优秀的将领,他不但作战勇敢,而且经验丰富,还能搞外交。

		当时的蒙古部落,已经成为后金军队的同盟,无论打劫打仗都跟着一起来,明军压力很大,而满桂的到来彻底改变了这一切。

		他利用自己的少数民族身份,对同胞进行了长时间耐心的劝说,对于不听劝说的,也进行了长时间耐心的攻打。很快,大家就被他又打又拉的诚恳态度所感动,全都服气了\footnote{桂善操纵,诸部咸服。}。

		此外,他很擅长堆砖头,经常亲自监工砌墙,还很喜欢练兵,经常把手下的兵练得七荤八素。

		就这样,在满桂的不懈努力下,宁远由当初一座较大的废墟,变成了一座较大的城市\footnote{军民五万余家,屯种远至五十里。}。

		而作为宁远地区的最高武官,他与袁崇焕的关系也相当好。

		其实矛盾还是有的,但问题不大,至少当时不大。

		必须说明一点,满桂当时的职务,是宁远总兵,而袁崇焕,是宁前道。就级别而言,满桂比袁崇焕要高,但明朝的传统,是以文制武,所以在宁远,袁崇焕的地位要略高于满桂,高一点点。

		而据史料记载,满桂是个不苟言笑,却极其自负的人。加上他本人是从小兵干起,平时干的都是砍人头的营生\footnote{一个五十两。},注重实践,最看不起的,就是那些空谈理论,没打过仗的文官,当然,这其中也包括袁崇焕。

		但有趣的是,他和袁崇焕相处得还不错,并不是他比较大度,而是袁崇焕比较能忍。

		袁大人是很有自知之明的。他很清楚,在辽东混的,大部分都是老兵油子,杀人放火的事情干惯了,在这些人看来,自己这种文化人兼新兵蛋子,是没有发言权的。

		所以他非常谦虚,非常能装孙子,还时常向老前辈们\footnote{如满桂。}虚心请教,满桂们也心知肚明,知道他是孙承宗的人,得罪不起,都给他几分面子。总之,大家混得都还不错。

		满桂,袁崇焕的第四个帮助者,三年后的共经生死的战友,七年后置于死地的对手。

		或许你觉得人已经够多了,可是孙承宗似乎不怎么看,不久之后,他又送来了第五个人。

		这个人,是他从刑场上救下来的,他的名字叫赵率教。

		赵率教,陕西人,此人当官很早,万历中期就已经是参将了,履历平平,战功平平,资质平平,什么都平平。

		表现一般不说,后来还吃了官司,工作都没了。后来也拜杨镐先生的福,武将死得太多没人补,他就自告奋勇,去补了缺,在袁应泰的手下,混了个副总兵。

		可是他的运气很不好,刚去没多久,辽阳就丢了,袁应泰自杀,他跑了。

		情急之下,他投奔了王化贞,一年后,广宁失陷,王化贞跑了,他也跑了。

		再后来,王在晋来了,他又投奔了王在晋。

		由于几年之中,他到了好几个地方,到哪,哪就倒霉,且全无责任心,遇事就跑,遇麻烦就溜,至此,他终于成为了明军之中有口皆碑的典型人物——反面典型。

		对此,赵率教没有说什么,也不能说什么。

		然而不久后,赵率教突然找到了王在晋,主动提出了一个要求:

		“我愿戴罪立功,率军收复失地。”

		王在晋认为,自己一定是听错了,然而当他再次听到同样坚定的话时,他认定,赵率教同志可能是受了什么刺激。

		因为在当时,失地这个概念,是比较宽泛的,明朝手中掌握的,只有山海关,往大了说,整个辽东都是失地,您要去收复哪里?

		赵率教回答:前屯。

		前屯,就在宁远附近,是明军的重要据点。

		在确定赵率教头脑清醒,没有寻死倾向之后,王在晋也说了实话:

		“收复实地固然是好,但眼下无余兵。”

		这就很实在了,我不是不想成全你,只是我也没法。

		然而赵率教的回答彻底出乎了王大人的意料:

		“无需派兵,我自己带人去即可。”

		老子是辽东经略,手下都没几号人,你还有私人武装?于是好奇的王在晋提出了问题:

		“你有多少人?”

		赵率教答:

		“三十八人。”

		王在晋彻底郁闷了,眼下大敌当前,努尔哈赤随时可能打过来,士气如此低落,平时能战斗的,也都躲了,这位平时特别能躲的,却突然站出来要战斗?

		这都啥时候了,你开什么玩笑?还嫌不够乱?

		于是一气之下,王在晋手一挥:你去吧!

		这是一句气话,可他万没想到,这哥们真去了。

		赵率教率领着他的家丁,三十八人,向前屯进发,去收复失地。

		这是一个有明显自杀迹象的举动,几乎所有的人都认为,赵率教疯了。

		但事实证明,赵先生没有疯,因为当他接近前屯,得知此地有敌军出现时,便停下了脚步。

		“前方已有敌军,不可继续前进,收复此地即可。”

		此地,就是他停下的地方,名叫中前所。

		中前所,地处宁远近郊,大致位于今天的辽宁省绥中县附近,赵率教在此扎营,就地召集难民,设置营地,挑选精壮充军,并组织屯田。

		王在晋得知了这个消息,却只是轻蔑地笑了笑,他认为,在那片遍布敌军的土地上,赵率教很快会故伎重演,丢掉一切再跑回来。

		几个月后,孙承宗来到了这个原本应该空无一人的据点,却看见了广阔的农田、房屋,以及手持武器、训练有素的士兵。

		在得知此前这里只有三十八人后,他找来了赵率教,问了他一个问题:

		“现在这里有多少人?”

		赵率教回答:

		“民六万有余,士兵上万人。”

		从三十八,到六万,面对这个让人难以置信的奇迹,孙承宗十分激动,他老人家原本是坐着马车来的,由于过于激动,当即把车送给了赵率教,自己骑马回去了。

		从此,他记住了这个人的名字。

		就赵率教同志的表现来看,他是一个知道羞耻的人,知耻近乎勇,在经历了无数犹豫、困顿后,他开始用行动,去证明自己的勇气。

		可他刚证明到一半,就差点被人给砍了。

		正当赵率教撩起袖子,准备大干一场的时候,兵部突然派人来找他,协助调查一件事情。

		赵率教明白,这回算活到头了。

		事情是这样的,当初赵率教在辽阳的时候,职务是副总兵,算是副司令员,掌管中军,这就意味着,当战争开始时,手握军队主力的赵率教应全力作战,然而他逃了,并直接导致了作战失败。

		换句话说,小兵可以跑,老百姓可以跑,但赵率教不能跑,也不应该跑,既然跑了,就要依法处理,根据明朝军法,此类情形必死无疑。

		但所谓必死无疑,还是有疑问的,特别是当有猛人求情的时候。

		孙承宗听说此事后,当即去找了兵部尚书,告诉他,此人万不可杀,兵部尚书自然不敢得罪内阁大学士,索性做了个人情,把赵率教先生放了。

		孙承宗并不是一个仁慈的人,他之所以放赵率教一马,是因为他认定,这人活着比死了好。

		而赵率教用实际行动,证明了孙承宗的判断,在不久后的那场大战中,他将起到至关重要的作用。

		赵率教,袁崇焕的第五个帮助者。

		\subsection{惊变}
		天启元年(1620年),孙承宗刚到辽东的时候,他所有的,只是山海关以及关外的八里地。

		天启五年(1624年),孙承宗巩固了山海关,收复了宁远,以及周边几百里土地。

		在收复宁远之后,孙承宗决定再进一步,占据另一个城市——锦州。他认定,这是一个至关重要的地点。

		但努尔哈赤似乎不这么看,锦州嘛,又小又穷,派兵守还要费粮食,谁要谁就拿去。

		就这样,不费吹灰之力,孙承宗得到了锦州。

		事后证明,自明朝军队进入锦州的那一刻起,努尔哈赤的悲惨命运便已注定。

		因为至此,孙承宗终于完成了他一生中最伟大的杰作——关锦防线。

		所谓关锦防线,是指由山海关——宁远——锦州组成的防御体系,该防线全长四百余里,深入后金区域,沿线均有明朝堡垒、据点,极为坚固。

		历史告诉我们,再坚固的防线,也有被攻陷的一天。

		历史还告诉我们,凡事总有例外,比如这条防线。

		事实上,直到明朝灭亡,它也未被突破。此后长达十余年时间里,后金军队用手刨,用嘴啃,用牙咬,都毫无效果,还搭上了努尔哈赤先生的一条老命。

		这是一个科学、富有哲理而又使人绝望的防御体系,因为它基本上没有弱点。

		锦州,辽东重镇,自古为入关要道,且地势险要,更重要的是,锦州城的一面,靠海。对于没有海军的后金而言,这又是一个噩梦。

		这就是说,只要海运充足,在大多数情况下,即使被围得水泄不通,锦州也是很难攻克的。

		既然难打,能不能不打呢?

		不能。

		我的一位住在锦州的朋友告诉我,他要回家十分方便,因为从北京出发,开往东三省,在锦州停靠的火车,有十八辆。

		我顿时不寒而栗,这意味着,三百多年前的明朝,要前往辽东,除个别缺心眼爬山坡的人外,锦州是唯一的选择。

		要想入关,必须攻克宁远,要攻克宁远,必须攻克锦州,要攻克锦州,攻克不了。

		当然,有人会说,锦州不过是个据点,何必一定要攻陷?只要把锦州围起来,借个道过去,继续攻击宁远,不就行了吗?

		是的,按照这个逻辑,也不一定要攻陷宁远,只要把宁远围起来,借个道过去,继续攻击山海关,不也行吗?

		这样看来,努尔哈赤实在太蠢了,这么简单的道理,为什么就没想到呢?

		我觉得,持有这种想法的人,应该去洗把脸,清醒清醒。

		假定你是努尔哈赤,带了几万兵,到了锦州,锦州没人打你,于是,你又到了宁远,宁远也没人打你,就这么一路顺风到了山海关,准备发动攻击。

		我相信,这个时候你会惊喜的发现,锦州和宁远的军队已经出现在你的后方,准备把你一锅端——除非这两地方的守将是白痴。

		现在你有大麻烦了,眼前是山海关,没准十天半月攻不下来,请屁股后面的军队别打你,估计人家不干,就算你横下一条心,用头把城墙撞破,冲进了关内,抢到了东西,你也总得回去吧。

		如果你没长翅膀,你回去的路线应该是山海关——宁远——锦州……

		看起来似乎比较艰难,不是吗?

		这就是为什么曹操同志多年来不怕孙权,不怕刘备,偏偏就怕马腾、马超——这两位先生的地盘在他的后方。

		这就是孙承宗的伟大成就,短短几年之间,他修建了若干据点,收复了若干失地,提拔了若干将领,训养了若干士兵。

		现在,在他手中的,是一条坚不可破的防线,一支精锐无比的军队,一群天赋异禀的卓越将领。

		但对于这一切,努尔哈赤并不清楚,至少不十分清楚。

		祖大寿、吴襄、满桂、赵率教、毛文龙以及袁崇焕,对努尔哈赤而言,这些名字毫无意义。

		自万历四十六年起兵以来,明朝能打的将领,他都打了,杨镐、刘綎、杜松、王化贞、袁应泰,全都是手下败将,无一例外,在他看来,新来的这拨人下场估计也差不多。

		但他终将失败,败在这几个无名小卒的手中,并永远失去翻盘的机会。

		话虽如此,努尔哈赤还是很有几把刷子的,他不了解目前的局势,却了解孙承宗的实力,很明显,这位督师大人比熊廷弼还难对付,所以几年之内,他都没有发动大的进攻。

		大的没有,小的还是有。

		在后金的军队中,最优秀的将领无疑是努尔哈赤,但正如孙承宗一样,他的属下,也有很多相当厉害的猛人。

		而在这些猛人里,最猛的,就是八大贝勒。

		所谓八大贝勒,分别是指代善、阿敏、莽古尔泰、皇太极、阿济格、多尔衮、多铎、济尔哈朗。

		在这八个人里,按照军功和资历,前四个大猛,故称四大贝勒,后四个小猛,故称四小贝勒。

		其中最有名的,无疑是两个人,皇太极、多尔衮。

		但最能打仗的,是三个人,除皇太极和多尔衮外,还有一个代善。

		多尔衮年纪还小,就不说了,皇太极很有名,也不说了,这位代善,虽然年纪很大,且不出名,但很有必要说一说。

		事实上,大贝勒代善是当时后金最为杰出的军事将领之一,此人非常勇猛,在与明朝作战时,经常身先士卒,且深通兵法,擅长伏击,极其能打。

		因为他很能打,所以努尔哈赤决定,挑选一个目标,由代善发动攻击,以试探孙承宗的虚实,而他选定的这个目标,就是锦州。

		当代善率军来到锦州城下的时候,他才意识到,这是个结结实实的黑锅。

		首先锦州非常坚固。在修城墙方面,孙承宗很有一套,城不但高,而且厚,光凭刀砍斧劈,那是没指望的,要想进城,没有大炮是不行的。

		大炮也是有的,不过不在城下,而在城头。

		其实一直以来,明朝的火器水平相当高。万历三大征打日本的时候也很经用,后来之所以荒废,不是技术问题,而是态度问题。

		万历前期,皇帝陛下精神头足,什么事都愿意折腾,后来不想干了,天天躲着不上朝,下面也开始消极怠工。外加火器工作危险性大,吃力不讨好,没准出个安全事故,是很麻烦的。

		孙承宗不怕麻烦,他不但为部队添置三眼火铳等先进装备,还购置了许多大炮,尝试用火炮守城。而锦州,就是他的试点城市。

		虽然情况不妙,但代善不走寻常路,也不走回头路,依然一根筋,找人架云梯、冲车往城里冲。

		此时的锦州守将,是赵率教。应该说,他的作战态度是很成问题的,面对着在城下张牙舞爪,极其激动的代善,他却心平气和,毫不激动,时不时在城头转两圈,放几炮,城下便会迅速传来凄厉的惨叫声,在赔上若干架云梯,若干条性命,却毫无所得的情况下,代善停止进攻。

		虽然停止进攻,但代善还不大想走,他还打算再看两天。

		可是孙承宗似乎是不欢迎参观的,代贝勒的屁股还没坐热,就得到一个可怕的消息,一支明军突然出现在自己的侧翼。

		这支部队是驻守前屯、松山的明军,听说客人来了,没赶上接风,特来送行。

		在短暂慌乱之后,代善恢复了平静,作为一名经验丰富的将领,他有信心击退这支突袭部队。

		可他刚带队发起反击,就看到自己屁股后面烟尘四起:城内的明军出动了。

		这就算是腹背受敌了,但代善依然很平静,作为一名经验丰富的将领,他很有信心。

		然后,很有信心的代善又得知了另一个消息——宁远、中前所等地的明军已经出动,正朝这边来,吃顿饭的功夫也就到了。

		但代善不愧是代善,作为一名经验丰富的将领,他非常自信,镇定地做出了一个英明的判断:快逃。

		可是来去自如只是一个幻想,很快代善就发现,自己已经陷入重围。明军毫不客气,一顿猛打,代善部伤亡十分惨重。好在来的多是骑兵,机动力强,拼死往外冲,总算奔出了条活路,一口气跑上百里,直到遇见接他的二贝勒阿敏,魂才算漂回来。

		此战明军大胜,击溃后金军千余人,战后清点斩获首级六百多颗,努尔哈赤为他的试探付出了惨痛的代价。

		在孙承宗督师辽东的几年里,双方很有点相敬如宾的意思,虽说时不时搞点小摩擦,但大仗没打过,孙承宗不动,努尔哈赤不动。

		可是孙承宗不动是可以的,努尔哈赤不动是不行的。

		因为孙大人的任务是防守,只要不让敌人进关抢东西,他就算赢了。

		努尔哈赤就不同了,他的任务是抢,虽说占了挺大一块地方,但人都跑光了,技术型人才不多,啥产业都没有。据说有些地方,连铁锅都造不出来。孙承宗到辽东算出差,有补助,还有朝廷送物资,时不时还能回去休个假,努先生完全是原生态,没人管没人疼,不抢怎么办?

		必须抢,然而不能抢,因为有孙承宗。

		作为世界超级大国,美国有一个非常有趣的形象代言人——山姆大叔。这位大叔的来历就不说了,他的具体特点是面相端正,勤劳乐观,处事低调埋头苦干,属于那种不怎么言语,却特能干事的类型,是许多美国人争相效仿的楷模。

		孙承宗就是一个山姆大叔型的人物,当然,按年龄算,应该叫山姆大爷,这位仁兄相貌奇伟\footnote{画像为证。},极富乐观主义精神\footnote{大家都不干,他干。},非常低调\footnote{从不出兵闹事。},经常埋头苦干\footnote{参见前文孙承宗业绩清单。}。

		刚开始的时候,努尔哈赤压根瞧不起孙大爷,因为这个人到任后毫无动静,一点不折腾,什么一举荡平,光复辽东,提都不提,别说出兵攻击,连挑衅斗殴都不来,实在没意思。

		但慢慢地,他才发现,这是一个极其厉害的人。

		就在短短几年内,明朝的领土以惊人的速度扩张,从关外的一亩三分地,到宁远,再到锦州,在不知不觉中,他已收复了辽东近千里土地。

		更为可怕的是,此人每走一步,都经过精心策划,步步为营稳扎稳打,趁你不注意,就刨你两亩地,每次都不多占,但占住了就不走,几乎找不到任何弱点。

		对于这种抬头望天,低头使坏的人,努尔哈赤是一点办法都没有,只能眼睁睁地看着对方大踏步的前进,自己大踏步地后退,直到天启五年(1625年)十月的那一天。

		这一天,努尔哈赤得到消息,孙承宗回京了。

		他之所以回去,不是探亲,不是述职,也不是做检讨,而是彻底退休。

		必须说明的是,他是主动提出退休的,却并不情愿,他不想走,却不能不走。

		因为他曾无比依赖的强大组织东林党,被毁灭了。
		\ifnum\theparacolNo=2
	\end{multicols}
\fi
\newpage
\section{一个监狱看守}
\ifnum\theparacolNo=2
	\begin{multicols}{\theparacolNo}
		\fi
		关于东林党的覆灭,许多史书上的说法比较类似:一群有道德的君子,在无比黑暗的政治斗争中,输给了一群毫无道德的小人,最终失败。

		我认为,这个说法,那是相当的胡扯。

		事实上,应该是一群精明的人,在无比黑暗的政治斗争中,输给了另一群更为精明的人,最终失败。

		许多年来,东林党的失败之所以很难说清楚,是由于东林党的成功没说清楚。

		而东林党的成功之所以没说清楚,是由于这个问题,很难说清楚。

		这不是顺口溜,其实一直以来,在东林党的兴亡之中,都隐藏着一些不足为人道的玄机,很多人不知道,知道的人不说。

		凑巧的是,我是一个比较较真的人,对于某些很难说清楚的问题,不足为人道的玄机,有着很难说清楚,不足为人道的兴趣。

		于是,在查阅分析了许多史籍资料后,我得到了这样一个结论:

		东林党之所以成功,是因为强大,之所以失败,是因为过于强大。

		万历四十八年(1620年),在杨涟、左光斗以及一系列东林党人的努力下,朱常洛顺利即位,成为了明光宗。

		虽然这位仁兄命短,只活了一个月,但东林党人再接再厉,经历千辛万苦,又把他的儿子推了上去,并最终控制了朝廷政权。

		用正面的话说,这是正义战胜了邪恶,意志顽强,坚持到底。

		用反面的话说,这是赌一把,运气好,找对了人,打对了架。

		无论正面反面,几乎所有人都认为,东林党能够掌控天下,全靠明光宗死后那几天里,杨涟的拼死一搏,以及继任皇帝的感恩图报。

		这是一个重要的原因,但绝不是唯一重要的原因。

		因为在中国历史上,一般而言,只要皇帝说话,什么事都好办,什么事都能办,可是明朝实在太不一般。

		明朝的皇帝,从来不是说了就算的,且不论张居正、刘瑾、魏忠贤之类的牛人,光是那帮六七品的小御史、给事中,天天上书骂人,想干啥都不让,能把人活活烦死。

		比如明武宗,就想出去转转,换换空气,麻烦马上就来,上百人跪在门口痛哭流涕,示威请愿,午觉都不让睡。闹得你死我活,最后也没去成。

		换句话说,皇帝大人连自己的事情都搞不定,你让他帮东林党控制朝政,那是不太现实的,充其量能帮个忙而已。

		东林党掌控朝廷的真正原因在于,他们打败了朝廷中所有的对手,具体说,是齐、楚、浙三党。

		众所周知,东林党中的许多成员是没有什么博爱精神的,经常耍二杆子性格,非我族类就是其心必异,什么人都敢惹,搞了几十年斗争,仇人越来越多,特别是三党,前仆后继,前人退休,后人接班,一代代接茬上,斗得不亦乐乎。

		这两方的矛盾,那叫一个苦大仇深。什么争国本、妖书案、梃击案,只要是个机会,能借着打击对手,就绝不放过,且从万历十几年就开始闹,真可谓是历史悠久。

		就实力而言,东林党势头大,人多,占据优势,而三党迫于压力,形成了联盟,共同对付东林党,所以多年以来此消彼长,什么京察、偷信,全往死里整。可由于双方实力差距不大,这么多年了,谁也没能整死谁。

		万历末年,一个人来到了京城,不久之后,在极偶然的情况下,他加入了其中一方。

		他加入的是东林党,于是,三党被整死了。

		这是一个不折不扣的小人物,然而,正是这个小人物的到来,打破了几十年的僵局,这个人名叫汪文言。

		如果你不了解这个人,那是正常的,如果你了解,那是不正常的。

		甚至很多熟读明清历史的人,也只知道这个名字,而不清楚这个名字背后隐藏的东西。

		因为这个人实在是太不起眼了。

		事实上,为查这位仁兄的生平,我吃了很大苦头,翻了很多书,还专门去查了历史文献检索,竟然都没能摸清他的底。

		在几乎所有的史籍中,对于此人的描述都只有只言片语,应该说,这是奇怪的现象。

		对于一个在历史上有一定知名度的人而言,介绍如此之少,是很不正常的,但从某个角度讲,又是很正常的。

		因为决定成败的关键人物,往往喜欢隐藏于幕后。

		汪文言,安徽人,不是进士,也不是举人,甚至不是秀才,他没有进过考场,没有当过官,只是个普通的老百姓。

		对于这位老百姓,后世曾有一个评价:以布衣之身,操控天下。

		汪布衣小时候情况如何不太清楚,从目前的材料看,是个很能混的人,他虽然不考科举,却还是当上了公务员——县吏。

		事实上,明代的公务员,并非都是政府官员,它分为两种:官与吏。

		参加科举考试,考入政府成为公务员的,是官员。就算层次最低、底子最差的举人\footnote{比如海瑞。},至少也能混个县教育局长。

		可问题在于,明朝的官员编制是很少的,按规定,一个县里有品级,吃皇粮的,只有知县\footnote{县长。}、县丞\footnote{县政府办公室主任。}几个人而已。

		而没有品级,也吃皇粮的,比如教谕\footnote{教育局长。}、驿丞\footnote{县招待所所长。},大都由举人担任,人数也不多。

		在一个县里,只有以上人员算是国家公务员,换句话说,他们是领国家工资的。

		然而一个县只靠这些人是不行的,县长大人日理万机,无论如何是忙不过来的,所以手下还要有跑腿的,偷奸耍滑的,老实办事的,端茶倒水的。

		这些被找来干活的人,就叫吏。

		吏没有官职、没有编制,国家也不给他们发工资,所有收入和办公费用都由县里解决,换句话说,这帮人国家是不管的。

		虽然国家不管,没有正式身份,也不给钱,但这份职业还是相当热门,每年都有无数热血青年前来报考,没关系还当不上,也着实吸引了许多杰出人才,比如阳谷县的都头武松同志,就是其中的优秀榜样。

		这是因为在吏的手中,掌握着一件最为重要的东西——权力。

		一般说来,县太爷都是上级派下来的,没有根基,也没有班底,而吏大都是地头蛇,熟悉业务,有权在手,熟门熟路,擅长贪污受贿,黑吃黑,除去个把像海瑞那种软硬不吃的极品知县外,谁都拿这帮编外公务员没办法。

		汪文言,就是编外公务员中,最狡猾,最会来事,最杰出的代表人物。

		汪文言的官场生涯,是从监狱开始的,那时候,他是监狱的看守。

		作为一名优秀的看守,他忠实履行了守护监狱,训斥犯人,收取贿赂、拿黑钱的职责。

		由于业务干得相当不错,在上级\footnote{收过钱的。}和同僚\footnote{都是同伙。}的一致推荐下,他进入了县衙,在新的岗位上继续开展自己的光辉事业。

		值得表扬的是,此人虽然长期和流氓地痞打交道,不光彩的事情也没少干,但为人还是很不错的,经常仗义疏财,接济朋友。但凡认识他的,就算走投无路,只要找上门来,他都能帮人一把,江湖朋友纷纷前来蹭饭,被誉为当代宋江。

		就这样,汪文言名头越来越响,关系越来越野,越来越能办事,连知县搞不定的事情,都要找他帮忙。家里跟宋江一样,经常宾客盈门,什么人都有,即有晁盖之类的江洋大盗,又有李逵之流的亡命之徒,上门的礼仪也差不多,总是“叩头就拜”,酒足饭饱拿钱之后,就甘心做小弟,四处传扬汪先生的优秀品格。

		在无数志愿宣传员的帮助下,汪先生逐渐威名远播,终于打出县城,走向全省,波及全国。

		但无论如何,他依然只是一个县衙的小人物,直到有一天,他的名声传到了一个人的耳中。

		这个人叫于与立,时任刑部郎中。

		这位于郎中官职不算太高,但想法不低,经常四处串门拉关系,他听说汪文言的名声后,便主动找上门去,特聘汪先生到京城,发挥特长,为他打探消息。

		汪先生岂是县中物,毫不犹豫就答应了,准备到京城大展拳脚。

		可几个月下来,汪文言发现,自己县里那套,在京城根本混不开。

		因为汪先生一无学历,二无来历,档次太低,压根就没人搭理他。无奈之下,他只好出钱,去捐了个监生,不知找了谁的门路,还混进了太学。

		这可就真了不得了,汪先生当即拿出当年跑江湖的手段,上下打点,四面逢源,短短几月,上至六部官员,下到穷学生,他都混熟了,没混熟的,也混个脸熟。

		一时之间,汪文言从县里的风云人物,变成了京城的风云人物。

		但这位风云人物,依然还是个小人物。

		因为真正掌控这个国家权力中枢的重要人物,是不会搭理他的,无论是东林党的君子,还是三党的小人,都看不上这位江湖人士。

		但他终究找到了一位可靠的朋友,并在他的帮助下,成功进入了这片禁区。

		这位不计较出身的朋友,名叫王安。

		要论出身,在朝廷里比汪文言还低的,估计也只有太监了,所以这两人交流起来,也没什么心理障碍。

		当时的王安,并非什么了不得的人物,虽说是太子朱常洛的贴身太监,可这位太子也不吃香,要什么没什么,老爹万历又不待见,所以王安同志混得相当不行,没人去搭理他。

		但汪文言恰恰相反,鞍前马后帮他办事,要钱给钱,要东西给东西,除了女人,什么都给了。

		王安很喜欢汪文言。

		当然,汪文言先生不是人道主义者,也不是慈善家,他之所以结交王安,只是想赌一把。

		一年后,他赌赢了。

		在万历四十八年(1620年)七月二十一日的那个夜晚,当杨涟秘密找到王安,通报老头子即将走人的消息时,还有第三个人在场——汪文言。

		杨涟说,皇上已经不行了,太子应立即入宫继位,以防有变。

		王安说,目前情形不明,没有皇上的谕令,如果擅自入宫,凶多吉少。

		杨涟说,皇上已经昏迷,不会再有谕令,时间紧急,绝不能再等!

		王安说,事关重大,再等等。

		僵持不下时,汪文言用自己几十年官海沉浮的经验,做出了一个判断。

		他对王安说:杨御史是对的,不能再等待,必须立即入宫。

		一直以来,王安对汪文言都极为信任,于是他同意了,并带领朱常洛,在未经许可的情况下进入了皇宫,成功即位。

		这件事不但加深了王安对汪文言的信任,还让东林党人第一次认清了这个编外公务员,江湖混混的实力。

		继杨涟之后,东林党的几位领导,大学士刘一璟、韩旷、尚书周嘉谟、御史左光斗等人,都和汪文言拉上了关系。

		就这样,汪文言加深了与东林党的联系,并最终成为了东林党的一员——瞎子都看得出,新皇帝要即位了,东林党要发达了。

		但当他真正踏入政治中枢的时候,才发现,局势远不像他想象的那么乐观。

		当时明光宗已经去世,虽说新皇帝也是东林党捧上去的,但三党势力依然很大,以首辅方从哲为首的浙党、以山东人给事中亓诗教为首的齐党、和以湖广人官应震、吴亮嗣为首的楚党,个个都不是省油的灯。

		三党的核心,是浙党,此党的创始人前任首辅沈一贯,一贯善于拉帮结派,后来的接班人,现任首辅方从哲充分发扬了这一精神,几十年下来,朝廷内外,浙党遍布。

		齐党和楚党也不简单,这两个党派的创始人和成员基本都是言官,不是给事中,就是御史,看上去级别不高,能量却不小,类似于今天的媒体舆论,动不动就上书弹劾兴风作浪。

		三党分工配合,通力协作,极不好惹,东林党虽有皇帝在手,明里暗里斗过几次,也没能搞定。

		关键时刻,汪文言出场。

		在仔细分析了敌我形势后,汪文言判定,以目前东林党的实力,就算和对方死拼,也只能死,没得拼。

		而最关键的问题在于,东林党的这帮大爷都是进士出身,个个都牛得不行,进了朝廷就人五人六,谁都瞧不上谁,看你不顺眼也不客套,恨不得操板砖上去就拍。

		汪文言认为,这是不对的,为了适应新的斗争形势,必须转变观念。

		由于汪先生之前在基层工作,从端茶倒水提包拍马开始,一直相当低调,相当能忍,所以在他看来,这个世界上没有永远的敌人,也没有永远的朋友,只要会来事,朋友和敌人,是可以相互转化的。

		秉持着这一理念,他拟定了一个计划,并开始寻找一个恰当的人选。

		很快,他就找到了这个人——梅之焕。

		梅之焕,字彬父,万历三十二年进士,选为庶吉士。后任吏科给事中。

		此人出身名门,文武双全,十几岁的时候,有一次朝廷阅兵,他骑匹马,没打招呼,稀里糊涂就跑了进去,又稀里糊涂地要走。

		阅兵的人不干,告诉他你要不露一手,今天就别想走。

		梅之焕二话不说,拿起弓就射,九发九中,射完啥也不说,摆了个特别酷的动作,就走人了\footnote{长揖上马而去。}。

		除上述优点外,这人还特有正义感,东厂坑人,他就骂东厂,沈一贯结党,他就骂沈一贯,是个相当强硬的人。

		但汪文言之所以找到这位仁兄,不是因为他会射箭,很正直,而是因为他的籍贯。

		梅之焕,是湖广人,具体地说,是湖北麻城人。

		明代官场里,最重要的两大关系,就是师生、老乡。一个地方出来的,都到京城来混饭吃,老乡关系一攀,就是兄弟了。所以自打进入朝廷,梅之焕认识的,大都是楚党成员。

		可这人偏偏是个东林党。

		有着坚定的东林党背景,又与楚党有着密切的联系,很好,这正是那个计划所需要的人。

		汪文言认为,遇到敌人,直接硬干是不对的,在操起板砖之前,应该先让他自己绊一跤。

		三党是不好下手的,只要找到一个突破口,把三党变成两党,就好下手了。

		在仔细衡量利弊后,他选择了楚党。

		因为在不久之前,发生过这样一件事情。

		虽然张居正大人已经死去多年,却依然被人怀念,于是朝中有人提议,要把这位大人从坟里再掘出来,修理一顿。

		这个建议的提出,充分说明朝廷里有一大帮吃饱了没事干,且心理极其阴暗变态的王八蛋,按说是没什么人理的,可不巧的是,提议的人,是浙党的成员。

		这下就热闹了,许多东林党人闻讯后,纷纷赶来骂仗,痛斥三党,支持张居正。

		说句实话,当年反对张居正的时候,东林党也没少掺合,之所以跑来伸张正义,无非是为了反对而反对,提议是什么并不重要,只要是三党提出的,就是错的,对人不对事,不必当真。

		梅之焕也进来插了句话,且相当不客气:

		“如果江陵\footnote{指张居正。}还在,你们这些无耻小人还敢这样吗?”

		话音刚落,就有人接连上书,表示同意,但让所有人都出乎意料的是,支持他的人,并不是东林党,而是官应震。

		官应震,是楚党的首领,他之所以支持梅之焕,除了两人是老乡,关系不错外,还有一个十分重要的原因:死去的张居正先生,是湖广人。

		这件事情让汪文言认识到,所谓三党,并不是铁板一块,只要动动手脚,就能将其彻底摧毁。

		所以,他找到了梅之焕,拉拢了官应震,开始搞小动作。

		至于他搞了什么小动作,我确实很想讲讲,可惜史书没写,我也不知道,只好省略,反正结论是三党被搞垮了。

		此后的事情,我此前已经讲过了,方从哲被迫退休,东林党人全面掌权,杨涟升任左副都御史,赵南星任吏部尚书,高攀龙任光禄丞,邹元标任左都御史等等。

		之所以让你再看一遍,是要告诉你,在这几个成功男人的背后,是一个沉默的男人。

		\ifnum\theparacolNo=2
	\end{multicols}
\fi
\newpage
\section{毁灭之路}
\ifnum\theparacolNo=2
	\begin{multicols}{\theparacolNo}
		\fi
		这就是东林党成功的全部奥秘,很明显,不太符合其一贯正面光辉的形象,所以如果有所隐晦,似乎可以理解。

		东林党的成功之路到此结束,同学们,现在我们来讲下一课:东林党的失败之路。

		在我看来,东林党之所以失败,是因为自大、狂妄,以及嚣张,不是一个,而是一群。

		如果要在这群人中寻找一个失败的代表,那这个人一定不是杨涟,也不是左光斗,而是赵南星。

		虽然前两个人很有名,但要论东林党内的资历跟地位,他们和赵先生压根就没法比。

		关于赵南星先生的简历,之前已经介绍过了,从东林党创始人顾宪成时代开始,他就是东林党的领导,原先干人事,回家呆了二十多年,人老心不老,又回来干人事。

		一直以来,东林党的最高领导人\footnote{或者叫精神领袖。},是三个人,他们分别是顾宪成、邹元标以及赵南星。

		顾宪成已经死了,天启二年,邹元标也退休了,现在只剩下了赵南星。

		赵先生不但在东林党内有着至高无上的地位,他在政府里,也占据着最牛的职务——吏部尚书。一手抓东林党,一手抓人事权,换句话说,赵南星就是朝廷的实际掌控者。

		但失败之根源,正是此人。

		天启三年(1623年),是一个很特殊的年份,因为这一年,是京察年。

		所谓京察年,也就是折腾年。六年一次,上级考核各级官吏,有冤报冤,有仇报仇,万历年间的几次京察,每年搞得不亦乐乎,今年也不例外。

		按照规定,主持折腾工作的,是吏部尚书,也就是说,是赵南星。

		赵南星是个很负责的人,经过仔细考察,列出了第一批名单,从朝廷滚蛋的名单,包括以下四人:亓诗教、官应震、吴亮嗣、赵兴邦。

		如果你记性好,应该记得这几位倒霉蛋的身份,亓诗教,齐党首领,赵兴邦,浙党骨干,官应震、吴亮嗣,楚党首领。

		此时的朝政局势,大致是这样的,东林党大权在握,三党一盘散沙,已经成了落水狗。

		很明显,虽然这几位兄弟已经很惨了,但赵先生并不干休,他一定要痛打落水狗。

		这是一个很过分的行为,不但要挤掉他们的政治地位,还要挤掉他们的饭碗,实在太不厚道。

		更不厚道的是,就在不久之前,楚党还曾是东林党的同盟,帮助他们掌控政权,结果官应震大人连屁股都没坐热,就被轰走了。

		这就意味着,汪文言先生连哄带骗,好不容易建立的牢固同盟,就此彻底崩塌。

		赵大人在把他们扫地出门的同时,也不忘给这四位下岗人员一个响亮的称号——四凶。为此,他还写了一篇评论文章《四凶论》,以示纪念。

		跟着这四位一起走人的,还有若干人,他们都有着共同的身份:三党成员、落水狗。

		此处不留爷,自有留爷处,既然赵大人不给饭吃,就只好另找饭馆开饭了。

		就在此时,一个人站在他们面前,体贴地对他们说,在这世界上,赵南星并不是唯一的饭馆老板。

		据史料记载,这个人言语温和,面目慈祥,是个亲切的胖老头。

		现在,让我们隆重介绍:明代太监中的极品,宦官制度的终极产物,让刘瑾、王振等先辈汗颜的后来者,比万岁只差一千岁的杰出坏人、恶棍、流氓地痞的综合体——魏忠贤。

		魏忠贤,北直隶\footnote{今河北。}肃宁县人,曾用名先是魏进忠,后是李进忠。

		对于魏公公的出身,历史上一直有两种说法。一种说,他的父母都是贫苦农民;另一种说,他的父母都是街头玩杂耍的。

		说法是不同的,结果是一样的,因为无论农民或杂耍,都是穷人。

		家里穷,自然就没钱给他读书,不读书,自然就不识字,也没法考取功名,升官发财,小孩不上学,父母又不管,只能整天在街上闲逛。

		就这样,少年魏忠贤成为了失学儿童、文盲、社会无业游荡人员。

		但这样的悲惨遭遇,丝毫没有影响魏忠贤的心情,因为他压根儿不觉得自己很惨。

		\subsection{混混的幸福}
		多年前,我曾研习过社会学,并从中发现了这样一条原理:社会垃圾\footnote{俗称混混。},是从来不会自卑的。

		虽然在别人眼中,他们是当之无愧的人渣、败类、计划生育的败笔,但在他们自己看来,能成为一个混混,是极其光荣且值得骄傲的。

		因为他们从不认为自己在混,对于这些人而言,打架、斗殴、闹事,都是美好生活的一部分,抢小孩的棒棒糖和完成一座建筑工程,都是人生意义的自我实现,没有任何区别。

		做了一件坏事,却绝不会后悔愧疚,并为之感到无比光辉与自豪的人,才是一个合格的坏人,一个纯粹的坏人,一个坏得掉渣的坏人。

		魏忠贤,就是这样一个坏人。

		根据史料记载,少年魏忠贤应该是个非常开朗的人,虽然他没钱上学,没法读书,没有工作,却从不唉声叹气,相当乐观。

		面对一没钱、二没前途的不利局面,魏忠贤不等不靠,毅然走上社会,大玩特玩,并在实际生活中确定了自己的人生性格\footnote{市井一无赖尔。}。

		他虽然是个文盲,却能言善辩\footnote{目不识丁,言辞犀利。},没读过书,却无师自通\footnote{性多狡诈。},更为难得的是,他虽然身无分文,却胸怀万贯,具体表现为明明吃饭的钱都没有,还敢跑去赌博\footnote{家无分文而一掷百万。},赌输后没钱给,被打得生活不能自理,依然无怨无悔,下次再来。

		混到这个份上,可算是登峰造极了。

		然而混混魏忠贤,也是有家庭的,至少曾经有过。

		在他十几岁的时候,家里就给他娶了老婆,后来还生了个女儿,一家人过得还不错。

		但为了快乐的混混生活,魏忠贤坚定地抛弃了家庭,在他尚未成为太监之前,四处寻花问柳,城中的大小妓院,都留下了他的足迹,家里仅有的一点钱财,也被他用光用尽。

		被债主逼上门的魏忠贤,终于幡然悔悟,经过仔细反省,他发现,原来自己并非一无所有——还有个女儿。

		于是,他义无反顾地卖掉了自己的女儿,以极其坚定的决心和勇气,为了还清赌债。

		能干出这种事情的人,也就不是人了,魏忠贤的老婆受不了了,离家出走改嫁了。应该说,这个决定很正确,因为按当时情形看,下一个被卖的,很可能是她。

		原本只有家,现在连家都没了,卖无可卖的魏忠贤再次陷入了困境。

		被债主逼上门的魏忠贤,再次幡然悔悟,经过再次反省,他再次发现,原来自己并非一无所有,事实上,还多了件东西。

		只要丢掉这件东西,就能找一份好工作——太监。

		这并非魏忠贤的个人想法,事实上在当地,这是许多人的共识。

		魏忠贤所在的直隶省河间府,一向盛产太监,由于此地距离京城很近,且比较穷,从来都是宫中太监的主要产地,并形成了固定产业,也算是当地创收的一种主要方式。

		混混都混不下去,人生失败到这个程度,必须豁出去了。

		经过短期的激烈思想斗争,魏忠贤树立了当太监的远大理想,然而当他决心在太监的大道上奋勇前进的时候,才惊奇地发现,原来要当一名太监,是很难的。

		一直以来,在人们的心目中,做太监,是迫于无奈,是没办法的办法。

		现在,我要严肃地告诉你,这种观点是错误的。太监,是一份工作,极其热门的工作,而想成为一名太监,是很难的。

		事实上,太监这个职业之所以出现,只是因为一个极其简单的原因——宫里只有女人。

		由于老婆太多,忙不过来,为保证皇帝陛下不戴绿帽子\footnote{这是很有可能的。},宫里不能进男人。可问题是,宫里太大,上千人吃喝拉撒,重活累活得有人干,女的干不了,男的不能进,只好不男不女了。

		换句话说,太监其实就是进城干活的劳工。唯一不同的是,他们的工作地点,是皇宫。

		既然是劳工,就有用工指标,毕竟太监也有个新陈代谢,老太监死了,新太监才能进,也就是说,每年录取太监比例相当低。

		有多低呢?我统计了一下,大致是百分之十到百分之十五,而且哪年招还说不准,今年要不缺人,就不招。

		对于有志于踏入这一热门行业,成为合格太监的众多有志青年而言,这是一个十分残酷的事实,因为这意味着,在一百个符合条件\footnote{割了。}的人中,只有十到十五人,能够成为光荣的太监。

		事实上,自明代中期,每年都有上千名符合条件\footnote{割过了。},却没法入宫的太监\footnote{候选。}在京城等着。

		要知道,万一切了,又当不了太监,那就惨了。虽说太监很吃香,但归根结底,吃香的只是太监的工资收入,不是太监本人。对于这类“割了”的人,人民群众是相当鄙视的。

		所以众多未能成功入选的太监候选人,既不能入宫,也不能回家,只能在京城混。后来混得人越来越多,严重影响京城社会治安的稳定,为此,明朝政府曾颁布法令:未经允许,不得擅自阉割。

		我一直相信,世事皆有可能。

		太监之所以如此热门,除了能够找工作,混饭吃外,还有一个重要的原因——权力。

		公正地讲,明代是一个公正的朝代。任何一个平凡的人,哪怕是八辈贫农,全家只有一条裤子,只要出个能读书的,就能当官,就能进入朝廷,最终掌控无数人的命运。

		唯一的问题在于,这条道路虽然公正,却不平坦。

		魏忠贤当政以后,对自己以前的历史万般遮掩,特别是他怎么当上太监,怎么进宫这一段,是绝口不提,搞得云里雾里,捉摸不透。

		但这种行为,就好比骂自己的儿子是王八蛋一样,最终只能自取其辱。

		他当年的死党,后来的死敌刘若愚太监告诉我们,魏公公不愿提及发家史,是因为违背了太监成长的正常程序——他是自宫的。

		我一直坚信,东方不败是这个世界上最伟大、最杰出,也最有可能的自宫者。

		这绝不仅仅因为他的自宫,绝无混饭吃、找工作的目的,而是为了中华武学的发展。

		真正的原因在于,当我考证了太监阉割的全过程后,才不禁由衷感叹,自宫不仅需要勇气,没准还真得要点功夫。

		很多人不知道,其实阉割是个技术工作,想一想就明白了,从人身上割点东西下来,还是重要部位,稍有不慎,命就没了。

		所以很多年以来,干这行的都是家族产业,代代相传,以割人为业,其中水平最高的,还能承包官方业务,获得官方认证。

		一般这种档次的,不但技术高,能达到庖丁解人的地步,快速切除,还有配套医治伤口,消毒处理,很有服务意识。

		所有说,东方不败能在完全外行的情况下,完成这一复杂的手术,且毫无后遗症\footnote{至少我没看出来。},没有几十年的内功修养,估计是白扯。

		魏忠贤不是武林高手\footnote{不算电影电视。},要他自我解决,实在勉为其难,于是只好寻到上述专业机构,找人帮忙。

		可到地方一问,才知道人家服务好,收费也高,割一个得四五两银子,我估算了一下,合人民币大概是三四千块。

		这可就为难魏公公了,身上要有这么多钱,早拿去赌博翻本,哪犯得着干这个?

		割还是不割,这不是一个问题,问题是,没钱。

		但现实摆在眼前,不找工作是不行了,魏公公心一横——自己动手,前程无忧。

		果不其然,业余的赶不上专业的,手术的后遗症十分严重,出血不止,幸亏好心人路过,帮他止了血。

		成功自宫后,魏忠贤跑去报名,可刚到报名处,问清楚录取条件,当时就晕了。

		事情是这样的,宫里招太监,是有年龄要求的,因为小孩进宫好管,也好教,可是魏忠贤同志自己扳指头一算,今年芳龄已近二十。

		这可要了命了,年龄是硬指标,跟你一起入宫的,都是几岁的孩子,哪个太监师傅愿意带你这么个五大三粗的小伙子,纯粹浪费粮食。

		魏忠贤急了,可急也没用,招聘规定是公开的,你不去问,还能怪谁?

		可事到如今,割也割了,又没法找回来,想再当混混,没指望了,要知道,混混虽然很混,也瞧不起人妖。

		宫进不去,家回不去,魏公公就此开始了他的流浪生涯,具体情况他本人不说,所以我也没法同情他,但据说是过得很惨,到后来,只能以讨饭为生,偶尔也打打杂工。

		万历十六年(1588年),穷困至极的魏忠贤来到了一户人家的府上,在这里,他找到了一份佣人的工作。

		他的命运就此改变。

		一般说来,寻常人家找佣人,是不会找阉人的,魏忠贤之所以成功应聘,是因为这户人家的主人,也是个阉人。

		这个人的名字,叫孙暹,是宫中的太监,准确地说是太监首领,他的职务,是司礼监秉笔太监。

		这个职务,是帮助皇帝批改奏章的,前面说过很多次,就不多说了。

		魏忠贤很珍惜这个工作机会,他起早贪黑,日干夜干,终于有一天,孙暹找他谈话,说是看在他比较老实的份上,愿意保举他进宫。

		万历十七年(1589年),在经历了无数波折之后,魏忠贤终于圆了他的梦,进宫当了一名太监。

		不好意思,纠正一下,是火者。

		实际上,包括魏忠贤在内的所有新阉人,在刚入宫的时候,只是宦官,并不是太监,某些人甚至一辈子也不是太监。

		因为太监,是很难当上的。

		宫里,能被称为“太监”的,都是宦官的最高领导,太监以下,是少监,少监以下,是监丞,监丞以下,还有长随、当差。

		当差以下,就是火者了。

		那么魏火者的主要工作是什么呢?大致包括以下几项:扫地、打水、洗马桶、开大门等等。

		很明显,这不是一份很有前途的工作,而且进宫这年,魏忠贤已经二十一岁了,所以在相当长的时间里,魏忠贤很不受人待见。

		一晃十几年过去了,魏忠贤没有任何成就,也没有任何名头,因为他的年龄比同期入宫的太监大,经常被人呼来喝去,人送外号“魏傻子”。

		但这一切,全都是假象。

		据调查\footnote{本人调查。},最装牛的傻人,与人接触时,一般不会被识破。

		而最装傻的牛人,在与人接触时,一辈子都不会被识破。

		魏忠贤就是后者的杰出代表。

		许多人评价魏忠贤时,总是一把鼻涕一把泪,说大明江山,太祖皇帝,怎么就被这么个文盲、傻子给废掉了。

		持有这种观点的人,才是傻子。

		能在明朝当官,且进入权力核心的这拨人,基本都是高智商的,加上官场沉浮,混了那么多年,生人一来,打量几眼,就能把这人摸得差不多,在他们面前耍花招,那就是自取其辱。

		而在他们的眼中,魏忠贤是一个标准的老实人,年纪大,傻不拉矶的,每天都呵呵笑,长相忠厚老实,人家让他干啥就干啥,欺负他,占他便宜,他都毫不在意,所以从明代,直到今天,很多人认定,这人就是个傻子,能混成后来那样,全凭运气。

		这充分说明,魏公公实在是威力无穷,在忽悠了明代的无数老狐狸后,还继续忽悠着现代群众。

		在我看来,魏忠贤固然是个文盲,却是一个有天赋的文盲,他的这种天赋,叫做伪装。

		一般人在骗人的时候,都知道自己在骗人,而据史料分析,魏公公骗人时,不知道自己在骗人,他骗人的态度,是极其真诚的。

		在宫里的十几年里,他就用这种天赋,骗过了无数老滑头,并暗中结交了很多朋友,其中一个叫做魏朝。

		这位魏朝,也是宫里的太监,对魏忠贤十分欣赏,还帮他找了份工作。这份工作的名字,叫做典膳。

		所谓典膳,就是后宫管伙食的,听起来似乎不怎么样,除了混吃混喝,没啥油水。

		管伙食固然没什么,可关键在于管谁的伙食。

		魏公公的服务对象,恰好就是后宫的王才人。这位王才人的名头虽然不响,但他儿子的名气很大——朱由校。

		正是在那里,魏忠贤第一次遇见了决定他未来命运的两位关键人物——朱常洛父子。

		虽然见到了大人物,但魏忠贤的命运仍无丝毫改变,因为王才人身边有很多太监,他不过是极其普通的一个,平时连跟主子说话的机会都没有。

		而且此时朱常洛还只是太子,且地位十分不稳,随时可能被拿下,所以他老婆王才人混得也不好,还经常被另一位老婆李选侍欺负。

		这么一来,魏忠贤自然也混得差,到万历四十七年(1619年),魏忠贤进宫二十周年纪念之际,他混到了人生的最低点:由于王才人去世,他失业了。

		失业后的魏忠贤无计可施,只能回到宫里,当了一个仓库保管员。

		但被命运挑选的人,注定是不会漏网的,在经过无数极为复杂的人事更替,误打误撞后,魏忠贤竟然摇身一变,又成了李选侍的太监。

		正是在这个女人的手下,魏忠贤第一次露出了他的狰狞面目。

		这位入宫三十年,已五十多岁的老太监突然焕发了青春,他不等不靠,主动接近李选侍,拍马擦鞋,无所不用其极,最终成为了李选侍的心腹。

		因为在他看来,这个掌握帝国未来继承人\footnote{朱由校。},且和他一样精明、自私、无耻的女人,将大有作为。

		万历四十八年(1620年),魏忠贤的机会到了。

		这一年七月,明神宗死了,明光宗即位,李选侍成了候选皇后,朱由校也成了后备皇帝。

		可是好景不长,只过了一个月,明光宗又死了,李选侍成了寡妇。

		当李寡妇不知所措之时,魏忠贤及时站了出来,开导了李寡妇,告诉她,其实你无需失望,因为一个更大的机会,就在你的眼前:只要紧紧抓住年幼的朱由校,成为幕后的操纵者,你得到的,将不仅仅是皇后甚至太后的头衔,而是整个天下。

		这是一个很好的想法,可惜绝非独创,朝廷里文官集团的老滑头们,也明白这一点。

		于是在东林党人的奋力拼杀下,朱由校又被抢了回去,李选侍就此彻底歇菜,魏忠贤虽然左蹦右跳,反应活跃,最终也没逃脱下岗的命运。

		正是在这次斗争中,魏忠贤认识了他宿命中的对手,杨涟。

		杨涟,是一个让魏忠贤寒毛直竖的人物。

		两人第一次相遇,是在抢人的路上。杨涟抢走朱由校,魏忠贤去反抢,结果被骂了回来,哆嗦了半天。

		第二次相遇,是他奉命去威胁杨涟,结果被杨涟威胁了,杨大人还告诉他,再敢作对,就连你一块收拾。

		魏忠贤相当识趣,掉头就走,从此以后,再不敢惹这人。

		总而言之,在魏忠贤的眼中,杨涟是个不贪财,不好色,不怕事,几乎没有任何弱点,还特能折腾的人,而要对付这种人,李选侍是不够分量的,必须寻找一个新的主人。

		然而很遗憾,在当时的宫里,比李选侍还狠的,只有东林党,就算魏太监想进,估计人家也不肯收。

		看起来是差不多了,毕竟魏公公都五十多了,你要告诉他,别灰心,不过从头再来,估计他能跟你玩命。

		但拯救他的人,终究还是出现了。

		许多人都知道,天启皇帝朱由校是很喜欢东林党的,也很够意思,继位一个月,就封了很多人,要官给官,要房子给房子。

		但许多人不知道,他第一个封的并不是东林党,继位后第十天,他就封了一个女人,封号“奉圣夫人”。

		这个女人姓客,原名客印月,史称“客氏”。

		客,是一个非常特别的姓氏,估计这辈子,你也很难遇上一个姓客的,而这位客小姐,那就更特别了,可谓五百年难得一遇的极品。

		进宫之前,客印月是北直隶保定府村民候二的老婆,相貌极其妖艳,且极其早熟,啥时候结婚没人知道,反正十八岁就生了儿子。

		她的命运就此彻底改变。因为就在同一年,宫里的王才人生出了朱由校。

		按照惯例,必须挑选合适的乳母去喂养朱由校,经过层层选拔,客印月战胜众多竞争对手,成功入宫。

		刚进宫时,客印月极为勤奋,随叫随到,两年后,她的丈夫不幸病逝,但客印月表现了充分的职业道德,依然兢兢业业完成工作,在宫里混得相当不错。

		但很快,宫里的人就发现,这是一个有问题的女人。

		有群众反映,客印月常缺勤出宫,行踪诡异,经常出入各种娱乐场所,后经调查,客印月有生活作风问题,时常借机外出幽会。

		作为宫中的乳母,如此行径,结论是清晰的,情节是严重的,但处罚是没有的。有人议论,没人告发。

		因为这个看似普通的乳母,一点也不普通。

		按说乳母这份活,也就是个临时工,孩子长大了就得走人,该干嘛干嘛去,可是客小姐是个例外,朱由校断奶,她没走,朱由校长大了,她也没走,朱由校十六岁,当了皇帝,她还是没走。

		根据明朝规定,皇子长到六岁,乳母必须出宫,但客印月偏偏不走,硬是多混了十多年,也没人管,因为皇帝不让她走。

		不但不让走,还封了个“奉圣夫人”,这位夫人的架子还很大,在宫中可以乘坐轿子,还有专人负责接送。要知道,内阁大学士刘一璟,二品大员,都六十多了,在朝廷混了一辈子,进出皇宫也得步行。

		非但如此,逢年过节,皇帝还要亲自前往祝贺,请她吃饭。夏天,给她搭棚子,送冰块;冬天给她挖坑,烧炭取暖。宫里给她分了房子,宫外也有房子,还是黄金地段,就在今天北京的正义路上,步行至天安门,只需十分钟,极具升值潜力。

		她家还有几百个仆人伺候,皇宫随意出入,想住哪里就住哪里,想怎么住就怎么住。

		所谓客小姐,说破天也就是个保姆,如此得势嚣张,实在很不对劲。

		一年之后,这位保姆干出了一件更不对劲的事情。

		天启二年(1622年),明熹宗朱由校结婚了,皇帝嘛,娶个老婆很正常,谁也没话说。

		可是客阿姨\footnote{三十五了。}不高兴了,突然跳了出来,说了一些不着边际的话,用史籍《明季北略》的话说,是“客氏不悦”。

		皇帝结婚,保姆不悦,这是一个相当无厘头的举动。更无厘头的是,朱由校同志非但没有“不悦”,还亲自跑到保姆家,说了半天好话,并当即表示,今后我临幸的事情,就交给你负责了,你安排哪个妃子,我就上哪过夜,绝对服从指挥。

		这也太过分了,很多人都极其不满,说你一个保姆,老是赖在宫里,还敢插手后宫,某些胆大的大臣先后上疏,要求客氏出宫。

		这事说起来,确实不大光彩,皇帝大人迫于舆论压力,就只好同意了。

		但在客氏出宫当天,人刚出门,熹宗就立刻传谕内阁,说了这样一段话:今日出宫,午膳至晚未进,暮思至晚,痛心不已,着时进宫奉慰,外廷不得烦激。

		这段话的意思是客氏今天出宫,我中午饭到现在都没吃,整天都在想念她,非常痛心。还是让她回来安慰我吧,你们这些大臣不要再烦我了!

		傻子都知道了,这两个人之间,必定存在着一种十分特殊的关系。

		对此,后半生竭力揭批魏忠贤,猛挖其人性污点的刘若愚同志曾在著作中,说过这样一句话:
		\begin{quote}
			\begin{spacing}{0.5}  %行間距倍率
				\textit{{\footnotesize
							\begin{description}
								\item[\textcolor{Gray}{\faQuoteRight}] 倏出倏入,人多讶之,道路流传,讹言不一,尚有非臣子之所忍言者。
							\end{description}
						}}
			\end{spacing}
		\end{quote}

		这句话的意思是,经常进进出出,许多人都惊讶,也有很多谣言,那些谣言,做臣子的是不忍心提的。

		此言非同小可。

		所谓臣子不忍心提,那是瞎扯,不敢提倒是真的。

		朱由校的母亲王才人死得很早,他爹当了几十年太子,自己命都难保,这一代人的事都搞不定,哪有时间关心下一代。所以朱由校基本算是客氏养大的。

		十几年朝夕相处,而且客氏又是“妖艳美貌,品行淫荡”,要有点什么瓜田李下,鸡鸣狗盗,似乎也能理解。

		就年龄而言,客氏比朱由校大十八岁,按说不该引发猜想,可惜明代皇帝在这方面,是有前科的。比如成化年间的明宪宗同志,他的保姆万贵妃,就比他大十九岁,后来还名正言顺地搬被子住到一起。就年龄差距而言,客氏也技不如人,没能打破万保姆的记录,如此看来,传点绯闻,实在比较正常。

		当然,这两人之间到底有没有猫腻,谁都不知道,知道也不能写,但可以肯定的是,皇帝陛下对于这位保姆,是十分器重的。

		客氏就是这么个人物,皇帝捧,大臣让,就连当时的东厂提督太监和内阁大臣都要给她几分面子。

		对年过半百的魏忠贤而言,这个女人,是他成功的唯一机会,也是最后的机会。

		于是,他下定决心,排除万难,一定要争取这个人。

		而争取这个人的最好方法,就是让她成为自己的老婆。

		你没有看错,我没有写错,事实就是如此。

		虽然魏忠贤是个太监,但他是可以找老婆的。

		作为古代宫廷的传统,太监找老婆,有着悠久的历史,事实上,还有专用名词——对食。

		对食,就是大家一起吃饭,但在宫里,你要跟人对食,人家不一定肯。

		历代宫廷里,有很多宫女,平时不能出宫,且没啥事干,且不能嫁人,长夜漫漫寂寞难耐,闲着也是闲着,许多人就在宫中找对象,可是宫里除皇帝外,又没男人,找来找去,长得像男人的,只有太监。

		没办法,就这么着吧。

		虽说太监不算男人,但毕竟不是女人,反正有名无实,大家一起过日子,说说话,也就凑合了。

		这种现象,即所谓对食。自明朝开国以来,就是后宫里的经典剧目,经常上演,一般皇帝也不怎么管,但要遇到凶恶型的,还是相当危险。比如明成祖朱棣,据说被他看见,当头就是一刀,眼睛都不眨。

		到明神宗这代,开始还管管,后来他都不上朝,自然就不管了。

		但魏忠贤要跟客氏“对食”,还有一个极大的障碍:客氏已经有对象了。

		其实对食,和谈恋爱也差不多,也有第三者插足,路边野花四处踩,寻死觅活等俗套剧情,但这一次,情况有点特殊。

		因为客氏的那位对食,恰好就是魏朝。

		之前我说过了,魏朝是魏忠贤的老朋友,还帮他介绍过工作,关系相当好,所谓“朋友妻,不可欺”,实在是个问题。

		但魏忠贤先生又一次用事实证明了他的无耻,面对朋友的老婆,二话不说,光膀子就上,毫无心理障碍。

		但人民群众都知道,要找对象,那是要条件的,客氏就不用说了,皇帝的乳母,宫里的红人,不到四十,“妖艳美貌,品行淫荡”,而魏朝是王安的下属,任职乾清宫管事太监,还管兵仗局,是太监里的成功人士,可谓门当户对。

		相比而言,魏忠贤就寒掺多了,就一管仓库的,靠山也倒了,要挖墙脚,希望相当渺茫。

		但魏忠贤没有妄自菲薄,因为他有一个魏朝没有的优点:胆儿大。

		作为曾经的赌徒,魏忠贤胆子相当大,相当敢赌。表现在客氏身上,就是敢花钱,明明没多少钱,还敢拼命花,不但拍客氏马屁,花言巧语,还经常给她送名贵时尚礼物,类似今天送法国化妆品,高级香水,相当有杀伤力。

		这还不算,他隔三差五请客氏吃饭。吃饭的档次是“六十肴一席,费至五百金”。翻译成白话就是,一桌六十个菜,要花五百两银子。

		五百两银子,大约是人民币四万多,就一顿饭,没落太监魏忠贤的消费水平大抵如此。

		人穷不要紧,只要胆子大,这就是魏忠贤公公的人生准则。其实这一招到今天,也还能用,比如你家不富裕,就六十万,但你要敢拿这六十万去买个戒指求婚,没准真能蒙个把人回来。

		外加魏太监不识字,看上去傻乎乎的,老实得不行,实在是宫中女性的不二选择,于是,在短短半年内,客氏就把老情人丢到脑后,接受了这位第三者。

		然而在另外一本史籍中,事情的真相并非如此。

		几年后,一个叫宋起凤的人跟随父亲到了京城。因为他家和宫里太监关系不错,所以经常进宫转悠,在这里他看到很多,也听到了很多。

		几十年后,他把自己当年的见闻写成了一本书,取名《稗说》。

		所谓稗,就是野草。宋起凤先生的意思是,他的这本书,是野路子,您看了爱信不信,就当图个乐,他不在乎。

		但就史料价值而言,这本书是相当靠谱的。因为宋起凤不是东林党,不是阉党,不存在立场问题,加上他在宫里混的时间长,许多事是亲身经历,没有必要胡说八道。

		这位公正的宋先生,在他的野草书里,告诉我们这样一句话:

		“魏虽腐余,势未尽,又挟房中术以媚,得客欢。”

		这句话,通俗点说就是,魏忠贤虽然割了,但没割干净。后半句儿童不宜,我不解释。

		按此说法,有这个优势,魏忠贤要抢魏朝的老婆,那简直是一定的。

		能说话,敢花钱,加上还有太监所不及的特长,魏忠贤顺利地打败了魏朝,成为了客氏的新对食。

		说穿了,对食就是谈恋爱,谈恋爱是讲规则的,你情我愿,谈崩了,女朋友没了,回头再找就是了。

		但魏朝比较惨,他找不到第二个女朋友。

		因为魏忠贤是个无赖,无赖从来不讲规则,他不但要抢魏朝的女朋友,还要他的命。天启元年(1620年),在客氏的配合下,魏朝被免职发配,并在发配的路上被暗杀。

		魏忠贤之所以能够除掉魏朝,是因为王安。

		作为三朝元老太监,王安已经走到了人生的顶点,现在的皇帝,乃至于皇帝他爹,都是他扶上去的,加上东林党都是他的好兄弟,那真是天下无敌,比东方不败猛了去了。

		可是王安也有一个致命的弱点——喜欢高帽子。

		高帽子,就是拍马屁。所谓“千穿万穿,马屁不穿”,真可谓是至理名言,无论这人多聪明,多精明,只要找得准,拍得狠,都不堪一击。

		自盘古开天辟地以来,我们就知道,马屁,是有声音的。

		但魏忠贤的马屁,打破了这个俗套,达到马屁的最高境界——无声之屁。

		每次见王安,魏忠贤从不主动吹捧,也不说话,只是磕头,王安不叫他,他就不去,王安不问他,他就不说话。王安跟他说话,他不多说,态度谦恭点到即止。

		他不来虚的,尽搞实在的,逢年过节送东西,还是猛送,礼物一车车往家里拉。于是当魏朝和魏忠贤发生争斗的时候,他全力支持了魏忠贤,赶走了魏朝。

		但他并不知道,魏忠贤的目标并不是魏朝,而是他自己。

		此时的魏忠贤已经站在了门槛上,只要再走一步,他就能获取至高无上的权力。

		但是王安,就站在他的面前。必须铲除此人,才能继续前进。

		跟之前对付魏朝一样,魏忠贤毫无思想障碍,朋友是可以出卖的,上级自然可以出卖,作为一个无赖、混混、人渣,无时无刻,他始终牢记自己的本性。

		可是怎么办呢?

		王安不是魏朝,这人不但地位高,资格老,跟皇帝关系好,路子也猛,东林党的杨涟、左光斗都经常去他家串门。

		要除掉他,似乎绝无可能。

		但是魏忠贤办到了,用一种匪夷所思的方式。

		天启元年(1620年),司礼监掌印太监卢受因为犯了事,被罢免了。

		在当时,卢受虽然地位高,势力却不大,所以这事并不起眼。

		王安,正是栽在了这件并不起眼的事情上。

		前面讲过,在太监里面,最牛的是司礼太监,包括掌印太监一人,秉笔太监若干人。

		作为司礼监的最高领导,按照惯例,如职位空缺,应该由秉笔太监接任。在当时而言,就是王安接任。

		必须说明,虽然王安始终是太监的实际领导,但他并不是掌印太监,具体原因无人知晓。可能是这位仁兄知道枪打出头鸟,所以死不出头,想找人去顶缸。

		但这次不同了,卢受出事后,最有资历的就剩下他,只能自己干了。

		但魏忠贤不想让他干,因为这个位置太过重要,要让王安坐上去,自己要出头,只能等下辈子了。

		可是事实如此,生米做成了熟饭,魏忠贤无计可施。

		王安也是这么想的,他打点好一切,并接受了任命。按照以往的惯例,写了一封给皇帝的上疏。主要意思无非是我无才无能,干不了,希望皇上另找贤能之类的话。

		接受任命后,再写这些,似乎比较虚伪,但这也是没办法,在我们这个有着光荣传统的地方,成功是不能得意的,得意是不能让人看见的。

		几天后,他得到了皇帝的回复:同意,换人。

		王安自幼入宫,从倒马桶干起,熬到了司礼监,一向是现实主义者,从不相信什么神话。但这次,他亲眼看见了神话。

		写这封奏疏,无非是跟皇帝客气客气,皇帝也客气客气,然后该干嘛干嘛,突然来这么一杠子,实在出人意料。

		但更出人意料的是,没过多久,他就被勒令退休,彻底赶出了朝廷。而那个他亲手捧起的朱由校,竟然毫无反应。

		魏忠贤,确实是一个聪明绝顶的人,在苦思冥想后,他终于找到了这个不是机会的机会:你要走,我批准,实在是再自然不过。

		但这个创意的先决条件是,皇帝必须批准,这是有难度的。因为皇帝大人虽说喜欢当木工,也没啥文化,但要他下手坑捧过他的王公公,实在需要一个理由。

		魏忠贤帮他找到了这个理由:客氏。

		乳母、保姆、外加还可能有一腿,凭如此关系,要他去办掉王公公,应该够了。

		王安失去了官职,就此退出政治舞台,凄惨离去。此时他才明白,几十年的宦海沉浮,尔虞我诈的权谋,扶植过两位皇帝的功勋,都抵不上一个保姆。

		心灰意冷的他打算回去养老,却未能如愿。因为一个人下定决心,要斩草除根,这人不是魏忠贤。

		以前曾有个人问我,在整死岳飞的那几个人里,谁最坏?

		我不假思索地回答,当然是秦桧。

		于是此人脸上带着欠揍的表情,微笑着对我说,不对,是秦桧他老婆。

		我想了一下,对他说:你是对的。

		我想起了当年读过的那段记载,秦桧想杀岳飞,却拿不定主意干不干,于是他的老婆,李清照的表亲王氏告诉他,一定要干,必须要干,不干不行,于是他干了。

		魏忠贤的情况大致如此,这位仁兄虽不认朋友,倒还认领导,想来想去,对老婆客氏说,算了吧。

		然后,客氏对他说了这样几句话:

		“移宫时,对外传递消息,说李选侍挟持太子的,是王安;东林党来抢人,把太子拉走的,是王安;和东林党串通,逼李选侍迁出乾清宫的,还是王安。此人非杀不可!”

		说这句话的时候,她的表情十分严肃,态度十分认真。

		女人比男人更凶残,信乎。

		魏忠贤听从了老婆的指示,他决定杀掉王安。

		这事很难办,皇帝大人比魏忠贤厚道,他固然不用王安,却绝不会下旨杀他。

		但在魏忠贤那里,就不难办了。因为接替王安,担任司礼监掌印太监的,是他的心腹王体乾,而他自己,是司礼监秉笔太监兼东厂提督太监,大权在握,想怎么折腾都行,反正皇帝大人每天都做木匠,也不大管。

		很快,王安就在做苦工的时候,发生了意外,夜里突然就死掉了,后来报了个自然死亡,也就结了。

		至此,魏忠贤通过不懈的无耻和卑劣,终于掌握了东厂的控制权,成为了最大的特务。皇帝的往来公文,都要经过他的审阅,才能通过,最少也是一言八鼎了。

		然而,每次有公文送到时,他都不看,因为他不识字。

		在文盲这一点上,魏忠贤是认账且诚实的,但他并没有因此耽误国家大事,总是把公文带回家,给他的狗头军师们研究,有用的用,没用的擦屁股垫桌脚,做到物尽其用。

		\ifnum\theparacolNo=2
	\end{multicols}
\fi
\newpage
\section{道统}
\ifnum\theparacolNo=2
	\begin{multicols}{\theparacolNo}
		\fi
		入宫三十多年后,魏忠贤终于走到了人生的高峰。

		但还不是顶峰。

		战胜了魏朝,除掉了王安,搞定了皇帝,但这还不够,要想成为这个国家的真正统治者,必须面对下一个,也是最后一个敌人——东林党。

		于是,在成为东厂提督太监后不久,魏忠贤经过仔细思考、精心准备,对东林党发动攻击。

		具体行动包括,派人联系东林党的要人,包括刘一璟、周嘉谟、杨涟等人,表示自己刚上来,许多事情还望多多关照,并多次附送礼物。

		此外,他还在公开场合,赞扬东林党的某些干将,兴奋之情溢于言表。

		更让人感动的是,他多次在皇帝面前进言,说东林党的赵南星是国家难得的人才,工作努力认真,值得信赖,还曾派自己的亲信上门拜访,表达敬意。

		除去遭遇车祸失忆,意外中风等不可抗力因素,魏忠贤突然变好的可能性,大致是0%,所以结论是,这些举动都是伪装。在假象的背后,隐藏着不可告人的秘密。

		这个不可告人的秘密就是:魏忠贤想跟东林党做朋友。

		有必要再申明一次,这句话我没有写错。

		其实我们这个国家的历史,一向是比较复杂的。所谓你中有我,我中有你,能凑合就凑合,能糊弄就糊弄。向上追溯,真正执着到底,绝不罢休的,估计只有山顶洞人。

		魏忠贤并不例外,他虽然不识字,却很识相。

		他非常清楚,东林党这帮人不但手握重权,且都是读书人,其实手握重权并不可怕,书呆子才可怕。

		自古以来,读书人大致分为两种,一种叫文人,另一种叫书生。文人是“文人相轻”,具体特点为比较无耻外加自卑。你好,他偏说坏;你行,他偏说不行;胆子还小,平时骂骂咧咧,遇上动真格的,又把头缩回去,实在是相当之扯淡。

		而书生的主要特点,是“书生意气”,表现为二杆子加一根筋。好就是好,不好就是不好,认死理,平时不惹事,事来了不怕死。关键时刻敢于玩命,文弱书生变身钢铁战士,不用找电话亭,不用换衣服,眨眼就行。

		当年的读书人,还算比较靠谱,所以在东林党里,这两种人都有,后者占绝大多数,形象代言人就是杨涟,咬住就不撒手,相当头疼。

		这种死脑筋,敢于乱来的人,对于见机行事、欺软怕硬的无赖魏忠贤而言,实在是天然的克星。

		所以魏忠贤死乞白赖地要巴结东林党,他实在是不想得罪这帮人。这世道,大家都不容易,混碗饭吃嘛,我又不想当皇帝,最多也就是个成功太监,你们之前跟王安合作愉快,现在我来了,不过是换个人,有啥不同的。

		对于魏忠贤的善意表示,东林党的反应是这样的:上门的礼物,全部退回去,上门拜访的,赶走。

		最不给面子的,是赵南星。

		在东林党人中,魏忠贤最喜欢赵南星,因为赵南星和他是老乡,容易上道,所以他多次拜见,还人前人后,逢人便夸赵老乡如何如何好。

		可是赵老乡非但不领情,拒不见面。有一次,还当着很多人的面,针对魏老乡的举动,说了这样一句话:宜各努力为善。

		联系前后关系,这句话的隐含意思是,各自干好各自的事就行了,别动歪心思,没事少烦我。

		魏忠贤就不明白了,王安你们都能合作,为什么不肯跟我合作呢?

		其实东林党之所以不肯和魏忠贤合作,不是因为魏忠贤是文盲,不是因为他是无赖,只是因为,他不是王安。

		没有办法,书生都是认死理的。虽然从本质和生理结构上讲,王安和魏忠贤实在没啥区别,都是太监,都是司礼监,都管公文,但东林党一向是做熟不如做生,对人不对事,像魏忠贤这种无赖出身,行为卑劣的社会垃圾,他们是极其鄙视的。

		应该说,这种思想是值得尊重的,值得敬佩的,却是绝对错误的。

		因为他们并不知道,政治的最高技巧,不是你死我活,而是妥协。

		魏忠贤愤怒了,他的愤怒是有道理的,不仅是因为东林党拒绝合作,更重要的是,他感觉自己被鄙视了。

		这个世上的人分很多类,魏忠贤属于江湖类,这种人从小混社会,狐朋狗友一大串,老婆可以不要,女儿可以不要,只有面子,是不能不要的。东林党的蔑视,给他那污浊不堪的心灵以极大的震撼,他痛定思痛,幡然悔悟,毅然做出了一个决定:

		既然不给脸,那就撕破脸吧!

		但魏公公很快就发现,要想撕破脸,一点也不容易。

		因为他是文盲。

		解决魏朝、王安,只要手够狠,心够黑就行,但东林党不同,这些人都是知识分子,至少也是个进士,擅长朝廷斗争,这恰好是魏公公的弱项。

		在朝廷里干仗,动刀动枪是不行的,一般都是骂人打笔仗,技术含量相当之高,多用典故成语,保证把你祖宗骂绝也没一脏字,对于字都不识的魏公公而言,要他干这活,实在有点勉为其难。

		为了适应新形势下的斗争,不至于被人骂死还哈哈笑,魏公公决定找几个助手,俗称走狗。

		最早加入,也最重要的两个走狗,分别是顾秉谦与魏广微。

		顾秉谦,万历二十三年(1595年)进士,坏人。

		此人翰林出身,学识过人,无耻也过人,无耻到魏忠贤没找他,他就自己上门去了。

		当时他的职务是礼部尚书,都七十一了,按说干几年就该退休,但这孙子偏偏人老心不老,想更进一步,大臣又瞧不上他,索性投了太监。

		改变门庭倒也无所谓,这人最无耻的地方在于,他干过这样一件事:

		有一次为了升官,顾秉谦先生不顾自己七十高龄,带着儿子登门拜访魏忠贤,说了这样一段话:

		“我希望认您做父亲,但又怕您觉得我年纪大,不愿意,索性让我的儿子给您做孙子吧!”

		顾秉谦,嘉靖二十九年(1550年)生,魏忠贤,隆庆二年(1568年)出生。顾秉谦比魏忠贤大十八岁。

		无耻,无语。

		魏广微,万历三十二年(1604年)进士,可好可坏的人。

		魏广微的父亲,叫做魏允贞,魏允贞有一个最好的朋友,叫做赵南星。

		万历年间,魏允贞曾当过侍郎。他和赵南星的关系很好,两人曾有八拜之交,用今天话说,是拜过把子的把兄弟。

		魏广微的仕途比较顺利,考中翰林,然后步步高升,天启年间,就当上了礼部侍郎。

		按说这个速度不算慢,可魏先生是个十分有上进心的人,为了实现跨越性发展,他找到了魏忠贤。

		魏公公自然求之不得,仅过两年,就给他提级别,从副部长升到部长,并让他进入内阁,当上了大学士。

		值得表扬的是,魏广微同志有了新朋友,也不忘老朋友。上任之后,第一件事就去拜会父亲当年的老战友赵南星。

		但赵南星没有见他,让他滚蛋的同时,送给了他四个字:

		“见泉无子!”

		魏广微之父魏允贞,字见泉。

		这是一句相当狠毒的话,你说我爹没有儿子,那我算啥?

		魏广微十分气愤。

		气愤归气愤,他还是第二次上门,要求见赵南星。

		赵南星还是没见他。

		接下来,魏广微做出了一个出人意料的举动,他又去了。

		魏先生不愧为名门之后,涵养很好,当年刘备请诸葛亮出山卖命,也就三次,魏广微不要赵大人卖命,吃顿饭聊聊天就好。

		但赵南星还是拒而不见。

		面对着紧闭的大门,魏广微怒不可遏,立誓,与赵南星势不两立。

		魏广微之所以愤怒,见不见面倒是其次,关键在于赵南星坏了规矩。

		当时的赵南星,是吏部尚书,人事部部长,魏广微却是礼部尚书,东阁大学士。虽说两人都是部长,但魏广微是内阁成员,相当于副总理,按规矩,赵部长还得叫他领导。

		但魏大学士不计较,亲自登门,还三次,您都不见,实在有点太不像话。

		就这样,这个可好可坏的人,在赵南星的无私帮助下,变成了一个彻底的坏人。

		除了这两人外,魏忠贤的党羽还有很多,如冯铨、施凤来、崔呈秀、许显纯等等,后人统称为:五虎、五彪、十狗、十孩儿,光这四拨人加起来,就已有三十个。

		这还是小儿科,魏公公的手下,还有二十孩儿、四十猴孙、五百义孙,作为一个太监,如此多子多孙,实在是有福气。

		我曾打算帮这帮太监子孙亮亮相,搞个简介,起码列个名,但看到“五百义孙”之类的字眼时,顿时失去了勇气。

		其实东林党在拉山头、搞团体等方面,也很有水平。可和魏公公比起来,那就差得多了。

		因为东林党的入伙标准较高,且渠道有限:要么是同乡\footnote{乡党。},同事\footnote{同科进士。},要么是座主\footnote{师生关系。},除个别有特长者外\footnote{如汪文言。},必须是高级知识分子\footnote{进士或翰林。},还要身家清白,没有案底\footnote{贪污受贿。}。

		而魏公公就开放得多了,他本来就是无赖、文盲,还兼职人贩子\footnote{卖掉女儿。},要找个比他素质还低的人,那是比较难的。

		所以他收人的时候,非常注意团结。所谓英雄莫问出处,富贵不思来由,阿猫阿狗无所谓,能干活就行,他手下这帮人也还相当知趣,纷纷用“虎”、“彪”、“狗”、“猴”自居,甭管是何禽兽,反正不是人类。

		这帮妖魔鬼怪构成很复杂,有太监、特务、六部官员、地方官、武将,涉及各个阶层,各个行业,百花齐放。

		虽然他们来自不同领域,但有一点是相同的:他们都是经过精挑细选,纯度极高的人渣。

		比如前面提到的四位仁兄,即很有代表性:

		崔呈秀,原本是一贪污犯,收了人家的钱,被检举丢了官,才投奔魏公公。

		施凤来,混迹朝廷十余年,毫无工作能力,唯一的长处是替人写碑文。

		许显纯,武进士出身,锦衣卫首领,残忍至极,喜欢刑讯逼供,并有独特习惯:杀死犯人后,将其喉骨挑出,作为凭证,或作纪念。

		但相对而言,以上三位还不够份,要论王八蛋程度,还是冯铨先生技高一筹。

		这位仁兄全靠贪污起家,并主动承担陷害杨涟、左光斗等人的任务,唯恐坏事干得不够多,更让人称奇的是,后来这人还主动投降了清朝,成为了不知名的汉奸。

		短短一生之中,竟能集贪官、阉党、汉奸于一体,如此无廉耻,如此无人格,说他是禽兽,那真是侮辱了禽兽。

		综上所述,魏忠贤手下这帮人,在工作和生活中,有着这样一个特点:

		什么都干,就是不干好事,什么都要,就是不要脸。

		其实阉党之中的大多数人,都曾是三党的成员,在彻底出卖自己的灵魂和躯体,加入这个温馨的集体,成为毫无廉耻的禽兽之前,他们曾经也是人。

		多年以前,当他们刚踏入朝廷的时候,都曾品行端正满怀理想,立志以身许国,匡扶天下,公正地对待每一个人,谨言慎行,并最终成为一个青史留名的伟人。

		但他们终究倒下了,在残酷的斗争、仕途的磨砺、党争的失败面前,他们失去了最后的勇气和尊严,并最终屈服,屈服于触手可及的钱财、权位和利益。

		魏忠贤明白,坚持理想的东林党,是绝不可能跟他合作的,要想继续好吃好喝混下去,就必须解决这些人,现在,他准备摊牌了。

		但想挑事,总得有个由头,东林党这帮人都是道德先生,也不怎么收黑钱,想找茬整顿他们,是有相当难度的。

		考虑再三之后,魏忠贤找到了一个看似完美的突破口——汪文言。

		作为东林党的智囊,汪文言起着极其关键的作用,左推右挡来回忽悠,拥立了皇帝,搞垮了三党,人送外号“天下第一布衣”。

		但在魏忠贤看来,这位布衣有个弱点:他没有功名,不能做官,只能算是地下党。对这个人下手,即不会太显眼,又能打垮东林党的支柱,实在是一举两得。

		所以在王安死后,魏忠贤当即指使顺天府府丞绍辅忠,弹劾汪文言。

		要整汪文言,是比较容易的。这人本就是个老油条,除东林党外,跟三党也很熟。后来三党垮了,他跟阉党中的许多人关系也很铁,经常来回倒腾事儿,收人钱财,替人消灾,底子实在太不干净。

		更重要的是,他的老东家王安倒了,靠山没了,自然好收拾。

		事实恰如所料,汪文言一弹就倒,监生的头衔没收,还被命令马上收拾包裹滚蛋。

		汪文言相当听话,也不闹,乖乖地走人了,可他还没走多远,京城里又来了人,从半道上把他请了回去——坐牢。

		赶走汪文言,是不够的,魏忠贤希望,能把这个神通广大又神秘莫测的人一棍子打死,于是他指使御史弹了汪先生第二下,把他直接弹进了牢房里。

		魏忠贤终于满意了,行动进行极其顺利,汪文言已成为阶下囚,一切都已准备妥当,下面……

		下面没有了。

		因为不久之后,汪文言就出狱了。

		此时的魏忠贤是东厂提督太监、掌控司礼监、党羽遍布天下,而汪先生是个没有功名,没有身份,失去靠山的犯人。并且魏公公很不喜欢汪文言,很想把他打翻在地,再踏上一只脚,这看上去,似乎是件十分容易的事情。毕竟连汪文言的后台王安,都死在了魏忠贤的手中。

		无论如何,他都不应该、不可能出狱。

		然而他就是出狱了。

		他到底是怎么出狱的,我不知道,反正是出来了,成功自救,魏公公也毫无反应,王安都没有办到的事情,他办到了。

		而且这位仁兄出狱之后,名声更大,赵南星、左光斗、杨涟都亲自前来拜会慰问,上门的人络绎不绝,用以往革命电影里的一句话:坐牢还坐出好来了。

		更出人意料的是,不久之后,朝廷首辅叶向高主动找到了他,并任命他为内阁中书。

		所谓内阁中书,大致相当于国务院办公厅主任,是个极为重要的职务。汪文言先生连举人都没考过,竟然捞到这个位置,实在耸人听闻。

		而对这个严重违背常规的任命,魏公公竟然沉默是金,什么话都不说。因为他已经意识到,自己还没有足够的实力,去战胜这个神通广大的人。

		于是,魏忠贤停止了行动,他知道,要打破目前的僵局,必须继续等待。

		此后的三年里,悄无声息之中,他不断排挤东林党,安插自己的亲信,投靠他的人越来越多,他的党羽越来越庞大,实力越来越强,但他仍在沉默中等待。

		因为他已看清,这个看似强大的东林党,实际上非常脆弱,吏部尚书赵南星不可怕,佥都御史左光斗不可怕,甚至首辅叶向高,也只是一个软弱的盟友。

		真正强大的,只有这个连举人都考不上,地位卑微,却机智过人,狡猾到底的汪文言,要解决东林党,必须除掉这个人,没有任何捷径。

		这是一件非常冒险的事,魏忠贤不喜欢冒险,所以他选择等待。

		但事情的发展,超出了所有人的预料,包括魏忠贤在内。

		天启四年(1624年)吏科给事中阮大铖上书,弹劾汪文言、左光斗互相勾结,祸乱朝政。

		热闹就此开始,阉党纷纷加入,趁机攻击东林党,左光斗也不甘示弱,参与论战,朝廷上下,口水滔滔,汪文言被免职,连首辅叶向高也申请辞职,乱得不可开交。

		但讽刺的是,对于这件事,魏忠贤事先可能并不知道。

		这事之所以闹起来,无非是因为吏科都给事中退了,位置空出来,阮大铖想要进步,就开始四处活动,拉关系。

		偏偏东林党不吃这套,人事部长赵南星听说这事后,索性直接让他滚出朝廷,连给事中都不给干。阮大铖知道后,十分愤怒,决定告左光斗的黑状。

		这是句看上去前言不搭后语的话,赵南星让他滚,关左光斗何事?

		原因在于,左光斗是阮大铖的老乡,当年阮大铖进京,就是左光斗抬举的。所以现在他升不了官,就要找左光斗的麻烦。

		看起来,这个说法仍然比较乱,不过跟“因为生在荆楚之地,所以就叫萌萌”之类的逻辑相比,这种想法还算正常。

		这位逻辑“还算正常”的阮大铖先生,真算是奇人。可以多说几句。后来他加入了阉党,跟着魏忠贤混,混砸了又跑到南京,跟着南明混,南明混砸了,他又加入满清,在满清军营里,他演出了人生中最精彩,最无耻的一幕。

		作为投降的汉奸,他毫无羞耻之心,还经常和满清将领说话。白天说完,晚上接着说,说得人家受不了,对他说:您口才真好,可我们明天早起还要打仗,早点洗了睡吧。

		此后不久,他因急于抢功跑得太快,猝死于军中。

		但在当时,阮大铖先生这个以怨报德的黑状,只是导火索。真正让魏公公极为愤怒,痛下杀手的,是另一件事。准确地说,是另一笔钱。

		其实一直以来,魏公公虽和东林党势不两立,却只有公愤,并无私仇。但几乎就在阮大铖上书的同一时刻,魏公公得到消息,他的一笔生意黄了,就黄在东林党的手上。

		这笔生意值四万两银子,和他做生意的人,叫熊廷弼。

		希望大家还记得这兄弟,自从回京后,他已经被关了两年多了,由于情节严重,上到皇帝下到刑部,倾向性意见相当一致——杀。

		事到如今,只能开展自救了,熊廷弼开始积极活动,找人疏通关系,希望能送点钱,救回这条命。

		七转八转,他终于找到了一位叫做汪文言的救星,据说此人神通广大,手到擒来。

		汪文言答应了,开始活动,他七转八转,找到了一个能办事的人——魏忠贤。

		当然,鉴于魏忠贤同志对他极度痛恨,干这件事的时候,他没有露面,而是找人代理。

		魏忠贤接到消息,欣然同意,并开出了价码——四万两,熊廷弼不死。

		汪文言非常高兴,立刻回复了熊廷弼,告诉他这个好消息,以及所需银子的数量\footnote{很可能不是四万两,毕竟中间人也要收费。}

		以汪文言的秉性,拿中介费是一定的,拿多少是不一定的,但这次,他一文钱也没拿到。因为熊廷弼拿不出四万两。

		拿不出钱来,事情没法办,也就没了下文。

		但魏忠贤不知是手头紧,还是办事认真负责,发现这事没消息了,就好了奇,派人去查。七转八转,终于发现那个托他办事的人,竟然是汪文言!

		过分了,实在过分了,魏忠贤感受到了出离的愤怒:和我作对也就罢了,竟然还要托我办事,吃我的中介费!

		拿不到钱,又被人耍了一把的魏忠贤国仇家恨顿时涌上心头,当即派人把汪文言抓了起来。

		汪文言入狱了,但这只是开始,魏忠贤的最终目标,是通过他,把东林党人拉下水。

		但事实再一次证明,冲动是魔鬼。一时冲动的魏公公惊奇地发现,他又撞见鬼了,汪文言入狱后,审来审去毫无进展,别说杨涟、左光斗,就连汪先生自己也在牢里过得相当滋润。

		之所以出现如此怪象,除汪先生自己特别能战斗外,另一个人的加入,也起了极大的作用。

		这个人名叫黄尊素,时任都察院监察御史。

		这是一个很有名的人,知道他的人比较多,但他还有个更有名的儿子——黄宗羲。如果连黄宗羲都不知道,应该回家多读点书。

		在以书生为主的东林党里,黄尊素是个异类。此人深谋远虑,凡事三思而行,擅长权谋,与汪文言并称为东林党两大智囊。

		得知汪文言被抓后,许多东林党人都很愤怒,但也就是发发牢骚,真正做出反应的,只有两个人,其中一个,就是黄尊素。

		他敏锐地感觉到,魏忠贤要动手了。

		抓汪文言只是个开头,很快,这场战火就将延伸到东林党的身上。到时一切都迟了。

		于是,他连夜找到了锦衣卫刘侨。

		刘侨,时任锦衣卫镇抚司指挥使,管理诏狱,汪文言就在他地盘坐牢。

		这人品格还算正派,所以黄尊素专程找到他,疏通关系。

		黄尊素表示,人你照抓照关,但万万不能牵涉到其他人,比如左光斗、杨涟等等。

		刘侨答应了。

		刘侨是个聪明人,他明白黄尊素的意思。便照此意思吩咐审讯工作,所以汪文言在牢里满口胡话,也没人找他麻烦。

		而另一个察觉魏忠贤企图的人,是叶向高。

		叶向高毕竟是见过世面的,几十年朝廷混下来,一看就明白。即刻上书表示汪文言是自己任命的,如果此人有问题,就是自己责任,与他人无关,特请退休回家养老。

		叶首辅不愧为老狐狸,他明知道,朝廷是不会让自己走的,却偏要以退为进,给魏忠贤施加压力,让他无法轻举妄动。

		看到对方摆出如此架势,魏忠贤退缩了。

		太冲动了,时候还没到。

		在这个回合里,东林党获得了暂时的胜利,却将迎来永远的失败。

		抓汪文言时,魏忠贤并没有获胜的把握,但到了天启四年(1624年)五月,连东林党都不再怀疑自己注定失败的命运。

		因为魏公公实在太能拉人了。

		几年之间,所谓“众正盈朝”已然变成了“众兽盈朝”。魏公公手下那些飞禽走兽已经遍布朝廷,王体乾掌控了司礼监,顾秉谦、魏广微进入内阁,许显纯、田尔耕控制锦衣卫。六部里,只有吏部部长赵南星还苦苦支撑,其余各部到处都是阉党,甚至管纪检监察的都察院六科,都成为了阉党的天下。

		对于这一转变,大多数书上的解释是世风日下,人心不古,道德沦丧,品质败坏等等等等。

		其实原因很简单,就一句话:实在。

		魏忠贤能拉人,因为他实在。

		你要人家给你卖命,拿碗白饭对他说,此去路远,多吃一点,那是没有效果的。毕竟千里迢迢,不要脸面,没有廉耻来投个太监,不见点干货,心理很难平衡。

		在这一点上,魏公公表现得很好,但凡投奔他的,要钱给钱,要官给官,真金实银,不打白条。

		相比而言,东林党的竞争力实在太差,什么都不给还难进,实在有点难度过高。

		如果有人让你选择如下两个选项:坚持操守,坚定信念和理想,一生默默无闻,家徒四壁,为国为民,辛劳一生。

		或是放弃原则,泯灭良心,少奋斗几十年,青云直上,升官发财,好吃好喝,享乐一生。
		\begin{quote}
			\begin{spacing}{0.5}  %行間距倍率
				\textit{{\footnotesize
							\begin{description}
								\item[\textcolor{Gray}{\faQuoteRight}] 嗟乎!大阉之乱,以缙绅之身而不改其志者,四海之大,有几人欤?——《五人墓碑记》
							\end{description}
						}}
			\end{spacing}
		\end{quote}

		不用回答,我们都知道答案。

		很久以前,我曾经看过一部电影,电影里的黑社会老大在向他的手下训话,他说,昨天晚上他做了一个梦,梦见这个世界上没有黑社会了。

		因为这个世界上的人,都变成了黑社会。

		这句话在魏忠贤那里,已不再是梦想。

		他不问出身,不问品格,将朝廷大权赋予所有和他一样卑劣无耻的人。

		而这些靠跪地磕头、自认孙子才掌握大权的人,自然没有什么造福人民的想法,受尽屈辱才得到的荣华富贵,不屈辱一下老百姓,怎么对得起自己呢?

		在这种良好愿望的驱使下,某些匪夷所思的事情,开始陆续发生。比如某县有位富翁,闲来无事杀了个人,知县秉公执法,判了死刑。这位仁兄不想死,就找到一位阉党官员,希望能够拿钱买条命。

		很快他就得到了答复:一万两。

		这位财主同意了,此外他还提出了一个要求:希望杀掉那位判他死刑的知县,因为这位县太爷太过公正,实在让他不爽。

		要说还是阉党的同志们实在,收钱之后立马放人,并当即捏造了罪名,把那位知县干掉了。

		无辜的被害者,正直的七品知县,司法、正义,全加在一起,也就一万两。

		事实上,这个价码还偏高。

		搞到后来,除封官许愿外,魏忠贤还开发了新业务:卖官!有些史料还告诉我们,当时的官职都是明码标价,买个知县,大致是两三千两,要买知府,五六千两也就够了。

		如此看来,那位草菅人命的财主,还真是不会算帐。索性找到魏公公,花一半钱买个知府,直接当那知县的上级,找个由头把他干掉,还能省五千两,亏了,真亏了。

		自开朝以来,大明最黑暗的时刻,终于到来!

		我们想干什么就干什么,我们想怎么干就怎么干,为了获取权力和财富,所付出的尊严和代价,要从那些更为弱小的人身上加倍掠夺。蹂躏、欺凌、劫掠,不用顾忌,不用考虑,我们可以为所欲为!

		因为在这个时代,没有人能阻止我们,没有人敢阻止我们!

		\subsection{道统}
		几年来,杨涟一直在看。

		他看见那个无恶不作的太监,抢走了朋友的情人,杀死了朋友,坑死了上司,却掌握了天下的大权,无需偿命,没有报应。

		那个叫天理的东西,似乎并不存在。

		他看见,一个无比强大的敌人,已经出现在自己的面前。

		在明代历史上,从来不缺重量级的坏人,比如刘瑾,比如严嵩,但刘瑾多少还读点书,知道做事要守规矩,至少有个底线,所以他明知李东阳和他作对,也没动手杀人。严嵩虽说杀了夏言,至少还善待自己的老婆。

		而魏忠贤,是一个文盲,逼走老婆,卖掉女儿,他没原则,没底线,阴险狡诈,不择手段,已达到了无耻无极限的境界。他绝了后,也空了前。

		当杨涟回过神来,他才发现,自己身边,已是空无一人,那些当年的敌人、甚至朋友、同僚都已抛弃良知,投入了这个人的怀抱。在利益的面前,良知实在太过脆弱。

		但他依然留在原地,一动不动,因为他依然坚持着一样东西——道统。

		所谓道统,是一种规则,一种秩序,是这个国家几千年来历经苦难挫折依旧前行的动力。

		杨涟和道统已经认识很多年了。

		小时候,道统告诉他,你要努力读书,研习圣人之道,将来报效国家。

		当知县时,道统告诉他,你要为官清廉,不能贪污,不能拿不该拿的钱,要造福百姓。

		京城,皇帝病危,野心家蠢蠢欲动,道统告诉他,国家危亡,你要挺身而出,即使你没有义务,没有帮手。

		一直以来,杨涟对道统的话都深信不疑,他照做了,并获得了成功:

		是你让我相信,一个普通的平民子弟,也能够通过自己的努力,坚持不懈,成就一番事业,成为千古留名的人物。

		你让我相信,即使身居高位,尊容加身,也不应滥用自己的权力,去欺凌那些依旧弱小的人。

		你让我相信,一个人活在这世上,不能只是为了自己。他应该清正廉洁,严于律己,坚守那条无数先贤走过的道路,继续走下去。

		但是现在,我有一个疑问:

		魏忠贤是一个不信道统的人,他无恶不作,肆无忌惮,没有任何原则,但他依然成为了胜利者,越来越多的人放弃了道统,投奔了他,只是因为他封官给钱,如同送白菜。

		我的朋友越来越少,敌人越来越多,在这条道路上,我已是孤身一人。

		道统说:是的,这条道路很艰苦,门槛高,规矩多,清廉自律,家徒四壁,还要立志为民请命,一生报效国家,实在太难。

		那我为何还要继续走下去呢?

		因为这是一条正确的道路,几千年来,一直有人走在这条孤独的道路上,无论经过多少折磨,他们始终相信规则,相信每个人都有着自己的尊严和价值,相信这个世界上,存在着公理与正义,相信千年之下,正气必定长存。

		是的,我明白了,现在轮到我了,我会坚守我的信念,我将对抗那个强大的敌人,战斗至最后一息,即使孤身一人。

		好吧,杨涟,现在我来问你,最后一个问题:

		为了你的道统,牺牲你的一切,可以吗?

		可以。

		\ifnum\theparacolNo=2
	\end{multicols}
\fi
\newpage
\section{杨涟}
\ifnum\theparacolNo=2
	\begin{multicols}{\theparacolNo}
		\fi
		天启四年(1624年)六月,左副都御史杨涟写就上疏,弹劾东厂提督太监魏忠贤二十四大罪。

		在这篇青史留名的檄文中,杨涟历数了魏忠贤的种种罪恶,从排除异己、陷害忠良、图谋不轨、杀害无辜,可谓世间万象,无所不包,且真实可信,字字见血。

		由此看来,魏忠贤确实是人才,短短几年里,跨行业、跨品种,坏事干得面面俱到,着实不易。

		这是杨涟的最后反击,与其说是反击,不如说是愤怒。因为连他自己都很清楚,此时的朝廷,从内阁到六部,都已是魏忠贤的爪牙。按照常理,这封奏疏只要送上去,必定会落入阉党之手,到时只能是废纸一张。

		杨涟虽然正直,却并非没有心眼,为了应对不利局面,他想出了两个办法。

		他写完这封奏疏后,并没有遵守程序,把它送到内阁,而是随身携带,等待着第二天的到来。

		因为在这一天,皇帝大人将上朝议事,那时,杨涟将拿出这封奏疏,亲口揭露魏忠贤的罪恶。

		在清晨的薄雾中,杨涟怀揣着奏疏,前去上朝,此时除极个别人外,无人知道他的计划,和他即将要做的事。

		然而当他来到大殿前的时候,却得到一个让人哭笑不得的消息:皇帝下令,今天不办公\footnote{免朝。}。

		紧绷的神经顿时松弛了下来,杨涟明白,这场生死决战又延迟了一天。

		只能明天再来了。

		但就在他准备打道回府之际,却突然意识到一个问题,于是他改变了主意。

		杨涟走到了会极门,按照惯例,将这封奏疏交给了负责递文书的官员。

		在交出文书的那一刻,杨涟已然确定,不久之后,这份奏疏就会放在魏忠贤的文案上。

		之所以做此选择,是因为他别无选择。

		杨涟是一个做事认真谨慎的人,他知道,虽然此事知情者很少,但难保不出个把叛徒,万一事情曝光,以魏公公的品行,派个把东厂特务把自己黑掉,也不是不可能的。

		不能再等了,不管魏忠贤何时看到,会不会在上面吐唾沫,都不能再等了。

		第一个办法失败了,杨涟没能绕开魏忠贤,直接上书。事实上,这封奏疏确实落到了魏忠贤的手中。

		魏忠贤知道这封奏疏是告他的,但不知是怎么告的,因为他不识字。

		所以,他找人读给他听。

		但当这位无恶不作、肆无忌惮的大太监听到一半时,便打断了朗读,不是歇斯底里的愤怒,而是面无人色的恐惧。

		魏忠贤害怕了,这位不可一世,手握大权的魏公公,竟然害怕了。

		据史料的记载,此时的魏公公面无人色,两手不由自主颤抖,并且半天沉默不语。

		他已经不是四年前那个站在杨涟面前,被骂得狗血淋头,哆哆嗦嗦的老太监了。

		现在他掌握了内阁,掌握了六部,甚至还掌握了特务,他一度以为,天下再无敌手。

		但当杨涟再次站在他面前的时候,他才明白,纵使这个人孤立无援、身无长物,他却依然畏惧这个人,深入骨髓的畏惧。

		极度的恐慌彻底搅乱了魏忠贤的神经,他的脑海中只剩下一个念头:绝对不能让这封奏疏传到皇帝的手中!

		奏疏倒还好说,魏公公一句话,说压就压了,反正皇帝也不管。但问题是,杨涟是左副都御史,朝廷高级官员,只要皇帝上朝,他就能够见到皇帝,揭露所有一切。

		怎么办呢?魏忠贤冥思苦想了很久,终于想出了一个没办法的办法:不让皇帝上朝。

		在接下来的三天里,皇帝都没有上朝。

		但这个办法实在有点蠢,因为天启皇帝到底是年轻人,到第四天,就不干了,偏要去上朝。

		魏忠贤头疼不已,但皇帝大人说要上朝,不让他去又不行,迫于无奈,竟然找了上百个太监,把皇帝大人围了起来,到大殿转了一圈,权当是给大家一个交代。

		此外,他还特意派人事先说明,不允许任何人发言。

		总之,他的对策是,先避风头,把这件事压下去,以后再跟杨涟算帐。

		得知皇帝三天没有上朝,且目睹了那场滑稽游行的杨涟并不吃惊,事情的发展,早在他意料之中。

		因为当他的第一步计划失败,被迫送出那份奏疏的时候,他就想好了第二个对策。

		虽然魏忠贤压住了杨涟的奏疏,但让他惊奇的是,这封文书竟然长了翅膀,没过几天,朝廷上下,除了皇帝没看过,大家基本是人手一份,还有个把缺心眼的,把词编成了歌,四处去唱,搞得魏公公没脸出门。

		杨涟充分发挥了东林党的优良传统,不坐地等待上级批复,就以讲学传道为主要途径,把魏忠贤的恶劣事迹广泛传播,并在短短几天之内,达到了妇孺皆知的效果。

		比如当时国子监里的几百号人,看到这封奏疏后,欢呼雀跃,连书都不读了,每天就抄这份二十四大罪,抄到手软,并广泛散发。

		吃过魏公公苦头的人民大众自不用说,大家一拥而上,反复传抄,当众朗诵,成为最流行的手抄本。据说最风光的时候,连抄书的纸都缺了货。

		左光斗是少数几个事先的知情者之一,此时自然不甘人后,联同朝廷里剩余的东林党官员共同上书,斥责魏忠贤。甚至某些退休在家的老先生,也来凑了把热闹。于是几天之内,全国各地弹劾魏忠贤的公文纸纷至沓来,堆积如山,足够把魏忠贤埋了再立个碑。

		眼看革命形势一片大好,许多原先是阉党的同志也坐不住了,唯恐局势变化自己垫背,一些人纷纷倒戈,掉头就骂魏公公,搞得魏忠贤极其狼狈。

		事实证明,广大人民群众对魏忠贤的愤怒之情,就如同那滔滔江水,延绵不绝。搞得连深宫之中的皇帝,都听说了这件事,专门找魏忠贤来问话,到了这个地步,事情已经瞒不住了。

		杨涟没有想到,自己的义愤之举,竟然会产生如此重大的影响,在他看来,照此形势发展,大事必成,忠贤必死。

		然而有一个人,不同意杨涟的看法。

		在写奏疏之前,为保证一击必中,杨涟曾跟东林党的几位重要人物,如赵南星、左光斗通过气,但有一个人,他没有通知,这个人是叶向高。

		由始至终,叶向高都是东林党的盟友,且身居首辅,是压制魏忠贤的最后力量,但杨先生就是不告诉他,偏不买他的帐。

		因为叶向高曾不只一次对杨涟表达过如下观点:

		对付魏忠贤,是不能硬来的。

		叶向高认为,魏忠贤根基深厚,身居高位,且内有奶妈\footnote{客氏。},外有特务\footnote{东厂。},以东林党目前的力量,是无法扳倒的。

		杨涟认为,叶向高的言论,是典型的投降主义精神。

		魏忠贤再强大,也不过是个太监。他手下的那帮人,无非是乌合之众,只要能够集中力量,击倒魏忠贤,就能将阉党这帮人渣一网打尽,维持社会秩序、世界和平。

		更何况,自古以来,邪不胜正。

		邪恶是必定失败的!基于这一基本判断,杨涟相信,自己是正确的,魏忠贤终究会被摧毁。

		历史已经无数次证明,邪不胜正是靠谱的,但杨涟不明白,这个命题有个前提条件——时间。

		其实在大多数时间里,除去超人、蝙蝠侠等不可抗力出来维护正义外,邪是经常胜正的。所谓好人、善人、老实人常常被整得凄惨无比,比如于谦、岳飞等等,都是死后多少年才翻身平反。

		只有岁月的沧桑,才能淘尽一切污浊,扫清人们眼帘上的遮盖与灰尘,看到那些殉道者无比璀璨的光芒,历千年而不灭。

		杨涟,下一个殉道者。

		很不幸,叶向高的话虽然不中听,却是对的。以东林党目前的实力,要干掉魏忠贤,是毫无胜算的。

		但决定他们必定失败宿命的,不是奶妈,也不是特务,而是皇帝。

		杨涟并不傻,他知道大臣靠不住,太监靠不住,所以他把所有的希望,都寄托在皇帝身上。希望皇帝陛下雷霆大怒,最好把魏公公五马分尸再拉出去喂狗。

		可惜,杨涟同志寄予厚望的天启皇帝,是靠不住的。

		自有皇帝以来,牛皇帝有之,熊皇帝有之,不牛不熊的皇帝也有之,而天启皇帝比较特别:他是木匠。

		身为一名优秀的木匠,明熹宗有着良好的职业素养,他经常摆弄宫里建筑。具体表现为在他当政的几年里,宫里经常搞工程,工程的设计单位、施工、监理、检验,全部由皇帝大人自己承担。

		更为奇特的是,工程的目的也很简单,修好了,就拆,拆完了,再修,以达到拆拆修修无穷尽之目的。总之,搞来搞去,只为图个乐。

		这是大工程,小玩意天启同志也搞过。据史料记载,他曾经造过一种木制模型,有山有水有人,据说木人身后有机关控制,还能动起来,纯手工制作,比起今天的遥控玩具有过之而无不及。

		为检验自己的实力,天启还曾把自己的作品放到市场上去卖,据称能卖近千两银子,合人民币几十万。要换在今天,这兄弟就不干皇帝,也早发了。

		可是,他偏偏就是皇帝。

		大明有无数木匠,但只有一个皇帝,无论是皇帝跑去做木匠,还是木匠跑来做皇帝,都是彻底地抓瞎。

		当然,许多书上说这位皇帝是低能儿,从来不管政务,不懂政治,那也是不对的,虽然他把权力交给了魏忠贤,也不看文件,不理朝廷,但他心里是很有数的。

		比如魏公公,看准了皇帝不想管事,就爱干木匠,每次有重要事情奏报,他都专挑朱木匠干得最起劲的时候去,朱木匠自然不高兴,把手一挥:我要你们是干什么的?

		这句话在手,魏公公自然欢天喜地,任意妄为。

		但在这句话后,朱木匠总会加上一句:好好干,莫欺我!

		这句话的表面意思是,你不要骗我,但隐含意思是,我知道,你可能会骗我。

		事实上,对魏忠贤的种种恶行,木匠多少还知道点,但在他看来,无论这人多好,只要对他坏,就是坏人;无论这人多坏,只要对他好,就是好人。

		基于这一观点,他对魏忠贤有着极深的信任,就算不信任他,也没有必要干掉他。

		叶向高正是认识到这一点,才认定,单凭这封奏疏,是无法解决魏忠贤的。

		而东林党里的另一位明白人黄尊素,事发后也问过这样一个问题:

		“清君侧者必有内援,杨公有乎?”

		这意思是,你要搞定皇帝身边的人,必须要有内应,当然没内应也行,像当年猛人朱棣,带几万人跟皇帝死磕,一直打到京城,想杀谁杀谁。

		杨涟没有,所以不行。

		但他依然充满自信,因为奏疏在社会上引起的强烈反响和广大声势让他相信:真理和正义是站在他这边的。

		但是实力,并不在他的一边。

		奏疏送上后的第五天,事情开始脱离杨涟的轨道,走上了叶向高预言的道路。

		\subsection{底线}
		焦头烂额的魏忠贤几乎绝望了,面对如潮水涌来的攻击,他束手无策,无奈之下,他只能跑去求内阁大臣,东林党人韩旷,希望他手下留情。

		韩旷给他的答复是:没有答复。

		这位东林党内除叶向高外的最高级别干部,对于魏公公的请求,毫无回应,别说赞成,连拒绝都没有。

		如此的态度让魏忠贤深信,如果不久之后自己被拉出去干掉,往尸体上吐唾沫的人群行列中,此人应该排在头几名。

		与韩旷不同,叶向高倒还比较温柔。他曾表示,对魏忠贤无须赶尽杀绝,能让他消停下来,洗手不干,也就罢了。

		这个观点后来被许多的史书引用,来说明叶向高那卑劣的投降主义和悲观主义思想,甚至还有些人把叶先生列入了阉党的行列。

		凡持此种观点者,皆为站着说话不腰疼、啃着馒头看窝头之流。

		因为就当时局势而言,叶向高说无须赶尽杀绝,那只是客气客气的,实际上,压根就无法赶尽杀绝。

		事情的下一步发展完美地印证了这一点。

		在被无情地拒绝后,魏忠贤丢掉了所有的幻想,他终于明白,对于自己的胡作非为,东林党人是无法容忍,也无法接纳的。

		正邪不能共存,那么好吧,我将把所有的一切,都拉入黑暗之中。

		魏忠贤立即找到了另一个人,一个能够改变一切的人。

		在皇帝的面前,魏忠贤表现得相当悲痛,一进去就哭,一边哭一边说:

		“现在外面有人要害我,而且还要害皇上,我无法承担重任,请皇上免去我的职务吧。”

		这种混淆是非,拉皇帝下水的伎俩,虽然并不高明,却比较实用,是魏公公的必备招数。

		面对着痛哭流涕的魏忠贤,天启皇帝只说了一句话,就打乱了魏公公的所有部署:

		“听说有人弹劾你,是怎么回事?”

		听到这句话时,魏忠贤知道,完蛋了。他压住杨涟的奏疏,煞费苦心封锁消息,这木匠还是知道了。

		对于朱木匠,魏忠贤还是比较了解的,虽不管事,绝不白痴,事到如今不说真话是不行了。

		于是他承认了奏疏的存在,并顺道沉重地控诉了对方的污蔑。

		但皇帝陛下似乎不太关心魏公公的痛苦,只说了一句话:

		“奏疏在哪里,拿来给我!”

		这句话再次把魏公公推入了深渊。因为在那封奏疏上,杨涟列举了很多内容,比如迫害后宫嫔妃,甚至害死怀有身孕的妃子,以及私自操练兵马\footnote{内操。},图谋不轨等等。

		贪污受贿,皇帝可以不管,坑皇帝的老婆,抢皇帝的座位,皇帝就生气了。

		更何况这些事,他确实也干过,只要皇帝知道,一查就一个准。

		奏疏拿来了,就在魏忠贤的意志即将崩溃的时候,他听到了皇帝陛下的指示:

		“读给我听。”

		魏忠贤笑了。

		因为他刚刚想起一件很重要的事——皇帝陛下,是不大识字的。

		如果说皇帝陛下的文化程度和魏公公差不多,似乎很残酷,但却是事实,天启之所以成长为准文盲\footnote{认字不多。},归根结底,还是万历惹的祸。

		万历几十年不立太子,太子几十年不安心,自己都搞不定,哪顾得上儿子,儿子都顾不上,哪顾得上儿子读书,就这么折腾来折腾去,把天启折腾成了木匠。

		所以现在,他并没有自己看,而是找了个人,读给他听。

		魏忠贤看到了那个读奏疏的人,他确定,东林党必将死无葬身之地。

		这个朗读者,是司礼监掌印太监,他的死党,王体乾。

		就这样,杨涟的二十四条大罪,在王太监的口里缩了水,为不让皇帝大人担心,有关他老婆和他个人安危的,都省略了,而魏公公一些过于恶心人的行为,出于善意,也不读了。

		所以一篇文章读下来,皇帝大人相当疑惑,听起来魏公公为人还不错,为何群众如此愤怒?

		但这也无所谓,反正也没什么大事,老子还要干木匠呢,就这么着吧。

		于是他对魏忠贤说,你接着干吧,没啥大事。

		魏忠贤彻底解脱了。

		正如叶向高所说的那样,正义和道德是打不倒魏忠贤的,能让这位无赖屈服的,只有实力。而唯一拥有这种实力的人,只有皇帝。

		现在皇帝表明了态度,事件的结局,已无悬念。

		天启四年(1624年)十月,看清虚实的魏忠贤,终于举起了屠刀。

		同月,在毫无预兆的情况下,皇帝下旨,训斥吏部尚书赵南星结党营私,此后皇帝又先后下文,批评杨涟、左光斗、高攀龙等人,最后索性给他们搞了个总结,一顿猛踩,矛头直指东林党。

		可以肯定的是,皇帝大人对此是不大清楚的,他老人家本不识字,且忙做木匠,考虑到情况比较特殊,为保证及时有力迫害忠良,魏公公越级包办了所有圣旨。

		大势已去,一切已然无可挽回。

		同月,心灰意冷的赵南星、杨涟、左光斗纷纷提出辞职,回了老家。东林党就此土崩瓦解。

		只剩下一个人——叶向高。

		叶向高很冷静,由始至终,他都极其低调,魏忠贤倒霉时,他不去踩,魏忠贤得意时,他不辞职,因为他知道,自己将是东林党最后的希望。

		必须忍耐下去,等待反攻的时机。

		但是,他错误地估计了一点——魏忠贤的身份。

		魏忠贤是一个无赖,无赖没有原则,他不是刘瑾,不会留着李东阳给自己刨坟。

		几天之后,叶向高的住宅迎来了一群不速之太监,每天在叶向高门口大吵大嚷,不让睡觉,无奈之下,叶向高只得辞职回家。

		两天后,内阁大学士韩旷辞职,魏忠贤的非亲生儿子顾秉谦接任首辅,至此,内阁彻底沦陷。

		东林党失败了,败得心灰意冷,按照以往的惯例,被赶出朝廷的人,唯一的选择是在家养老。

		但这一次,魏公公给他们提供了第二个选择——赶尽杀绝。

		因为魏公公不是政治家,他是无赖流氓,政治家搞人,搞倒搞臭也就罢了,无赖流氓搞人,都是搞死为止。

		杀死那些毫无抵抗能力的人,这就是魏忠贤的品格。

		但要办到这一点,是有难度的。

		大明毕竟是法制社会,要干掉某些人,必须要罪名,至少要个借口,但魏公公查遍了杨涟等人的记录,作风问题、经济问题,都是统统的没有。

		东林党用实际行动证明了这样一点:他们或许狭隘、或许偏激,却不贪污,不受贿,不仗势欺民,他们的所有举动,都是为了百姓的生计,为了这个国家的未来。

		什么生计、未来,魏公公是不关心的,他关心的是,如何合理地把东林党人整死:抓来打死不行,东林党人都有知名度,社会压力太大,抓来死打套取口供,估计也不行,这帮人是出了名的硬骨头,攻坚难度太大。

		于是,另一个人进入了魏忠贤的视线,他相信,从此人的身上,他将顺利地打开突破口。

		虽然在牢里,但汪文言仍然清楚地感觉到,世界变了,刘侨走了,魏忠贤的忠实龟孙,五彪之一的许显纯接替了他的位置,原先好吃好喝,现在没吃没喝,审讯次数越来越多,态度越来越差。

		但他并不知道,地狱之门才刚刚打开。

		魏忠贤明白,东林党的人品是清白的,把柄是没有的,但这位汪文言是个例外,这人自打进朝廷以来,有钱就拿,有利就贪,东林党熟,阉党也熟,牛鬼蛇神全不耽误,谈不上什么原则。只要从他身上获取杨涟等人贪污的口供,就能彻底摧毁东林党。

		面对左右逢源、投机取巧的汪文言,这似乎不是什么难事。

		天启五年(1625年),许显纯接受魏忠贤的指示,审讯汪文言。

		史料反映,许显纯很可能是个心理比较变态的人,他不但喜欢割取犯人的喉骨,还想出了许多花样繁多的酷刑,比如用铁钩扎穿琵琶骨,把人吊起来,或是用沾着盐水的铁刷去刷犯人,皮肤会随着惨叫声一同脱落。所谓审讯,就是赤裸裸的折磨。

		第一次审讯后,汪文言已经是遍体鳞伤,半死不活。

		但许显纯并不甘休,之后他又进行了第二次、第三次审讯,十几次审下来,审到他都体力不支,依然乐此不疲。

		因为无论他怎么殴打、侮辱、拷问汪文言,逼他交代东林党的罪行,这个不起眼的小人物始终重复一句话:

		“不知道。”

		无论拷打多少次,折磨多少回,穷凶极恶的质问,丧心病狂的酷刑,这就是他唯一的回答。

		当汪文言的侄子买通了看守,在牢中看到不成人形的汪文言时,禁不住痛哭流涕。

		然而汪文言用镇定地语气对他说:

		“不要哭,我必死,却并不怕死!”

		许显纯急眼了,在众多的龟孙之中,魏公公把如此重要的任务交给他,实在是莫大的信任,为不让太监爷爷失望,他必须继续拷打。

		终于有一天,在拷打中,奄奄一息的汪文言用微弱的声音对许显纯说:

		“你要我承认什么,说吧,我承认就是了。”

		许显纯欣喜万分,说道:

		“只要你说杨涟收取贿赂,作口供为证,就放了你。”

		在短暂的沉默之后,一个微弱却坚定的声音响起:

		“这世上,没有贪赃的杨涟。”

		六年前,他之所以加入东林党,不是为了正义,是为了混饭吃。

		混社会的游民,油滑的县吏,唯利是图,狡猾透顶的官僚汪文言,为了在这丑恶的世界上生存下去,他的一生,都在虚伪、圆滑、欺骗中度过,他的每次选择,都是为了利益,都是妥协的产物。

		但在这人生的最后时刻,他做出了最后的抉择:面对黑暗,绝不妥协。

		付出生命,亦在所不惜。

		许显纯无计可施,所以他决定,用一种更不要脸的方式解决问题——伪造口供。

		在这个问题上,许显纯再次显示了他的变态心理,他一边拷打汪文言,一边在他的眼前伪造证词,意思很明白:我就在你的面前,伪造你的口供,你又能怎么样呢?

		但当他洋洋得意地伪造供词的时候,对面阴暗的角落里,那个遍体鳞伤,奄奄一息的人发出了声音。

		无畏的东林党人汪文言,用尽他最后的力气,向这个黑暗的世界,迸发出愤怒的控诉:

		“不要乱写,就算我死了,也要与你对质!”

		这是他留在世间的最后一句话。

		这句话告诉我们,追逐权位,利益至上的老油条汪文言,经历几十年官场沉浮、尔虞我诈之后,拒绝了诱惑,选择了理想,并最终成为了一个正直无私的人。

		\subsection{血书}
		许显纯怕了,他怕汪文言的诅咒,于是,他找到了一个解决方法:杀死汪文言。

		死后对质还在其次,如果让他活着对质,下一步计划将无法进行。

		天启五年(1625年)四月,汪文言被害于狱中,他始终没有屈服。

		同月,魏忠贤的第二步计划开始,杨涟、左光斗、魏大中等东林党人被逮捕,他们的罪名是受贿,而行贿者是已经处决的熊廷弼。

		受贿的证据自然是汪文言的那份所谓口供,在这份无耻的文书中,杨涟被认定受贿两万两,左光斗等人也人人有份。

		审讯开始了,作为最主要的对象,杨涟被首先提审。

		许显纯拿出了那份伪造的证词,问:

		“熊廷弼是如何行贿的?”

		杨涟答:

		“辽阳失陷前,我就曾上书弹劾此人,他战败后,我怎会帮他出狱?文书尚在可以对质。”

		许显纯无语。

		很明显,许锦衣卫背地耍阴招有水平,当面胡扯还差点,既然无法在沉默中发言,只能在沉默中变态:

		“用刑!”

		下面是杨涟的反应:

		“用什么刑?有死而已!”

		许显纯想让他死,但他必须找到死的理由。

		拷打如期进行,拷打规律是每五天一次,打到不能打为止,杨涟的下颌脱落,牙齿打掉,却依旧无一字供词。

		于是许显纯用上了钢刷,几次下来,杨涟体无完肤,史料有云:“皮肉碎裂如丝”。

		然“骂不绝口”,死不低头。

		在一次严酷的拷打后,杨涟回到监房,写下了《告岳武穆疏》。

		在这封文书中,杨涟没有无助的报怨,也没有愤怒的咒骂,他说:

		“此行定知不测,自受已是甘心。”

		他说:

		“涟一身一家其何足道,而国家大体大势所伤实多。”

		昏暗的牢房中,惨无人道的迫害,无法形容的痛苦,死亡边缘的挣扎,却没有仇恨,没有愤懑。

		只有坦然,从容,以天下为己任。

		在无数次的尝试失败后,许显纯终于认识到,要让这个人低头认罪,是绝不可能的。

		栽赃不管用的时候,暗杀就上场了。

		魏忠贤很清楚,杨涟是极为可怕的对手,是绝对不能放走的。无论如何,必须将他杀死,且不可走漏风声。

		许显纯接到了指令,他信心十足地表示,杨涟将死在他的监狱里,悄无声息,他的冤屈和酷刑将永无人知晓。

		事实确实如此,朝廷内外只知道杨涟有经济问题,被弄进去了,所谓拷打、折磨,闻所未闻。

		对于这一点,杨涟自己也很清楚,他可以死,但不想死得不明不白。

		所以,在暗无天日的监房中,杨涟用被打得几近残废的手,颤抖地写下了两千字的绝笔遗书。在遗书中,他写下了事情的真相,以及自己坎坷的一生。

		遗书写完了,却没用,因为送不出去。

		为保证杨涟死得不清不楚,许显纯加派人手,经常检查杨涟的牢房,如无意外,这封绝笔最终会落入许显纯手中,成为灶台的燃料。

		于是,杨涟将这封绝笔交给了同批入狱的东林党人顾大章。

		顾大章接受了,但他也没办法,因为他是东林重犯,如果杨涟被杀,他必难逃一死。且此封绝笔太过重要,如若窝藏必是重犯,推来推去,谁都不敢收。

		更麻烦的是,看守查狱的时候,发现了这封绝笔,顾大章已别无选择。

		他面对监狱的看守,坦然告诉他所有的一切,然后从容等待结局。

		短暂的沉寂后,他看见那位看守面无表情地收起绝笔,平静地告诉他:这封绝笔,绝不会落到魏忠贤的手中。

		这封绝笔开始被藏在牢中关帝像的后面,此后被埋在牢房的的墙角下,杨涟被杀后,那位看守将其取出,并最终公告于天下。

		无论何时何地,正义终究是存在的。

		天启五年(1625年)七月,许显纯开始了谋杀。

		不能留下证据,所以不能刀砍,不能剑刺,不能有明显的皮外伤。

		于是许显纯用铜锤砸杨涟的胸膛,几乎砸断了他的所有肋骨。

		然而杨涟没有死。

		他随即用上了监狱里最著名的杀人技巧——布袋压身。

		所谓布袋压身,是监狱里杀人的不二法门,专门用来处理那些不好杀,却又不能不杀的犯人。具体操作程序是:找到一只布袋,里面装满土,晚上趁犯人睡觉时压在他身上。按照清代桐城派著名学者方苞的说法\footnote{当年曾经蹲过黑牢。},基本上是晚上压住,天亮就死,品质有保障。

		然而杨涟还是没死,每晚在他身上压布袋,就当是盖被子,白天拍土又站起来。

		口供问不出来倒也罢了,居然连人都干不掉,许显纯快疯了。

		于是这个疯狂的人,使用了丧心病狂的手段。

		他派人把铁钉钉入了杨涟的耳朵。

		具体的操作方法,我不知道。我只知道,这不是人能干出来的事情。

		铁钉入耳的杨涟依然没有死,但例外不会再发生了,毫无人性的折磨、耳内的铁钉已经重创了杨涟,他的神智开始模糊。

		杨涟知道,自己活不了多久了,于是他咬破手指,对这个世界,写下了最后的血书。

		此时的杨涟已处于濒死状态,他没有力气将血书交给顾大章,在那个寂静无声的黑夜里,凭借着顽强的意志,他拖着伤残的身体,用颤抖的双手,将血书藏在了枕头里。

		结束吧,杨涟微笑着,等待着最后的结局。

		许显纯来了,用人间的言语来形容他的卑劣与无耻,已经力不从心了。

		看着眼前这个有着顽强信念,和坚韧生命力的人,许显纯真的害怕了,敲碎他全身的肋骨,他没有死,用土袋压,他没有死,用钉子钉进耳朵,也没有死。

		无比恐惧的许显纯决定,使用最后,也是最残忍的一招。

		天启五年(1625年)七月二十四日夜。

		许显纯把一根大铁钉,钉入了杨涟的头顶。

		这一次,奇迹没有再次出现,杨涟当场死亡,年五十四。

		伟大的殉道者,就此走完了他光辉的一生!

		杨涟希望,他的血书能够在他死后清理遗物时,被亲属发现。

		然而这注定是个破灭的梦想,因为这一点,魏忠贤也想到了。

		为消灭证据,他下令对杨涟的所有遗物进行仔细检查,绝不能遗漏。

		很明显,杨涟藏得不好,在检查中,一位看守轻易地发现了这封血书。

		他十分高兴,打算把血书拿去请赏。

		但当他看完这封血迹斑斑的遗言后,便改变了主意。

		他藏起了血书,把它带回了家,他的妻子知道后,非常恐慌,让他交出去。

		牢头并不理会,只是紧握着那份血书,一边痛哭,一边重复着这样一句话:

		“我要留着它,将来,它会赎清我的罪过。”

		三年后,当真相大白时,他拿出了这份血书,并昭示天下。

		如下:
		\begin{quote}
			\begin{spacing}{0.5}  %行間距倍率
				\textit{{\footnotesize
							\begin{description}
								\item[\textcolor{Gray}{\faQuoteRight}] 仁义一生,死于诏狱,难言不得死所,何憾于天,何怨于人?唯我身副宪臣,曾受顾命,孔子云:托孤寄命,临大节而不可夺。持此一念终可见先帝于在天,对二祖十宗于皇天后土,天下万世矣!
								\item[\textcolor{Gray}{\faQuoteRight}] 大笑大笑还大笑,刀砍东风,于我何有哉!
							\end{description}
						}}
			\end{spacing}
		\end{quote}

		他不知道自己还能活多久,不知道死后何人知晓,不知道能否平反,也不知道这份血书能否被人看见。

		毫无指望,只有彻底的孤独和无助。

		这就是阴森恐怖的牢房里,肋骨尽碎的杨涟,在最为绝望的时刻,写下的文字,每一个字,都闪烁着希望和光芒。

		拷打、折磨,毫无人性的酷刑,制服了他的身体,却没有征服他的意志。无论何时,他都坚持着自己的信念,那个他写在绝笔中的信念,那个崇高、光辉、唯一的信念:
		\begin{quote}
			\begin{spacing}{0.5}  %行間距倍率
				\textit{{\footnotesize
							\begin{description}
								\item[\textcolor{Gray}{\faQuoteRight}] 涟即身无完骨,尸供蛆蚁,原所甘心。
								\item[\textcolor{Gray}{\faQuoteRight}] 但愿国家强固,圣德刚明,海内长享太平之福。
								\item[\textcolor{Gray}{\faQuoteRight}] 此痴愚念头,至死不改。
							\end{description}
						}}
			\end{spacing}
		\end{quote}

		有人曾质问我,遍读史书如你,所见皆为帝王将相之家谱,有何意义?

		千年之下,可有一人,不求家财万贯,不求出将入相,不求青史留名,唯以天下、以国家、以百姓为任,甘受屈辱,甘受折磨,视死如归?

		我答:曾有一人,不求钱财,不求富贵,不求青史留名,有慨然雄浑之气,万刃加身不改之志。

		杨涟,千年之下,终究不朽!

		\ifnum\theparacolNo=2
	\end{multicols}
\fi
\newpage
\section{殉道}
\ifnum\theparacolNo=2
	\begin{multicols}{\theparacolNo}
		\fi
		\subsection{老师}
		左光斗只比杨涟多活了一天。

		身为都察院高级长官,左光斗也是许显纯拷打的重点对象,杨涟挨过的酷刑,左光斗一样都没少。

		而他的态度,也和杨涟一样,绝不退让,绝不屈服。

		虽然被打得随时可能断气,左光斗却毫不在乎,死不低头。

		他不在乎,有人在乎。

		先是左光斗家里的老乡们开始凑钱,打算把人弄出来,至少保住条命。无效不退款后,他的家属和学生就准备进去探监,至少再见个面。

		但这个要求也被拒绝了。

		最后,他的一位学生费尽浑身解数,才买通了一位看守,进入了监牢。

		他换上了破衣烂衫,化装成捡垃圾的,在黑不隆冬的诏狱里摸了半天,才摸到了左光斗的牢房。

		左光斗是坐着的,因为他的腿已经被打没了\footnote{筋骨尽脱。}。面对自己学生的到访,他没有表现出任何惊讶,因为他根本不知道——脸已被烙铁烙坏,连眼睛都睁不开。

		他的学生被惊呆了,于是他跪了下来,抱住老师,失声痛哭。

		左光斗听到了哭声,他醒了过来,没有惊喜,没有哀叹,只有愤怒,出离的愤怒:

		“蠢人!这是什么地方,你竟然敢来\footnote{此何地也,而汝前来。}!国家已经到了这个地步,我死就死了,你却如此轻率,万一出了事,将来国家的事情谁来管!?”

		学生呆住了,呆若木鸡。

		左光斗的愤怒似乎越发激烈,他摸索着地上的镣铐,做出投掷的动作,并说出了最后的话:

		“你还不走?!再不走,无需奸人动手,我自己杀了你\footnote{扑杀汝。}!”

		面对着世界上最温暖的威胁,学生眼含着热泪,快步退了出去。

		临死前,左光斗用自己的行动,给这名学生上了最后一课:

		一个人应该坚持信念,至死也不动摇。

		天启五年(1625年)七月二十五日,左光斗在牢中遇害,年五十一。

		二十年后,扬州。

		南京兵部尚书,内阁大学士,南明政权的头号重臣史可法,站在城头眺望城外的清军,时为南明弘光元年(1645年)二月。

		雪很大,史可法却一直站在外面,安排部署,他的部下几次劝他进屋躲雪,他的回复总是同一句话:

		“我不能对不起我的老师,我不能对不起我的老师\footnote{愧于吾师。}!”

		史可法最终做到了,他的行为,足以让他的老师为之自豪。

		左光斗死后,同批入狱的东林党人魏大中、袁化中,周朝瑞先后被害。

		活着的人,只剩下顾大章。

		顾大章,时任礼部郎中,算是正厅级干部,在这六人里就官职而言并不算大,但他还是有来头的,他的老师就是叶向高,加上平时活动比较积极,所以这次也被当作要犯抓了进来。

		抓进来六个,其他五个都死了,他还活着,不是他地位高,只是因为他曾经担任过一个特殊的官职——刑部主事。

		刑部主事,大致相当于司法部的一个处长,但凑巧的是,他这个部门恰好就是管监狱的,所谓刑部天牢、锦衣诏狱的看守,原先都是他的部下。

		现在老上级进去了,遇到了老下级,这就好比是路上遇到劫道的,一看,原来你是我小学时候的同学,还一起罚过站,这就不好下手了。咬咬牙,哥们你过去吧,这单生意我不做了,下次注意点,别再到我的营业区域里转悠。

		外加顾主事平时为人厚道,对牢头看守们都很照顾,所以他刚进去的时候,看守都向他行礼,对他非常客气,点头哈腰,除了人渣许显纯例行拷打外,基本没吃什么亏。

		但其他人被杀后,他的处境就危险了,毕竟一共六个,五个都死了,留你一个似乎不太像话。更重要的是,这些惨无人道的严刑拷打,是不能让人知道的,要是让他出狱,笔杆子一挥全国人民都知道了,舆论压力比较大。

		事实上,许显纯和魏忠贤确实打算把顾大章干掉,且越快越好。顾大章去阎王那里伸冤的日子已经不远了。

		然而这个世界上,意外的事情总是经常发生的。

		一般说来,管牢房的人交际都比较广泛。特别是天牢、诏狱这种高档次监狱,进来的除了窦娥、忠良外,大都有点水平,或是特殊技能,江洋大盗之类的牛人也不少见。

		我们有理由相信,顾大章认识一些这样的人。

		因为就在九月初,处死他的决议刚刚通过,监狱看守就知道了。

		但是这位看守没有把消息告诉顾大章,却通知了另一个人。

		这个人的姓名不详,人称燕大侠,也在诏狱里混,但既不是犯人,也不是看守,每天就混在里面,据说还是主动混进来的,几个月了都没人管。

		他怎么进来的,不得而知,为什么没人管,不太清楚,但他之所以进来,只是为了救顾大章。为什么要救顾大章,也不太清楚,反正他是进来了。

		得知处决消息,他并不慌张,只是找到报信的看守,问了他一个问题:

		“我给你钱,能缓几天吗?”

		看守问:

		“几天?”

		燕大侠答:

		“五天。”

		看守答:

		“可以。”

		五天之后,看守跑来找燕大侠:

		“我已尽力,五日已满,今晚无法再保证顾大章的安全,怎么办?”

		燕大侠并不紧张:

		“今晚定有转机。”

		看守认为,燕大侠在做梦,他笑着走了。

		几个时辰之后,他接到了命令,将顾大章押往刑部。

		还没等他缓过神来,许显纯又来了。

		许显纯急匆匆跑来,把顾大章从牢里提出来,声色俱厉地说了句话:

		“你几天以后,还是要回来的!”

		然后,他又急匆匆地走了。

		顾大章很高兴。

		作为官场老手,他很理解许显纯这句话的隐含意义——自己即将脱离诏狱,而许显纯无能为力。

		因为所谓锦衣卫、东厂,都是特务机关,并非司法机构。这件案子被转交刑部,公开审判,就意味着许显纯们搞不定了。

		很明显,他们受到了压力。

		但为什么搞不定,又是什么压力,他不知道。

		这是个相当诡异的问题:魏公公权倾天下,连最能搞关系的汪文言都整死了,然而燕大侠横空出世,又把事情解决了,实在让人难以理解。

		顾大章不知道答案,看守不知道答案,许显纯也未必知道。

		燕大侠知道,可是他没告诉我,所以我也不知道。

		之前我曾介绍过许多此类幕后密谋,对于这种鬼才知道的玩意,我的态度是,不知道就说不知道,绝不猜。

		我倒是想猜,因为这种暗箱操作,还是能猜的。如当年太史公司马迁先生,就很能猜的,秦始皇死后,李斯和赵高密谋干掉太子,他老人家并不在场,上百年前的事,天知地知你知我知,对话都能猜出来。过了几千年,也没人说他猜得不对,毕竟事情后来就是那么干的。

		可这件事实在太过复杂,许显纯没招,魏公公不管\footnote{或是管不了。},他们商量的时候也没叫我去,实在是不敢乱猜。

		无论事实真相如何,反正顾大章是出来了。在经历几十天痛苦的折磨后,他终于走出了地狱。

		按说到了刑部,就是顾大人的天下了,可实情并非如此。

		因为刑部尚书李养正也投了阉党,部长大人尚且如此,顾大人就没辙了。

		天启五年(1625年)九月十二日,刑部会审。

		李养正果然不负其阉党之名,一上来就喝斥顾大章,让他老实交代。更为搞笑的是,他手里拿的罪状,就是许显纯交给他的,一字都没改,底下的顾大章都能背出来,李尚书读错了,顾大人时不时还提他两句。

		审讯的过程也很简单,李尚书要顾大章承认,顾大章不承认,并说出了不承认的理由:

		“我不能代死去的人,承受你们的诬陷。”

		李尚书沉默了,他知道这位曾经的下属是冤枉的,但他依然做出了判决:

		杨涟、左光斗、顾大章等六人,因收受贿赂,结交疆臣,处以斩刑。

		这是一份相当无聊的判决,因为判决书里的六个人,有五个已经挂了,实际上是把顾大章先生拉出来单练,先在诏狱里一顿猛打,打完再到刑部,说明打你的合法理由。

		形势急转直下,燕大侠也慌了手脚,一天夜里,他找到顾大章,告诉他情况不妙。

		然而出乎意料的是,顾大章并不惊慌,恰恰相反,他用平静的口吻,向燕大侠揭示了一个秘密——出狱的秘密。

		第二天,在刑部大堂上,顾大章公开了这个秘密。

		顾大章招供了,他供述的内容,包括如下几点,杨涟的死因,左光斗的死因,许显纯的刑罚操作方法,绝笔,无人性的折磨,无耻的谋杀。

		刑部知道了,朝廷知道了,全天下人都知道了。

		魏忠贤不明白,许显纯不明白,甚至燕大侠也不明白,顾大章之所以忍辱负重,活到今天,不是心存侥幸,不是投机取巧。

		他早就想死了,和其他五位舍生取义的同志一起,光荣地死去,但他不能死。

		当杨涟把绝笔交给他的那一刻,他的生命就不再属于他自己,他知道自己有义务活下去,有义务把这里发生的一切,把邪恶的丑陋,正义的光辉,告诉世上所有的人。

		所以他隐忍、等待,直至出狱,不为偷生,只为永存。

		正如那天夜里,他对燕大侠所说的话:

		“我要把凶手的姓名传播于天下\footnote{播之天下。},等到来日世道清明,他们一个都跑不掉\footnote{断无遗种。}!”

		“吾目暝矣。”

		这才是他最终的目的。

		他做到了,是以今日之我们,可得知当年之一切。

		一天之后,他用残废的手\footnote{三个指头已被打掉。}写下了自己的遗书,并于当晚自缢而死。

		杨涟,当日你交付于我之重任,我已完成。

		“吾目暝矣。”

		至此,杨涟、左光斗、魏大中、袁化中、周朝瑞、顾大章六人全部遇害,史称“六君子之狱”。

		就算是最恶俗的电视剧,演到这里,坏人也该休息了。

		但魏忠贤实在是个超一流的反派,他还列出了另一张杀人名单。

		在这份名单上,有七个人的名字,分别是高攀龙、李应升、黄遵素、周宗建,缪昌期、周起元、周顺昌。

		这七位仁兄地位说高不高,就是平时骂魏公公时狠了点,但魏公公一口咬死,要把他们组团送到阎王那里去。

		六君子都搞定了,搞个七君子不成问题。

		春风得意、无往不胜的魏公公认为,他已经天下无敌了,可以把事情做绝做尽。

		魏忠贤错了。

		在一部相当胡扯的香港电影中,某大师曾反复说过句不太胡扯的话:凡事太尽,缘分必定早尽。

		刚开始的时候,事情是很顺利的,东林党的人势力没有,气节还是有的,不走也不逃,坐在家里等人来抓,李应升、周宗建,缪昌期、周起元等四人相继被捕,上路的时候还特高兴。

		因为在他们看来,坚持信念,被魏忠贤抓走,是光辉的荣誉。

		高攀龙更厉害,抓他的东厂特务还没来,他就上路了——自尽。

		在被捕前的那个夜晚,他整理衣冠,向北叩首,然后投水自杀。

		死前留有遗书一封,有言如下:可死,不可辱。

		在这七个人中,高攀龙是都察院左都御史,李应升、周宗建、黄尊素都是御史,缪昌期是翰林院谕德,周起元是应天巡抚,说起来,不太起眼的,就数周顺昌了。

		这位周先生曾吏部员外郎,论资历、权势,都是小字辈,但事态变化,正是由他而起。

		周顺昌,字景文,万历四十一年进士,嫉恶如仇。

		说起周兄,还有个哭笑不得的故事,当初他在外地当官,有一次人家请他看戏,开始挺高兴,结果看到一半,突然怒发冲冠,众目睽睽之下跳上舞台,抓住演员一顿暴打,打完就走。

		这位演员之所以被打,只是因为那天,他演的是秦桧。

		听说当年演白毛女的时候,通常是演着演着,下面突来一枪,把黄世仁同志干掉,看来是有历史传统的。

		连几百年前的秦桧都不放过,现成的魏忠贤当然没问题。

		其实最初名单上只有六个人,压根就没有周顺昌,他之所以成为候补,是因为当初魏大中过境时,他把魏先生请到家里,好吃好喝,还结了亲家,东厂特务想赶他走,结果他说:

		“你不知道世上有不怕死的人吗?!回去告诉魏忠贤,我叫周顺昌,只管找我!”

		后来东厂抓周起元的时候,他又站出来大骂魏忠贤,于是魏公公不高兴了,就派人去抓他。

		周顺昌是南直隶吴县人,也就是今天的江苏苏州,周顺昌为人清廉,家里很穷,还很讲义气,经常给人帮忙,在当地名声很好。

		东厂特务估计不太了解这个情况,又觉得苏州人文绉绉的,好欺负,所以一到地方就搞潜规则,要周顺昌家给钱,还公开扬言,如果不给,就在半道把周顺昌给黑了。

		可惜周顺昌是真没钱,他本人也看得开,同样扬言:一文钱不给,能咋样?

		但是人民群众不干了,他们开始凑钱,有些贫困家庭把衣服都当了,只求东厂高抬贵手。

		这次带队抓人的东厂特务,名叫文之炳,可谓是王八蛋中的王八蛋,得寸进尺,竟然加价,要了还要。

		这就过于扯淡了,但为了周顺昌的安全,大家忍了。

		第二天,为抗议逮捕周顺昌,苏州举行罢市活动。

		要换个明白人,看到这个苗头,就该跑路,可这帮特务实在太过嚣张\footnote{或是太傻。},一点不消停,还招摇过市欺负老百姓,为不连累周顺昌,大家又忍了。

		一天后,苏州市民涌上街头,为周顺昌送行,整整十几万人,差点把县衙挤垮,巡抚毛一鹭吓得不行,表示有话好好说。有人随即劝他,众怒难犯,不要抓周顺昌,上奏疏说句公道话。

		毛一鹭胆子比较小,得罪群众是不敢的,得罪魏忠贤自然也不敢,想来想去,一声都不敢出。

		所谓干柴烈火,大致就是这个样子,十几万人气势汹汹,就等一把火。

		于是文之炳先生挺身而出了,他大喊一声:

		“东厂逮人,鼠辈敢尔?”

		火点燃了。

		勒索、收钱不办事、欺负老百姓,十几万人站在眼前,还敢威胁人民群众,人蠢到这个份上,就无须再忍了。

		短暂的平静后,一个人走到了人群的前列,面对文之炳,问出了一个问题:

		“东厂逮人,是魏忠贤\footnote{魏监。}的命令吗?”

		问话的人,是一个当时寂寂无名,后来名垂青史的人,他叫颜佩韦。

		颜佩韦是一个平民,一个无权无势的平民,所以当文特务确定他的身份后,顿时勃然大怒:

		“割了你的舌头!东厂的命令又怎么样?”

		他穿着官服,手持武器,他认为,手无寸铁的老百姓颜佩韦会害怕,会退缩。

		然而,这是个错误的判断。

		颜佩韦振臂而起:

		“我还以为是天子下令,原来是东厂的走狗!”

		然后他抓住眼前这个卑劣无耻、飞扬跋扈的特务,拳打脚踢,发泄心中的怒火。

		文之炳被打蒙了,但其他特务反应很快,纷纷拔刀,准备上来砍死这个胆大包天的人。

		然而接下来,他们看见了让他们恐惧一生的景象,十几万个胆大包天的人,已向他们冲来。

		这些此前沉默不语,任人宰割的羔羊,已经变成了恶狼,纷纷一拥而上,逮住就是一顿暴打。由于人太多,只有离得近的能踩上几脚,距离远的就脱鞋,看准了就往里砸\footnote{提示:时人好穿木屐。}。

		东厂的人疯了,平时大爷当惯了,高官看到他们都打哆嗦,这帮平民竟敢反抗,由于反差太大,许多人思想没转过弯来,半天还在发愣。

		但他们不愧训练有素,在现实面前,迅速地完成了思想斗争,并认清了自己的逃跑路线,四散奔逃,有的跑进民宅,有的跳进厕所,有位身手好的,还跳到房梁上。

		说实话,我认为跳到房梁上的人,脑筋有点问题,人民群众又不是野生动物,你以为他们不会爬树?

		对于这种缺心眼的人,群众们使用了更为简洁的方法,一顿猛揣,连房梁都揣动了,直接把那人摇了下来,一顿群殴,当场毙命。

		相对而言,另一位东厂特务就惨得多了,他是被人踹倒的,还没反应过来,又是一顿猛踩,被踩死了,连肇事者都找不着。

		值得夸奖的是,苏州的市民们除了有血性外,也很讲策略。所有特务都被抓住暴打,但除个别人外,都没打死——半死。这样既出了气,又不至于连累周顺昌。

		打完了特务,群众还不满意,又跑去找巡抚毛一鹭算帐。

		其实毛巡抚比较冤枉,他不过是执行命令,胆子又小,吓得魂不附体,只能躲进粪坑里,等到地方官出来说情,稳定秩序,才把浑身臭气的毛巡抚捞出来。

		这件事件中,东厂特务被打得晕头转向,许多人被打残,还留下了极深的心理创伤。据说有些人回京后,一辈子都只敢躲在小黑屋里,怕光怕声,活像得了狂犬病。

		气是出够了,事也闹大了。

		东厂抓人,人没抓到还被打死几个,魏公公如此窝囊,实在耸人听闻,几百年来都没出过这事。

		按说接下来就该是腥风血雨,可十几天过去,别说反攻倒算,连句话都没有。

		因为魏公公也吓坏了。

		事发后,魏忠贤得知事态严重,当时就慌了,马上把首辅顾秉谦抓来一顿痛骂,说他本不想抓人,听了你的馊主意,才去干的,闹到这个地步,怎么办?

		魏忠贤的意思很明白,他不喜欢这个黑锅,希望顾秉谦帮他背。但顾大人岂是等闲之辈,只磕头不说话,回去就养病,索性不来了。

		魏公公无计可施,想来想去,只好下令,把周顺昌押到京城,参与群众一概不问。

		说是这么说,过了几天,顾秉谦看风声过了,又跳了出来,说要追究此事。

		还没等他动手,就有人自首了。

		自首的,是当天带头的五个人,他们主动找到巡抚毛一鹭,告诉他,事情就是自己干的,与旁人无关,不要株连无辜。

		这五个人的名字是:颜佩韦、杨念如、沈扬、周文元、马杰。

		五人中,周文元是周顺昌的轿夫,其余四人并未见过周顺昌,与他也无任何关系。

		几天后,周顺昌被押解到京,被许显纯严刑拷打,不屈而死。

		几月后,周顺昌的灵柩送回苏州安葬,群情激奋,为平息事端,毛一鹭决定处决五人。

		处斩之日,五人神态自若。

		沈扬说:无憾!

		马杰大笑:

		“吾等为魏奸阉党所害,未必不千载留名,去,去!”

		颜佩韦大笑:

		“列位请便,学生去了!”

		遂英勇就义。

		五人死后,明代著名文人张傅感其忠义,挥笔写就一文,是为《五人墓碑记》,四百年余后,被编入中华人民共和国中学语文课本。
		\begin{quote}
			\begin{spacing}{0.5}  %行間距倍率
				\textit{{\footnotesize
							\begin{description}
								\item[\textcolor{Gray}{\faQuoteRight}] 嗟夫!大阉之乱,以缙绅之身而不改其志者,四海之大,有几人欤?
								\item[\textcolor{Gray}{\faQuoteRight}] 而五人生于编伍之间,素不闻诗书之训,激昂大义,蹈死不顾。——《五人墓碑记》
							\end{description}
						}}
			\end{spacing}
		\end{quote}

		颜佩韦和马杰是商人,沈扬是贸易行中间人,周文元是轿夫,杨念如是卖布的。

		不要以为渺小的,就没有力量;不要以为卑微的,就没有尊严。

		弱者和强者之间唯一的差别,只在信念是否坚定。

		这五位平民英雄的壮举直接导致了两个后果:

		一、魏忠贤害怕了,他以及他的阉党,受到了极大的震动,用历史书上的话说,是为粉碎阉党集团奠定了群众基础。

		相比而言,第二个结果有点歪打正着:七君子里最后的幸存者黄尊素,逃过了一劫。

		东林党两大智囊之一的黄尊素之所以能幸免,倒不是他足智多谋,把事情都搞定了,也不是魏忠贤怕事,不敢抓他,只是因为连颜佩韦等人都不知道,那天被他们打的人里,有几位兄弟是无辜的。

		其实民变发生当天,抓周顺昌的特务和群众对峙时,有一批人恰好正经过苏州,这批人恰好也是特务——抓黄尊素的特务。

		黄尊素是浙江余姚人,要到余姚,自然要经过苏州,于是就赶上了。

		实在有点冤枉,这帮人既没捞钱,也没勒索,无非是过个路,可由于群众过于激动,过于能打,见到东厂装束的人就干,就把他们顺道也干了。

		要说还是特务,那反应真是快,看见一群人朝自己冲过来,虽说不知怎么回事,立马就闪人了,被逼急了就往河里跳,总算是逃过了一劫。

		可从河里出来后一摸,坏了,驾帖丢了。

		所谓驾帖,大致相当于身份证加逮捕证,照眼下这情景,要是没有驾帖就跑去,能活着回来是不太正常的。想来想去,也就不去了。

		于是黄尊素纳闷了,他早就得到消息,在家等人来抓,结果等十几天,人影都没有。

		但黄尊素是个聪明人,聪明人明白一个道理——覆巢之下,岂有完卵。

		躲是躲不过去的,大家都死了,一个人怎能独活呢?

		于是他自己穿上了囚服,到衙门去报到,几个月后,他被许显纯拷打至死。

		在黄尊素走前,叫来了自己的家人,向他们告别。

		大家都很悲痛,只有一个人例外。

		他的儿子黄宗羲镇定地说道:

		“父亲若一去不归,儿子来日自当报仇!”

		一年之后,他用比较残忍的方式,实现了自己的诺言。

		黄尊素死了,东林党覆灭,“六君子”、“七君子”全部殉难,无一幸免,天下再无人与魏忠贤争锋。

		纵观东林党的失败过程,其斗争策略,就是毫无策略,除了愤怒,还是愤怒,输得那真叫彻底,局势基本是一边倒,朝廷是魏公公的,皇帝听魏公公的,似乎毫无胜利的机会。

		事实上,机会还是有的,一个。
		\ifnum\theparacolNo=2
	\end{multicols}
\fi
\newpage
\section{袁崇焕}
\ifnum\theparacolNo=2
	\begin{multicols}{\theparacolNo}
		\fi
		\subsection{犹豫的人}
		在东林党里,有一个特殊的人,此人既有皇帝的信任,又有足以扳倒魏忠贤的实力——孙承宗。

		在得知杨涟被抓后,孙承宗非常愤怒,当即决定弹劾魏忠贤。

		但他想了一下,便改变了主意。

		孙承宗很狡猾,他明白上书是毫无作用的,他不会再犯杨涟的错误,决定使用另一个方法。

		天启四年(1625年)十一月,孙承宗开始向京城进发,他此行的目的,是去找皇帝上访告状。

		对一般人而言,这是不可能的,因为朱木匠天天干木匠活,不大见人,还有魏管家帮他闭门谢客,想见他老人家一面,实在难如登天。

		但孙承宗不存在这个问题,打小他就教朱木匠读书,虽说没啥效果\footnote{认字不多。},但两人感情很好,魏公公几次想挑事,想干掉孙承宗,朱木匠都笑而不答,从不理会,因为他很清楚魏公公的目的。

		他并不傻,这种借刀杀人的小把戏,是不会上当的。

		于是魏忠贤慌了,他很清楚,孙承宗极不简单,不但狡猾大大的,和皇帝关系铁,还手握兵权,如果让他进京打小报告,那就真没戏了,就算没告倒,只要带兵进京来个武斗,凭自己手下这帮废物,是没指望的。

		魏忠贤正心慌,魏广微又来凑热闹了,这位仁兄不知从哪得到的小道消息,说孙承宗带了几万人,打算进京修理魏公公。

		为说明事态的严重性,他还打了个生动的比方:一旦让孙大人进了京,魏公公立马就成粉了\footnote{公立齑粉矣。}。

		魏公公疯了,二话不说,马上跑到皇帝那里,苦苦哀求,不要让孙承宗进京,当然他的理由很正当:孙承宗带兵进京是要干掉皇帝,身为忠臣,必须阻止此种不道德的行为。

		但出乎他意料的是,皇帝大人毫不慌张,他还安慰魏公公,孙老师靠得住,就算带兵,也不会拿自己开刀的。

		这个判断充分说明,皇帝大人非但不傻,还相当地幽默,魏公公被涮得一点脾气都没有。

		话说完,皇帝还要做木匠,就让魏公公走人,可是魏公公不走。

		他知道,今天要不讨个说法,等孙老师进京,没准就真成粉末了。所以他开始哭,且哭出了花样——“绕床痛哭”。

		也就是说,魏公公赖在皇帝的床边,不停地哭。皇帝在床头,他就哭到床头,皇帝到床尾,他就哭到床尾,孜孜不倦,锲而不舍。

		皇帝也是人,也要睡觉,哭来哭去,真没法了,只好发话:

		“那就让他回去吧。”

		有了这句话,魏忠贤胆壮了,他随即命人去关外传令,让孙承宗回去。

		然而不久之后,有人告诉了他一个消息,于是他又下达了第二道命令:

		“孙承宗若入九门,即刻逮捕!”

		那个消息的内容是,孙承宗没有带兵。

		孙承宗确实没有带兵,他只想上访,不想造反。

		所以魏忠贤改变了主意,他希望孙承宗违抗命令,大胆反抗来到京城,并最终落入他的圈套。

		事实上,这是很有可能的,鉴于地球人都知道,魏公公一向惯于假传圣旨,所以愤怒的孙承宗必定会拒绝这个无理的命令,进入九门,光荣被捕。

		然而他整整等了一夜,也没有看到这一幕。

		孙承宗十分愤怒,他急匆匆地赶到了通州,却接到让他返回的命令。他的愤怒到达了顶点,于是他没有丝毫犹豫——返回了。

		孙承宗实在聪明绝顶,虽然他知道魏忠贤有假传圣旨的习惯,但这道让他返回的谕令,却不可能是假的。

		因为魏忠贤知道他和皇帝的关系,他见皇帝,就跟到邻居家串门一样,说来就来了,胡说八道是没用的。

		然而现在他收到了谕令,这就代表着皇帝听从了魏忠贤的忽悠,如果继续前进,后果不堪设想,所以跑路是最好的选择。

		现在摆在他面前的,有两个选择:

		一,回去睡觉,老老实实呆着。

		二,索性带兵进京,干他娘一票,解决问题。

		孙承宗是一个几乎毫无缺陷的人,政治上面很会来事,谁也动不了,军事上稳扎稳打,眼光独到,且一贯小心谨慎,老谋深算,所以多年来,他都是魏忠贤和努尔哈赤最为害怕的敌人。

		但在这一刻,他暴露出了自己人生中的最大弱点——犹豫。

		孙承宗是典型的谋略型统帅,他的处事习惯是如无把握,绝不应战,所以他到辽东几年,收复无数失地,却很少打仗。

		而眼前的这一仗,他没有必胜的把握,所以他放弃。

		无论这个决定正确与否,东林党已再无回天之力。

		三十年前,面对黑暗污浊的现实,意志坚定的吏部员外郎顾宪成相信,对的终究是对的,错的终究是错的。于是他决心,建立一个合理的秩序,维护世上的公义,使那些身居高位者,不能随意践踏他人,让那些平凡的人,有生存的权利。

		为了这个理想,他励精图治,忍辱负重,从那个小小的书院开始,经历几十年起起落落,坚持道统,至死不渝。在他的身后,有无数的追随者杀身成仁。

		然而杀身固然成仁,却不能成事。

		以天下为己任的东林党,终究再无回天之力。

		其实我并不喜欢东林党,因为这些人都是书呆子,自命清高,还空谈阔论,缺乏实干能力。

		小时候,历史老师讲到东林党时,曾说道:东林党人并不是进步的象征,因为他们都是封建士大夫。

		我曾问:何谓封建士大夫?

		老师答:封建士大夫,就是封建社会里,局限、落后,腐朽的势力,而他们的精神,绝不代表历史的发展方向。

		多年以后,我亲手翻开历史,看到了另一个真相。

		所谓封建士大夫,如王安石、如张居正、如杨涟、如林则徐。

		所谓封建士大夫精神,就是没落,守旧,不懂变通,不识时务,给脸不要脸,瞧不起劳动人民,自命清高,即使一穷二白,被误解,污蔑,依然坚持原则、坚持信念、坚持以天下为己任的人。

		他们坚信自己的一生与众不同,高高在上,无论对方反不反感。

		坚信自己生来就有责任和义务,去关怀与自己毫不相干的人,无论对方接不接受。

		坚信国家危亡之际,必须挺身而出,去捍卫那些自己不认识,或许永远不会认识的芸芸众生,并为之奋斗一生,无论对方是否知道,是否理解。

		坚信无论经过多少黑暗与苦难,那传说了无数次,忽悠了无数回,却始终未见的太平盛世,终会到来。

		\subsection{遗弃}
		孙承宗失望而归,他没有能够拯救东林党,只能拯救辽东。

		魏忠贤曾经想把孙老师一同干掉,可他反复游说,皇帝就是不松口,还曾经表示,如果孙老师出了事,就唯你是问。

		魏公公只好放弃了,但让孙老师呆在辽东,手里握着十几万人,实在有点睡不安稳,就开始拿辽东战局说事,还找了几十个言官,日夜不停告黑状。

		孙承宗撑不下去了。

		天启五年(1625年)十月,他提出了辞呈。

		可是他提了很多次,也没得到批准。

		倒不是魏忠贤不想他走,是他实在走不了,因为没人愿意接班。

		按魏忠贤的意思,接替辽东经略的人,应该是高第。

		高第,万历十七年进士,是个相当厉害的人。

		明代的官员,如果没有经济问题,进士出身,十几年下来,至少也能混个四品。而高先生的厉害之处在于,他混了整整三十三年,熬死两个皇帝,连作风问题都没有,到天启三年(1623年),也才当了个兵部侍郎,非常人所能及。

		更厉害的是,高先生只当了一年副部长,第二年就退休了。

		魏忠贤本不想用这人,但算来算去,兵部混过的,阉党里也只有他了。于是二话不说,把他找来,说,我要提你的官,去当辽东经略。

		高先生一贯胆小,但这次也胆大了,当即回复:不干,死都不干。

		为说明他死都不干的决心,他当众给魏忠贤下跪,往死了磕头\footnote{叩头岂免。}:我都这把老骨头了,就让我在家养老吧。

		魏忠贤觉得很空虚。

		费了那么多精神,给钱给官,就拉来这么个废物。所以他气愤了:必须去!

		混吃等死不可能了,高第擦干眼泪,打起精神,到辽东赴任了。

		在辽东,高第用实际行动证实,他既胆小,也很无耻。

		到地方后,高先生立即上了第一封奏疏:弹劾孙承宗,罪名:吃空额。

		经过孙承宗的整顿,当时辽东部队,已达十余万人,对此高第是有数的,但这位兄弟睁眼说瞎话,说他数下来,只有五万人。其余那几万人的工资,都是孙承宗领了。

		对此严重指控,孙承宗欣然表示,他没有任何异议。

		他同时提议,今后的军饷,就按五万人发放。

		这就意味着,每到发工资时,除五万人外,辽东的其余几万苦大兵就要拿着刀,奔高经略要钱。

		高第终于明白,为什么东林党都倒了,孙承宗还没倒,要论狡猾,他才刚起步。

		但高先生的劣根性根深蒂固,整人不成,又开始整地方。

		他一直认为,把防线延伸到锦州、宁远,是不明智的行为,害得经略大人暴露在辽东如此危险的地方,有家都回不去,于心何忍?

		还不如放弃整个辽东,退守到山海关,就算失去纵深阵地,就算敌人攻破关卡,至少自己是有时间跑路的。

		他不但这么想,也这么干。

		天启五年(1625年)十一月,高第下令,撤退。

		撤退的地方包括锦州、松山、杏山、宁远、右屯、塔山、大小凌河,总之关外的一切据点,全部撤走。

		撤退的物资包括:军队、平民、枪械、粮食,以及所有能搬走的物件。

		他想回家,且不想再来。

		但老百姓不想走,他们的家就在这里,他们已经失去很多,这是他们仅存的希望。

		但他们没有选择,因为高先生说了,必须要走,“家毁田亡,嚎哭震天”,也得走。

		高第逃走的时候,并没有追兵,但他逃走的动作实在太过逼真,跑得飞快,看到司令跑路,小兵自然也跑,孙承宗积累了几年的军事物资、军粮随即丢弃一空。

		数年辛苦努力,收复四百余里江山,十余万军队,几百个据点,就这样毁于一旦。

		希望已经断绝,东林党垮了,孙承宗走了,所谓关宁防线,已名存实亡,时局已无希望,很快,努尔哈赤的铁蹄,就会毫不费力地踩到这片土地上。

		没有人想抵抗,也没有人能抵抗,跑路,是唯一的选择。

		有一个人没有跑。

		他看着四散奔逃的人群,无法控制的混乱,说出了这样的话:

		“我是宁前道,必与宁前共存亡!我绝不入关,就算只我一人,也要守在此处\footnote{独卧孤城。},迎战敌人!”

		宁前道者,文官袁崇焕。

		\subsection{袁崇焕}
		\begin{quote}
			\begin{spacing}{0.5}  %行間距倍率
				\textit{{\footnotesize
							\begin{description}
								\item[\textcolor{Gray}{\faQuoteRight}] 若夫以一身之言动、进退、生死,关系国家之安危、民族之隆替者,于古未始有之。有之,则袁督师其人也。——梁启超
							\end{description}
						}}
			\end{spacing}
		\end{quote}

		关于袁崇焕的籍贯,是有纠纷的。他的祖父是广东东莞人,后来去了广西滕县,这就有点麻烦,名人就是资源,就要猛抢,东莞说他是东莞人,滕县说他是藤县人,争到今天都没消停。

		但无论是东莞,还是滕县,当年都不是啥好地方。

		明代的进士不少,但广东和广西的很少,据统计,70%以上都是江西、福建、浙江人。特别是广西,明代二百多年,一个状元都没出过。

		袁崇焕就在广西读书,且自幼读书,因为他家是做生意的,那年头做生意的没地位,要想出人头地,只有读书。

		就智商而言,袁崇焕是不低的,他二十三岁参加广西省统一考试,中了举人,当时他很得意,写了好几首诗庆祝,以才子自居。

		一年后他才知道,自己还差得很远。

		袁崇焕去北京考进士了,不久之后,他就回来了。

		三年后,他又去了,不久之后,又回来了。

		三年后,他又去了,不久之后,又回来了。

		以上句式重复四遍,就是袁崇焕同学的考试成绩。

		从二十三岁,一直考到三十五岁,考了四次,四次落榜。

		万历四十七年(1619年),袁崇焕终于考上了进士,他的运气很好。

		他的运气确实很好,因为他的名次,是三甲第四十名。

		明代的进士录取名额,大致是一百多人,是按成绩高低录取的,排到三甲第四十名,说明他差点没考上。

		关于这一点,我曾去国子监的进士题名碑上看过,在袁崇焕的那科石碑上,我找了很久,才在相当靠下的位置\footnote{按名次,由上往下排。},找到他的名字。

		在当时,考成这样,前途就算是交代了,因为在他之前,但凡建功立业、匡扶社稷,如徐阶、张居正、孙承宗等人,不是一甲榜眼,就是探花,最次也是个二甲庶吉士。

		所谓出将入相,名留史册,对位于三甲中下层的袁崇焕同志而言,是一个梦想。

		当然,如同许多成功人士\footnote{参见朱重八、张居正。}一样,袁崇焕小的时候,也有许多征兆,预示他将来必定有大出息。比如他放学回家,路过土地庙,当即精神抖擞,开始教育土地公:土地公,为何不去守辽东?!

		虽然我很少跟野史较真,但这个野史的胡说八道程度,是相当可以的。

		袁崇焕是万历十二年(1584年)生人,据称此事发生于他少年时期,往海了算,二十八岁时说了这话,也才万历四十年,努尔哈赤先生是万历四十六年才跟明朝干仗的,按此推算,袁崇焕不但深谋远虑,还可能会预知未来。

		话虽如此,但这种事总有人信,总有人讲,忽悠个上千年都不成问题。

		比如那位著名的预言家查诺丹马斯,几百年前说世纪末全体人类都要完蛋,传了几百年,相关书籍、预言一大堆,无数人信,搞得政府还公开辟谣。

		我曾研习欧洲史,对这位老骗子,倒还算比较了解,几百年后不去管它,当年他曾给法兰西国王查理二世算命,说:国王您身体真是好,能活到九十岁。

		查理二世很高兴,后来挂了,时年二十四岁。

		总之,就当时而言,袁崇焕肯定是个人才\footnote{全国能考前一百名,自然是个人才。},但相比而言,不算特别显眼的人才。

		接下来的事充分说明了这点,由于太不起眼,吏部分配工作的时候,竟然把这位仁兄给漏了,说是没有空闲职位,让他再等一年。

		于是袁崇焕在家待业一年,万历四十八年(1620年),他终于得到了人生中的第一个职务:福建邵武知县。

		邵武,今天还叫邵武,位于福建西北,在武夷山旁边,换句话说,是山区。

		在这个山区县城,袁崇焕干得很起劲,很积极,丰功伟绩倒说不上,但他曾经爬上房梁,帮老百姓救火,作为一个县太爷,无论如何,这都是不容易的。

		至于其他光辉业绩,就不得而知了,毕竟是个县城,要干出什么惊天动地的好事,很难。

		天启二年(1622年),袁崇焕接到命令,三年任职期满,要去北京述职。

		改变命运的时刻到来了。

		明代的官员考核制度,是十分严格的,京城的就不说了,京察六年一次,每次都掉层皮。即使是外面天高皇帝远的县太爷,无论是偏远山区,还是茫茫沙漠,只要你还活着,轮到你了,就得到本省布政使那里报到,然后由布政使组团,大家一起上路,去北京接受考核。

		考核结果分五档,好的晋升,一般的留任,差点的调走,没用的退休,乱来的滚蛋。

		袁崇焕的成绩大致是前两档,按常理,他最好的结局应该是回福建,升一级,到地级市接着干慢慢熬。

		但袁崇焕的运气实在是好得没了边,他不但升了官,还是京官。

		因为一个人看中了他。

		这个人的名字叫侯恂,时任都察院御史,东林党人。

		侯恂是个不出名的人,级别也低,但很擅长看人,是骡子是马,都不用拉出来,看一眼就明白。

		当他第一次看到袁崇焕的时候,就认定此人非同寻常,必可大用,这一点,袁崇焕自己都未必知道。

		更重要的是,他的职务虽不高,却是御史,可以直接向皇帝上书。所以他随即写了封奏疏,说我发现了个人才,叫袁崇焕,希望把他留用。

		当时正值东林党当政,皇帝大人还管管事,看到奏疏,顺手就给批了。

		几天后,袁崇焕接到通知,他不用再回福建当知县了,从今天起,他的职务是,兵部职方司主事,六品。

		顺便说句,提拔了袁崇焕的这位无名侯恂,有个著名的儿子,叫做侯方域,如果不知道此人,可以去翻翻《桃花扇》。

		接下来的事情十分有名,各种史料上都有记载:兵部职方司主事袁崇焕突然失踪,大家都很着急,四处寻找,后来才知道,刚上任的袁主事去山海关考察了。

		这件事有部分是真的,袁崇焕确实去了山海关。但猫腻在于,袁大人失踪绝不是什么大事,也没那么多人找他。当时广宁刚刚失陷,皇帝拉着叶向高的衣服,急得直哭,乱得不行,袁主事无非是个处级干部,鬼才管他去哪。

		袁崇焕回来了,并用一句话概括了他之后十余年的命运:

		“予我兵马钱粮,我一人足守此!”

		在当时说这句话,胆必须很壮,因为当时大家认定,辽东必然丢掉,山海关迟早失守,而万恶的朝廷正四处寻找背黑锅的替死鬼往那里送,守辽东相当于判死刑,闯辽东相当于闯刑场。这时候放话,是典型的没事找死。

		事情确实如此,袁崇焕刚刚放话,就升官了。因为朝廷听说了袁崇焕的话,大为高兴,把他提为正五品山东按察司佥事,山海关监军,以表彰他勇于背黑锅的勇敢精神。

		大家听到这个消息,不管认识的,还是不认识的,都纷纷来为袁崇焕送行,有的还带上了自己的子女,以达到深刻的教育意义:看到了吧,这人就要上刑场了,看你还敢胡乱说话!

		在一片哀叹声中,袁崇焕高高兴兴地走了,几个月后,他遇到了上司王在晋,告了他的黑状,又几个月后,他见到了孙承宗。

		且慢,且慢,在见到这两个人之前,他还遇见了另一个人,而这次会面是绝不能忽略的。

		因为在会面中,袁崇焕确定了一个秘诀,四年后,努尔哈赤就败在了这个秘诀之上。

		离开京城之前,袁崇焕去拜见了熊廷弼。

		熊廷弼当时刚回来,还没进号子,袁崇焕上门的时候,他并未在意。在他看来,这位袁处长,不过是前往辽东挨踹的另一个菜鸟。

		所以他问:

		“你去辽东,有什么办法吗\footnote{操何策以往。}?”

		袁崇焕思考片刻,回答:

		“主守,后战。”

		熊廷弼跳了起来,他兴奋异常,因为他知道,眼前的这个人已经找到了制胜的道路。

		所谓主守后战,就是先守再攻,说白了就是先让人打,再打人。

		这是句十分简单的话。

		真理往往都很简单。

		正如毛泽东同志那句著名的军事格言:打得赢就打,打不赢就走。很简单,很管用。

		一直以来,明朝的将领们绞尽脑汁,挖坑,造枪,练兵,修碉堡,只求能挡住后金军前进的步伐。

		其实要战胜天下无双的努尔哈赤和他那可怕的骑兵,只要这四个字。

		这四个字他们并非不知道,只是不想知道。

		作为大明天朝的将领,对付辽东地区的小小后金,即使丢了铁岭、丢了沈阳、辽阳,哪怕辽东都丢干净,也要打。

		所以就算萨尔浒死十万人,沈阳死六万人,也要攻。

		这不是智力问题,而是态度问题。

		后金军队不过是抢东西的强盗,努尔哈赤是强盗头,对付这类货色,怎么能当缩头乌龟呢?

		然而袁崇焕明白,按努尔哈赤的实力和级别,就算是强盗,也是巨盗。

		他还明白,缩头的,并非一定是乌龟,毒蛇在攻击之前,也要收脖子。

		后金骑兵很强大,强大到明朝骑兵已经无法与之对阵,努尔哈赤很聪明,聪明到这个世上已无几人可与之抗衡。

		抱持着此种理念,袁崇焕来到辽东,接受了孙承宗的教导。在那里,他掌握胜利的手段,寻找胜利的帮手,坚定胜利的信念。而与此同时,局势也在一步步好转,袁崇焕相信,在孙承宗的指挥下,他终将看到辽东的光复。

		然而这一切注定都是幻想。

		天启五年(1625年)十月,他所信赖和依靠的孙承宗走了。

		走时,袁崇焕前去送行,失声痛哭,然而孙承宗只能说:事已至此,我已无能为力。

		然而高第来了,很快,他就看见高大人丢弃了几年来,他为之奋斗的一切,土地、防线、军队、平民,毫不吝惜,只为保住自己的性命。

		袁崇焕不撤退,虽然他只是个无名小卒,无足轻重,但他有保国的志向,制胜的方法,以及坚定的决心。

		在过去的几年里,我一直这里,默默学习,默默进步,直到有一天,我看到了胜利的希望。

		所以我不会撤退,即使你们全都逃走,我也绝不撤退。

		“我一人足守此!”

		“独卧孤城,以当虏耳!”

		现在,履行诺言的时候到了。

		但这个诺言注定是很难兑现的,因为两个月后,他获知了一个可怕的消息。

		天启六年(1626年)正月十四日,努尔哈赤来了,带着全部家当来了。

		根据史料分析,当时后金的全部兵力,如果加上老头、小孩、残疾人,大致在十万左右,而真正的精锐部队,约有六七万人。

		努尔哈赤的军队,人数共计六万人,号称二十万。

		按某些军事专家的说法,这是当时世界上最为强大的骑兵部队,对于这个说法,我认为比较正确。

		理由十分简单:对他们而言,战争是一种乐趣。

		由于处于半开化状态,也不在乎什么诗书礼仪,传统道德,工作单位,打小就骑马,骁勇无畏,说打就打绝不含糊,更绝的是,家属也大力支持:

		据史料记载,后金骑兵出去拼命前,家里人从不痛哭流涕,悲哀送行,也不报怨政府,老老少少都高兴得不行,跟过节似的,千言万语化作一句话,多抢点东西回来!

		坦白地讲,我很能理解这种心情,啥产业结构都没有,又不大会种地,做生意也不在行,不抢怎么办?

		所以他们来了,带着抢掠的意图、锋锐的马刀和胜利的把握。

		努尔哈赤是很有把握的,此前,他已等待了四年,自孙承宗到任时起。

		一个卓越的战略家,从不会轻易冒险,努尔哈赤符合这个条件,他知道孙承宗的可怕,所以从不敢惹这人,但是现在孙承宗走了。

		当年秦桧把岳飞坑死了,多少还议了和,签了合同,现在魏忠贤把孙承宗整走,却是毫无附加值,还附送了许多礼物,礼单包括锦州、松山、杏山、右屯、塔山、大小凌河以及关外的所有据点。

		这一年,努尔哈赤六十七岁,就目前史料看,没有老年痴呆的迹象,他还有梦想,梦想抢掠更多的人口、牲畜、土地,壮大自己的子民。

		公正地讲,站在他的立场上,这一切都无可厚非。

		孙承宗走了,明军撤退了,眼前已是无人之地,很明显,他们已经失去了抵抗的勇气。

		进军吧,进军到前所未至的地方,取得前所未有的胜利,无人可挡!

		一切都很顺利,后金军毫不费力地占领了大大小小的据点,没有付出任何代价,直到正月二十三日那一天。

		天启六年(1626年)正月二十三日,努尔哈赤抵达了宁远城郊,惊奇地发现,这座城市竟然有士兵驻守,于是他派出了使者。

		他毫不掩饰自己的得意,写出了如下的话:

		“我带二十万人前来攻城,必破此城!如果你们投降,我给你们官做。”

		在这封信中,他没有提及守将袁崇焕的姓名,要么是他不知道这个人,要么是他知道,却觉得此人不值一提。

		总之在他看来,袁崇焕还是方崇焕都不重要,这座城市很快就会投降,并成为努尔哈赤旅游团路经的又一个观光景点。

		三天之后,他会永远记住袁崇焕这个名字。

		他原以为要等一天,然而下午,城内的无名小卒袁崇焕就递来了回信:

		“这里原本就是你不要的地方,我既然恢复,就应当死守,怎么能够投降呢?”

		然后是幽默感:

		“你说有二十万人,我知道是假的,只有十三万而已,不过我也不嫌少!”

		\ifnum\theparacolNo=2
	\end{multicols}
\fi
\newpage
\section{决心}
\ifnum\theparacolNo=2
	\begin{multicols}{\theparacolNo}
		\fi
		\subsection{胜利之路}
		努尔哈赤决定,要把眼前这座不听话的城市,以及那个敢调侃他的无名小卒彻底灭掉。

		他相信自己能够做到这一点,因为他已确知,这是一座孤城,在它的前方和后方,没有任何援军,也不会有援军,而在城中抵挡的,只是一名不听招呼的将领,和一万多孤立无援的明军。

		六年前,在萨尔浒,他用四万多人,击溃了明朝最为精锐的十二万军队,连在朝鲜打得日本人屁滚尿流的名将刘綎,也死在了他的手上。

		现在,他率六万精锐军队,一路所向披靡,来到了这座小城,面对着仅一万多人的守军,和一个叫袁崇焕的无名小卒。

		胜负毫无悬念。

		对于这一点,无论是努尔哈赤以及他手下的四大贝勒,还是明朝的高第、甚至孙承宗,都持相同观点。
		\begin{quote}
			\begin{spacing}{0.5}  %行間距倍率
				\textit{{\footnotesize
							\begin{description}
								\item[\textcolor{Gray}{\faQuoteRight}] 我们的同志在困难的时候,要看到成绩,要看到光明,要提高我们的勇气。——毛泽东
							\end{description}
						}}
			\end{spacing}
		\end{quote}

		袁崇焕是相信光明的,因为在他的手中,有四种制胜的武器。

		第一种武器叫死守,简单说来就是死不出城,任你怎么打,就不出去,死也死在城里。

		虽然这个战略比较怂,但很有效,你有六万人,我只有一万人,凭什么出去让你打?有种你打进来,我就认输。

		他的第二种武器,叫红夷大炮。

		大炮,是明朝的看家本领,当年打日本的时候,就全靠这玩意,把上万鬼子送上天,杀人还兼带毁尸功能,实在是驱赶害虫的不二利器。

		但这招在努尔哈赤身上,就不大中用了,因为日军的主力是步兵,而后金都是骑兵,速度极快,以明代大炮的射速和质量,没打几炮马刀就招呼过来了。

		袁崇焕清楚这一点,但他依然用上了大炮——进口大炮。

		红夷大炮,也叫红衣大炮,纯进口产品,国外生产,国外组装。

		我并非瞧不起国货,但就大炮而言,还是外国的好。其实明代的大炮也还凑合,在小型手炮上面\footnote{小佛郎机。},还有一定技术优势,但像大将军炮这种大型火炮,就出问题了。

		这是一个无法攻克的技术问题——炸膛。

		大家要知道,当时的火炮,想把炮弹打出去,就要装火药,炮弹越重,火药越多,如果火药装少了,没准炮弹刚出炮膛就掉地上了,最大杀伤力也就是砸人脚,可要是装多了,由于炮管是一个比较封闭的空间,就会内部爆炸,即炸膛。

		用哲学观点讲,这是一个把炸药填入炮膛,却只允许其冲击力向一个方向\footnote{前方。}前进的二律背反悖论。

		这个问题到底怎么解决,我不知道,袁崇焕应该也不知道,但外国人知道,他们造出了不炸膛的大炮,并几经辗转,落在了葡萄牙人的手里。

		至于这炮到底是哪产的,史料有不同说法。有的说是荷兰,有的说是英国,罗尔斯罗伊斯还是飞利浦,都无所谓,好用就行。

		据说这批火炮共有三十门,经葡萄牙倒爷的手,卖给了明朝。拿回来试演,当场就炸膛了一门\footnote{绝不能迷信外国货。},剩下的倒还能用,经袁崇焕请求,十门炮调到宁远,剩下的留在京城装样子。

		这十门大炮里,有一门终将和努尔哈赤结下不解之缘。

		为保证大炮好用,袁崇焕还专门找来了一个叫孙元化的人。按照惯例,买进口货,都要配发中文说明书,何况是大炮。葡萄牙人很够意思,虽说是二道贩子,没有说明书,但可以搞培训,就专门找了几个中国人,集中教学,而孙元化就是葡萄牙教导班的优秀学员。

		袁崇焕的第三种武器,叫做坚壁清野。

		为了保证不让敌人抢走一粒粮,喝到一滴水,袁崇焕命令,烧毁城外的一切房屋、草料,将所有居民转入城内。此外,他还干了一件此前所有努尔哈赤的对手都没有干过的事——清除内奸。

		努尔哈赤是个比较喜欢耍阴招的人,对派奸细里应外合很有兴趣,此前的抚顺、铁岭、辽阳、沈阳、广宁都是这么拿下的。

		努尔哈赤不了解袁崇焕,袁崇焕却很了解努尔哈赤,他早摸透了这招,便组织了除奸队,挨家挨户查找外来人口,遇到奸细立马干掉,并且派民兵在城内站岗,预防奸细破坏。

		死守、大炮、坚壁清野,但这还不够,远远不够,努尔哈赤手下的六万精兵,已经把宁远团团围住,突围是没有希望的,死守是没有援兵的,即使击溃敌人,他们还会再来,又能支撑多久呢?

		所以最终将他带上胜利之路的,是最后一种武器。

		这件武器,从一道命令开始。

		布置外防务后,袁崇焕叫来下属,让他立即到山海关,找到高第,向他请求一件事。

		这位部下清楚,这是去讨援兵,但他也很迷茫,高先生跑得比兔子都快,才把兵撤回去,怎么可能派兵呢?

		“此行必定无果,援兵是不会来的。”

		袁崇焕镇定地回答:

		“我要你去,不是讨援兵的。”

		“请你转告高大人,我不要他的援兵,只希望他做一件事。”

		“如发现任何自宁远逃回的士兵或将领,格杀勿论!”

		这件武器的名字,叫做决心。

		我没有朝廷的支持,我没有老师的指导,我没有上级的援兵,我没有胜利的把握,我没有幸存的希望。

		但是,我有一个坚定的信念。

		我不会后退,我会坚守在这里,战斗到最后一个人,即使同归于尽,也绝不后退。

		这就是我的决心。

		正月二十四日的那一天,战争即将开始之前,袁崇焕召集了他的所有部下,在一片惊愕声中,向他们跪拜。

		他坦白地告诉所有人,不会有援兵,不会有帮手,宁远已经被彻底抛弃。

		但是我不想放弃,我将坚守在这里,直到最后一刻。

		然后他咬破中指写下血书,郑重地立下了这个誓言。

		我不知道士兵们的反应,但我知道,在那场战斗中,在所有坚守城池的人身上,只有勇气、坚定和无畏,没有懦弱。

		天启六年正月二十四日晨,努尔哈赤带着轻蔑的神情,发动了进攻的命令,声势浩大的精锐后金军随即涌向孤独的宁远城。

		必须说明,后金军攻城,不是光膀子去的,他们也很清楚,骑着马是冲不上城墙的,事实上,他们有一套相当完整的战术系统,大致有三拨人。

		每逢攻击时,后金军的前锋,都由一种特别的兵种担任——楯兵。所有的楯兵都推着楯车。所谓楯车,是一种木车,在厚木板的前面裹上几层厚牛皮,泼上水,由于木板和牛皮都相当皮实,明军的火器和弓箭无法射破,这是第一拨人。

		第二拨是弓箭手,躲在楯车后面,以斜四十五度角向天上射箭\footnote{射程很远。},甭管射不射得中,射完就走人。

		最后一拨就是骑兵,等前面都忙活完了,距离也就近了,冲出去砍人效果相当好。

		无数明军就是这样被击败的,火器不管用,骑兵砍不过人家,只好就此覆灭。

		这次的流程大致相同,无数的楯兵推着木车,向着城下挺进,他们相信,城中的明军和以往没有区别,火器和弓箭将在牛皮面前屈服。

		然而牛皮破了。

		架着云梯的后金军躲在木板和牛皮的后面,等待靠近城墙的时刻,但他们等到的,只是晴天的霹雳声,以及从天而降的不明物体。

		值得庆祝的是,他们中的许多人还是俯瞰到了宁远城的全貌——在半空中。

		宁远城头的红夷大炮,以可怕的巨响,喷射着灿烂的火焰,把无数的后金军,他们破碎的楯车,以及无数张牛皮,都送上了天空——然后是地府。

		关于红夷大炮的效果,史书中的形容相当贴切且耸人听闻:“至处遍地开花,尽皆糜烂”。

		当第一声炮响的时候,袁崇焕不在城头,他正在接见外国朋友——朝鲜翻译韩瑗。

		巨响吓坏了朝鲜同志,他惊恐地看着袁崇焕,却只见到一张笑脸,以及轻松的三个字:

		“贼至矣!”

		几个月前,当袁崇焕决心抵抗之时,就已安排了防守体系,总兵满桂守东城,参将祖大寿守南城,副将朱辅守西城,副总兵朱梅守北城,袁崇焕坐镇中楼,居高指挥。

		四人之中,以满桂和祖大寿的能力最强,他们守护的东城和南城,也最为坚固。

		后金军是很顽强的,在经历了重大打击后,他们毫不放弃,踩着前辈的尸体,继续向城池挺进。

		他们选择的主攻方向,是西南面。

		这个选择不是太好,因为西边的守将是朱辅,南边的守将是祖大寿,所以守护西南面的,是朱辅和祖大寿。

		更麻烦的是,后金军刚踏着同志们的尸体冲到了城墙边,就陷入了一个奇怪的境地。

		攻城的方法,大抵是一方架云梯,拼命往上爬,一方扔石头,拼命不让人往上爬,只要皮厚硬头皮,冲上去就赢了。

		可是这次不同,城下的后金军惊奇地发现,除顶头挨炮外,他们的左侧、右侧、甚至后方都有连绵不断的炮火袭击,可谓全方位、全立体,无处躲闪,痛不欲生。

		这个痛不欲生的问题,曾让我百思不得其解,后来我去了一趟兴城\footnote{今宁远。},又查了几张地图,解了。

		简单地讲,这是一个建筑学问题。

		要说清这个问题,应该画几个图,可惜我画得太差,不好拿出来丢人,只好用汉字代替了,看懂就行。

		大家知道,一般的城池,是“口”字型,四四方方,一方爬,一方不让爬,比较厚道。

		更猛一点的设计,是“凹”字型,敌军进攻此类城池时,如进入凹口,就会受到左中右三个方向的攻击,相当难受。

		这种设计常见于大城的内城,比如北京的午门,西安古城墙的瓮城,就是这个造型。

		或者是城内有点兵,没法拉出去打,又不甘心挨打的,也这么修城,杀点敌人好过把瘾。

		但我查过资料兼实地观查之后,才知道,创意是没有止境的。

		宁远的城墙,大致是个“山”字。

		也就是说,在城墙的外面,伸出去一道城楼,在这座城楼上派兵驻守,会有很多好处,比如敌人刚进入山字的两个入口时,就打他们的侧翼,敌人完全进入后,就打他们的屁股。如果敌人还没有进来,在城头上架门炮,可以提前把他们送上天。

		此外,这个设计还有个好处,敌人冲过来的时候,有这个玩意,可以把敌人分流成两截,分开打。

		当然疑问也是有的,比如把城楼修得如此靠前,几面受敌,如果敌人集中攻打城楼,该怎么办呢?

		答案:随便打,无所谓。

		因为这座城楼伸出去,就是让人打的。而且我查了一下,这座城楼可能是实心的,下面没有通道,士兵调遣都在城头上进行,也就是说,即使你把城楼拆了,还得接着啃城墙,压根就进不了城。

		我不知道这城楼是谁设计的,只觉得这人比较狠。

		除地面外,后金军承受了来自前、后、左、右、上\footnote{天上。}五个方向的打击,他们能够得到的唯一遮挡,就是同伴的尸体,所以片刻之间,已经尸横遍野,血流成河。

		然而进攻者没有退缩,无功而返,努尔哈赤的面子且不管,啥都没弄到,回去怎么跟老婆孩子交代?

		在残酷的现实面前,后金军终于爆发了。

		虽然不断有战友飞上天空,但他们在尸体的掩护下,终究还是来到了城下,开始架云梯。

		然而炮火实在太猛,天上还不断掉石头,弓箭火枪不停地打,刚架上去,就被推下来,几次三番,他们爬墙的积极性受到了沉重的打击,于是决定改变策略——钻洞。

		具体施工方法是,在头上盖牛皮木板,用大斧、刀剑对着城墙猛劈,最终的工程目的,是把城墙凿穿。

		这是一个难度很大的工程,头顶上经常高空抛物不说,还缺乏重型施工机械,就凭人刨,那真是相当之困难。

		但后金军用施工成绩证明,他们之前的一切胜利,都不是侥幸取得的。

		在寒冷的正月,后金挖墙队顶着炮火,凭借刀劈手刨,竟然把坚固的城墙挖出了几个大洞,按照史料的说法,是“凿墙缺二丈者三四处”,也就是说,二丈左右的缺口,挖出了三四个。

		明军毫无反应。

		不是没反应,而是没办法反应,因为城头的大炮是有射程的,敌人若贴近城墙,就会进入射击死角,炮火是打不着的,而火枪、弓箭都无法穿透后金军的牛皮,只能眼睁睁地看着对方紧张施工,毫无办法。

		就古代城墙而言,凿开两丈大的洞,就算是致命伤了,一般都能塌掉,但奇怪的是,洞凿开了,城墙却始终不垮。

		原因在于天冷,很冷。

		按史料分析,当时的温度大致在零下几十度,城墙的地基被冰冻住,所以不管怎么凿,就是垮不下来。

		但袁崇焕很着急,因为指望老天爷,毕竟是不靠谱的,按照这个工程进度,没过多久,城墙就会被彻底凿塌,六万人涌进来,说啥都没用了。

		当务之急,要干掉城下的那帮牛皮护身的工兵,然而大炮打不着,火枪没有用,如之奈何?

		关键时刻,群众的智慧发挥了最为重要的作用。

		城墙即将被攻破之际,城头上的明军突然想出了一个反击的方法。

		这个方法有如下步骤,先找来一张棉被,铺上稻草,并在里面裹上火药,拿火点燃,扔到城下。

		棉被、稻草加上火药,无论是材料,还是操作方法,都是平淡无奇的,但是效果,是非常恐怖的。

		几年前,我曾找来少量材料,亲手试验过一次,这次实验的直接结果是,我再没有试过第二次,因为其燃烧的速度和猛烈程度,只能用可怕两个字形容。\footnote{特别提示,该实验相当危险,切勿轻易尝试,切勿模仿,特此声明。。}

		明军把棉被卷起来,点上火,扔下去,转瞬间,壮观的一幕出现了。

		沾满了火药的棉被开始剧烈燃烧,开始四处飘散,漂到哪里,就烧到哪里,只要沾上,就会陷入火海,即使就地翻滚,也毫无作用。

		在冰天雪地的严寒中,伴随着恐怖的大炮轰鸣声,一道火海包围了宁远城,把无数的后金军送入了地狱,英勇的后金工程队全军覆没。

		这种临时发明的武器,就是鼎鼎大名的“万人敌”,从此,它被载入史册,并成为世界上最早的燃烧瓶的雏形。

		\subsection{战斗,直至最后一人}
		眼前的一切,都超出了努尔哈赤的想象,以及心理承受程度。

		万历十二年(1584年),他二十五岁,以十三副盔甲起兵,最终杀掉了仇人尼堪外兰,而那一年,袁崇焕才刚刚出生。

		他跟随过李成梁,打败过杨镐,杀掉了刘綎、杜松,吓走了王化贞,当他完成这些丰功伟业,名声大振的时候,袁崇焕只是个四品文官,无名小卒。

		之前几乎每一次战役,他都以少打多,以弱胜强,然而现在他带着前所未有的强大兵力,势不可挡之气魄,进攻兵力只有自己六分之一的小人物袁崇焕,输了。

		战无不胜,攻无不克,小本起家的天命大汗是不会输的,也是不能输的,即使伤亡惨重,即使血流成河,用尸体堆,也要堆上城头!

		所以,观察片刻之后,他决定改变攻击的方向——南城。

		这个决定充分证明,努尔哈赤同志是一位相当合格的指挥官。

		他认为,南城就快顶不住了。

		南城守将祖大寿同意这个观点。

		就实力而言,如果后金军全力攻击城池一面,明军即使有大炮,也盖不住对方人多,失守只是个时间问题。

		好在此前后金军缺心眼,好好的城墙不去,偏要往夹脚里跑,西边打,南边也打,被打了个乱七八糟,现在,他们终于觉醒了。

		知错就改的后金军转换方向,向南城涌去。

		我到宁远时,曾围着宁远城墙走了一圈,没掐表,但至少得半小时,宁远城里就一万多人,分摊到四个城头,也就两千多人。以每面城墙一公里长计算,每米守兵大致是两人。

		这是最乐观的估算。

		所以根据数学测算,面对六万人的拼死攻击,明军是抵挡不住的。

		事情发展与数学模型差不多,初期惊喜之后,后金军终于呈现出了可怕的战斗力,鉴于上面经常扔“万人敌”,墙就不去凿了,改爬云梯。

		冲过来的路上,被大炮轰死一批,冲到城脚,被烧死一批,爬墙,被弓箭、火枪射死一批。

		没被轰死、烧死,射死的,接着爬。

		与此同时,后金军开始组织弓箭队,对城头射箭,提供火力支援。

		在这种拼死的猛攻下,明军开始大量伤亡,南城守军损失达三分之一以上,许多后金军爬上城墙,与明军肉搏,形势十分危急。

		祖大寿战败前,袁崇焕赶到了。

		袁崇焕并不在城头,他所处的位置,在宁远城正中心的高楼。这个地方,我曾经去过,登上这座高楼,可以清晰地看到四城的战况。

		袁崇焕率军赶到南城,在那里,他投入了最后的预备队。

		长久以来的训练终于显现了效果,在强敌面前,明军毫无畏惧,与后金军死战,把爬上城头的人赶了回去。

		与此同时,为遏制后金军的攻势,明军采用了新战略——火攻。

		明军开始大量使用火具,除大炮、万人敌、火枪外,火球甚至火把,但凡是能点燃的,就往城下扔。

		这个战略是有道理的,你要知道,这是冬天,而冬天时,后金士兵是有几件棉衣的。

		战争是智慧的源泉,很快,更缺德的武器出现了,不知是谁提议,拉出了几条长铁索,用火烧红,甩到城下用来攻击爬墙的后金士兵。

		于是壮丽的一幕出现了,在北风呼啸中,几条红色的锁链在南城飘扬,它甩向哪里,惨叫就出现在哪里。

		在熊熊的烈火之中,后金的攻势被遏制了,尸体堆满宁远城下,却始终未能前进一步,直至黄昏。

		至此,宁远战役已进行一天,后金军伤亡惨重,死伤达一千余人,却只换来了几块城砖。

		然而战斗并没有结束。

		愤怒至极的努尔哈赤下达了一个出人意料的命令:夜战。

		夜战并不是后金的优势,但仗打到这个份上,缩头就跑,就是一个严肃的面子问题,努尔哈赤认定,敌人城池受损,兵力已经到达极限,只要再攻一次,宁远城就会彻底崩塌。

		在领导的召唤下,后金士兵举着火把,开始了夜间的进攻。

		正如努尔哈赤所料,他很快就等到了崩溃的消息,后金军的崩溃。

		几次拼死进攻后,后金的士兵们终于发现,他们确实在逐渐逼近胜利——用一种最为残酷的方法:

		攻击无果,伤亡很大,尸体越来越多,越来越厚,如果他们全都死光,是可以踩着尸体爬上去的。

		沉默久了,就会爆发,爆发久了,就会崩溃,在又一轮的火烧、炮轰、箭射后,后金军终于违背了命令,全部后撤。

		正月二十四日深夜,无奈的努尔哈赤接受了这个事实,他压抑住心中怒火,准备明天再来。

		但他不知道的是,如果他不放弃进攻,第二天历史将会彻底改变。

		袁崇焕也已顶不住了,他已经投入了所有的预备队,连他自己也亲自上阵,左手还负了伤,如果努尔哈赤豁出去再干一次,后果将不堪设想。

		努尔哈赤放弃了,他坚持了,所以他守住了宁远。

		而下一个问题是,能否击溃后金,守住宁远。

		从当天后金军的表现看,这个问题的答案是肯定的——不能。

		没有帮助,没有援军,修了几年的坚城,只用一天,就被打成半成品,敌人战斗力太过强悍,很明显,如果后金军豁出去,在这里待上几月,就是用手刨也刨下来了。

		对于这个答案,袁崇焕的心里是有数的。

		于是,他来到了最后一个问题:既然必定失守,还守不守?

		他决定坚守下去,即使全军覆没,毫无希望,也要坚持到底,坚持到最后一个人。
		\begin{quote}
			\begin{spacing}{0.5}  %行間距倍率
				\textit{{\footnotesize
							\begin{description}
								\item[\textcolor{Gray}{\faQuoteRight}] 军队应该具有一往无前的精神,它要压倒一切敌人,而决不被敌人所屈服。不论在任何艰难困苦的场合,只要还有一个人,这个人就要继续战斗下去。——毛泽东
							\end{description}
						}}
			\end{spacing}
		\end{quote}

		袁崇焕很清楚,明天城池或许失守,或许不失守,但终究是要失守的。以努尔哈赤的操行成绩,接踵而来的,必定是杀戮和死亡。

		然而袁崇焕不打算放弃,因为他是一个没有援军、没有粮食、没有理想、没有希望,依然能够坚持下去的人。

		四十二岁年前,袁崇焕出生于穷乡僻壤,一直以来,他都很平凡,平凡的中了秀才,平凡的中了举人,平凡的落榜,平凡的再次赶考,平凡的再次落榜,平凡的最终上榜。

		然后是平凡的知县,平凡的处级干部,平凡的四品文官,平凡的学生,直至他违抗命令,孤身一人,面对那个不可一世、强大无比的对手。

		四十年平凡的生活,不断的磨砺,沉默的进步,坚定的信念,无比的决心:

		只为一天的不朽。

		\ifnum\theparacolNo=2
	\end{multicols}
\fi
\newpage
\section{胜利结局}
\ifnum\theparacolNo=2
	\begin{multicols}{\theparacolNo}
		\fi
		正月二十五日。

		以前有个人对我说过这样一句话:

		只要你不放弃自己,上天就不会放弃你。

		绝境中的袁崇焕,在沉思中等来了正月二十五日的清晨,他终究没有放弃。

		于是,他等来了奇迹。

		天启六年(1626年)正月二十五日,改变历史的一天。

		努尔哈赤怀着满腔的愤怒,发动了新的进攻。他认为,经过前一天的攻击,宁远已近崩溃,只要最后一击,胜利触手可得。

		然而他想不到的是,战斗是以一种不可思议的形式开始的。

		第一轮进攻被火炮打退后,他看见勇猛的后金士兵们怂了。

		无论将领们怒吼,还是威胁,以往工作积极性极高的后金军竟然不买账了,任你怎么说,就是不冲。

		这是可以理解的,大家出来打仗,说到底是想抢点东西,发发小财,现在人家炮架上了,打死上千人,尸体都堆在那儿,还要往上冲,你当我们白内障看不见啊。

		勇敢,也是要有点智商的。

		努尔哈赤是很地道的,为了消除士兵们的恐惧心理,他毅然决定,停止进攻,把尸体捞回来先。

		为一了百了,他还特事特办,在城外开办了简易火葬场,什么遗体告别,追悼会都省了,但凡抢回来的尸体,往里一丢了事。

		烧完,接着打。

		努尔哈赤已近乎疯狂了,现在他所要的,并不是宁远,也不是辽东,而是脸面,起兵三十年,纵横天下无人可敌,竟然攻不下一座孤城,太丢人了,实在太丢人了。

		所以他发誓,无论如何,一定要争回这个面子。

		不想丢人,就只能丢命。

		面对蜂拥而上的后金军,袁崇焕的策略还是老一套——大炮。

		要说这外国货还是靠谱,顶在城头上轰了一天,非但没有炸膛,还越打越有劲,东一炮“尽皆糜烂”,西一炮“尽皆糜烂”,相当皮实。

		但是意外还是有的,具体说来是一起安全事故。

		很多古装电视剧里,大炮发射大致是这么个过程:一人站在大炮后,拿一火把点引线,引线点燃后轰一声,炮口一圈白烟,远处一片黑烟,这炮就算打出去了。

		可以肯定的是,如按此方式发射红夷大炮,必死无疑。

		我认为,葡萄牙人之所以卖了大炮还要教打炮,绝不仅是服务意识强,说到底,是怕出事。

		由于红夷大炮的威力太大,在大炮轰击时,炮尾炸药爆炸时,会产生巨大的后座力,巨大到震死人不成问题,所以每次发射时,都要从炮签出一条引线,人躲得远远的,拿火点燃再打出去。

		经过孙元化的培训,城头的明军大都熟悉规程,严格按安全规定办事,然而在二十五日这一天,由于城头忙不过来,一位通判也上去凑热闹,一手拿线,一手举火,就站在炮尾处点火,结果被当场震死。

		但除去这起安全事故外,整体情况还算正常,大炮不停地轰,后金军不停地死,然后是抢尸体,抢完再烧,烧完再打,打完再死,死完再抢、再烧,死死烧烧无穷尽也。

		直至那历史性的一炮。

		到底是哪一炮,谁都说不清,但可以肯定的是,在那寒冷的一天,漫天的炮火轰鸣声中,有一炮射向了城下,伴随着一片惊叫和哀嚎,命中了一个目标。

		这个目标到底是谁,至今不得要领,但可以肯定是相当重要的,因为一个不重要的人,不会坐在黄帐子里\footnote{并及黄龙幕。},也不会让大家如此悲痛\footnote{嚎哭奔去。}。

		对于此人身份,有多种说法,明朝这边,说是努尔哈赤,清朝那边,是压根不提。

		这也不奇怪,如果战无不胜的努尔哈赤,在一座孤城面前,对阵一个无名小卒,被一颗无名炮弹重伤,实在太不体面,换我,我也不说。

		于是接下来,袁崇焕看到了让他百思不得其解的景象,冲了两天的后金军退却了,退到了五里之外。

		很明显,坐在黄帐子里的那人,是个大人物,但按照后金的道德标准,死个把领导也不是什么大事,这实在是件相当奇怪的事情。

		第二天,当袁崇焕站在城头的时候,他终于确信,自己已经创造了奇迹。

		后金军仍然在攻城,攻势比前两天更为猛烈,但长期的军事经验告诉袁崇焕,这是撤退的前兆。

		几个时辰之后,后金军开始总退却。

		当然努尔哈赤是不会甘心的,所以在临走之前,他把所有的怒火发泄到了宁远城边的觉华岛上,那里还驻扎着几千明军,以及上万名无辜的百姓。

		那一年的冬天很冷,原本相隔几十里的大海,结上了厚厚的冰,失落的后金军踏着冰层,向岛上发动猛攻,毫无遮挡的明军全军覆没,此外,士兵屠杀了岛上所有的百姓\footnote{逢人立碎。},以显示努尔哈赤的雄才大略,并向世间证明,努尔哈赤先生并不是无能的,他至少还能杀害手无寸铁的平民。

		宁远之战就此结束,率领全部主力,拼死攻击的名将努尔哈赤,最终败给了仅有一万多人,驻守孤城的袁崇焕,铩羽而归。

		此战后金损失极为惨重,虽然按照后金的统计,仅伤亡将领两人,士兵五百人,但很明显,这是个相当谦虚的数字。

		数学应用题1:十门大炮轰六万人,轰了两天半,每炮每天只轰二十炮\footnote{最保守的数字。},问:总共轰多少炮?

		答:以两天计算,至少四百炮。

		数学应用题2:后金军总共伤亡五百人,以明军攻击数计算,平均每炮轰死多少人?

		答:以五百除以四百,平均每炮轰死1.25人。

		参考史料:“红夷大炮者,周而不停,每炮所中,糜烂数十尺,断无生理。”

		综合由应用题一、应用题二及参考资料,得出结论如下:每一个后金士兵,都有高厚度的装甲保护,是不折不扣的钢铁战士。

		扯淡就此结束,根据保守统计,在宁远战役中,后金军伤亡的人数,大致在四千人以上,损失大量攻城车辆、兵器。

		这是自万历四十六年以来,后金军的第一次总退却,战无不胜的努尔哈赤终于迎来了他人生的第一次战败。

		或许直到最后,他也没弄明白,到底是谁击败了他,那座孤独的宁远城,那几门外国进口的大炮,还是那一万多陷入绝境的明军。

		他不知道,他的真正对手,是一种信念。

		即使绝望,毫无生机,永不放弃。

		在那座孤独的城市里,有一个叫袁崇焕的人,在过去的几十年中,一直坚守着这样的信念。

		他不知道,也永远不会知道了。

		因为七个月后,他就翘辫子了。

		天启六年(1626年)八月十一日,征战半生的努尔哈赤终于逝世了。

		他的死因,有很多说法,有说是被炮弹打坏的,也有的说是病死的,但无论是病死还是打死,都跟袁崇焕有着莫大的关系。

		挨炮就不说了,那么大一铁陀子,外加各类散弹,穿几个窟窿不说,再加上破伤风,这人就废定了。

		就算他没挨炮,精神上也受到了严重的损害,有点心理障碍十分正常,外加努先生自打出道以来,从没吃过亏,败在无名小卒的手上,实在太丢面子,就这么憋屈死,也是很有可能的。

		在这一点上,袁崇焕也做出了很大贡献,在击退努尔哈赤后,他立即派出了使者,给努老先生送去了一封信,内容如下:

		“你横行天下这么久,今天竟然败在我的手里,应该是天命吧!”

		努尔哈赤很有礼貌,还派人回了礼,表示下次再跟你小子算帐\footnote{约期再战。}。

		至于努先生的内心活动,用他自己的话说,是这样的:

		“我自二十五岁起兵以来,攻无不克,战无不胜,小小的宁远,竟然攻不下来,这是命啊!”

		说完不久就死了。

		一代枭雄努尔哈赤死了,对于这个人的评价,众多纷纭,有些人说他代表了先进的,进步的势力,冲击了腐败的明朝,为历史的发展做出了贡献云云。

		我才疏学浅,不敢说通晓古今,但基本道理还是懂的,遍览他的一生,我没有看到进步、发展、只看到了抢掠、杀戮和破坏。

		我不清楚什么伟大的历史意义,我只明白,他的马队所到之处,没有先进生产力,没有国民生产指数,没有经济贸易,只有尸横遍野、残屋破瓦,农田变成荒地,平民成为奴隶。

		我不知道什么必定取代的新兴霸业,我只知道,说这种话的人,应该自己到后金军的马刀下面亲身体验。

		马刀下的冤魂和马鞍上的得意,没有丝毫区别,所有的生命,都是平等的,任何人都没有无故剥夺的权力。
		\ifnum\theparacolNo=2
	\end{multicols}
\fi
